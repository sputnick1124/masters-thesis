\chapter{Two-cart Flexible System}\label{c:acc}
\section{Introduction}
The 1992 American Control Conference published a set of benchmark problems to explore the capabilities of
modern control techniques\cite{wie1992benchmark}. The second stated problem solicited solutions to the
stabilization of a two-body spring mass system which was robust to uncertain masses and spring constants. This
benchmark problem focuses on disturbance rejection in minimum time. Stengel\cite{stengel1992robustness}
surveyed the best performing control strategies and quantified their stability and performance robustness
using stochastic robustness analysis. Cohen et. al \cite{cohen:01jgcd} independently proposed a solution using
fuzzy logic which has more robust stability, performance and uses less control effort.

This problem extends the benchmark problem with the addition of automatically tuned fuzzy controllers for
various plant models via a GA. The goal of the controller is to move the dual-mass system from a stationary
initial condition to a stationary end goal in minimum time. Simplifications to the problem include a
collocated sensor and actuator, removed plant disturbance, and removed sensor noise. The model dynamics are
similar to those experienced by a robotic arm moving from one control point to another. The system exhibits
both rigid and flexible body modes and thus provides a good test case for a nonlinear fuzzy controller. The
controller is first developed by hand and then automatic tuning strategies are employed using a GA. The plant
model masses are then changed significantly and the controller is retrained to test the efficacy of the GA
tuning process and to demonstrate the utility of having an automated tuning method. The result is a system
which can be quickly trained to accommodate new plant dynamics within this class of models. The performance of
the hand- and GA-tuned controllers are compared. 

\section{Coupled Spring-Mass Simulation Model}\label{s:model}
A simulated environment is necessary to test the efficacy of the proposed FIS. The simulation consists of two
cars connected by a spring; this system is expected to traverse a given distance as quickly as possible
without exceeding a certain boundary represented by a wall. The system is propelled by a force on the leading car
which represents the actuating output of the controller. At each time instant, the controller must determine
how much force to exert on the carts, and in which direction, to get as close as possible to the wall in
minimum time. The carts are represented by point masses which
may occupy the same position in space. The constants for the model are the weight of each car, the spring constant, and the distance
which must be traversed to the wall. A diagram of the simulation is shown in \cref{f:model}.

\begin{displaymath}
    m_{1}=\SI{1}{\kilogram}, \quad m_{2}=\SI{2}{\kilogram}, \quad K =
    \SI{250}{\newton\per\metre}, \quad L = \SI{100}{\metre}
\end{displaymath}
\nomenclature{$m$}{Cart mass,\si{\kilogram}}
\nomenclature{$K$}{Spring Constant, \si{\newton\per\metre}}
\nomenclature{$L$}{Simulation length, \si{\metre}}
\nomenclature[b$1$]{$1$}{Denotes property of trailing cart}
\nomenclature[b$2$]{$2$}{Denotes property of leading cart}
\nomenclature{$F$}{Control force, \si{\newton}}
\nomenclature{$t$}{Time, \si{\second}}

\begin{figure}
    \centering
    \includegraphics[width=0.9\textwidth]{images/model.png}
    \caption{Diagram of two rigid bodies connected by a spring traversing distance \textrm{L} in minimum
    time.} \label{f:model}
\end{figure}

The modeled system contains displacement and velocity sensors on each cart; therefore, the four inputs to the
FIS are the distances traveled and the velocities of both carts. These inputs represent the state vector of
the system. The output of the controller is the force, $F(t)$, applied to cart 2, limited to $\pm
\SI{1}{\newton}$. At each time step of the simulation, the FIS uses the state of cart 2 to determine the force
which must be exerted according to a fuzzy rule base. 

\begin{displaymath}
    \mathrm{Input:}\quad \vec{y}(t)= \begin{bmatrix} x_{1}(t)\\ x_{2}(t)\\ \dot{x}_{1}(t)\\
\dot{x}_{2}(t) \end{bmatrix}
\end{displaymath}
\begin{displaymath} \mathrm{Output:}\quad |F(t)|\le
\SI{1}{\newton}
\end{displaymath}
\nomenclature[1\(y\)]{$y$}{System state vector}
\nomenclature[1\(xx\)]{$\dot{x}(t)$}{Cart velocity, \si{\metre\per\second}}
\nomenclature[1\(xb\)]{$x(t)$}{Cart position, \si{\meter}}
At time $t=0$, both carts are at rest at position
$x=0$. The maximum allowed run time of the simulation is \SI{500}{\second}. The system requirement for the
final condition is that both carts come to rest between \SIrange{99}{100}{\metre} with minimal oscillation.
Neither cart is allowed at any point in the simulation to exceed \SI{100}{\metre}.
\begin{displaymath}
    \vec{y}_0=\begin{bmatrix}
\SI{0}{\metre}\\\SI{0}{\metre}\\\SI{0}{\metre\per\second}\\\SI{0}{\metre\per\second} \end{bmatrix},\quad
\vec{y}(500)=\begin{bmatrix} \SI{99}{\metre}<x_1<\SI{100}{\metre}\\ \SI{99}{\metre}<x_2<\SI{100}{\metre}\\ \SI{0}{\metre\per\second}\\
\SI{0}{\metre\per\second} \end{bmatrix}
\end{displaymath}
\nomenclature[b$0$]{$0$}{Denotes initial condition}

The acceleration of cart 1 is determined by the displacement between the two carts, the spring constant, and
the mass of the cart. The acceleration of cart 2 is also a function of these parameters as well as the control
force. The equations of motion for the system are represented by simple second-order differential equations.
\begin{equation}
\label{e:cart1} \mathrm{Cart 1:}\quad \ddot{x}_1=\frac{K}{m_1}(x_2-x_1) 
\end{equation}
\begin{equation}
    \label{e:cart2} \mathrm{Cart 2:}\quad \ddot{x}_2=\frac{K}{m_2}(x_1-x_2)+\frac{F}{m_2}
\end{equation} \nomenclature[1\(xxx\)]{$\ddot{x}(t)$}{Cart acceleration, \si{\metre\per\second\squared}}

\subsection{System Performance Cost}
The efficiency of a FIS's control of the system is based upon the amount of time
it expends traversing the distance to the wall and how close the carts are to the wall when they settle. Any
breach of the wall results in immediate system failure. A cost function, $J$, is defined by the settling time
$t_f$, the time taken to settle within \SI{1}{\metre} of the wall and the steady state error, the distance
between the leading cart and the wall. The control system that produces the lowest $J$ value will be proven
to be the most fit solution for the simulation. This cost function in particular was provided in order to
compare to results previously obtained\cite{walker:13p,vick:13p,mitchell:13p,stimetz:13p}.
\begin{equation}\label{e:timesettle}
t_f=|L-\bar{x}(t_f)|\le\SI{1}{\metre}
\end{equation} \nomenclature[b$f$]{$f$}{Denotes condition at settling time}
\begin{equation}\label{e:cost}
J=\frac{t_f}{100}+2[L-x_2(500)]
\end{equation}
where the constants 100 and 2 are scaling factors. In order to provide a frame of reference for the
performance of any control system, the theoretical limits of the simulation were calculated to provide a lower
bound for the value of \cref{e:cost}. Assuming a single rigid body assembly with no dynamic coupling, the
model is greatly simplified to a single equation where the acceleration of the body is a function of only the
control force. Since no energy is lost to a spring, the optimal solution to this system is assumed to be a
lower bound on the flexible system with losses.
\begin{equation}
    \label{e:simplemodel} \ddot{x}=\frac{F}{M}
\end{equation}
\nomenclature{$M$}{Total system mass, \si{\kilogram}}
where $M$ is the total mass of the system. The total mass of the system is \SI{3}{\kilogram}, whereas the
maximum input force is limited to \SI{1}{\newton}, rendering \cref{e:simplemodel}
\begin{displaymath}
        \ddot{x}=\frac{\SI{1}{\newton}}{\SI{3}{\metre}}=\SI[quotient-mode=fraction]{1/3}{\metre\per\second\squared}
\end{displaymath}
as the maximum acceleration. Given this acceleration and the distance to be traversed to the wall, the minimum
time to complete the trip can be calculated. There are, however, two scenarios to consider. 
\begin{enumerate} 
    \item Applying maximum force over the entire distance and instantaneously
        stopping the carts at (but not touching) the wall provides an absolute, if unfeasible, optimum
        simulation completed in minimum time. Given an initial velocity of zero and a constant accelerating
        force of \SI{1}{\newton}, the traversal time is calculated. Note that once the carts have breached
        \SI{99}{\metre}, the system may come to rest and be considered settled. 
        \begin{displaymath}
        x(t)=\dot{x}_0t+\frac{1}{2}\ddot{x}t^2,\quad \dot{x}_0=0
        \end{displaymath}
        \begin{displaymath}
        x(t)=\frac{1}{2}\ddot{x}t^2
        \end{displaymath}
        Letting $x(t) = \SI{99}{\metre}$
        \begin{displaymath}
            t=\sqrt{\frac{2x(t)}{\ddot{x}}}=\sqrt{\frac{2\cdot
            \SI{99}{\metre}}{\SI[quotient-mode=fraction]{1/3}{\metre\per\second\squared}}}=\SI{24.37}{\second}
        \end{displaymath}
    \item Applying maximum force over half of the distance and then applying maximum
        negative force in the second half to slow the velocities of the carts to zero at the wall position.

First half: \begin{displaymath}
t=\sqrt{\frac{2x(t)}{\ddot{x}}}=\sqrt{\frac{2\cdot\SI{50}{\metre}}{\SI[quotient-mode=fraction]{1/3}{\meter\per\second\squared}}}=\SI{17.23}{\second}
\end{displaymath} Traversing the second half and stopping at the wall takes the same amount of time, therefore
the total time expended in reaching \SI{100}{\metre} is \SI{34.46}{\second}; however, the interest lies in the
time taken to breach the \SI{99}{\metre} mark. The time taken to travel the last meter is \SI{2.45}{\second},
so the best possible time to reach the \SI{99}{\metre} position is : \begin{displaymath} t=\SI{32.19}{\second}
\end{displaymath} As this is a much more realistic scenario, this is the limit used as the benchmark in this
research. Using this time to evaluate \cref{e:cost} \begin{displaymath}
    J=\frac{32.19}{100}+2[100-100]=0.3219 \end{displaymath} results in the minimum possible cost. The addition
    of harmonic oscillation and non-linear dynamics ensures that this limit will not be reached, but merely
    provides a standard against which a controller may be measured.
\end{enumerate}

\section{Fuzzy Inference System}The FIS built during this project uses two measured inputs: the position and velocity of cart 2. The output of
the FIS is the force exerted on cart 2 at a given point in the simulation.

\subsection{Membership Functions and Rule Base} \subsubsection{Position} The first input variable, position,
is composed of three membership functions to which it can map. The position of the car, from
\SIrange{0}{100}{\metre}, is described in human-understandable language as far away, close to, or very close
to the wall. If the simulation has just begun and the cart is as far away from the wall as it can be, the
``Far'' membership function will map to 1 and the ``Close'' and ``VeryClose'' functions will evaluate to 0.
Conversely, at the end of the traverse, as the car approaches the wall, ``VeryClose'' will map to 1 and
``Far'' to 0. ``Close'' may evaluate to some value in between. The membership functions are shown in
\cref{f:x2mfs}. As the system predominately behaves as a rigid body on a large scale, the membership
functions mirror the ideal simulation of a rigid body for the majority of the carts' travel. Each function is
represented by a vector of values expressing the points at which the function switches from 0 to 1 or 1 to 0.
For the FIS used in this research, trapezoidal- and triangular-shaped membership functions are utilized.
Trapezoids are represented by a four-element vector as they start at 0, rise to 1, remain at 1, and finally
fall to 0. Likewise, triangular functions are expressed as three-element vectors. The parameters of the
position membership functions shown in \cref{f:x2mfs} are: \begin{displaymath} \mathrm{Far:}\quad
\begin{bmatrix} \SI{0}{\metre}\\\SI{0}{\metre}\\\SI{49.8}{\metre}\\\SI{50.1}{\metre}
\end{bmatrix}, \quad \mathrm{Close:}\quad \begin{bmatrix}
    \SI{49.8}{\metre}\\\SI{50.1}{\metre}\\\SI{99.9}{\metre}\\\SI{100}{\metre} \end{bmatrix}, \end{displaymath}
\begin{displaymath} \mathrm{VeryClose:}\quad \begin{bmatrix}
\SI{99.9}{\metre}\\\SI{100}{\metre}\\\SI{100.1}{\metre} \end{bmatrix} \end{displaymath}

\subsubsection{Velocity} The second input variable, velocity, is composed of three membership functions. The
velocity of the cart, within a range from \SIrange{-6}{6}{\metre\per\second}, can either be classified as
``Negative'', ``Zero'', or ``Positive''. When the simulation starts, the carts are at rest and the degree of
membership to ``Zero'' velocity will be 1 whereas ``Negative'' and ``Positive'' will be 0. For the majority of
the simulation, the carts are moving forward with a fast pace; therefore, the ``Negative'' membership function
does not come into play until close to the very end of the  simulation when the oscillatory effects dominate
the motion of the cart system. It is at this point that the true dynamic nature of the system is exhibited and
the controller does the most calculation in an attempt to damp the oscillations. The membership functions for
$\ddot{x}_2$ are shown in \cref{f:x4mfs}.
%\begin{figure}
    %\begin{subfigmatrix}{2}
        %%\subfigure[$x_2$ membership functions\label{f:x2mfs}]{\includegraphics{images/x2_mfs.pdf}}
        %\subfigure[$x_2$ membership functions\label{f:x2mfs}]{\input{tikz/accx2plot}}
        %%\subfigure[$\dot{x}_2$ membership functions\label{f:x4mfs}]{\includegraphics{images/x4_mfs.pdf}}
        %\subfigure[$\dot{x}_2$ membership functions\label{f:x4mfs}]{\input{tikz/accx2dotplot}}
    %\end{subfigmatrix} \caption{Input membership functions}\label{f:mfs}
%\end{figure}

\begin{figure}[ht]
    \centering
    \input{tikz/accx2plot}
    \caption{$x_2$ membership functions}\label{f:x2mfs}
\end{figure}

\begin{figure}[ht]
    \centering
    \input{tikz/accx2dotplot}
    \caption{$\dot{x}_2$ membership functions}\label{f:x4mfs}
\end{figure}


The velocity membership function parameters are: \begin{displaymath} \mathrm{Negative:}\quad
\begin{bmatrix}
\SI{-6}{\metre\per\second}\\\SI{-6}{\metre\per\second}\\\SI{-.2792}{\metre\per\second}\\\SI{0}{\metre\per\second}
\end{bmatrix}, \quad \mathrm{Zero:}\quad \begin{bmatrix}
    \SI{-0.2792}{\metre\per\second}\\\SI{0}{\metre\per\second}\\\SI{0.2792}{\metre\per\second} \end{bmatrix},
\end{displaymath} \begin{displaymath} \mathrm{Positive:}\quad \begin{bmatrix}
\SI{0}{\metre\per\second}\\\SI{0.2792}{\metre\per\second}\\\SI{6}{\metre\per\second}\\\SI{6}{\metre\per\second}
\end{bmatrix} \end{displaymath}
\subsubsection{Rule Base}\label{ss:rulebase}
As previously discussed, the rule base of an FIS is a series of IF-THEN (antecedent-consequent) conditions
that use the membership functions of all the inputs in order to determine the output variable. There are also
several membership functions for the output variable that the rule base maps to. Each of the rules has an
influence on what the output of the system should be. The weight of these influences again varies from 0 to 1
and the final output value is determined by calculating the centroid of all of the rules. For instance, one
rule dictates that if the car is far away then the output force should be positive and large so that the car
will move towards the wall. Another rule will say that when the car is close to the wall the output force
should be negative in order to slow the car down. All of these rules have influence over all input ranges
determined by how the input variables map to the given membership functions in that specific rule.

The antecedent of each statement contains memberships of both input variables and the consequent maps to the
output membership functions. The rules for this FIS were developed based upon intuitive decision making. The
rules developed for our FIS are shown in \cref{tab:rulebase}.  
\begin{table}[ht]
    \begin{center}
        \caption{Rule Base of the Fuzzy Inference System}\label{tab:rulebase}
        \begin{tabular}{ccccc} \multicolumn{2}{c}{}  & \multicolumn{3}{c}{Velocity Measurement}\\ \cline{3-5}
            \multicolumn{2}{c|}{}  & \multicolumn{1}{c|}{\textbf{Negative}} & \multicolumn{1}{c|}{\textbf{Zero}} & \multicolumn{1}{c|}{\textbf{Positive}} \\ \cline{2-5}
            \multicolumn{1}{c|}{\multirow{3}{*}{\parbox{3cm}{\centering Position\\Measurement}}} & \multicolumn{1}{c|}{\textbf{Far}} & \multicolumn{3}{c|}{Positive} \\ \cline{2-5}
            \multicolumn{1}{c|}{} & \multicolumn{1}{c|}{\textbf{Close}} & \multicolumn{1}{c|}{Positive} & \multicolumn{1}{c|}{Zero} & \multicolumn{1}{c|}{Negative}\\\cline{2-5}
            \multicolumn{1}{c|}{} & \multicolumn{1}{c|}{\textbf{VeryClose}} & \multicolumn{1}{c|}{Positive} & \multicolumn{1}{c|}{Zero} & \multicolumn{1}{c|}{Negative} \\ \cline{2-5}
            \multicolumn{2}{c}{}  & \multicolumn{3}{c}{Control Force}
        \end{tabular}
    \end{center}
\end{table}

\subsubsection{Control Force}
\begin{figure}[ht]
    \centering
    \input{tikz/accoutplot}
    \caption{Control force membership functions.}
    \label{f:fmfs}
\end{figure}
The output variable, force, is
also composed of three membership functions. The output force is bounded from \SIrange{-1}{1}{\newton} thus
the membership functions allow for the output to be ``Negative'', ``Zero'', or ``Positive''. As the centroid
of the area beneath each function is used to evaluate the control force, the upper and lower bounds are
centered over 1 and -1 respectively. These functions represent the ``defuzzification'' stage of the FIS and
these outputs are determined by evaluating each rule in the rule base (see \cref{ss:rulebase}) in
parallel\cite{matlab:12tb}. The controller emulates bang-bang control for the majority of the simulation
lifetime, sharply transitioning from full acceleration to full deceleration. As the cart system nears the
wall, the oscillation damping rules will dictate the force to apply on the system. Typically, the output force
will be directed opposite to the velocity of cart 2. The membership functions for the control force are shown
in \cref{f:fmfs}.
	
	
\begin{displaymath}
    \mathrm{Negative:}\quad 
        \begin{bmatrix}
            \SI{-2}{\newton}\\
            \SI{-1}{\newton}\\
            \SI{0}{\newton}
        \end{bmatrix}, \quad
    \mathrm{Zero:}\quad 
        \begin{bmatrix}
            \SI{-1}{\newton}\\
            \SI{0}{\newton}\\
            \SI{1}{\newton}
        \end{bmatrix}
\end{displaymath}
\begin{displaymath}
    \mathrm{Positive:}
    \quad 
        \begin{bmatrix}
            \SI{0}{\newton}\\
            \SI{1}{\newton}\\
            \SI{2}{\newton}
        \end{bmatrix}
\end{displaymath}

\subsection{Simulation Performance}\label{ss:simperf}
\begin{figure}[ht]
    \centering
    \begin{tikzpicture}[gnuplot]
%% generated with GNUPLOT 5.0p3 (Lua 5.1; terminal rev. 99, script rev. 100)
%% Thu 29 Mar 2018 12:41:19 AM EDT
\gpmonochromelines
\path (0.000,0.000) rectangle (12.500,8.750);
\gpcolor{color=gp lt color border}
\gpsetlinetype{gp lt border}
\gpsetdashtype{gp dt solid}
\gpsetlinewidth{1.00}
\draw[gp path] (1.504,1.022)--(1.684,1.022);
\draw[gp path] (11.947,1.022)--(11.767,1.022);
\node[gp node right] at (1.320,1.022) {$-1$};
\draw[gp path] (1.504,2.852)--(1.684,2.852);
\draw[gp path] (11.947,2.852)--(11.767,2.852);
\node[gp node right] at (1.320,2.852) {$-0.5$};
\draw[gp path] (1.504,4.683)--(1.684,4.683);
\draw[gp path] (11.947,4.683)--(11.767,4.683);
\node[gp node right] at (1.320,4.683) {$0$};
\draw[gp path] (1.504,6.514)--(1.684,6.514);
\draw[gp path] (11.947,6.514)--(11.767,6.514);
\node[gp node right] at (1.320,6.514) {$0.5$};
\draw[gp path] (1.504,8.344)--(1.684,8.344);
\draw[gp path] (11.947,8.344)--(11.767,8.344);
\node[gp node right] at (1.320,8.344) {$1$};
\draw[gp path] (1.504,0.985)--(1.504,1.165);
\draw[gp path] (1.504,8.381)--(1.504,8.201);
\node[gp node center] at (1.504,0.677) {$0$};
\draw[gp path] (2.548,0.985)--(2.548,1.165);
\draw[gp path] (2.548,8.381)--(2.548,8.201);
\node[gp node center] at (2.548,0.677) {$5$};
\draw[gp path] (3.593,0.985)--(3.593,1.165);
\draw[gp path] (3.593,8.381)--(3.593,8.201);
\node[gp node center] at (3.593,0.677) {$10$};
\draw[gp path] (4.637,0.985)--(4.637,1.165);
\draw[gp path] (4.637,8.381)--(4.637,8.201);
\node[gp node center] at (4.637,0.677) {$15$};
\draw[gp path] (5.681,0.985)--(5.681,1.165);
\draw[gp path] (5.681,8.381)--(5.681,8.201);
\node[gp node center] at (5.681,0.677) {$20$};
\draw[gp path] (6.726,0.985)--(6.726,1.165);
\draw[gp path] (6.726,8.381)--(6.726,8.201);
\node[gp node center] at (6.726,0.677) {$25$};
\draw[gp path] (7.770,0.985)--(7.770,1.165);
\draw[gp path] (7.770,8.381)--(7.770,8.201);
\node[gp node center] at (7.770,0.677) {$30$};
\draw[gp path] (8.814,0.985)--(8.814,1.165);
\draw[gp path] (8.814,8.381)--(8.814,8.201);
\node[gp node center] at (8.814,0.677) {$35$};
\draw[gp path] (9.858,0.985)--(9.858,1.165);
\draw[gp path] (9.858,8.381)--(9.858,8.201);
\node[gp node center] at (9.858,0.677) {$40$};
\draw[gp path] (10.903,0.985)--(10.903,1.165);
\draw[gp path] (10.903,8.381)--(10.903,8.201);
\node[gp node center] at (10.903,0.677) {$45$};
\draw[gp path] (11.947,0.985)--(11.947,1.165);
\draw[gp path] (11.947,8.381)--(11.947,8.201);
\node[gp node center] at (11.947,0.677) {$50$};
\draw[gp path] (1.504,8.381)--(1.504,0.985)--(11.947,0.985)--(11.947,8.381)--cycle;
\node[gp node center,rotate=-270] at (0.246,4.683) {Force, N};
\node[gp node center] at (6.725,0.215) {Time, s};
\node[gp node right] at (10.479,8.047) {FIS Force Output};
\gpsetlinewidth{2.00}
\draw[gp path] (10.663,8.047)--(11.579,8.047);
\draw[gp path] (1.504,8.344)--(1.505,8.344)--(1.506,8.344)--(1.507,8.344)--(1.508,8.344)%
  --(1.509,8.344)--(1.510,8.344)--(1.511,8.344)--(1.512,8.344)--(1.513,8.344)--(1.514,8.344)%
  --(1.516,8.344)--(1.517,8.344)--(1.518,8.344)--(1.520,8.344)--(1.521,8.344)--(1.522,8.344)%
  --(1.524,8.344)--(1.525,8.344)--(1.527,8.344)--(1.528,8.344)--(1.530,8.344)--(1.532,8.344)%
  --(1.534,8.344)--(1.535,8.344)--(1.536,8.344)--(1.538,8.344)--(1.539,8.344)--(1.540,8.344)%
  --(1.542,8.344)--(1.543,8.344)--(1.544,8.344)--(1.545,8.344)--(1.547,8.344)--(1.548,8.344)%
  --(1.549,8.344)--(1.551,8.344)--(1.552,8.344)--(1.553,8.344)--(1.554,8.344)--(1.556,8.344)%
  --(1.557,8.344)--(1.559,8.344)--(1.560,8.344)--(1.562,8.344)--(1.563,8.344)--(1.565,8.344)%
  --(1.567,8.344)--(1.568,8.344)--(1.569,8.344)--(1.571,8.344)--(1.572,8.344)--(1.573,8.344)%
  --(1.575,8.344)--(1.576,8.344)--(1.578,8.344)--(1.579,8.344)--(1.580,8.344)--(1.581,8.344)%
  --(1.583,8.344)--(1.584,8.344)--(1.585,8.344)--(1.586,8.344)--(1.588,8.344)--(1.589,8.344)%
  --(1.591,8.344)--(1.592,8.344)--(1.593,8.344)--(1.595,8.344)--(1.597,8.344)--(1.598,8.344)%
  --(1.600,8.344)--(1.601,8.344)--(1.603,8.344)--(1.604,8.344)--(1.605,8.344)--(1.607,8.344)%
  --(1.608,8.344)--(1.609,8.344)--(1.611,8.344)--(1.612,8.344)--(1.613,8.344)--(1.615,8.344)%
  --(1.616,8.344)--(1.617,8.344)--(1.618,8.344)--(1.620,8.344)--(1.621,8.344)--(1.622,8.344)%
  --(1.624,8.344)--(1.625,8.344)--(1.626,8.344)--(1.628,8.344)--(1.630,8.344)--(1.631,8.344)%
  --(1.633,8.344)--(1.634,8.344)--(1.636,8.344)--(1.637,8.344)--(1.638,8.344)--(1.640,8.344)%
  --(1.641,8.344)--(1.643,8.344)--(1.644,8.344)--(1.645,8.344)--(1.647,8.344)--(1.648,8.344)%
  --(1.649,8.344)--(1.650,8.344)--(1.652,8.344)--(1.653,8.344)--(1.654,8.344)--(1.656,8.344)%
  --(1.657,8.344)--(1.658,8.344)--(1.660,8.344)--(1.661,8.344)--(1.663,8.344)--(1.664,8.344)%
  --(1.666,8.344)--(1.667,8.344)--(1.669,8.344)--(1.670,8.344)--(1.672,8.344)--(1.673,8.344)%
  --(1.674,8.344)--(1.676,8.344)--(1.677,8.344)--(1.679,8.344)--(1.680,8.344)--(1.681,8.344)%
  --(1.683,8.344)--(1.684,8.344)--(1.685,8.344)--(1.686,8.344)--(1.688,8.344)--(1.689,8.344)%
  --(1.690,8.344)--(1.692,8.344)--(1.693,8.344)--(1.694,8.344)--(1.696,8.344)--(1.697,8.344)%
  --(1.699,8.344)--(1.700,8.344)--(1.702,8.344)--(1.703,8.344)--(1.705,8.344)--(1.706,8.344)%
  --(1.708,8.344)--(1.709,8.344)--(1.711,8.344)--(1.712,8.344)--(1.714,8.344)--(1.715,8.344)%
  --(1.716,8.344)--(1.717,8.344)--(1.719,8.344)--(1.720,8.344)--(1.721,8.344)--(1.723,8.344)%
  --(1.724,8.344)--(1.725,8.344)--(1.727,8.344)--(1.728,8.344)--(1.730,8.344)--(1.731,8.344)%
  --(1.732,8.344)--(1.734,8.344)--(1.736,8.344)--(1.737,8.344)--(1.739,8.344)--(1.740,8.344)%
  --(1.742,8.344)--(1.743,8.344)--(1.745,8.344)--(1.746,8.344)--(1.747,8.344)--(1.749,8.344)%
  --(1.750,8.344)--(1.751,8.344)--(1.752,8.344)--(1.754,8.344)--(1.755,8.344)--(1.756,8.344)%
  --(1.758,8.344)--(1.759,8.344)--(1.760,8.344)--(1.762,8.344)--(1.763,8.344)--(1.765,8.344)%
  --(1.766,8.344)--(1.768,8.344)--(1.769,8.344)--(1.771,8.344)--(1.772,8.344)--(1.774,8.344)%
  --(1.775,8.344)--(1.777,8.344)--(1.778,8.344)--(1.780,8.344)--(1.781,8.344)--(1.782,8.344)%
  --(1.784,8.344)--(1.785,8.344)--(1.786,8.344)--(1.787,8.344)--(1.789,8.344)--(1.790,8.344)%
  --(1.791,8.344)--(1.793,8.344)--(1.794,8.344)--(1.795,8.344)--(1.797,8.344)--(1.798,8.344)%
  --(1.800,8.344)--(1.801,8.344)--(1.803,8.344)--(1.804,8.344)--(1.806,8.344)--(1.808,8.344)%
  --(1.809,8.344)--(1.811,8.344)--(1.812,8.344)--(1.814,8.344)--(1.815,8.344)--(1.816,8.344)%
  --(1.817,8.344)--(1.819,8.344)--(1.820,8.344)--(1.821,8.344)--(1.823,8.344)--(1.824,8.344)%
  --(1.825,8.344)--(1.826,8.344)--(1.828,8.344)--(1.829,8.344)--(1.831,8.344)--(1.832,8.344)%
  --(1.834,8.344)--(1.835,8.344)--(1.837,8.344)--(1.838,8.344)--(1.840,8.344)--(1.841,8.344)%
  --(1.843,8.344)--(1.845,8.344)--(1.846,8.344)--(1.847,8.344)--(1.849,8.344)--(1.850,8.344)%
  --(1.851,8.344)--(1.853,8.344)--(1.854,8.344)--(1.855,8.344)--(1.856,8.344)--(1.858,8.344)%
  --(1.859,8.344)--(1.860,8.344)--(1.862,8.344)--(1.863,8.344)--(1.865,8.344)--(1.866,8.344)%
  --(1.868,8.344)--(1.869,8.344)--(1.871,8.344)--(1.872,8.344)--(1.874,8.344)--(1.875,8.344)%
  --(1.877,8.344)--(1.878,8.344)--(1.880,8.344)--(1.881,8.344)--(1.883,8.344)--(1.884,8.344)%
  --(1.885,8.344)--(1.886,8.344)--(1.888,8.344)--(1.889,8.344)--(1.890,8.344)--(1.892,8.344)%
  --(1.893,8.344)--(1.894,8.344)--(1.896,8.344)--(1.897,8.344)--(1.898,8.344)--(1.900,8.344)%
  --(1.901,8.344)--(1.903,8.344)--(1.904,8.344)--(1.906,8.344)--(1.908,8.344)--(1.909,8.344)%
  --(1.911,8.344)--(1.912,8.344)--(1.914,8.344)--(1.915,8.344)--(1.916,8.344)--(1.918,8.344)%
  --(1.919,8.344)--(1.920,8.344)--(1.922,8.344)--(1.923,8.344)--(1.924,8.344)--(1.925,8.344)%
  --(1.927,8.344)--(1.928,8.344)--(1.929,8.344)--(1.931,8.344)--(1.932,8.344)--(1.934,8.344)%
  --(1.935,8.344)--(1.937,8.344)--(1.938,8.344)--(1.940,8.344)--(1.941,8.344)--(1.943,8.344)%
  --(1.945,8.344)--(1.946,8.344)--(1.948,8.344)--(1.949,8.344)--(1.950,8.344)--(1.952,8.344)%
  --(1.953,8.344)--(1.954,8.344)--(1.955,8.344)--(1.957,8.344)--(1.958,8.344)--(1.959,8.344)%
  --(1.961,8.344)--(1.962,8.344)--(1.963,8.344)--(1.965,8.344)--(1.966,8.344)--(1.968,8.344)%
  --(1.969,8.344)--(1.971,8.344)--(1.972,8.344)--(1.974,8.344)--(1.975,8.344)--(1.977,8.344)%
  --(1.978,8.344)--(1.980,8.344)--(1.982,8.344)--(1.983,8.344)--(1.984,8.344)--(1.985,8.344)%
  --(1.987,8.344)--(1.988,8.344)--(1.989,8.344)--(1.991,8.344)--(1.992,8.344)--(1.993,8.344)%
  --(1.995,8.344)--(1.996,8.344)--(1.997,8.344)--(1.999,8.344)--(2.000,8.344)--(2.002,8.344)%
  --(2.003,8.344)--(2.005,8.344)--(2.006,8.344)--(2.008,8.344)--(2.009,8.344)--(2.011,8.344)%
  --(2.012,8.344)--(2.014,8.344)--(2.015,8.344)--(2.017,8.344)--(2.018,8.344)--(2.019,8.344)%
  --(2.021,8.344)--(2.022,8.344)--(2.023,8.344)--(2.025,8.344)--(2.026,8.344)--(2.027,8.344)%
  --(2.028,8.344)--(2.030,8.344)--(2.031,8.344)--(2.033,8.344)--(2.034,8.344)--(2.035,8.344)%
  --(2.037,8.344)--(2.038,8.344)--(2.040,8.344)--(2.042,8.344)--(2.043,8.344)--(2.045,8.344)%
  --(2.046,8.344)--(2.048,8.344)--(2.049,8.344)--(2.051,8.344)--(2.052,8.344)--(2.053,8.344)%
  --(2.055,8.344)--(2.056,8.344)--(2.057,8.344)--(2.058,8.344)--(2.060,8.344)--(2.061,8.344)%
  --(2.062,8.344)--(2.064,8.344)--(2.065,8.344)--(2.066,8.344)--(2.068,8.344)--(2.069,8.344)%
  --(2.071,8.344)--(2.072,8.344)--(2.074,8.344)--(2.075,8.344)--(2.077,8.344)--(2.079,8.344)%
  --(2.080,8.344)--(2.082,8.344)--(2.083,8.344)--(2.085,8.344)--(2.086,8.344)--(2.087,8.344)%
  --(2.088,8.344)--(2.090,8.344)--(2.091,8.344)--(2.092,8.344)--(2.094,8.344)--(2.095,8.344)%
  --(2.096,8.344)--(2.097,8.344)--(2.099,8.344)--(2.100,8.344)--(2.102,8.344)--(2.103,8.344)%
  --(2.105,8.344)--(2.106,8.344)--(2.108,8.344)--(2.109,8.344)--(2.111,8.344)--(2.112,8.344)%
  --(2.114,8.344)--(2.116,8.344)--(2.117,8.344)--(2.118,8.344)--(2.120,8.344)--(2.121,8.344)%
  --(2.122,8.344)--(2.124,8.344)--(2.125,8.344)--(2.126,8.344)--(2.127,8.344)--(2.129,8.344)%
  --(2.130,8.344)--(2.131,8.344)--(2.133,8.344)--(2.134,8.344)--(2.136,8.344)--(2.137,8.344)%
  --(2.139,8.344)--(2.140,8.344)--(2.142,8.344)--(2.143,8.344)--(2.145,8.344)--(2.146,8.344)%
  --(2.148,8.344)--(2.149,8.344)--(2.151,8.344)--(2.152,8.344)--(2.154,8.344)--(2.155,8.344)%
  --(2.156,8.344)--(2.157,8.344)--(2.159,8.344)--(2.160,8.344)--(2.161,8.344)--(2.163,8.344)%
  --(2.164,8.344)--(2.165,8.344)--(2.167,8.344)--(2.168,8.344)--(2.170,8.344)--(2.171,8.344)%
  --(2.172,8.344)--(2.174,8.344)--(2.176,8.344)--(2.177,8.344)--(2.179,8.344)--(2.180,8.344)%
  --(2.182,8.344)--(2.183,8.344)--(2.185,8.344)--(2.186,8.344)--(2.187,8.344)--(2.189,8.344)%
  --(2.190,8.344)--(2.191,8.344)--(2.193,8.344)--(2.194,8.344)--(2.195,8.344)--(2.197,8.344)%
  --(2.198,8.344)--(2.199,8.344)--(2.200,8.344)--(2.202,8.344)--(2.203,8.344)--(2.205,8.344)%
  --(2.206,8.344)--(2.208,8.344)--(2.209,8.344)--(2.211,8.344)--(2.213,8.344)--(2.214,8.344)%
  --(2.216,8.344)--(2.217,8.344)--(2.219,8.344)--(2.220,8.344)--(2.221,8.344)--(2.223,8.344)%
  --(2.224,8.344)--(2.225,8.344)--(2.227,8.344)--(2.228,8.344)--(2.229,8.344)--(2.230,8.344)%
  --(2.232,8.344)--(2.233,8.344)--(2.234,8.344)--(2.236,8.344)--(2.237,8.344)--(2.239,8.344)%
  --(2.240,8.344)--(2.242,8.344)--(2.243,8.344)--(2.245,8.344)--(2.246,8.344)--(2.248,8.344)%
  --(2.250,8.344)--(2.251,8.344)--(2.253,8.344)--(2.254,8.344)--(2.255,8.344)--(2.257,8.344)%
  --(2.258,8.344)--(2.259,8.344)--(2.260,8.344)--(2.262,8.344)--(2.263,8.344)--(2.264,8.344)%
  --(2.266,8.344)--(2.267,8.344)--(2.268,8.344)--(2.270,8.344)--(2.271,8.344)--(2.273,8.344)%
  --(2.274,8.344)--(2.276,8.344)--(2.277,8.344)--(2.279,8.344)--(2.280,8.344)--(2.282,8.344)%
  --(2.283,8.344)--(2.285,8.344)--(2.287,8.344)--(2.288,8.344)--(2.289,8.344)--(2.290,8.344)%
  --(2.292,8.344)--(2.293,8.344)--(2.294,8.344)--(2.296,8.344)--(2.297,8.344)--(2.298,8.344)%
  --(2.299,8.344)--(2.301,8.344)--(2.302,8.344)--(2.304,8.344)--(2.305,8.344)--(2.307,8.344)%
  --(2.308,8.344)--(2.310,8.344)--(2.311,8.344)--(2.313,8.344)--(2.314,8.344)--(2.316,8.344)%
  --(2.317,8.344)--(2.319,8.344)--(2.320,8.344)--(2.322,8.344)--(2.323,8.344)--(2.324,8.344)%
  --(2.326,8.344)--(2.327,8.344)--(2.328,8.344)--(2.329,8.344)--(2.331,8.344)--(2.332,8.344)%
  --(2.333,8.344)--(2.335,8.344)--(2.336,8.344)--(2.337,8.344)--(2.339,8.344)--(2.340,8.344)%
  --(2.342,8.344)--(2.343,8.344)--(2.345,8.344)--(2.347,8.344)--(2.348,8.344)--(2.350,8.344)%
  --(2.351,8.344)--(2.353,8.344)--(2.354,8.344)--(2.356,8.344)--(2.357,8.344)--(2.358,8.344)%
  --(2.360,8.344)--(2.361,8.344)--(2.362,8.344)--(2.363,8.344)--(2.365,8.344)--(2.366,8.344)%
  --(2.367,8.344)--(2.369,8.344)--(2.370,8.344)--(2.371,8.344)--(2.373,8.344)--(2.374,8.344)%
  --(2.376,8.344)--(2.377,8.344)--(2.379,8.344)--(2.380,8.344)--(2.382,8.344)--(2.384,8.344)%
  --(2.385,8.344)--(2.387,8.344)--(2.388,8.344)--(2.389,8.344)--(2.391,8.344)--(2.392,8.344)%
  --(2.393,8.344)--(2.395,8.344)--(2.396,8.344)--(2.397,8.344)--(2.398,8.344)--(2.400,8.344)%
  --(2.401,8.344)--(2.402,8.344)--(2.404,8.344)--(2.405,8.344)--(2.407,8.344)--(2.408,8.344)%
  --(2.410,8.344)--(2.411,8.344)--(2.413,8.344)--(2.414,8.344)--(2.416,8.344)--(2.417,8.344)%
  --(2.419,8.344)--(2.420,8.344)--(2.422,8.344)--(2.423,8.344)--(2.425,8.344)--(2.426,8.344)%
  --(2.427,8.344)--(2.429,8.344)--(2.430,8.344)--(2.431,8.344)--(2.432,8.344)--(2.434,8.344)%
  --(2.435,8.344)--(2.436,8.344)--(2.438,8.344)--(2.439,8.344)--(2.441,8.344)--(2.442,8.344)%
  --(2.444,8.344)--(2.445,8.344)--(2.447,8.344)--(2.448,8.344)--(2.450,8.344)--(2.451,8.344)%
  --(2.453,8.344)--(2.454,8.344)--(2.456,8.344)--(2.457,8.344)--(2.459,8.344)--(2.460,8.344)%
  --(2.461,8.344)--(2.462,8.344)--(2.464,8.344)--(2.465,8.344)--(2.466,8.344)--(2.468,8.344)%
  --(2.469,8.344)--(2.470,8.344)--(2.472,8.344)--(2.473,8.344)--(2.474,8.344)--(2.476,8.344)%
  --(2.477,8.344)--(2.479,8.344)--(2.481,8.344)--(2.482,8.344)--(2.484,8.344)--(2.485,8.344)%
  --(2.487,8.344)--(2.488,8.344)--(2.490,8.344)--(2.491,8.344)--(2.492,8.344)--(2.494,8.344)%
  --(2.495,8.344)--(2.496,8.344)--(2.498,8.344)--(2.499,8.344)--(2.500,8.344)--(2.501,8.344)%
  --(2.503,8.344)--(2.504,8.344)--(2.505,8.344)--(2.507,8.344)--(2.508,8.344)--(2.510,8.344)%
  --(2.511,8.344)--(2.513,8.344)--(2.514,8.344)--(2.516,8.344)--(2.517,8.344)--(2.519,8.344)%
  --(2.521,8.344)--(2.522,8.344)--(2.524,8.344)--(2.525,8.344)--(2.526,8.344)--(2.528,8.344)%
  --(2.529,8.344)--(2.530,8.344)--(2.531,8.344)--(2.533,8.344)--(2.534,8.344)--(2.535,8.344)%
  --(2.537,8.344)--(2.538,8.344)--(2.539,8.344)--(2.541,8.344)--(2.542,8.344)--(2.544,8.344)%
  --(2.545,8.344)--(2.547,8.344)--(2.548,8.344)--(2.550,8.344)--(2.551,8.344)--(2.553,8.344)%
  --(2.554,8.344)--(2.556,8.344)--(2.558,8.344)--(2.559,8.344)--(2.560,8.344)--(2.561,8.344)%
  --(2.563,8.344)--(2.564,8.344)--(2.565,8.344)--(2.567,8.344)--(2.568,8.344)--(2.569,8.344)%
  --(2.571,8.344)--(2.572,8.344)--(2.573,8.344)--(2.575,8.344)--(2.576,8.344)--(2.578,8.344)%
  --(2.579,8.344)--(2.581,8.344)--(2.582,8.344)--(2.584,8.344)--(2.585,8.344)--(2.587,8.344)%
  --(2.588,8.344)--(2.590,8.344)--(2.591,8.344)--(2.593,8.344)--(2.594,8.344)--(2.595,8.344)%
  --(2.597,8.344)--(2.598,8.344)--(2.599,8.344)--(2.601,8.344)--(2.602,8.344)--(2.603,8.344)%
  --(2.604,8.344)--(2.606,8.344)--(2.607,8.344)--(2.609,8.344)--(2.610,8.344)--(2.611,8.344)%
  --(2.613,8.344)--(2.614,8.344)--(2.616,8.344)--(2.618,8.344)--(2.619,8.344)--(2.621,8.344)%
  --(2.622,8.344)--(2.624,8.344)--(2.625,8.344)--(2.627,8.344)--(2.628,8.344)--(2.629,8.344)%
  --(2.631,8.344)--(2.632,8.344)--(2.633,8.344)--(2.634,8.344)--(2.636,8.344)--(2.637,8.344)%
  --(2.638,8.344)--(2.640,8.344)--(2.641,8.344)--(2.642,8.344)--(2.644,8.344)--(2.645,8.344)%
  --(2.647,8.344)--(2.648,8.344)--(2.650,8.344)--(2.651,8.344)--(2.653,8.344)--(2.655,8.344)%
  --(2.656,8.344)--(2.658,8.344)--(2.659,8.344)--(2.661,8.344)--(2.662,8.344)--(2.663,8.344)%
  --(2.664,8.344)--(2.666,8.344)--(2.667,8.344)--(2.668,8.344)--(2.670,8.344)--(2.671,8.344)%
  --(2.672,8.344)--(2.674,8.344)--(2.675,8.344)--(2.676,8.344)--(2.678,8.344)--(2.679,8.344)%
  --(2.681,8.344)--(2.682,8.344)--(2.684,8.344)--(2.685,8.344)--(2.687,8.344)--(2.688,8.344)%
  --(2.690,8.344)--(2.692,8.344)--(2.693,8.344)--(2.694,8.344)--(2.696,8.344)--(2.697,8.344)%
  --(2.698,8.344)--(2.700,8.344)--(2.701,8.344)--(2.702,8.344)--(2.703,8.344)--(2.705,8.344)%
  --(2.706,8.344)--(2.707,8.344)--(2.709,8.344)--(2.710,8.344)--(2.712,8.344)--(2.713,8.344)%
  --(2.715,8.344)--(2.716,8.344)--(2.718,8.344)--(2.719,8.344)--(2.721,8.344)--(2.722,8.344)%
  --(2.724,8.344)--(2.725,8.344)--(2.727,8.344)--(2.728,8.344)--(2.730,8.344)--(2.731,8.344)%
  --(2.732,8.344)--(2.733,8.344)--(2.735,8.344)--(2.736,8.344)--(2.737,8.344)--(2.739,8.344)%
  --(2.740,8.344)--(2.741,8.344)--(2.743,8.344)--(2.744,8.344)--(2.746,8.344)--(2.747,8.344)%
  --(2.748,8.344)--(2.750,8.344)--(2.752,8.344)--(2.753,8.344)--(2.755,8.344)--(2.756,8.344)%
  --(2.758,8.344)--(2.759,8.344)--(2.761,8.344)--(2.762,8.344)--(2.763,8.344)--(2.765,8.344)%
  --(2.766,8.344)--(2.767,8.344)--(2.769,8.344)--(2.770,8.344)--(2.771,8.344)--(2.773,8.344)%
  --(2.774,8.344)--(2.775,8.344)--(2.777,8.344)--(2.778,8.344)--(2.779,8.344)--(2.781,8.344)%
  --(2.782,8.344)--(2.784,8.344)--(2.785,8.344)--(2.787,8.344)--(2.789,8.344)--(2.790,8.344)%
  --(2.792,8.344)--(2.793,8.344)--(2.795,8.344)--(2.796,8.344)--(2.797,8.344)--(2.799,8.344)%
  --(2.800,8.344)--(2.801,8.344)--(2.803,8.344)--(2.804,8.344)--(2.805,8.344)--(2.806,8.344)%
  --(2.808,8.344)--(2.809,8.344)--(2.810,8.344)--(2.812,8.344)--(2.813,8.344)--(2.815,8.344)%
  --(2.816,8.344)--(2.818,8.344)--(2.819,8.344)--(2.821,8.344)--(2.822,8.344)--(2.824,8.344)%
  --(2.826,8.344)--(2.827,8.344)--(2.829,8.344)--(2.830,8.344)--(2.831,8.344)--(2.833,8.344)%
  --(2.834,8.344)--(2.835,8.344)--(2.836,8.344)--(2.838,8.344)--(2.839,8.344)--(2.840,8.344)%
  --(2.842,8.344)--(2.843,8.344)--(2.844,8.344)--(2.846,8.344)--(2.847,8.344)--(2.849,8.344)%
  --(2.850,8.344)--(2.852,8.344)--(2.853,8.344)--(2.855,8.344)--(2.856,8.344)--(2.858,8.344)%
  --(2.859,8.344)--(2.861,8.344)--(2.863,8.344)--(2.864,8.344)--(2.865,8.344)--(2.866,8.344)%
  --(2.868,8.344)--(2.869,8.344)--(2.870,8.344)--(2.872,8.344)--(2.873,8.344)--(2.874,8.344)%
  --(2.876,8.344)--(2.877,8.344)--(2.878,8.344)--(2.880,8.344)--(2.881,8.344)--(2.883,8.344)%
  --(2.884,8.344)--(2.886,8.344)--(2.887,8.344)--(2.889,8.344)--(2.890,8.344)--(2.892,8.344)%
  --(2.893,8.344)--(2.895,8.344)--(2.896,8.344)--(2.898,8.344)--(2.899,8.344)--(2.900,8.344)%
  --(2.902,8.344)--(2.903,8.344)--(2.904,8.344)--(2.905,8.344)--(2.907,8.344)--(2.908,8.344)%
  --(2.909,8.344)--(2.911,8.344)--(2.912,8.344)--(2.913,8.344)--(2.915,8.344)--(2.916,8.344)%
  --(2.918,8.344)--(2.919,8.344)--(2.921,8.344)--(2.923,8.344)--(2.924,8.344)--(2.926,8.344)%
  --(2.927,8.344)--(2.929,8.344)--(2.930,8.344)--(2.932,8.344)--(2.933,8.344)--(2.934,8.344)%
  --(2.936,8.344)--(2.937,8.344)--(2.938,8.344)--(2.939,8.344)--(2.941,8.344)--(2.942,8.344)%
  --(2.943,8.344)--(2.945,8.344)--(2.946,8.344)--(2.947,8.344)--(2.949,8.344)--(2.950,8.344)%
  --(2.952,8.344)--(2.953,8.344)--(2.955,8.344)--(2.956,8.344)--(2.958,8.344)--(2.960,8.344)%
  --(2.961,8.344)--(2.963,8.344)--(2.964,8.344)--(2.965,8.344)--(2.967,8.344)--(2.968,8.344)%
  --(2.969,8.344)--(2.971,8.344)--(2.972,8.344)--(2.973,8.344)--(2.974,8.344)--(2.976,8.344)%
  --(2.977,8.344)--(2.978,8.344)--(2.980,8.344)--(2.981,8.344)--(2.983,8.344)--(2.984,8.344)%
  --(2.986,8.344)--(2.987,8.344)--(2.989,8.344)--(2.990,8.344)--(2.992,8.344)--(2.993,8.344)%
  --(2.995,8.344)--(2.997,8.344)--(2.998,8.344)--(2.999,8.344)--(3.001,8.344)--(3.002,8.344)%
  --(3.003,8.344)--(3.005,8.344)--(3.006,8.344)--(3.007,8.344)--(3.008,8.344)--(3.010,8.344)%
  --(3.011,8.344)--(3.012,8.344)--(3.014,8.344)--(3.015,8.344)--(3.017,8.344)--(3.018,8.344)%
  --(3.020,8.344)--(3.021,8.344)--(3.023,8.344)--(3.024,8.344)--(3.026,8.344)--(3.027,8.344)%
  --(3.029,8.344)--(3.030,8.344)--(3.032,8.344)--(3.033,8.344)--(3.035,8.344)--(3.036,8.344)%
  --(3.037,8.344)--(3.038,8.344)--(3.040,8.344)--(3.041,8.344)--(3.042,8.344)--(3.044,8.344)%
  --(3.045,8.344)--(3.046,8.344)--(3.048,8.344)--(3.049,8.344)--(3.051,8.344)--(3.052,8.344)%
  --(3.053,8.344)--(3.055,8.344)--(3.057,8.344)--(3.058,8.344)--(3.060,8.344)--(3.061,8.344)%
  --(3.063,8.344)--(3.064,8.344)--(3.066,8.344)--(3.067,8.344)--(3.068,8.344)--(3.070,8.344)%
  --(3.071,8.344)--(3.072,8.344)--(3.074,8.344)--(3.075,8.344)--(3.076,8.344)--(3.077,8.344)%
  --(3.079,8.344)--(3.080,8.344)--(3.081,8.344)--(3.083,8.344)--(3.084,8.344)--(3.086,8.344)%
  --(3.087,8.344)--(3.089,8.344)--(3.090,8.344)--(3.092,8.344)--(3.094,8.344)--(3.095,8.344)%
  --(3.097,8.344)--(3.098,8.344)--(3.100,8.344)--(3.101,8.344)--(3.102,8.344)--(3.104,8.344)%
  --(3.105,8.344)--(3.106,8.344)--(3.108,8.344)--(3.109,8.344)--(3.110,8.344)--(3.111,8.344)%
  --(3.113,8.344)--(3.114,8.344)--(3.115,8.344)--(3.117,8.344)--(3.118,8.344)--(3.120,8.344)%
  --(3.121,8.344)--(3.123,8.344)--(3.124,8.344)--(3.126,8.344)--(3.127,8.344)--(3.129,8.344)%
  --(3.130,8.344)--(3.132,8.344)--(3.134,8.344)--(3.135,8.344)--(3.136,8.344)--(3.138,8.344)%
  --(3.139,8.344)--(3.140,8.344)--(3.141,8.344)--(3.143,8.344)--(3.144,8.344)--(3.145,8.344)%
  --(3.147,8.344)--(3.148,8.344)--(3.149,8.344)--(3.151,8.344)--(3.152,8.344)--(3.154,8.344)%
  --(3.155,8.344)--(3.157,8.344)--(3.158,8.344)--(3.160,8.344)--(3.161,8.344)--(3.163,8.344)%
  --(3.164,8.344)--(3.166,8.344)--(3.168,8.344)--(3.169,8.344)--(3.170,8.344)--(3.171,8.344)%
  --(3.173,8.344)--(3.174,8.344)--(3.175,8.344)--(3.177,8.344)--(3.178,8.344)--(3.179,8.344)%
  --(3.180,8.344)--(3.182,8.344)--(3.183,8.344)--(3.185,8.344)--(3.186,8.344)--(3.188,8.344)%
  --(3.189,8.344)--(3.191,8.344)--(3.192,8.344)--(3.194,8.344)--(3.195,8.344)--(3.197,8.344)%
  --(3.198,8.344)--(3.200,8.344)--(3.201,8.344)--(3.203,8.344)--(3.204,8.344)--(3.205,8.344)%
  --(3.207,8.344)--(3.208,8.344)--(3.209,8.344)--(3.210,8.344)--(3.212,8.344)--(3.213,8.344)%
  --(3.214,8.344)--(3.216,8.344)--(3.217,8.344)--(3.218,8.344)--(3.220,8.344)--(3.221,8.344)%
  --(3.223,8.344)--(3.224,8.344)--(3.226,8.344)--(3.228,8.344)--(3.229,8.344)--(3.231,8.344)%
  --(3.232,8.344)--(3.234,8.344)--(3.235,8.344)--(3.237,8.344)--(3.238,8.344)--(3.239,8.344)%
  --(3.241,8.344)--(3.242,8.344)--(3.243,8.344)--(3.244,8.344)--(3.246,8.344)--(3.247,8.344)%
  --(3.248,8.344)--(3.250,8.344)--(3.251,8.344)--(3.252,8.344)--(3.254,8.344)--(3.255,8.344)%
  --(3.257,8.344)--(3.258,8.344)--(3.260,8.344)--(3.261,8.344)--(3.263,8.344)--(3.265,8.344)%
  --(3.266,8.344)--(3.268,8.344)--(3.269,8.344)--(3.271,8.344)--(3.272,8.344)--(3.273,8.344)%
  --(3.274,8.344)--(3.276,8.344)--(3.277,8.344)--(3.278,8.344)--(3.280,8.344)--(3.281,8.344)%
  --(3.282,8.344)--(3.284,8.344)--(3.285,8.344)--(3.286,8.344)--(3.288,8.344)--(3.289,8.344)%
  --(3.291,8.344)--(3.292,8.344)--(3.294,8.344)--(3.295,8.344)--(3.297,8.344)--(3.299,8.344)%
  --(3.300,8.344)--(3.302,8.344)--(3.303,8.344)--(3.304,8.344)--(3.306,8.344)--(3.307,8.344)%
  --(3.308,8.344)--(3.310,8.344)--(3.311,8.344)--(3.312,8.344)--(3.313,8.344)--(3.315,8.344)%
  --(3.316,8.344)--(3.317,8.344)--(3.319,8.344)--(3.320,8.344)--(3.322,8.344)--(3.323,8.344)%
  --(3.325,8.344)--(3.326,8.344)--(3.328,8.344)--(3.329,8.344)--(3.331,8.344)--(3.332,8.344)%
  --(3.334,8.344)--(3.336,8.344)--(3.337,8.344)--(3.338,8.344)--(3.340,8.344)--(3.341,8.344)%
  --(3.342,8.344)--(3.344,8.344)--(3.345,8.344)--(3.346,8.344)--(3.347,8.344)--(3.349,8.344)%
  --(3.350,8.344)--(3.351,8.344)--(3.353,8.344)--(3.354,8.344)--(3.356,8.344)--(3.357,8.344)%
  --(3.359,8.344)--(3.360,8.344)--(3.362,8.344)--(3.363,8.344)--(3.365,8.344)--(3.366,8.344)%
  --(3.368,8.344)--(3.369,8.344)--(3.371,8.344)--(3.372,8.344)--(3.374,8.344)--(3.375,8.344)%
  --(3.376,8.344)--(3.377,8.344)--(3.379,8.344)--(3.380,8.344)--(3.381,8.344)--(3.383,8.344)%
  --(3.384,8.344)--(3.385,8.344)--(3.387,8.344)--(3.388,8.344)--(3.390,8.344)--(3.391,8.344)%
  --(3.392,8.344)--(3.394,8.344)--(3.396,8.344)--(3.397,8.344)--(3.399,8.344)--(3.400,8.344)%
  --(3.402,8.344)--(3.403,8.344)--(3.405,8.344)--(3.406,8.344)--(3.408,8.344)--(3.409,8.344)%
  --(3.410,8.344)--(3.411,8.344)--(3.413,8.344)--(3.414,8.344)--(3.415,8.344)--(3.417,8.344)%
  --(3.418,8.344)--(3.419,8.344)--(3.421,8.344)--(3.422,8.344)--(3.423,8.344)--(3.425,8.344)%
  --(3.426,8.344)--(3.428,8.344)--(3.430,8.344)--(3.431,8.344)--(3.433,8.344)--(3.434,8.344)%
  --(3.436,8.344)--(3.437,8.344)--(3.439,8.344)--(3.440,8.344)--(3.441,8.344)--(3.443,8.344)%
  --(3.444,8.344)--(3.445,8.344)--(3.447,8.344)--(3.448,8.344)--(3.449,8.344)--(3.450,8.344)%
  --(3.452,8.344)--(3.453,8.344)--(3.454,8.344)--(3.456,8.344)--(3.457,8.344)--(3.459,8.344)%
  --(3.460,8.344)--(3.462,8.344)--(3.463,8.344)--(3.465,8.344)--(3.467,8.344)--(3.468,8.344)%
  --(3.470,8.344)--(3.471,8.344)--(3.473,8.344)--(3.474,8.344)--(3.475,8.344)--(3.477,8.344)%
  --(3.478,8.344)--(3.479,8.344)--(3.480,8.344)--(3.482,8.344)--(3.483,8.344)--(3.484,8.344)%
  --(3.486,8.344)--(3.487,8.344)--(3.488,8.344)--(3.490,8.344)--(3.491,8.344)--(3.493,8.344)%
  --(3.494,8.344)--(3.496,8.344)--(3.497,8.344)--(3.499,8.344)--(3.500,8.344)--(3.502,8.344)%
  --(3.504,8.344)--(3.505,8.344)--(3.507,8.344)--(3.508,8.344)--(3.509,8.344)--(3.511,8.344)%
  --(3.512,8.344)--(3.513,8.344)--(3.514,8.344)--(3.516,8.344)--(3.517,8.344)--(3.518,8.344)%
  --(3.520,8.344)--(3.521,8.344)--(3.522,8.344)--(3.524,8.344)--(3.525,8.344)--(3.527,8.344)%
  --(3.528,8.344)--(3.530,8.344)--(3.531,8.344)--(3.533,8.344)--(3.534,8.344)--(3.536,8.344)%
  --(3.537,8.344)--(3.539,8.344)--(3.541,8.344)--(3.542,8.344)--(3.543,8.344)--(3.544,8.344)%
  --(3.546,8.344)--(3.547,8.344)--(3.548,8.344)--(3.550,8.344)--(3.551,8.344)--(3.552,8.344)%
  --(3.554,8.344)--(3.555,8.344)--(3.556,8.344)--(3.558,8.344)--(3.559,8.344)--(3.561,8.344)%
  --(3.562,8.344)--(3.564,8.344)--(3.565,8.344)--(3.567,8.344)--(3.568,8.344)--(3.570,8.344)%
  --(3.571,8.344)--(3.573,8.344)--(3.574,8.344)--(3.576,8.344)--(3.577,8.344)--(3.578,8.344)%
  --(3.580,8.344)--(3.581,8.344)--(3.582,8.344)--(3.583,8.344)--(3.585,8.344)--(3.586,8.344)%
  --(3.587,8.344)--(3.589,8.344)--(3.590,8.344)--(3.592,8.344)--(3.593,8.344)--(3.594,8.344)%
  --(3.596,8.344)--(3.598,8.344)--(3.599,8.344)--(3.601,8.344)--(3.602,8.344)--(3.604,8.344)%
  --(3.605,8.344)--(3.607,8.344)--(3.608,8.344)--(3.610,8.344)--(3.611,8.344)--(3.612,8.344)%
  --(3.613,8.344)--(3.615,8.344)--(3.616,8.344)--(3.617,8.344)--(3.619,8.344)--(3.620,8.344)%
  --(3.621,8.344)--(3.623,8.344)--(3.624,8.344)--(3.625,8.344)--(3.627,8.344)--(3.629,8.344)%
  --(3.630,8.344)--(3.631,8.344)--(3.633,8.344)--(3.635,8.344)--(3.636,8.344)--(3.638,8.344)%
  --(3.639,8.344)--(3.641,8.344)--(3.642,8.344)--(3.644,8.344)--(3.645,8.344)--(3.646,8.344)%
  --(3.648,8.344)--(3.649,8.344)--(3.650,8.344)--(3.651,8.344)--(3.653,8.344)--(3.654,8.344)%
  --(3.655,8.344)--(3.657,8.344)--(3.658,8.344)--(3.659,8.344)--(3.661,8.344)--(3.662,8.344)%
  --(3.664,8.344)--(3.665,8.344)--(3.667,8.344)--(3.668,8.344)--(3.670,8.344)--(3.672,8.344)%
  --(3.673,8.344)--(3.675,8.344)--(3.676,8.344)--(3.678,8.344)--(3.679,8.344)--(3.680,8.344)%
  --(3.681,8.344)--(3.683,8.344)--(3.684,8.344)--(3.685,8.344)--(3.687,8.344)--(3.688,8.344)%
  --(3.689,8.344)--(3.691,8.344)--(3.692,8.344)--(3.693,8.344)--(3.695,8.344)--(3.696,8.344)%
  --(3.698,8.344)--(3.699,8.344)--(3.701,8.344)--(3.702,8.344)--(3.704,8.344)--(3.706,8.344)%
  --(3.707,8.344)--(3.709,8.344)--(3.710,8.344)--(3.711,8.344)--(3.713,8.344)--(3.714,8.344)%
  --(3.715,8.344)--(3.717,8.344)--(3.718,8.344)--(3.719,8.344)--(3.720,8.344)--(3.722,8.344)%
  --(3.723,8.344)--(3.724,8.344)--(3.726,8.344)--(3.727,8.344)--(3.729,8.344)--(3.730,8.344)%
  --(3.732,8.344)--(3.733,8.344)--(3.735,8.344)--(3.736,8.344)--(3.738,8.344)--(3.739,8.344)%
  --(3.741,8.344)--(3.743,8.344)--(3.744,8.344)--(3.745,8.344)--(3.747,8.344)--(3.748,8.344)%
  --(3.749,8.344)--(3.750,8.344)--(3.752,8.344)--(3.753,8.344)--(3.754,8.344)--(3.756,8.344)%
  --(3.757,8.344)--(3.758,8.344)--(3.760,8.344)--(3.761,8.344)--(3.763,8.344)--(3.764,8.344)%
  --(3.766,8.344)--(3.767,8.344)--(3.769,8.344)--(3.770,8.344)--(3.772,8.344)--(3.773,8.344)%
  --(3.775,8.344)--(3.776,8.344)--(3.778,8.344)--(3.779,8.344)--(3.781,8.344)--(3.782,8.344)%
  --(3.783,8.344)--(3.784,8.344)--(3.786,8.344)--(3.787,8.344)--(3.788,8.344)--(3.790,8.344)%
  --(3.791,8.344)--(3.792,8.344)--(3.794,8.344)--(3.795,8.344)--(3.797,8.344)--(3.798,8.344)%
  --(3.800,8.344)--(3.801,8.344)--(3.803,8.344)--(3.804,8.344)--(3.806,8.344)--(3.807,8.344)%
  --(3.809,8.344)--(3.810,8.344)--(3.812,8.344)--(3.813,8.344)--(3.814,8.344)--(3.816,8.344)%
  --(3.817,8.344)--(3.818,8.344)--(3.820,8.344)--(3.821,8.344)--(3.822,8.344)--(3.823,8.344)%
  --(3.825,8.344)--(3.826,8.344)--(3.828,8.344)--(3.829,8.344)--(3.830,8.344)--(3.832,8.344)%
  --(3.833,8.344)--(3.835,8.344)--(3.837,8.344)--(3.838,8.344)--(3.840,8.344)--(3.841,8.344)%
  --(3.843,8.344)--(3.844,8.344)--(3.846,8.344)--(3.847,8.344)--(3.848,8.344)--(3.850,8.344)%
  --(3.851,8.344)--(3.852,8.344)--(3.853,8.344)--(3.855,8.344)--(3.856,8.344)--(3.857,8.344)%
  --(3.859,8.344)--(3.860,8.344)--(3.861,8.344)--(3.863,8.344)--(3.864,8.344)--(3.866,8.344)%
  --(3.867,8.344)--(3.869,8.344)--(3.870,8.344)--(3.872,8.344)--(3.874,8.344)--(3.875,8.344)%
  --(3.877,8.344)--(3.878,8.344)--(3.880,8.344)--(3.881,8.344)--(3.882,8.344)--(3.884,8.344)%
  --(3.885,8.344)--(3.886,8.344)--(3.888,8.344)--(3.889,8.344)--(3.890,8.344)--(3.891,8.344)%
  --(3.893,8.344)--(3.894,8.344)--(3.895,8.344)--(3.897,8.344)--(3.898,8.344)--(3.900,8.344)%
  --(3.901,8.344)--(3.903,8.344)--(3.904,8.344)--(3.906,8.344)--(3.907,8.344)--(3.909,8.344)%
  --(3.910,8.344)--(3.912,8.344)--(3.914,8.344)--(3.915,8.344)--(3.916,8.344)--(3.918,8.344)%
  --(3.919,8.344)--(3.920,8.344)--(3.921,8.344)--(3.923,8.344)--(3.924,8.344)--(3.925,8.344)%
  --(3.927,8.344)--(3.928,8.344)--(3.929,8.344)--(3.931,8.344)--(3.932,8.344)--(3.934,8.344)%
  --(3.935,8.344)--(3.937,8.344)--(3.938,8.344)--(3.940,8.344)--(3.941,8.344)--(3.943,8.344)%
  --(3.945,8.344)--(3.946,8.344)--(3.947,8.344)--(3.949,8.344)--(3.950,8.344)--(3.951,8.344)%
  --(3.953,8.344)--(3.954,8.344)--(3.955,8.344)--(3.956,8.344)--(3.958,8.344)--(3.959,8.344)%
  --(3.960,8.344)--(3.962,8.344)--(3.963,8.344)--(3.965,8.344)--(3.966,8.344)--(3.968,8.344)%
  --(3.969,8.344)--(3.971,8.344)--(3.972,8.344)--(3.974,8.344)--(3.975,8.344)--(3.977,8.344)%
  --(3.978,8.344)--(3.980,8.344)--(3.981,8.344)--(3.983,8.344)--(3.984,8.344)--(3.985,8.344)%
  --(3.987,8.344)--(3.988,8.344)--(3.989,8.344)--(3.990,8.344)--(3.992,8.344)--(3.993,8.344)%
  --(3.994,8.344)--(3.996,8.344)--(3.997,8.344)--(3.999,8.344)--(4.000,8.344)--(4.002,8.344)%
  --(4.003,8.344)--(4.005,8.344)--(4.006,8.344)--(4.008,8.344)--(4.009,8.344)--(4.011,8.344)%
  --(4.012,8.344)--(4.014,8.344)--(4.015,8.344)--(4.017,8.344)--(4.018,8.344)--(4.019,8.344)%
  --(4.021,8.344)--(4.022,8.344)--(4.023,8.344)--(4.024,8.344)--(4.026,8.344)--(4.027,8.344)%
  --(4.028,8.344)--(4.030,8.344)--(4.031,8.344)--(4.032,8.344)--(4.034,8.344)--(4.035,8.344)%
  --(4.037,8.344)--(4.039,8.344)--(4.040,8.344)--(4.042,8.344)--(4.043,8.344)--(4.045,8.344)%
  --(4.046,8.344)--(4.048,8.344)--(4.049,8.344)--(4.050,8.344)--(4.052,8.344)--(4.053,8.344)%
  --(4.054,8.344)--(4.056,8.344)--(4.057,8.344)--(4.058,8.344)--(4.060,8.344)--(4.061,8.344)%
  --(4.062,8.344)--(4.064,8.344)--(4.065,8.344)--(4.066,8.344)--(4.068,8.344)--(4.069,8.344)%
  --(4.071,8.344)--(4.072,8.344)--(4.074,8.344)--(4.075,8.344)--(4.077,8.344)--(4.079,8.344)%
  --(4.080,8.344)--(4.082,8.344)--(4.083,8.344)--(4.084,8.344)--(4.086,8.344)--(4.087,8.344)%
  --(4.088,8.344)--(4.089,8.344)--(4.091,8.344)--(4.092,8.344)--(4.093,8.344)--(4.095,8.344)%
  --(4.096,8.344)--(4.097,8.344)--(4.099,8.344)--(4.100,8.344)--(4.102,8.344)--(4.103,8.344)%
  --(4.105,8.344)--(4.106,8.344)--(4.108,8.344)--(4.109,8.344)--(4.111,8.344)--(4.112,8.344)%
  --(4.114,8.344)--(4.115,8.344)--(4.117,8.344)--(4.118,8.344)--(4.120,8.344)--(4.121,8.344)%
  --(4.122,8.344)--(4.124,8.344)--(4.125,8.344)--(4.126,8.344)--(4.127,8.344)--(4.129,8.344)%
  --(4.130,8.344)--(4.131,8.344)--(4.133,8.344)--(4.134,8.344)--(4.136,8.344)--(4.137,8.344)%
  --(4.139,8.344)--(4.140,8.344)--(4.142,8.344)--(4.143,8.344)--(4.145,8.344)--(4.146,8.344)%
  --(4.148,8.344)--(4.149,8.344)--(4.151,8.344)--(4.152,8.344)--(4.153,8.344)--(4.155,8.344)%
  --(4.156,8.344)--(4.157,8.344)--(4.159,8.344)--(4.160,8.344)--(4.161,8.344)--(4.163,8.344)%
  --(4.164,8.344)--(4.165,8.344)--(4.167,8.344)--(4.168,8.344)--(4.169,8.344)--(4.171,8.344)%
  --(4.173,8.344)--(4.174,8.344)--(4.176,8.344)--(4.177,8.344)--(4.179,8.344)--(4.180,8.344)%
  --(4.182,8.344)--(4.183,8.344)--(4.185,8.344)--(4.186,8.344)--(4.187,8.344)--(4.189,8.344)%
  --(4.190,8.344)--(4.191,8.344)--(4.192,8.344)--(4.194,8.344)--(4.195,8.344)--(4.196,8.344)%
  --(4.198,8.344)--(4.199,8.344)--(4.200,8.344)--(4.202,8.344)--(4.203,8.344)--(4.205,8.344)%
  --(4.206,8.344)--(4.208,8.344)--(4.209,8.344)--(4.211,8.344)--(4.212,8.344)--(4.214,8.344)%
  --(4.216,8.344)--(4.217,8.344)--(4.219,8.344)--(4.220,8.344)--(4.221,8.344)--(4.223,8.344)%
  --(4.224,8.344)--(4.225,8.344)--(4.226,8.344)--(4.228,8.344)--(4.229,8.344)--(4.230,8.344)%
  --(4.232,8.344)--(4.233,8.344)--(4.234,8.344)--(4.236,8.344)--(4.237,8.344)--(4.239,8.344)%
  --(4.240,8.344)--(4.242,8.344)--(4.243,8.344)--(4.245,8.344)--(4.246,8.344)--(4.248,8.344)%
  --(4.249,8.344)--(4.251,8.344)--(4.253,8.344)--(4.254,8.344)--(4.255,8.344)--(4.256,8.344)%
  --(4.258,8.344)--(4.259,8.344)--(4.260,8.344)--(4.262,8.344)--(4.263,8.344)--(4.264,8.344)%
  --(4.266,8.344)--(4.267,8.344)--(4.268,8.344)--(4.270,8.344)--(4.271,8.344)--(4.273,8.344)%
  --(4.274,8.344)--(4.276,8.344)--(4.277,8.344)--(4.279,8.344)--(4.280,8.344)--(4.282,8.344)%
  --(4.283,8.344)--(4.285,8.344)--(4.286,8.344)--(4.288,8.344)--(4.289,8.344)--(4.290,8.344)%
  --(4.292,8.344)--(4.293,8.344)--(4.294,8.344)--(4.295,8.344)--(4.297,8.344)--(4.298,8.344)%
  --(4.299,8.344)--(4.301,8.344)--(4.302,8.344)--(4.304,8.344)--(4.305,8.344)--(4.306,8.344)%
  --(4.308,8.344)--(4.310,8.344)--(4.311,8.344)--(4.313,8.344)--(4.314,8.344)--(4.316,8.344)%
  --(4.317,8.344)--(4.319,8.344)--(4.320,8.344)--(4.322,8.344)--(4.323,8.344)--(4.324,8.344)%
  --(4.325,8.344)--(4.327,8.344)--(4.328,8.344)--(4.329,8.344)--(4.331,8.344)--(4.332,8.344)%
  --(4.333,8.344)--(4.334,8.344)--(4.336,8.344)--(4.337,8.344)--(4.339,8.344)--(4.341,8.344)%
  --(4.342,8.344)--(4.343,8.344)--(4.345,8.344)--(4.347,8.344)--(4.348,8.344)--(4.350,8.344)%
  --(4.351,8.344)--(4.353,8.344)--(4.354,8.344)--(4.356,8.344)--(4.357,8.344)--(4.358,8.344)%
  --(4.359,8.344)--(4.361,8.344)--(4.362,8.344)--(4.363,8.344)--(4.365,8.344)--(4.366,8.344)%
  --(4.367,8.344)--(4.369,8.344)--(4.370,8.344)--(4.371,8.344)--(4.373,8.344)--(4.374,8.344)%
  --(4.376,8.344)--(4.377,8.344)--(4.379,8.344)--(4.380,8.344)--(4.382,8.344)--(4.384,8.344)%
  --(4.385,8.344)--(4.386,8.344)--(4.388,8.344)--(4.389,8.344)--(4.391,8.344)--(4.392,8.344)%
  --(4.393,8.344)--(4.394,8.344)--(4.396,8.344)--(4.397,8.344)--(4.398,8.344)--(4.400,8.344)%
  --(4.401,8.344)--(4.402,8.344)--(4.404,8.344)--(4.405,8.344)--(4.407,8.344)--(4.408,8.344)%
  --(4.410,8.344)--(4.411,8.344)--(4.413,8.344)--(4.414,8.344)--(4.416,8.344)--(4.417,8.344)%
  --(4.419,8.344)--(4.420,8.344)--(4.422,8.344)--(4.423,8.344)--(4.424,8.344)--(4.426,8.344)%
  --(4.427,8.344)--(4.428,8.344)--(4.430,8.344)--(4.431,8.344)--(4.432,8.344)--(4.433,8.344)%
  --(4.435,8.344)--(4.436,8.344)--(4.438,8.344)--(4.439,8.344)--(4.441,8.344)--(4.442,8.344)%
  --(4.443,8.344)--(4.445,8.344)--(4.447,8.344)--(4.448,8.344)--(4.450,8.344)--(4.451,8.344)%
  --(4.453,8.344)--(4.454,8.344)--(4.456,8.344)--(4.457,8.344)--(4.458,8.344)--(4.460,8.344)%
  --(4.461,8.344)--(4.462,8.344)--(4.464,8.344)--(4.465,8.344)--(4.466,8.344)--(4.468,8.344)%
  --(4.469,8.344)--(4.470,8.344)--(4.471,8.344)--(4.473,8.344)--(4.474,8.344)--(4.476,8.344)%
  --(4.478,8.344)--(4.479,8.344)--(4.480,8.344)--(4.482,8.344)--(4.484,8.344)--(4.485,8.344)%
  --(4.487,8.344)--(4.488,8.344)--(4.490,8.344)--(4.491,8.344)--(4.492,8.344)--(4.494,8.344)%
  --(4.495,8.344)--(4.496,8.344)--(4.497,8.344)--(4.499,8.344)--(4.500,8.344)--(4.501,8.344)%
  --(4.503,8.344)--(4.504,8.344)--(4.505,8.344)--(4.507,8.344)--(4.508,8.344)--(4.510,8.344)%
  --(4.511,8.344)--(4.513,8.344)--(4.515,8.344)--(4.516,8.344)--(4.517,8.344)--(4.519,8.344)%
  --(4.521,8.344)--(4.522,8.344)--(4.523,8.344)--(4.525,8.344)--(4.526,8.344)--(4.527,8.344)%
  --(4.529,8.344)--(4.530,8.344)--(4.531,8.344)--(4.533,8.344)--(4.534,8.344)--(4.535,8.344)%
  --(4.536,8.344)--(4.538,8.344)--(4.539,8.344)--(4.541,8.344)--(4.542,8.344)--(4.544,8.344)%
  --(4.545,8.344)--(4.547,8.344)--(4.548,8.344)--(4.550,8.344)--(4.551,8.344)--(4.553,8.344)%
  --(4.554,8.344)--(4.556,8.344)--(4.558,8.344)--(4.559,8.344)--(4.560,8.344)--(4.561,8.344)%
  --(4.563,8.344)--(4.564,8.344)--(4.565,8.344)--(4.567,8.344)--(4.568,8.344)--(4.569,8.344)%
  --(4.570,8.344)--(4.572,8.344)--(4.573,8.344)--(4.575,8.344)--(4.576,8.344)--(4.578,8.344)%
  --(4.579,8.344)--(4.581,8.344)--(4.582,8.344)--(4.584,8.344)--(4.585,8.344)--(4.587,8.344)%
  --(4.588,8.344)--(4.590,8.344)--(4.591,8.344)--(4.593,8.344)--(4.594,8.344)--(4.595,8.344)%
  --(4.597,8.344)--(4.598,8.344)--(4.599,8.344)--(4.600,8.344)--(4.602,8.344)--(4.603,8.344)%
  --(4.604,8.344)--(4.606,8.344)--(4.607,8.344)--(4.608,8.344)--(4.610,8.344)--(4.611,8.344)%
  --(4.613,8.344)--(4.615,8.344)--(4.616,8.344)--(4.618,8.344)--(4.619,8.344)--(4.621,8.344)%
  --(4.622,8.344)--(4.624,8.344)--(4.625,8.344)--(4.626,8.344)--(4.628,8.344)--(4.629,8.344)%
  --(4.630,8.344)--(4.632,8.344)--(4.633,8.344)--(4.634,8.344)--(4.636,8.344)--(4.637,8.344)%
  --(4.638,8.344)--(4.640,8.344)--(4.641,8.344)--(4.642,8.344)--(4.644,8.344)--(4.645,8.344)%
  --(4.647,8.344)--(4.648,8.344)--(4.650,8.344)--(4.652,8.344)--(4.653,8.344)--(4.654,8.344)%
  --(4.656,8.344)--(4.658,8.344)--(4.659,8.344)--(4.660,8.344)--(4.662,8.344)--(4.663,8.344)%
  --(4.664,8.344)--(4.666,8.344)--(4.667,8.344)--(4.668,8.344)--(4.669,8.344)--(4.671,8.344)%
  --(4.672,8.344)--(4.673,8.344)--(4.675,8.344)--(4.676,8.344)--(4.678,8.344)--(4.679,8.344)%
  --(4.681,8.344)--(4.682,8.344)--(4.684,8.344)--(4.685,8.344)--(4.687,8.344)--(4.688,8.344)%
  --(4.690,8.344)--(4.691,8.344)--(4.693,8.344)--(4.694,8.344)--(4.696,8.344)--(4.697,8.344)%
  --(4.698,8.344)--(4.700,8.344)--(4.701,8.344)--(4.702,8.344)--(4.703,8.344)--(4.705,8.344)%
  --(4.706,8.344)--(4.707,8.344)--(4.709,8.344)--(4.710,8.344)--(4.712,8.344)--(4.713,8.344)%
  --(4.715,8.344)--(4.716,8.344)--(4.718,8.344)--(4.719,8.344)--(4.721,8.344)--(4.722,8.344)%
  --(4.724,8.344)--(4.725,8.344)--(4.727,8.344)--(4.728,8.344)--(4.729,8.344)--(4.731,8.344)%
  --(4.732,8.344)--(4.733,8.344)--(4.735,8.344)--(4.736,8.344)--(4.737,8.344)--(4.738,8.344)%
  --(4.740,8.344)--(4.741,8.344)--(4.743,8.344)--(4.744,8.344)--(4.746,8.344)--(4.747,8.344)%
  --(4.748,8.344)--(4.750,8.344)--(4.752,8.344)--(4.753,8.344)--(4.755,8.344)--(4.756,8.344)%
  --(4.758,8.344)--(4.759,8.344)--(4.761,8.344)--(4.762,8.344)--(4.763,8.344)--(4.765,8.344)%
  --(4.766,8.344)--(4.767,8.344)--(4.768,8.344)--(4.770,8.344)--(4.771,8.344)--(4.772,8.344)%
  --(4.774,8.344)--(4.775,8.344)--(4.776,8.344)--(4.778,8.344)--(4.780,8.344)--(4.781,8.344)%
  --(4.782,8.344)--(4.784,8.344)--(4.785,8.344)--(4.787,8.344)--(4.789,8.344)--(4.790,8.344)%
  --(4.792,8.344)--(4.793,8.344)--(4.795,8.344)--(4.796,8.344)--(4.797,8.344)--(4.799,8.344)%
  --(4.800,8.344)--(4.801,8.344)--(4.802,8.344)--(4.804,8.344)--(4.805,8.344)--(4.806,8.344)%
  --(4.808,8.344)--(4.809,8.344)--(4.810,8.344)--(4.812,8.344)--(4.813,8.344)--(4.815,8.344)%
  --(4.816,8.344)--(4.818,8.344)--(4.819,8.344)--(4.821,8.344)--(4.822,8.344)--(4.824,8.344)%
  --(4.826,8.344)--(4.827,8.344)--(4.829,8.344)--(4.830,8.344)--(4.831,8.344)--(4.832,8.344)%
  --(4.834,8.344)--(4.835,8.344)--(4.836,8.344)--(4.838,8.344)--(4.839,8.344)--(4.840,8.344)%
  --(4.842,8.344)--(4.843,8.344)--(4.844,8.344)--(4.846,8.344)--(4.847,8.344)--(4.849,8.344)%
  --(4.850,8.344)--(4.852,8.344)--(4.853,8.344)--(4.855,8.344)--(4.856,8.344)--(4.858,8.344)%
  --(4.859,8.344)--(4.861,8.344)--(4.862,8.344)--(4.864,8.344)--(4.865,8.344)--(4.866,8.344)%
  --(4.867,8.344)--(4.869,8.344)--(4.870,8.344)--(4.871,8.344)--(4.873,8.344)--(4.874,8.344)%
  --(4.875,8.344)--(4.877,8.344)--(4.878,8.344)--(4.880,8.344)--(4.881,8.344)--(4.883,8.344)%
  --(4.884,8.344)--(4.886,8.344)--(4.887,8.344)--(4.889,8.344)--(4.890,8.344)--(4.892,8.344)%
  --(4.893,8.344)--(4.895,8.344)--(4.896,8.344)--(4.898,8.344)--(4.899,8.344)--(4.900,8.344)%
  --(4.902,8.344)--(4.903,8.344)--(4.904,8.344)--(4.905,8.344)--(4.907,8.344)--(4.908,8.344)%
  --(4.909,8.344)--(4.911,8.344)--(4.912,8.344)--(4.913,8.344)--(4.915,8.344)--(4.917,8.344)%
  --(4.918,8.344)--(4.919,8.344)--(4.921,8.344)--(4.923,8.344)--(4.924,8.344)--(4.926,8.344)%
  --(4.927,8.344)--(4.929,8.344)--(4.930,8.344)--(4.931,8.344)--(4.933,8.344)--(4.934,8.344)%
  --(4.935,8.344)--(4.937,8.344)--(4.938,8.344)--(4.939,8.344)--(4.940,8.344)--(4.942,8.344)%
  --(4.943,8.344)--(4.944,8.344)--(4.946,8.344)--(4.947,8.344)--(4.949,8.344)--(4.950,8.344)%
  --(4.952,8.344)--(4.953,8.344)--(4.955,8.344)--(4.956,8.344)--(4.958,8.344)--(4.960,8.344)%
  --(4.961,8.344)--(4.963,8.344)--(4.964,8.344)--(4.965,8.344)--(4.967,8.344)--(4.968,8.344)%
  --(4.969,8.344)--(4.971,8.344)--(4.972,8.344)--(4.973,8.344)--(4.974,8.344)--(4.976,8.344)%
  --(4.977,8.344)--(4.978,8.344)--(4.980,8.344)--(4.981,8.344)--(4.983,8.344)--(4.984,8.344)%
  --(4.986,8.344)--(4.987,8.344)--(4.989,8.344)--(4.990,8.344)--(4.992,8.344)--(4.993,8.344)%
  --(4.995,8.344)--(4.996,8.344)--(4.998,8.344)--(4.999,8.344)--(5.001,8.344)--(5.002,8.344)%
  --(5.003,8.344)--(5.004,8.344)--(5.006,8.344)--(5.007,8.344)--(5.008,8.344)--(5.010,8.344)%
  --(5.011,8.344)--(5.012,8.344)--(5.013,8.344)--(5.015,8.344)--(5.017,8.344)--(5.018,8.344)%
  --(5.020,8.344)--(5.021,8.344)--(5.023,8.344)--(5.024,8.344)--(5.026,8.344)--(5.027,8.344)%
  --(5.029,8.344)--(5.030,8.344)--(5.032,8.344)--(5.033,8.344)--(5.035,8.344)--(5.036,8.344)%
  --(5.037,8.344)--(5.038,8.344)--(5.040,8.344)--(5.041,8.344)--(5.042,8.344)--(5.044,8.344)%
  --(5.045,8.344)--(5.046,8.344)--(5.048,8.344)--(5.049,8.344)--(5.051,8.344)--(5.052,8.344)%
  --(5.054,8.344)--(5.055,8.344)--(5.057,8.344)--(5.058,8.344)--(5.060,8.344)--(5.061,8.344)%
  --(5.063,8.344)--(5.064,8.344)--(5.066,8.344)--(5.067,8.344)--(5.068,8.344)--(5.069,8.344)%
  --(5.071,8.344)--(5.072,8.344)--(5.073,8.344)--(5.075,8.344)--(5.076,8.344)--(5.077,8.344)%
  --(5.079,8.344)--(5.080,8.344)--(5.081,8.344)--(5.083,8.344)--(5.084,8.344)--(5.086,8.344)%
  --(5.087,8.344)--(5.089,8.344)--(5.090,8.344)--(5.092,8.344)--(5.093,8.344)--(5.095,8.344)%
  --(5.097,8.344)--(5.098,8.344)--(5.100,8.344)--(5.101,8.344)--(5.102,8.344)--(5.104,8.344)%
  --(5.105,8.344)--(5.106,8.344)--(5.107,8.344)--(5.109,8.344)--(5.110,8.344)--(5.111,8.344)%
  --(5.112,8.344)--(5.113,8.344)--(5.114,8.344)--(5.115,7.087)--(5.117,6.217)--(5.118,5.549)%
  --(5.119,4.976)--(5.120,4.434)--(5.121,3.867)--(5.123,3.211)--(5.124,2.366)--(5.124,2.219)%
  --(5.124,2.066)--(5.124,1.902)--(5.125,1.728)--(5.125,1.545)--(5.125,1.344)--(5.125,1.131)%
  --(5.125,1.022)--(5.126,1.022)--(5.127,1.022)--(5.128,1.022)--(5.129,1.022)--(5.130,1.022)%
  --(5.131,1.022)--(5.132,1.022)--(5.133,1.022)--(5.134,1.022)--(5.135,1.022)--(5.136,1.022)%
  --(5.137,1.022)--(5.139,1.022)--(5.140,1.022)--(5.141,1.022)--(5.142,1.022)--(5.143,1.022)%
  --(5.144,1.022)--(5.146,1.022)--(5.147,1.022)--(5.148,1.022)--(5.150,1.022)--(5.151,1.022)%
  --(5.152,1.022)--(5.154,1.022)--(5.155,1.022)--(5.156,1.022)--(5.157,1.022)--(5.158,1.022)%
  --(5.159,1.022)--(5.160,1.022)--(5.161,1.022)--(5.162,1.022)--(5.163,1.022)--(5.164,1.022)%
  --(5.165,1.022)--(5.166,1.022)--(5.168,1.022)--(5.169,1.022)--(5.170,1.022)--(5.171,1.022)%
  --(5.172,1.022)--(5.173,1.022)--(5.174,1.022)--(5.175,1.022)--(5.176,1.022)--(5.178,1.022)%
  --(5.179,1.022)--(5.180,1.022)--(5.182,1.022)--(5.183,1.022)--(5.185,1.022)--(5.186,1.022)%
  --(5.187,1.022)--(5.188,1.022)--(5.189,1.022)--(5.190,1.022)--(5.192,1.022)--(5.193,1.022)%
  --(5.194,1.022)--(5.195,1.022)--(5.196,1.022)--(5.197,1.022)--(5.198,1.022)--(5.199,1.022)%
  --(5.200,1.022)--(5.201,1.022)--(5.202,1.022)--(5.203,1.022)--(5.204,1.022)--(5.205,1.022)%
  --(5.206,1.022)--(5.208,1.022)--(5.209,1.022)--(5.210,1.022)--(5.211,1.022)--(5.213,1.022)%
  --(5.214,1.022)--(5.215,1.022)--(5.216,1.022)--(5.218,1.022)--(5.219,1.022)--(5.220,1.022)%
  --(5.221,1.022)--(5.223,1.022)--(5.224,1.022)--(5.225,1.022)--(5.226,1.022)--(5.227,1.022)%
  --(5.228,1.022)--(5.229,1.022)--(5.230,1.022)--(5.231,1.022)--(5.232,1.022)--(5.233,1.022)%
  --(5.234,1.022)--(5.235,1.022)--(5.237,1.022)--(5.238,1.022)--(5.239,1.022)--(5.240,1.022)%
  --(5.241,1.022)--(5.242,1.022)--(5.243,1.022)--(5.244,1.022)--(5.246,1.022)--(5.247,1.022)%
  --(5.248,1.022)--(5.250,1.022)--(5.251,1.022)--(5.252,1.022)--(5.254,1.022)--(5.255,1.022)%
  --(5.256,1.022)--(5.258,1.022)--(5.259,1.022)--(5.260,1.022)--(5.261,1.022)--(5.262,1.022)%
  --(5.263,1.022)--(5.264,1.022)--(5.265,1.022)--(5.266,1.022)--(5.267,1.022)--(5.268,1.022)%
  --(5.270,1.022)--(5.271,1.022)--(5.272,1.022)--(5.273,1.022)--(5.274,1.022)--(5.275,1.022)%
  --(5.276,1.022)--(5.277,1.022)--(5.279,1.022)--(5.280,1.022)--(5.281,1.022)--(5.283,1.022)%
  --(5.284,1.022)--(5.286,1.022)--(5.287,1.022)--(5.289,1.022)--(5.290,1.022)--(5.291,1.022)%
  --(5.293,1.022)--(5.294,1.022)--(5.295,1.022)--(5.296,1.022)--(5.297,1.022)--(5.298,1.022)%
  --(5.299,1.022)--(5.300,1.022)--(5.301,1.022)--(5.303,1.022)--(5.304,1.022)--(5.305,1.022)%
  --(5.306,1.022)--(5.307,1.022)--(5.308,1.022)--(5.309,1.022)--(5.310,1.022)--(5.311,1.022)%
  --(5.313,1.022)--(5.314,1.022)--(5.315,1.022)--(5.317,1.022)--(5.318,1.022)--(5.320,1.022)%
  --(5.322,1.022)--(5.323,1.022)--(5.324,1.022)--(5.326,1.022)--(5.327,1.022)--(5.328,1.022)%
  --(5.329,1.022)--(5.330,1.022)--(5.331,1.022)--(5.332,1.022)--(5.333,1.022)--(5.334,1.022)%
  --(5.336,1.022)--(5.337,1.022)--(5.338,1.022)--(5.339,1.022)--(5.340,1.022)--(5.341,1.022)%
  --(5.342,1.022)--(5.343,1.022)--(5.344,1.022)--(5.345,1.022)--(5.347,1.022)--(5.348,1.022)%
  --(5.349,1.022)--(5.350,1.022)--(5.352,1.022)--(5.353,1.022)--(5.355,1.022)--(5.356,1.022)%
  --(5.357,1.022)--(5.359,1.022)--(5.360,1.022)--(5.361,1.022)--(5.362,1.022)--(5.363,1.022)%
  --(5.364,1.022)--(5.365,1.022)--(5.366,1.022)--(5.367,1.022)--(5.369,1.022)--(5.370,1.022)%
  --(5.371,1.022)--(5.372,1.022)--(5.373,1.022)--(5.374,1.022)--(5.375,1.022)--(5.376,1.022)%
  --(5.377,1.022)--(5.378,1.022)--(5.379,1.022)--(5.381,1.022)--(5.382,1.022)--(5.383,1.022)%
  --(5.385,1.022)--(5.386,1.022)--(5.388,1.022)--(5.389,1.022)--(5.390,1.022)--(5.391,1.022)%
  --(5.392,1.022)--(5.393,1.022)--(5.394,1.022)--(5.396,1.022)--(5.397,1.022)--(5.398,1.022)%
  --(5.399,1.022)--(5.400,1.022)--(5.401,1.022)--(5.402,1.022)--(5.403,1.022)--(5.404,1.022)%
  --(5.405,1.022)--(5.406,1.022)--(5.407,1.022)--(5.408,1.022)--(5.409,1.022)--(5.411,1.022)%
  --(5.412,1.022)--(5.413,1.022)--(5.414,1.022)--(5.416,1.022)--(5.417,1.022)--(5.418,1.022)%
  --(5.419,1.022)--(5.421,1.022)--(5.422,1.022)--(5.423,1.022)--(5.424,1.022)--(5.426,1.022)%
  --(5.427,1.022)--(5.428,1.022)--(5.429,1.022)--(5.430,1.022)--(5.431,1.022)--(5.432,1.022)%
  --(5.433,1.022)--(5.435,1.022)--(5.436,1.022)--(5.437,1.022)--(5.438,1.022)--(5.439,1.022)%
  --(5.440,1.022)--(5.441,1.022)--(5.442,1.022)--(5.443,1.022)--(5.444,1.022)--(5.445,1.022)%
  --(5.446,1.022)--(5.447,1.022)--(5.449,1.022)--(5.450,1.022)--(5.451,1.022)--(5.452,1.022)%
  --(5.454,1.022)--(5.455,1.022)--(5.456,1.022)--(5.458,1.022)--(5.459,1.022)--(5.461,1.022)%
  --(5.462,1.022)--(5.463,1.022)--(5.464,1.022)--(5.465,1.022)--(5.466,1.022)--(5.467,1.022)%
  --(5.468,1.022)--(5.469,1.022)--(5.470,1.022)--(5.472,1.022)--(5.473,1.022)--(5.474,1.022)%
  --(5.475,1.022)--(5.476,1.022)--(5.477,1.022)--(5.478,1.022)--(5.479,1.022)--(5.480,1.022)%
  --(5.482,1.022)--(5.483,1.022)--(5.484,1.022)--(5.485,1.022)--(5.487,1.022)--(5.488,1.022)%
  --(5.490,1.022)--(5.491,1.022)--(5.493,1.022)--(5.494,1.022)--(5.496,1.022)--(5.497,1.022)%
  --(5.498,1.022)--(5.499,1.022)--(5.500,1.022)--(5.501,1.022)--(5.502,1.022)--(5.503,1.022)%
  --(5.504,1.022)--(5.505,1.022)--(5.506,1.022)--(5.507,1.022)--(5.508,1.022)--(5.510,1.022)%
  --(5.511,1.022)--(5.512,1.022)--(5.513,1.022)--(5.514,1.022)--(5.515,1.022)--(5.516,1.022)%
  --(5.518,1.022)--(5.519,1.022)--(5.521,1.022)--(5.522,1.022)--(5.524,1.022)--(5.525,1.022)%
  --(5.526,1.022)--(5.527,1.022)--(5.528,1.022)--(5.529,1.022)--(5.530,1.022)--(5.532,1.022)%
  --(5.533,1.022)--(5.534,1.022)--(5.535,1.022)--(5.536,1.022)--(5.537,1.022)--(5.538,1.022)%
  --(5.539,1.022)--(5.540,1.022)--(5.541,1.022)--(5.542,1.022)--(5.543,1.022)--(5.544,1.022)%
  --(5.545,1.022)--(5.547,1.022)--(5.548,1.022)--(5.549,1.022)--(5.550,1.022)--(5.551,1.022)%
  --(5.553,1.022)--(5.554,1.022)--(5.555,1.022)--(5.557,1.022)--(5.558,1.022)--(5.559,1.022)%
  --(5.560,1.022)--(5.561,1.022)--(5.563,1.022)--(5.564,1.022)--(5.565,1.022)--(5.566,1.022)%
  --(5.567,1.022)--(5.568,1.022)--(5.569,1.022)--(5.570,1.022)--(5.571,1.022)--(5.572,1.022)%
  --(5.573,1.022)--(5.574,1.022)--(5.576,1.022)--(5.577,1.022)--(5.578,1.022)--(5.579,1.022)%
  --(5.580,1.022)--(5.581,1.022)--(5.582,1.022)--(5.583,1.022)--(5.585,1.022)--(5.586,1.022)%
  --(5.587,1.022)--(5.588,1.022)--(5.590,1.022)--(5.591,1.022)--(5.592,1.022)--(5.594,1.022)%
  --(5.595,1.022)--(5.597,1.022)--(5.598,1.022)--(5.599,1.022)--(5.600,1.022)--(5.601,1.022)%
  --(5.602,1.022)--(5.603,1.022)--(5.604,1.022)--(5.605,1.022)--(5.606,1.022)--(5.607,1.022)%
  --(5.609,1.022)--(5.610,1.022)--(5.611,1.022)--(5.612,1.022)--(5.613,1.022)--(5.614,1.022)%
  --(5.615,1.022)--(5.616,1.022)--(5.617,1.022)--(5.619,1.022)--(5.620,1.022)--(5.622,1.022)%
  --(5.623,1.022)--(5.625,1.022)--(5.626,1.022)--(5.628,1.022)--(5.629,1.022)--(5.631,1.022)%
  --(5.632,1.022)--(5.633,1.022)--(5.634,1.022)--(5.635,1.022)--(5.636,1.022)--(5.637,1.022)%
  --(5.638,1.022)--(5.639,1.022)--(5.640,1.022)--(5.642,1.022)--(5.643,1.022)--(5.644,1.022)%
  --(5.645,1.022)--(5.646,1.022)--(5.647,1.022)--(5.648,1.022)--(5.649,1.022)--(5.650,1.022)%
  --(5.652,1.022)--(5.653,1.022)--(5.654,1.022)--(5.656,1.022)--(5.657,1.022)--(5.658,1.022)%
  --(5.659,1.022)--(5.661,1.022)--(5.662,1.022)--(5.663,1.022)--(5.665,1.022)--(5.666,1.022)%
  --(5.667,1.022)--(5.668,1.022)--(5.669,1.022)--(5.670,1.022)--(5.671,1.022)--(5.672,1.022)%
  --(5.673,1.022)--(5.675,1.022)--(5.676,1.022)--(5.677,1.022)--(5.678,1.022)--(5.679,1.022)%
  --(5.680,1.022)--(5.681,1.022)--(5.682,1.022)--(5.683,1.022)--(5.685,1.022)--(5.686,1.022)%
  --(5.687,1.022)--(5.688,1.022)--(5.690,1.022)--(5.691,1.022)--(5.693,1.022)--(5.694,1.022)%
  --(5.695,1.022)--(5.696,1.022)--(5.697,1.022)--(5.698,1.022)--(5.700,1.022)--(5.701,1.022)%
  --(5.702,1.022)--(5.703,1.022)--(5.704,1.022)--(5.705,1.022)--(5.706,1.022)--(5.707,1.022)%
  --(5.708,1.022)--(5.709,1.022)--(5.710,1.022)--(5.711,1.022)--(5.712,1.022)--(5.713,1.022)%
  --(5.715,1.022)--(5.716,1.022)--(5.717,1.022)--(5.718,1.022)--(5.719,1.022)--(5.720,1.022)%
  --(5.722,1.022)--(5.723,1.022)--(5.724,1.022)--(5.726,1.022)--(5.727,1.022)--(5.728,1.022)%
  --(5.729,1.022)--(5.730,1.022)--(5.732,1.022)--(5.733,1.022)--(5.734,1.022)--(5.735,1.022)%
  --(5.736,1.022)--(5.737,1.022)--(5.738,1.022)--(5.739,1.022)--(5.741,1.022)--(5.742,1.022)%
  --(5.743,1.022)--(5.744,1.022)--(5.745,1.022)--(5.746,1.022)--(5.747,1.022)--(5.748,1.022)%
  --(5.749,1.022)--(5.750,1.022)--(5.751,1.022)--(5.753,1.022)--(5.754,1.022)--(5.755,1.022)%
  --(5.756,1.022)--(5.758,1.022)--(5.759,1.022)--(5.760,1.022)--(5.762,1.022)--(5.763,1.022)%
  --(5.764,1.022)--(5.766,1.022)--(5.767,1.022)--(5.768,1.022)--(5.769,1.022)--(5.770,1.022)%
  --(5.771,1.022)--(5.772,1.022)--(5.773,1.022)--(5.774,1.022)--(5.775,1.022)--(5.777,1.022)%
  --(5.778,1.022)--(5.779,1.022)--(5.780,1.022)--(5.781,1.022)--(5.782,1.022)--(5.783,1.022)%
  --(5.784,1.022)--(5.785,1.022)--(5.786,1.022)--(5.788,1.022)--(5.789,1.022)--(5.790,1.022)%
  --(5.792,1.022)--(5.794,1.022)--(5.795,1.022)--(5.796,1.022)--(5.798,1.022)--(5.799,1.022)%
  --(5.801,1.022)--(5.802,1.022)--(5.803,1.022)--(5.804,1.022)--(5.805,1.022)--(5.806,1.022)%
  --(5.807,1.022)--(5.808,1.022)--(5.809,1.022)--(5.810,1.022)--(5.811,1.022)--(5.813,1.022)%
  --(5.814,1.022)--(5.815,1.022)--(5.816,1.022)--(5.817,1.022)--(5.818,1.022)--(5.819,1.022)%
  --(5.820,1.022)--(5.822,1.022)--(5.823,1.022)--(5.824,1.022)--(5.826,1.022)--(5.827,1.022)%
  --(5.829,1.022)--(5.830,1.022)--(5.831,1.022)--(5.832,1.022)--(5.833,1.022)--(5.835,1.022)%
  --(5.836,1.022)--(5.837,1.022)--(5.838,1.022)--(5.839,1.022)--(5.840,1.022)--(5.841,1.022)%
  --(5.842,1.022)--(5.843,1.022)--(5.844,1.022)--(5.845,1.022)--(5.846,1.022)--(5.847,1.022)%
  --(5.848,1.022)--(5.849,1.022)--(5.851,1.022)--(5.852,1.022)--(5.853,1.022)--(5.854,1.022)%
  --(5.855,1.022)--(5.856,1.022)--(5.858,1.022)--(5.859,1.022)--(5.861,1.022)--(5.862,1.022)%
  --(5.863,1.022)--(5.864,1.022)--(5.865,1.022)--(5.866,1.022)--(5.868,1.022)--(5.869,1.022)%
  --(5.870,1.022)--(5.871,1.022)--(5.872,1.022)--(5.873,1.022)--(5.874,1.022)--(5.875,1.022)%
  --(5.876,1.022)--(5.878,1.022)--(5.879,1.022)--(5.880,1.022)--(5.881,1.022)--(5.882,1.022)%
  --(5.883,1.022)--(5.884,1.022)--(5.885,1.022)--(5.886,1.022)--(5.887,1.022)--(5.889,1.022)%
  --(5.890,1.022)--(5.891,1.022)--(5.892,1.022)--(5.894,1.022)--(5.895,1.022)--(5.896,1.022)%
  --(5.898,1.022)--(5.899,1.022)--(5.901,1.022)--(5.902,1.022)--(5.903,1.022)--(5.904,1.022)%
  --(5.905,1.022)--(5.906,1.022)--(5.907,1.022)--(5.908,1.022)--(5.909,1.022)--(5.911,1.022)%
  --(5.912,1.022)--(5.913,1.022)--(5.914,1.022)--(5.915,1.022)--(5.916,1.022)--(5.917,1.022)%
  --(5.918,1.022)--(5.919,1.022)--(5.920,1.022)--(5.922,1.022)--(5.923,1.022)--(5.924,1.022)%
  --(5.925,1.022)--(5.927,1.022)--(5.928,1.022)--(5.929,1.022)--(5.931,1.022)--(5.932,1.022)%
  --(5.933,1.022)--(5.935,1.022)--(5.936,1.022)--(5.937,1.022)--(5.938,1.022)--(5.939,1.022)%
  --(5.940,1.022)--(5.941,1.022)--(5.942,1.022)--(5.944,1.022)--(5.945,1.022)--(5.946,1.022)%
  --(5.947,1.022)--(5.948,1.022)--(5.949,1.022)--(5.950,1.022)--(5.951,1.022)--(5.952,1.022)%
  --(5.953,1.022)--(5.954,1.022)--(5.956,1.022)--(5.957,1.022)--(5.958,1.022)--(5.959,1.022)%
  --(5.961,1.022)--(5.963,1.022)--(5.964,1.022)--(5.965,1.022)--(5.966,1.022)--(5.967,1.022)%
  --(5.968,1.022)--(5.970,1.022)--(5.971,1.022)--(5.972,1.022)--(5.973,1.022)--(5.974,1.022)%
  --(5.975,1.022)--(5.976,1.022)--(5.977,1.022)--(5.978,1.022)--(5.979,1.022)--(5.980,1.022)%
  --(5.981,1.022)--(5.983,1.022)--(5.984,1.022)--(5.985,1.022)--(5.986,1.022)--(5.987,1.022)%
  --(5.988,1.022)--(5.989,1.022)--(5.990,1.022)--(5.992,1.022)--(5.993,1.022)--(5.994,1.022)%
  --(5.996,1.022)--(5.997,1.022)--(5.998,1.022)--(5.999,1.022)--(6.001,1.022)--(6.002,1.022)%
  --(6.003,1.022)--(6.004,1.022)--(6.005,1.022)--(6.006,1.022)--(6.007,1.022)--(6.008,1.022)%
  --(6.010,1.022)--(6.011,1.022)--(6.012,1.022)--(6.013,1.022)--(6.014,1.022)--(6.015,1.022)%
  --(6.016,1.022)--(6.017,1.022)--(6.018,1.022)--(6.019,1.022)--(6.020,1.022)--(6.021,1.022)%
  --(6.022,1.022)--(6.024,1.022)--(6.025,1.022)--(6.026,1.022)--(6.027,1.022)--(6.029,1.022)%
  --(6.030,1.022)--(6.031,1.022)--(6.033,1.022)--(6.034,1.022)--(6.035,1.022)--(6.037,1.022)%
  --(6.038,1.022)--(6.039,1.022)--(6.040,1.022)--(6.041,1.022)--(6.043,1.022)--(6.044,1.022)%
  --(6.045,1.022)--(6.046,1.022)--(6.047,1.022)--(6.048,1.022)--(6.049,1.022)--(6.050,1.022)%
  --(6.051,1.022)--(6.052,1.022)--(6.053,1.022)--(6.054,1.022)--(6.055,1.022)--(6.057,1.022)%
  --(6.058,1.022)--(6.059,1.022)--(6.060,1.022)--(6.062,1.022)--(6.063,1.022)--(6.065,1.022)%
  --(6.066,1.022)--(6.068,1.022)--(6.069,1.022)--(6.071,1.022)--(6.072,1.022)--(6.073,1.022)%
  --(6.074,1.022)--(6.075,1.022)--(6.076,1.022)--(6.077,1.022)--(6.078,1.022)--(6.079,1.022)%
  --(6.081,1.022)--(6.082,1.022)--(6.083,1.022)--(6.084,1.022)--(6.085,1.022)--(6.086,1.022)%
  --(6.087,1.022)--(6.088,1.022)--(6.089,1.022)--(6.090,1.022)--(6.092,1.022)--(6.093,1.022)%
  --(6.094,1.022)--(6.096,1.022)--(6.097,1.022)--(6.099,1.022)--(6.100,1.022)--(6.102,1.022)%
  --(6.103,1.022)--(6.104,1.022)--(6.106,1.022)--(6.107,1.022)--(6.108,1.022)--(6.109,1.022)%
  --(6.110,1.022)--(6.111,1.022)--(6.112,1.022)--(6.113,1.022)--(6.114,1.022)--(6.115,1.022)%
  --(6.116,1.022)--(6.118,1.022)--(6.119,1.022)--(6.120,1.022)--(6.121,1.022)--(6.122,1.022)%
  --(6.123,1.022)--(6.124,1.022)--(6.125,1.022)--(6.127,1.022)--(6.128,1.022)--(6.129,1.022)%
  --(6.131,1.022)--(6.133,1.022)--(6.134,1.022)--(6.135,1.022)--(6.137,1.022)--(6.138,1.022)%
  --(6.140,1.022)--(6.141,1.022)--(6.142,1.022)--(6.143,1.022)--(6.144,1.022)--(6.145,1.022)%
  --(6.146,1.022)--(6.147,1.022)--(6.148,1.022)--(6.149,1.022)--(6.150,1.022)--(6.151,1.022)%
  --(6.152,1.022)--(6.153,1.022)--(6.155,1.022)--(6.156,1.022)--(6.157,1.022)--(6.158,1.022)%
  --(6.159,1.022)--(6.160,1.022)--(6.162,1.022)--(6.163,1.022)--(6.165,1.022)--(6.166,1.022)%
  --(6.168,1.022)--(6.169,1.022)--(6.171,1.022)--(6.172,1.022)--(6.173,1.022)--(6.175,1.022)%
  --(6.176,1.022)--(6.177,1.022)--(6.178,1.022)--(6.179,1.022)--(6.180,1.022)--(6.181,1.022)%
  --(6.182,1.022)--(6.183,1.022)--(6.184,1.022)--(6.185,1.022)--(6.186,1.022)--(6.187,1.022)%
  --(6.189,1.022)--(6.190,1.022)--(6.191,1.022)--(6.192,1.022)--(6.193,1.022)--(6.194,1.022)%
  --(6.195,1.022)--(6.197,1.022)--(6.199,1.022)--(6.200,1.022)--(6.201,1.022)--(6.203,1.022)%
  --(6.204,1.022)--(6.205,1.022)--(6.206,1.022)--(6.207,1.022)--(6.208,1.022)--(6.209,1.022)%
  --(6.210,1.022)--(6.211,1.022)--(6.213,1.022)--(6.214,1.022)--(6.215,1.022)--(6.216,1.022)%
  --(6.217,1.022)--(6.218,1.022)--(6.219,1.022)--(6.220,1.022)--(6.221,1.022)--(6.222,1.022)%
  --(6.223,1.022)--(6.224,1.022)--(6.225,1.022)--(6.227,1.022)--(6.228,1.022)--(6.229,1.022)%
  --(6.231,1.022)--(6.232,1.022)--(6.233,1.022)--(6.234,1.022)--(6.236,1.022)--(6.237,1.022)%
  --(6.238,1.022)--(6.239,1.022)--(6.240,1.022)--(6.242,1.022)--(6.243,1.022)--(6.244,1.022)%
  --(6.245,1.022)--(6.246,1.022)--(6.247,1.022)--(6.248,1.022)--(6.249,1.022)--(6.250,1.022)%
  --(6.251,1.022)--(6.252,1.022)--(6.253,1.022)--(6.254,1.022)--(6.255,1.022)--(6.257,1.022)%
  --(6.258,1.022)--(6.259,1.022)--(6.260,1.022)--(6.261,1.022)--(6.262,1.022)--(6.264,1.022)%
  --(6.265,1.022)--(6.266,1.022)--(6.268,1.022)--(6.269,1.022)--(6.270,1.022)--(6.272,1.022)%
  --(6.273,1.022)--(6.274,1.022)--(6.276,1.022)--(6.277,1.022)--(6.278,1.022)--(6.279,1.022)%
  --(6.280,1.022)--(6.281,1.022)--(6.282,1.022)--(6.283,1.022)--(6.284,1.022)--(6.285,1.022)%
  --(6.286,1.022)--(6.288,1.022)--(6.289,1.022)--(6.290,1.022)--(6.291,1.022)--(6.292,1.022)%
  --(6.293,1.022)--(6.294,1.022)--(6.295,1.022)--(6.297,1.022)--(6.298,1.022)--(6.299,1.022)%
  --(6.301,1.022)--(6.302,1.022)--(6.303,1.022)--(6.305,1.022)--(6.306,1.022)--(6.307,1.022)%
  --(6.309,1.022)--(6.310,1.022)--(6.311,1.022)--(6.312,1.022)--(6.313,1.022)--(6.314,1.022)%
  --(6.315,1.022)--(6.316,1.022)--(6.317,1.022)--(6.318,1.022)--(6.319,1.022)--(6.321,1.022)%
  --(6.322,1.022)--(6.323,1.022)--(6.324,1.022)--(6.325,1.022)--(6.326,1.022)--(6.327,1.022)%
  --(6.328,1.022)--(6.330,1.022)--(6.331,1.022)--(6.332,1.022)--(6.334,1.022)--(6.335,1.022)%
  --(6.337,1.022)--(6.338,1.022)--(6.339,1.022)--(6.340,1.022)--(6.341,1.022)--(6.342,1.022)%
  --(6.343,1.022)--(6.345,1.022)--(6.346,1.022)--(6.347,1.022)--(6.348,1.022)--(6.349,1.022)%
  --(6.350,1.022)--(6.351,1.022)--(6.352,1.022)--(6.353,1.022)--(6.354,1.022)--(6.355,1.022)%
  --(6.356,1.022)--(6.357,1.022)--(6.358,1.022)--(6.360,1.022)--(6.361,1.022)--(6.362,1.022)%
  --(6.363,1.022)--(6.364,1.022)--(6.366,1.022)--(6.367,1.022)--(6.368,1.022)--(6.369,1.022)%
  --(6.371,1.022)--(6.372,1.022)--(6.373,1.022)--(6.374,1.022)--(6.375,1.022)--(6.377,1.022)%
  --(6.378,1.022)--(6.379,1.022)--(6.380,1.022)--(6.381,1.022)--(6.382,1.022)--(6.383,1.022)%
  --(6.384,1.022)--(6.385,1.022)--(6.386,1.022)--(6.387,1.022)--(6.388,1.022)--(6.390,1.022)%
  --(6.391,1.022)--(6.392,1.022)--(6.393,1.022)--(6.394,1.022)--(6.395,1.022)--(6.396,1.022)%
  --(6.398,1.022)--(6.399,1.022)--(6.400,1.022)--(6.401,1.022)--(6.403,1.022)--(6.404,1.022)%
  --(6.405,1.022)--(6.407,1.022)--(6.408,1.022)--(6.409,1.022)--(6.411,1.022)--(6.412,1.022)%
  --(6.413,1.022)--(6.414,1.022)--(6.415,1.022)--(6.416,1.022)--(6.417,1.022)--(6.418,1.022)%
  --(6.419,1.022)--(6.420,1.022)--(6.421,1.022)--(6.422,1.022)--(6.423,1.022)--(6.425,1.022)%
  --(6.426,1.022)--(6.427,1.022)--(6.428,1.022)--(6.429,1.022)--(6.430,1.022)--(6.431,1.022)%
  --(6.433,1.022)--(6.434,1.022)--(6.436,1.022)--(6.437,1.022)--(6.439,1.022)--(6.440,1.022)%
  --(6.442,1.022)--(6.443,1.022)--(6.444,1.022)--(6.446,1.022)--(6.447,1.022)--(6.448,1.022)%
  --(6.449,1.022)--(6.450,1.022)--(6.451,1.022)--(6.452,1.022)--(6.453,1.022)--(6.454,1.022)%
  --(6.455,1.022)--(6.456,1.022)--(6.457,1.022)--(6.458,1.022)--(6.460,1.022)--(6.461,1.022)%
  --(6.462,1.022)--(6.463,1.022)--(6.464,1.022)--(6.465,1.022)--(6.467,1.022)--(6.468,1.022)%
  --(6.470,1.022)--(6.471,1.022)--(6.473,1.022)--(6.474,1.022)--(6.475,1.022)--(6.476,1.022)%
  --(6.477,1.022)--(6.478,1.022)--(6.479,1.022)--(6.481,1.022)--(6.482,1.022)--(6.483,1.022)%
  --(6.484,1.022)--(6.485,1.022)--(6.486,1.022)--(6.487,1.022)--(6.488,1.022)--(6.489,1.022)%
  --(6.490,1.022)--(6.491,1.022)--(6.492,1.022)--(6.493,1.022)--(6.494,1.022)--(6.496,1.022)%
  --(6.497,1.022)--(6.498,1.022)--(6.499,1.022)--(6.500,1.022)--(6.502,1.022)--(6.503,1.022)%
  --(6.505,1.022)--(6.506,1.022)--(6.507,1.022)--(6.508,1.022)--(6.509,1.022)--(6.510,1.022)%
  --(6.511,1.022)--(6.513,1.022)--(6.514,1.022)--(6.515,1.022)--(6.516,1.022)--(6.517,1.022)%
  --(6.518,1.022)--(6.519,1.022)--(6.520,1.022)--(6.521,1.022)--(6.522,1.022)--(6.523,1.022)%
  --(6.524,1.022)--(6.526,1.022)--(6.527,1.022)--(6.528,1.022)--(6.529,1.022)--(6.530,1.022)%
  --(6.531,1.022)--(6.532,1.022)--(6.534,1.022)--(6.535,1.022)--(6.536,1.022)--(6.538,1.022)%
  --(6.539,1.022)--(6.540,1.022)--(6.541,1.022)--(6.543,1.022)--(6.544,1.022)--(6.545,1.022)%
  --(6.546,1.022)--(6.547,1.022)--(6.548,1.022)--(6.549,1.022)--(6.550,1.022)--(6.551,1.022)%
  --(6.552,1.022)--(6.553,1.022)--(6.555,1.022)--(6.556,1.022)--(6.557,1.022)--(6.558,1.022)%
  --(6.559,1.022)--(6.560,1.022)--(6.561,1.022)--(6.562,1.022)--(6.563,1.022)--(6.564,1.022)%
  --(6.566,1.022)--(6.567,1.022)--(6.568,1.022)--(6.569,1.022)--(6.571,1.022)--(6.572,1.022)%
  --(6.574,1.022)--(6.575,1.022)--(6.576,1.022)--(6.577,1.022)--(6.578,1.022)--(6.580,1.022)%
  --(6.581,1.022)--(6.582,1.022)--(6.583,1.022)--(6.584,1.022)--(6.585,1.022)--(6.586,1.022)%
  --(6.587,1.022)--(6.588,1.022)--(6.589,1.022)--(6.590,1.022)--(6.591,1.022)--(6.592,1.022)%
  --(6.594,1.022)--(6.595,1.022)--(6.596,1.022)--(6.597,1.022)--(6.598,1.022)--(6.599,1.022)%
  --(6.600,1.022)--(6.602,1.022)--(6.603,1.022)--(6.604,1.022)--(6.606,1.022)--(6.607,1.022)%
  --(6.608,1.022)--(6.609,1.022)--(6.610,1.022)--(6.612,1.022)--(6.613,1.022)--(6.614,1.022)%
  --(6.615,1.022)--(6.616,1.022)--(6.617,1.022)--(6.618,1.022)--(6.619,1.022)--(6.620,1.022)%
  --(6.621,1.022)--(6.623,1.022)--(6.624,1.022)--(6.625,1.022)--(6.626,1.022)--(6.627,1.022)%
  --(6.628,1.022)--(6.629,1.022)--(6.630,1.022)--(6.631,1.022)--(6.632,1.022)--(6.634,1.022)%
  --(6.635,1.022)--(6.636,1.022)--(6.637,1.022)--(6.639,1.022)--(6.640,1.022)--(6.641,1.022)%
  --(6.643,1.022)--(6.644,1.022)--(6.645,1.022)--(6.647,1.022)--(6.648,1.022)--(6.649,1.022)%
  --(6.650,1.022)--(6.651,1.022)--(6.652,1.022)--(6.653,1.022)--(6.654,1.022)--(6.656,1.022)%
  --(6.657,1.022)--(6.658,1.022)--(6.659,1.022)--(6.660,1.022)--(6.661,1.022)--(6.662,1.022)%
  --(6.663,1.022)--(6.664,1.022)--(6.665,1.022)--(6.666,1.022)--(6.668,1.022)--(6.669,1.022)%
  --(6.670,1.022)--(6.672,1.022)--(6.673,1.022)--(6.675,1.022)--(6.676,1.022)--(6.678,1.022)%
  --(6.679,1.022)--(6.680,1.022)--(6.682,1.022)--(6.683,1.022)--(6.684,1.022)--(6.685,1.022)%
  --(6.686,1.022)--(6.687,1.022)--(6.688,1.022)--(6.689,1.022)--(6.690,1.022)--(6.691,1.022)%
  --(6.693,1.022)--(6.694,1.022)--(6.695,1.022)--(6.696,1.022)--(6.697,1.022)--(6.698,1.022)%
  --(6.699,1.022)--(6.700,1.022)--(6.701,1.022)--(6.703,1.022)--(6.704,1.022)--(6.705,1.022)%
  --(6.707,1.022)--(6.709,1.022)--(6.710,1.022)--(6.712,1.022)--(6.713,1.022)--(6.714,1.022)%
  --(6.716,1.022)--(6.717,1.022)--(6.718,1.022)--(6.719,1.022)--(6.720,1.022)--(6.721,1.022)%
  --(6.722,1.022)--(6.723,1.022)--(6.724,1.022)--(6.726,1.022)--(6.727,1.022)--(6.728,1.022)%
  --(6.729,1.022)--(6.730,1.022)--(6.731,1.022)--(6.732,1.022)--(6.733,1.022)--(6.734,1.022)%
  --(6.735,1.022)--(6.737,1.022)--(6.738,1.022)--(6.739,1.022)--(6.741,1.022)--(6.743,1.022)%
  --(6.744,1.022)--(6.746,1.022)--(6.747,1.022)--(6.748,1.022)--(6.750,1.022)--(6.751,1.022)%
  --(6.752,1.022)--(6.753,1.022)--(6.754,1.022)--(6.755,1.022)--(6.756,1.022)--(6.757,1.022)%
  --(6.758,1.022)--(6.759,1.022)--(6.760,1.022)--(6.761,1.022)--(6.762,1.022)--(6.764,1.022)%
  --(6.765,1.022)--(6.766,1.022)--(6.767,1.022)--(6.768,1.022)--(6.769,1.022)--(6.770,1.022)%
  --(6.772,1.022)--(6.773,1.022)--(6.775,1.022)--(6.776,1.022)--(6.778,1.022)--(6.779,1.022)%
  --(6.781,1.022)--(6.782,1.022)--(6.784,1.022)--(6.785,1.022)--(6.786,1.022)--(6.787,1.022)%
  --(6.788,1.022)--(6.789,1.022)--(6.790,1.022)--(6.791,1.022)--(6.792,1.022)--(6.793,1.022)%
  --(6.794,1.022)--(6.795,1.022)--(6.796,1.022)--(6.798,1.022)--(6.799,1.022)--(6.800,1.022)%
  --(6.801,1.022)--(6.802,1.022)--(6.803,1.022)--(6.804,1.022)--(6.805,1.022)--(6.807,1.022)%
  --(6.809,1.022)--(6.810,1.022)--(6.812,1.022)--(6.813,1.022)--(6.815,1.022)--(6.816,1.022)%
  --(6.817,1.022)--(6.818,1.022)--(6.820,1.022)--(6.821,1.022)--(6.822,1.022)--(6.823,1.022)%
  --(6.824,1.022)--(6.825,1.022)--(6.826,1.022)--(6.827,1.022)--(6.828,1.022)--(6.829,1.022)%
  --(6.830,1.022)--(6.831,1.022)--(6.832,1.022)--(6.833,1.022)--(6.835,1.022)--(6.836,1.022)%
  --(6.837,1.022)--(6.838,1.022)--(6.840,1.022)--(6.841,1.022)--(6.843,1.022)--(6.844,1.022)%
  --(6.846,1.022)--(6.847,1.022)--(6.849,1.022)--(6.850,1.022)--(6.851,1.022)--(6.852,1.022)%
  --(6.854,1.022)--(6.855,1.022)--(6.856,1.022)--(6.857,1.022)--(6.858,1.022)--(6.859,1.022)%
  --(6.860,1.022)--(6.861,1.022)--(6.862,1.022)--(6.863,1.022)--(6.864,1.022)--(6.865,1.022)%
  --(6.866,1.022)--(6.868,1.022)--(6.869,1.022)--(6.870,1.022)--(6.871,1.022)--(6.872,1.022)%
  --(6.873,1.022)--(6.875,1.022)--(6.877,1.022)--(6.878,1.022)--(6.880,1.022)--(6.881,1.022)%
  --(6.882,1.022)--(6.884,1.022)--(6.885,1.022)--(6.886,1.022)--(6.887,1.022)--(6.888,1.022)%
  --(6.890,1.022)--(6.891,1.022)--(6.892,1.022)--(6.893,1.022)--(6.894,1.022)--(6.895,1.022)%
  --(6.896,1.022)--(6.897,1.022)--(6.898,1.022)--(6.899,1.022)--(6.900,1.022)--(6.901,1.022)%
  --(6.902,1.022)--(6.904,1.022)--(6.905,1.022)--(6.906,1.022)--(6.907,1.022)--(6.909,1.022)%
  --(6.910,1.022)--(6.912,1.022)--(6.914,1.022)--(6.915,1.022)--(6.916,1.022)--(6.918,1.022)%
  --(6.919,1.022)--(6.920,1.022)--(6.921,1.022)--(6.922,1.022)--(6.923,1.022)--(6.925,1.022)%
  --(6.926,1.022)--(6.927,1.022)--(6.928,1.022)--(6.929,1.022)--(6.930,1.022)--(6.931,1.022)%
  --(6.932,1.022)--(6.933,1.022)--(6.934,1.022)--(6.935,1.022)--(6.936,1.022)--(6.937,1.022)%
  --(6.939,1.022)--(6.940,1.022)--(6.941,1.022)--(6.943,1.022)--(6.944,1.022)--(6.946,1.022)%
  --(6.948,1.022)--(6.949,1.022)--(6.950,1.022)--(6.952,1.022)--(6.953,1.022)--(6.954,1.022)%
  --(6.955,1.022)--(6.956,1.022)--(6.957,1.022)--(6.958,1.022)--(6.960,1.022)--(6.961,1.022)%
  --(6.962,1.022)--(6.963,1.022)--(6.964,1.022)--(6.965,1.022)--(6.966,1.022)--(6.967,1.022)%
  --(6.968,1.022)--(6.969,1.022)--(6.970,1.022)--(6.972,1.022)--(6.973,1.022)--(6.974,1.022)%
  --(6.975,1.022)--(6.977,1.022)--(6.978,1.022)--(6.980,1.022)--(6.982,1.022)--(6.983,1.022)%
  --(6.984,1.022)--(6.986,1.022)--(6.987,1.022)--(6.988,1.022)--(6.989,1.022)--(6.990,1.022)%
  --(6.991,1.022)--(6.992,1.022)--(6.993,1.022)--(6.995,1.022)--(6.996,1.022)--(6.997,1.022)%
  --(6.998,1.022)--(6.999,1.022)--(7.000,1.022)--(7.001,1.022)--(7.002,1.022)--(7.003,1.022)%
  --(7.004,1.022)--(7.005,1.022)--(7.007,1.022)--(7.008,1.022)--(7.009,1.022)--(7.011,1.022)%
  --(7.012,1.022)--(7.014,1.022)--(7.016,1.022)--(7.017,1.022)--(7.018,1.022)--(7.020,1.022)%
  --(7.021,1.022)--(7.022,1.022)--(7.023,1.022)--(7.024,1.022)--(7.025,1.022)--(7.026,1.022)%
  --(7.028,1.022)--(7.029,1.022)--(7.030,1.022)--(7.031,1.022)--(7.032,1.022)--(7.033,1.022)%
  --(7.034,1.022)--(7.035,1.022)--(7.036,1.022)--(7.037,1.022)--(7.038,1.022)--(7.039,1.022)%
  --(7.041,1.022)--(7.042,1.022)--(7.043,1.022)--(7.045,1.022)--(7.046,1.022)--(7.048,1.022)%
  --(7.050,1.022)--(7.051,1.022)--(7.052,1.022)--(7.054,1.022)--(7.055,1.022)--(7.056,1.022)%
  --(7.057,1.022)--(7.058,1.022)--(7.059,1.022)--(7.061,1.022)--(7.062,1.022)--(7.063,1.022)%
  --(7.064,1.022)--(7.065,1.022)--(7.066,1.022)--(7.067,1.022)--(7.068,1.022)--(7.069,1.022)%
  --(7.070,1.022)--(7.071,1.022)--(7.072,1.022)--(7.074,1.022)--(7.075,1.022)--(7.076,1.022)%
  --(7.077,1.022)--(7.079,1.022)--(7.080,1.022)--(7.081,1.022)--(7.083,1.022)--(7.084,1.022)%
  --(7.085,1.022)--(7.087,1.022)--(7.088,1.022)--(7.089,1.022)--(7.090,1.022)--(7.091,1.022)%
  --(7.092,1.022)--(7.094,1.022)--(7.095,1.022)--(7.096,1.022)--(7.097,1.022)--(7.098,1.022)%
  --(7.099,1.022)--(7.100,1.022)--(7.101,1.022)--(7.102,1.022)--(7.103,1.022)--(7.104,1.022)%
  --(7.105,1.022)--(7.106,1.022)--(7.108,1.022)--(7.109,1.022)--(7.110,1.022)--(7.111,1.022)%
  --(7.113,1.022)--(7.114,1.022)--(7.116,1.022)--(7.117,1.022)--(7.118,1.022)--(7.119,1.022)%
  --(7.120,1.022)--(7.121,1.022)--(7.123,1.022)--(7.124,1.022)--(7.125,1.022)--(7.126,1.022)%
  --(7.127,1.022)--(7.128,1.022)--(7.129,1.022)--(7.130,1.022)--(7.131,1.022)--(7.132,1.022)%
  --(7.133,1.022)--(7.134,1.022)--(7.135,1.022)--(7.137,1.022)--(7.138,1.022)--(7.139,1.022)%
  --(7.140,1.022)--(7.141,1.022)--(7.142,1.022)--(7.144,1.022)--(7.145,1.022)--(7.146,1.022)%
  --(7.148,1.022)--(7.149,1.022)--(7.150,1.022)--(7.151,1.022)--(7.152,1.022)--(7.154,1.022)%
  --(7.155,1.022)--(7.156,1.022)--(7.157,1.022)--(7.158,1.022)--(7.159,1.022)--(7.161,1.022)%
  --(7.162,1.022)--(7.163,1.022)--(7.164,1.022)--(7.165,1.022)--(7.166,1.022)--(7.167,1.022)%
  --(7.168,1.022)--(7.169,1.022)--(7.170,1.022)--(7.171,1.022)--(7.172,1.022)--(7.173,1.022)%
  --(7.175,1.022)--(7.176,1.022)--(7.177,1.022)--(7.178,1.022)--(7.179,1.022)--(7.181,1.022)%
  --(7.182,1.022)--(7.183,1.022)--(7.185,1.022)--(7.186,1.022)--(7.187,1.022)--(7.189,1.022)%
  --(7.190,1.022)--(7.191,1.022)--(7.192,1.022)--(7.193,1.022)--(7.194,1.022)--(7.195,1.022)%
  --(7.196,1.022)--(7.198,1.022)--(7.199,1.022)--(7.200,1.022)--(7.201,1.022)--(7.202,1.022)%
  --(7.203,1.022)--(7.204,1.022)--(7.205,1.022)--(7.206,1.022)--(7.207,1.022)--(7.208,1.022)%
  --(7.210,1.022)--(7.211,1.022)--(7.212,1.022)--(7.214,1.022)--(7.215,1.022)--(7.217,1.022)%
  --(7.218,1.022)--(7.220,1.022)--(7.221,1.022)--(7.222,1.022)--(7.224,1.022)--(7.225,1.022)%
  --(7.226,1.022)--(7.227,1.022)--(7.228,1.022)--(7.229,1.022)--(7.230,1.022)--(7.231,1.022)%
  --(7.232,1.022)--(7.234,1.022)--(7.235,1.022)--(7.236,1.022)--(7.237,1.022)--(7.238,1.022)%
  --(7.239,1.022)--(7.240,1.022)--(7.241,1.022)--(7.242,1.022)--(7.243,1.022)--(7.245,1.022)%
  --(7.246,1.022)--(7.247,1.022)--(7.249,1.022)--(7.251,1.022)--(7.252,1.022)--(7.253,1.022)%
  --(7.255,1.022)--(7.256,1.022)--(7.258,1.022)--(7.259,1.022)--(7.260,1.022)--(7.261,1.022)%
  --(7.262,1.022)--(7.263,1.022)--(7.264,1.022)--(7.265,1.022)--(7.266,1.022)--(7.267,1.022)%
  --(7.269,1.022)--(7.270,1.022)--(7.271,1.022)--(7.272,1.022)--(7.273,1.022)--(7.274,1.022)%
  --(7.275,1.022)--(7.276,1.022)--(7.277,1.022)--(7.279,1.022)--(7.280,1.022)--(7.281,1.022)%
  --(7.283,1.022)--(7.284,1.022)--(7.286,1.022)--(7.287,1.022)--(7.289,1.022)--(7.290,1.022)%
  --(7.292,1.022)--(7.293,1.022)--(7.294,1.022)--(7.295,1.022)--(7.296,1.022)--(7.297,1.022)%
  --(7.298,1.022)--(7.299,1.022)--(7.300,1.022)--(7.301,1.022)--(7.302,1.022)--(7.303,1.022)%
  --(7.304,1.022)--(7.306,1.022)--(7.307,1.022)--(7.308,1.022)--(7.309,1.022)--(7.310,1.022)%
  --(7.311,1.022)--(7.312,1.022)--(7.313,1.022)--(7.315,1.022)--(7.317,1.022)--(7.318,1.022)%
  --(7.320,1.022)--(7.321,1.022)--(7.322,1.022)--(7.323,1.022)--(7.324,1.022)--(7.325,1.022)%
  --(7.326,1.022)--(7.327,1.022)--(7.328,1.022)--(7.330,1.022)--(7.331,1.022)--(7.332,1.022)%
  --(7.333,1.022)--(7.334,1.022)--(7.335,1.022)--(7.336,1.022)--(7.337,1.022)--(7.338,1.022)%
  --(7.339,1.022)--(7.340,1.022)--(7.341,1.022)--(7.342,1.022)--(7.344,1.022)--(7.345,1.022)%
  --(7.346,1.022)--(7.347,1.022)--(7.349,1.022)--(7.350,1.022)--(7.351,1.022)--(7.352,1.022)%
  --(7.354,1.022)--(7.355,1.022)--(7.356,1.022)--(7.357,1.022)--(7.358,1.022)--(7.360,1.022)%
  --(7.361,1.022)--(7.362,1.022)--(7.363,1.022)--(7.364,1.022)--(7.365,1.022)--(7.366,1.022)%
  --(7.367,1.022)--(7.368,1.022)--(7.369,1.022)--(7.370,1.022)--(7.372,1.022)--(7.373,1.022)%
  --(7.374,1.022)--(7.375,1.022)--(7.376,1.022)--(7.377,1.022)--(7.378,1.022)--(7.379,1.022)%
  --(7.380,1.022)--(7.382,1.022)--(7.383,1.022)--(7.384,1.022)--(7.386,1.022)--(7.387,1.022)%
  --(7.388,1.022)--(7.390,1.022)--(7.391,1.022)--(7.392,1.022)--(7.394,1.022)--(7.395,1.022)%
  --(7.396,1.022)--(7.397,1.022)--(7.398,1.022)--(7.399,1.022)--(7.400,1.022)--(7.401,1.022)%
  --(7.402,1.022)--(7.403,1.022)--(7.405,1.022)--(7.406,1.022)--(7.407,1.022)--(7.408,1.022)%
  --(7.409,1.022)--(7.410,1.022)--(7.411,1.022)--(7.412,1.022)--(7.413,1.022)--(7.415,1.022)%
  --(7.416,1.022)--(7.417,1.022)--(7.419,1.022)--(7.421,1.022)--(7.422,1.022)--(7.424,1.022)%
  --(7.425,1.022)--(7.427,1.022)--(7.428,1.022)--(7.429,1.022)--(7.430,1.022)--(7.431,1.022)%
  --(7.432,1.022)--(7.433,1.022)--(7.434,1.022)--(7.435,1.022)--(7.436,1.022)--(7.438,1.022)%
  --(7.439,1.022)--(7.440,1.022)--(7.441,1.022)--(7.442,1.022)--(7.443,1.022)--(7.444,1.022)%
  --(7.445,1.022)--(7.446,1.022)--(7.448,1.022)--(7.449,1.022)--(7.450,1.022)--(7.451,1.022)%
  --(7.453,1.022)--(7.454,1.022)--(7.455,1.022)--(7.457,1.022)--(7.458,1.022)--(7.459,1.022)%
  --(7.461,1.022)--(7.462,1.022)--(7.463,1.022)--(7.464,1.022)--(7.465,1.022)--(7.466,1.022)%
  --(7.467,1.022)--(7.468,1.022)--(7.469,1.022)--(7.471,1.022)--(7.472,1.022)--(7.473,1.022)%
  --(7.474,1.022)--(7.475,1.022)--(7.476,1.022)--(7.477,1.022)--(7.478,1.022)--(7.479,1.022)%
  --(7.481,1.022)--(7.482,1.022)--(7.483,1.022)--(7.484,1.022)--(7.486,1.022)--(7.487,1.022)%
  --(7.489,1.022)--(7.490,1.022)--(7.491,1.022)--(7.492,1.022)--(7.493,1.022)--(7.495,1.022)%
  --(7.496,1.022)--(7.497,1.022)--(7.498,1.022)--(7.499,1.022)--(7.500,1.022)--(7.501,1.022)%
  --(7.502,1.022)--(7.503,1.022)--(7.504,1.022)--(7.505,1.022)--(7.506,1.022)--(7.507,1.022)%
  --(7.508,1.022)--(7.510,1.022)--(7.511,1.022)--(7.512,1.022)--(7.513,1.022)--(7.514,1.022)%
  --(7.515,1.022)--(7.516,1.022)--(7.518,1.022)--(7.519,1.022)--(7.521,1.022)--(7.522,1.022)%
  --(7.523,1.022)--(7.524,1.022)--(7.525,1.022)--(7.526,1.022)--(7.528,1.022)--(7.529,1.022)%
  --(7.530,1.022)--(7.531,1.022)--(7.532,1.022)--(7.533,1.022)--(7.534,1.022)--(7.535,1.022)%
  --(7.537,1.022)--(7.538,1.022)--(7.539,1.022)--(7.540,1.022)--(7.541,1.022)--(7.542,1.022)%
  --(7.543,1.022)--(7.544,1.022)--(7.545,1.022)--(7.546,1.022)--(7.547,1.022)--(7.549,1.022)%
  --(7.550,1.022)--(7.551,1.022)--(7.552,1.022)--(7.554,1.022)--(7.555,1.022)--(7.556,1.022)%
  --(7.558,1.022)--(7.559,1.022)--(7.560,1.022)--(7.562,1.022)--(7.563,1.022)--(7.564,1.022)%
  --(7.565,1.022)--(7.566,1.022)--(7.567,1.022)--(7.568,1.022)--(7.570,1.022)--(7.571,1.022)%
  --(7.572,1.022)--(7.573,1.022)--(7.574,1.022)--(7.575,1.022)--(7.576,1.022)--(7.577,1.022)%
  --(7.578,1.022)--(7.579,1.022)--(7.580,1.022)--(7.581,1.022)--(7.583,1.022)--(7.584,1.022)%
  --(7.585,1.022)--(7.587,1.022)--(7.588,1.022)--(7.590,1.022)--(7.591,1.022)--(7.593,1.022)%
  --(7.594,1.022)--(7.595,1.022)--(7.597,1.022)--(7.598,1.022)--(7.599,1.022)--(7.600,1.022)%
  --(7.601,1.022)--(7.602,1.022)--(7.603,1.022)--(7.604,1.022)--(7.605,1.022)--(7.606,1.022)%
  --(7.608,1.022)--(7.609,1.022)--(7.610,1.022)--(7.611,1.022)--(7.612,1.022)--(7.613,1.022)%
  --(7.614,1.022)--(7.615,1.022)--(7.616,1.022)--(7.618,1.022)--(7.619,1.022)--(7.620,1.022)%
  --(7.622,1.022)--(7.624,1.022)--(7.625,1.022)--(7.627,1.022)--(7.628,1.022)--(7.629,1.022)%
  --(7.631,1.022)--(7.632,1.022)--(7.633,1.022)--(7.634,1.022)--(7.635,1.022)--(7.636,1.022)%
  --(7.637,1.022)--(7.638,1.022)--(7.639,1.022)--(7.641,1.022)--(7.642,1.022)--(7.643,1.022)%
  --(7.644,1.022)--(7.645,1.022)--(7.646,1.022)--(7.647,1.022)--(7.648,1.022)--(7.649,1.022)%
  --(7.651,1.022)--(7.652,1.022)--(7.653,1.022)--(7.655,1.022)--(7.656,1.022)--(7.658,1.022)%
  --(7.660,1.022)--(7.661,1.022)--(7.662,1.022)--(7.663,1.022)--(7.665,1.022)--(7.666,1.022)%
  --(7.667,1.022)--(7.668,1.022)--(7.669,1.022)--(7.670,1.022)--(7.671,1.022)--(7.672,1.022)%
  --(7.674,1.022)--(7.675,1.022)--(7.676,1.022)--(7.677,1.022)--(7.678,1.022)--(7.679,1.022)%
  --(7.680,1.022)--(7.681,1.022)--(7.682,1.022)--(7.683,1.022)--(7.685,1.022)--(7.686,1.022)%
  --(7.687,1.022)--(7.688,1.022)--(7.690,1.022)--(7.691,1.022)--(7.693,1.022)--(7.694,1.022)%
  --(7.695,1.022)--(7.696,1.022)--(7.698,1.022)--(7.699,1.022)--(7.700,1.022)--(7.701,1.022)%
  --(7.702,1.022)--(7.703,1.022)--(7.704,1.022)--(7.705,1.022)--(7.707,1.022)--(7.708,1.022)%
  --(7.709,1.022)--(7.710,1.022)--(7.711,1.022)--(7.712,1.022)--(7.713,1.022)--(7.714,1.022)%
  --(7.715,1.022)--(7.716,1.022)--(7.717,1.022)--(7.719,1.022)--(7.720,1.022)--(7.721,1.022)%
  --(7.723,1.022)--(7.724,1.022)--(7.726,1.022)--(7.727,1.022)--(7.728,1.022)--(7.729,1.022)%
  --(7.730,1.022)--(7.731,1.022)--(7.733,1.022)--(7.734,1.022)--(7.735,1.022)--(7.736,1.022)%
  --(7.737,1.022)--(7.738,1.022)--(7.739,1.022)--(7.740,1.022)--(7.741,1.022)--(7.742,1.022)%
  --(7.743,1.022)--(7.744,1.022)--(7.745,1.022)--(7.746,1.022)--(7.747,1.022)--(7.749,1.022)%
  --(7.750,1.022)--(7.751,1.022)--(7.752,1.022)--(7.754,1.022)--(7.755,1.022)--(7.756,1.022)%
  --(7.757,1.022)--(7.759,1.022)--(7.760,1.022)--(7.761,1.022)--(7.762,1.022)--(7.764,1.022)%
  --(7.765,1.022)--(7.766,1.022)--(7.767,1.022)--(7.768,1.022)--(7.769,1.022)--(7.770,1.022)%
  --(7.771,1.022)--(7.773,1.022)--(7.774,1.022)--(7.775,1.022)--(7.776,1.022)--(7.777,1.022)%
  --(7.778,1.022)--(7.779,1.022)--(7.780,1.022)--(7.781,1.022)--(7.782,1.022)--(7.783,1.022)%
  --(7.784,1.022)--(7.785,1.022)--(7.787,1.022)--(7.788,1.022)--(7.789,1.022)--(7.790,1.022)%
  --(7.792,1.022)--(7.793,1.022)--(7.795,1.022)--(7.796,1.022)--(7.797,1.022)--(7.799,1.022)%
  --(7.800,1.022)--(7.801,1.022)--(7.802,1.022)--(7.803,1.022)--(7.804,1.022)--(7.805,1.022)%
  --(7.806,1.022)--(7.807,1.022)--(7.808,1.022)--(7.809,1.022)--(7.811,1.022)--(7.812,1.022)%
  --(7.813,1.022)--(7.814,1.022)--(7.815,1.022)--(7.816,1.022)--(7.817,1.022)--(7.818,1.022)%
  --(7.820,1.022)--(7.821,1.022)--(7.822,1.022)--(7.823,1.022)--(7.825,1.022)--(7.827,1.022)%
  --(7.828,1.022)--(7.830,1.022)--(7.831,1.022)--(7.832,1.022)--(7.834,1.022)--(7.835,1.022)%
  --(7.836,1.022)--(7.837,1.022)--(7.838,1.022)--(7.839,1.022)--(7.840,1.022)--(7.841,1.022)%
  --(7.842,1.022)--(7.843,1.022)--(7.844,1.022)--(7.845,1.022)--(7.846,1.022)--(7.848,1.022)%
  --(7.849,1.022)--(7.850,1.022)--(7.851,1.022)--(7.852,1.022)--(7.853,1.022)--(7.855,1.022)%
  --(7.856,1.022)--(7.857,1.022)--(7.859,1.022)--(7.860,1.022)--(7.862,1.022)--(7.863,1.022)%
  --(7.864,1.022)--(7.865,1.022)--(7.866,1.022)--(7.867,1.022)--(7.869,1.022)--(7.870,1.022)%
  --(7.871,1.022)--(7.872,1.022)--(7.873,1.022)--(7.874,1.022)--(7.875,1.022)--(7.876,1.022)%
  --(7.877,1.022)--(7.878,1.022)--(7.879,1.022)--(7.880,1.022)--(7.881,1.022)--(7.882,1.022)%
  --(7.883,1.022)--(7.885,1.022)--(7.886,1.022)--(7.887,1.022)--(7.888,1.022)--(7.889,1.022)%
  --(7.891,1.022)--(7.892,1.022)--(7.894,1.022)--(7.895,1.022)--(7.896,1.022)--(7.897,1.022)%
  --(7.898,1.022)--(7.900,1.022)--(7.901,1.022)--(7.902,1.022)--(7.903,1.022)--(7.904,1.022)%
  --(7.905,1.022)--(7.906,1.022)--(7.907,1.022)--(7.908,1.022)--(7.909,1.022)--(7.910,1.022)%
  --(7.911,1.022)--(7.912,1.022)--(7.914,1.022)--(7.915,1.022)--(7.916,1.022)--(7.917,1.022)%
  --(7.918,1.022)--(7.919,1.022)--(7.920,1.022)--(7.921,1.022)--(7.923,1.022)--(7.924,1.022)%
  --(7.925,1.022)--(7.927,1.022)--(7.928,1.022)--(7.929,1.022)--(7.931,1.022)--(7.932,1.022)%
  --(7.933,1.022)--(7.935,1.022)--(7.936,1.022)--(7.937,1.022)--(7.938,1.022)--(7.939,1.022)%
  --(7.940,1.022)--(7.941,1.022)--(7.943,1.022)--(7.944,1.022)--(7.945,1.022)--(7.946,1.022)%
  --(7.947,1.022)--(7.948,1.022)--(7.949,1.022)--(7.950,1.022)--(7.951,1.022)--(7.952,1.022)%
  --(7.953,1.022)--(7.955,1.022)--(7.956,1.022)--(7.957,1.022)--(7.958,1.022)--(7.960,1.022)%
  --(7.961,1.022)--(7.962,1.022)--(7.964,1.022)--(7.965,1.022)--(7.966,1.022)--(7.968,1.022)%
  --(7.969,1.022)--(7.970,1.022)--(7.971,1.022)--(7.972,1.022)--(7.973,1.022)--(7.974,1.022)%
  --(7.976,1.022)--(7.977,1.022)--(7.978,1.022)--(7.979,1.022)--(7.980,1.022)--(7.981,1.022)%
  --(7.982,1.022)--(7.983,1.022)--(7.984,1.022)--(7.985,1.022)--(7.986,1.022)--(7.987,1.022)%
  --(7.989,1.022)--(7.990,1.022)--(7.991,1.022)--(7.992,1.022)--(7.994,1.022)--(7.995,1.022)%
  --(7.997,1.022)--(7.998,1.022)--(7.999,1.022)--(8.000,1.022)--(8.001,1.022)--(8.002,1.022)%
  --(8.004,1.022)--(8.005,1.022)--(8.006,1.022)--(8.007,1.022)--(8.008,1.022)--(8.009,1.022)%
  --(8.010,1.022)--(8.011,1.022)--(8.012,1.022)--(8.013,1.022)--(8.014,1.022)--(8.015,1.022)%
  --(8.016,1.022)--(8.018,1.022)--(8.019,1.022)--(8.020,1.022)--(8.021,1.022)--(8.022,1.022)%
  --(8.023,1.022)--(8.025,1.022)--(8.026,1.022)--(8.027,1.022)--(8.029,1.022)--(8.030,1.022)%
  --(8.031,1.022)--(8.032,1.022)--(8.033,1.022)--(8.035,1.022)--(8.036,1.022)--(8.037,1.022)%
  --(8.038,1.022)--(8.039,1.022)--(8.040,1.022)--(8.042,1.022)--(8.043,1.022)--(8.044,1.022)%
  --(8.045,1.022)--(8.046,1.022)--(8.047,1.022)--(8.048,1.022)--(8.049,1.022)--(8.050,1.022)%
  --(8.051,1.022)--(8.052,1.022)--(8.053,1.022)--(8.054,1.022)--(8.056,1.022)--(8.057,1.022)%
  --(8.058,1.022)--(8.059,1.022)--(8.060,1.022)--(8.062,1.022)--(8.063,1.022)--(8.065,1.022)%
  --(8.066,1.022)--(8.067,1.022)--(8.068,1.022)--(8.070,1.022)--(8.071,1.022)--(8.072,1.022)%
  --(8.073,1.022)--(8.074,1.022)--(8.075,1.022)--(8.076,1.022)--(8.077,1.022)--(8.078,1.022)%
  --(8.080,1.022)--(8.081,1.022)--(8.082,1.022)--(8.083,1.022)--(8.084,1.022)--(8.085,1.022)%
  --(8.086,1.022)--(8.087,1.022)--(8.088,1.022)--(8.090,1.022)--(8.091,1.022)--(8.092,1.022)%
  --(8.093,1.022)--(8.095,1.022)--(8.096,1.022)--(8.098,1.022)--(8.099,1.022)--(8.101,1.022)%
  --(8.102,1.022)--(8.104,1.022)--(8.105,1.022)--(8.106,1.022)--(8.107,1.022)--(8.108,1.022)%
  --(8.109,1.022)--(8.110,1.022)--(8.111,1.022)--(8.113,1.022)--(8.114,1.022)--(8.115,1.022)%
  --(8.116,1.022)--(8.117,1.022)--(8.118,1.022)--(8.119,1.022)--(8.120,1.022)--(8.121,1.022)%
  --(8.122,1.022)--(8.123,1.022)--(8.124,1.022)--(8.126,1.022)--(8.127,1.022)--(8.128,1.022)%
  --(8.130,1.022)--(8.132,1.022)--(8.133,1.022)--(8.134,1.022)--(8.136,1.022)--(8.137,1.022)%
  --(8.139,1.022)--(8.140,1.022)--(8.141,1.022)--(8.142,1.022)--(8.143,1.022)--(8.144,1.022)%
  --(8.145,1.022)--(8.146,1.022)--(8.147,1.022)--(8.148,1.022)--(8.150,1.022)--(8.151,1.022)%
  --(8.152,1.022)--(8.153,1.022)--(8.154,1.022)--(8.155,1.022)--(8.156,1.022)--(8.157,1.022)%
  --(8.158,1.022)--(8.160,1.022)--(8.161,1.022)--(8.162,1.022)--(8.164,1.022)--(8.165,1.022)%
  --(8.167,1.022)--(8.168,1.022)--(8.170,1.022)--(8.171,1.022)--(8.172,1.022)--(8.174,1.022)%
  --(8.175,1.022)--(8.176,1.022)--(8.177,1.022)--(8.178,1.022)--(8.179,1.022)--(8.180,1.022)%
  --(8.181,1.022)--(8.182,1.022)--(8.183,1.022)--(8.184,1.022)--(8.185,1.022)--(8.186,1.022)%
  --(8.188,1.022)--(8.189,1.022)--(8.190,1.022)--(8.191,1.022)--(8.192,1.022)--(8.193,1.022)%
  --(8.195,1.022)--(8.196,1.022)--(8.198,1.022)--(8.199,1.022)--(8.201,1.022)--(8.202,1.022)%
  --(8.204,1.022)--(8.205,1.022)--(8.206,1.022)--(8.207,1.022)--(8.209,1.022)--(8.210,1.022)%
  --(8.211,1.022)--(8.212,1.022)--(8.213,1.022)--(8.214,1.022)--(8.215,1.022)--(8.216,1.022)%
  --(8.217,1.022)--(8.218,1.022)--(8.219,1.022)--(8.220,1.022)--(8.221,1.022)--(8.222,1.022)%
  --(8.224,1.022)--(8.225,1.022)--(8.226,1.022)--(8.227,1.022)--(8.228,1.022)--(8.230,1.022)%
  --(8.231,1.022)--(8.233,1.022)--(8.235,1.022)--(8.236,1.022)--(8.237,1.022)--(8.238,1.022)%
  --(8.239,1.022)--(8.240,1.022)--(8.241,1.022)--(8.242,1.022)--(8.243,1.022)--(8.245,1.022)%
  --(8.246,1.022)--(8.247,1.022)--(8.248,1.022)--(8.249,1.022)--(8.250,1.022)--(8.251,1.022)%
  --(8.252,1.022)--(8.253,1.022)--(8.254,1.022)--(8.255,1.022)--(8.256,1.022)--(8.257,1.022)%
  --(8.259,1.022)--(8.260,1.022)--(8.261,1.022)--(8.262,1.022)--(8.264,1.022)--(8.265,1.022)%
  --(8.266,1.022)--(8.268,1.022)--(8.269,1.022)--(8.270,1.022)--(8.271,1.022)--(8.272,1.022)%
  --(8.273,1.022)--(8.274,1.022)--(8.276,1.022)--(8.277,1.022)--(8.278,1.022)--(8.279,1.022)%
  --(8.280,1.022)--(8.281,1.022)--(8.282,1.022)--(8.283,1.022)--(8.284,1.022)--(8.285,1.022)%
  --(8.286,1.022)--(8.287,1.022)--(8.288,1.022)--(8.289,1.022)--(8.291,1.022)--(8.292,1.022)%
  --(8.293,1.022)--(8.294,1.022)--(8.296,1.022)--(8.297,1.022)--(8.298,1.022)--(8.299,1.022)%
  --(8.301,1.022)--(8.302,1.022)--(8.303,1.022)--(8.305,1.022)--(8.306,1.022)--(8.307,1.022)%
  --(8.309,1.022)--(8.310,1.022)--(8.311,1.022)--(8.312,1.022)--(8.313,1.022)--(8.314,1.022)%
  --(8.315,1.022)--(8.316,1.022)--(8.317,1.022)--(8.318,1.022)--(8.319,1.022)--(8.320,1.022)%
  --(8.321,1.022)--(8.323,1.022)--(8.324,1.022)--(8.325,1.022)--(8.326,1.022)--(8.327,1.022)%
  --(8.328,1.022)--(8.330,1.022)--(8.331,1.022)--(8.332,1.022)--(8.334,1.022)--(8.336,1.022)%
  --(8.337,1.022)--(8.339,1.022)--(8.340,1.022)--(8.342,1.022)--(8.343,1.022)--(8.344,1.022)%
  --(8.345,1.022)--(8.346,1.022)--(8.347,1.022)--(8.348,1.022)--(8.349,1.022)--(8.350,1.022)%
  --(8.351,1.022)--(8.353,1.022)--(8.354,1.022)--(8.355,1.022)--(8.356,1.022)--(8.357,1.022)%
  --(8.358,1.022)--(8.359,1.022)--(8.360,1.022)--(8.362,1.022)--(8.363,1.022)--(8.364,1.022)%
  --(8.365,1.022)--(8.367,1.022)--(8.368,1.022)--(8.369,1.022)--(8.370,1.022)--(8.372,1.022)%
  --(8.373,1.022)--(8.374,1.022)--(8.376,1.022)--(8.377,1.022)--(8.378,1.022)--(8.379,1.022)%
  --(8.380,1.022)--(8.381,1.022)--(8.382,1.022)--(8.383,1.022)--(8.384,1.022)--(8.386,1.022)%
  --(8.387,1.022)--(8.388,1.022)--(8.389,1.022)--(8.390,1.022)--(8.391,1.022)--(8.392,1.022)%
  --(8.393,1.022)--(8.394,1.022)--(8.395,1.022)--(8.397,1.022)--(8.398,1.022)--(8.399,1.022)%
  --(8.401,1.022)--(8.402,1.022)--(8.404,1.022)--(8.405,1.022)--(8.406,1.022)--(8.407,1.022)%
  --(8.408,1.022)--(8.410,1.022)--(8.411,1.022)--(8.412,1.022)--(8.413,1.022)--(8.414,1.022)%
  --(8.415,1.022)--(8.416,1.022)--(8.417,1.022)--(8.418,1.022)--(8.419,1.022)--(8.420,1.022)%
  --(8.421,1.022)--(8.422,1.022)--(8.423,1.022)--(8.425,1.022)--(8.426,1.022)--(8.427,1.022)%
  --(8.428,1.022)--(8.429,1.022)--(8.430,1.022)--(8.432,1.022)--(8.433,1.022)--(8.434,1.022)%
  --(8.436,1.022)--(8.437,1.022)--(8.438,1.022)--(8.439,1.022)--(8.440,1.022)--(8.441,1.022)%
  --(8.443,1.022)--(8.444,1.022)--(8.445,1.022)--(8.446,1.022)--(8.447,1.022)--(8.448,1.022)%
  --(8.449,1.022)--(8.450,1.022)--(8.452,1.022)--(8.453,1.022)--(8.454,1.022)--(8.455,1.022)%
  --(8.456,1.022)--(8.457,1.022)--(8.458,1.022)--(8.459,1.022)--(8.460,1.022)--(8.461,1.022)%
  --(8.462,1.022)--(8.464,1.022)--(8.465,1.022)--(8.466,1.022)--(8.467,1.022)--(8.469,1.022)%
  --(8.470,1.022)--(8.471,1.022)--(8.473,1.022)--(8.474,1.022)--(8.475,1.022)--(8.477,1.022)%
  --(8.478,1.022)--(8.479,1.022)--(8.480,1.022)--(8.481,1.022)--(8.482,1.022)--(8.483,1.022)%
  --(8.485,1.022)--(8.486,1.022)--(8.487,1.022)--(8.488,1.022)--(8.489,1.022)--(8.490,1.022)%
  --(8.491,1.022)--(8.492,1.022)--(8.493,1.022)--(8.494,1.022)--(8.495,1.022)--(8.497,1.022)%
  --(8.498,1.022)--(8.499,1.022)--(8.500,1.022)--(8.502,1.022)--(8.504,1.022)--(8.505,1.022)%
  --(8.507,1.022)--(8.508,1.022)--(8.510,1.022)--(8.511,1.022)--(8.512,1.022)--(8.513,1.022)%
  --(8.514,1.022)--(8.516,1.022)--(8.517,1.022)--(8.518,1.022)--(8.519,1.022)--(8.520,1.022)%
  --(8.521,1.022)--(8.522,1.022)--(8.523,1.022)--(8.524,1.022)--(8.525,1.022)--(8.526,1.022)%
  --(8.527,1.022)--(8.528,1.022)--(8.530,1.022)--(8.531,1.022)--(8.532,1.022)--(8.533,1.022)%
  --(8.534,1.022)--(8.536,1.022)--(8.537,1.022)--(8.538,1.022)--(8.540,1.022)--(8.541,1.022)%
  --(8.543,1.022)--(8.544,1.022)--(8.545,1.022)--(8.546,1.022)--(8.547,1.022)--(8.548,1.022)%
  --(8.550,1.022)--(8.551,1.022)--(8.552,1.022)--(8.553,1.022)--(8.554,1.022)--(8.555,1.022)%
  --(8.556,1.022)--(8.557,1.022)--(8.558,1.022)--(8.559,1.022)--(8.560,1.022)--(8.561,1.022)%
  --(8.562,1.022)--(8.564,1.022)--(8.565,1.022)--(8.566,1.022)--(8.567,1.022)--(8.569,1.022)%
  --(8.570,1.022)--(8.572,1.091)--(8.573,1.183)--(8.574,1.249)--(8.576,1.314)--(8.577,1.378)%
  --(8.578,1.440)--(8.579,1.499)--(8.580,1.557)--(8.581,1.613)--(8.582,1.666)--(8.583,1.714)%
  --(8.584,1.761)--(8.585,1.803)--(8.586,1.844)--(8.587,1.882)--(8.589,1.917)--(8.590,1.949)%
  --(8.591,1.979)--(8.592,2.007)--(8.593,2.031)--(8.594,2.053)--(8.595,2.072)--(8.596,2.089)%
  --(8.598,2.102)--(8.599,2.113)--(8.600,2.121)--(8.601,2.126)--(8.603,2.127)--(8.604,2.125)%
  --(8.606,2.120)--(8.607,2.114)--(8.608,2.106)--(8.609,2.098)--(8.610,2.088)--(8.612,2.078)%
  --(8.613,2.067)--(8.614,2.056)--(8.615,2.046)--(8.616,2.037)--(8.617,2.028)--(8.618,2.021)%
  --(8.619,2.015)--(8.620,2.010)--(8.621,2.007)--(8.623,2.006)--(8.624,2.007)--(8.625,2.010)%
  --(8.626,2.016)--(8.627,2.024)--(8.628,2.034)--(8.629,2.047)--(8.630,2.064)--(8.632,2.082)%
  --(8.633,2.104)--(8.634,2.132)--(8.635,2.162)--(8.637,2.196)--(8.638,2.231)--(8.639,2.266)%
  --(8.641,2.301)--(8.642,2.338)--(8.643,2.375)--(8.644,2.413)--(8.645,2.450)--(8.647,2.488)%
  --(8.648,2.525)--(8.649,2.557)--(8.650,2.589)--(8.651,2.620)--(8.652,2.649)--(8.653,2.677)%
  --(8.655,2.703)--(8.656,2.727)--(8.657,2.750)--(8.658,2.772)--(8.659,2.791)--(8.660,2.809)%
  --(8.661,2.825)--(8.662,2.839)--(8.663,2.850)--(8.665,2.860)--(8.666,2.867)--(8.667,2.872)%
  --(8.668,2.874)--(8.670,2.874)--(8.671,2.871)--(8.672,2.865)--(8.674,2.858)--(8.675,2.849)%
  --(8.676,2.839)--(8.678,2.827)--(8.679,2.815)--(8.680,2.802)--(8.681,2.789)--(8.683,2.779)%
  --(8.684,2.768)--(8.685,2.759)--(8.686,2.750)--(8.687,2.743)--(8.688,2.737)--(8.689,2.732)%
  --(8.690,2.728)--(8.691,2.726)--(8.693,2.726)--(8.694,2.728)--(8.695,2.731)--(8.696,2.737)%
  --(8.697,2.745)--(8.698,2.756)--(8.699,2.768)--(8.701,2.785)--(8.702,2.804)--(8.703,2.825)%
  --(8.705,2.849)--(8.706,2.876)--(8.707,2.905)--(8.709,2.936)--(8.710,2.968)--(8.711,3.001)%
  --(8.713,3.035)--(8.714,3.069)--(8.716,3.103)--(8.717,3.132)--(8.718,3.160)--(8.719,3.188)%
  --(8.720,3.214)--(8.721,3.239)--(8.722,3.262)--(8.723,3.284)--(8.724,3.305)--(8.726,3.324)%
  --(8.727,3.341)--(8.728,3.357)--(8.729,3.370)--(8.730,3.382)--(8.731,3.391)--(8.732,3.399)%
  --(8.733,3.403)--(8.735,3.406)--(8.736,3.405)--(8.737,3.401)--(8.739,3.392)--(8.740,3.381)%
  --(8.742,3.368)--(8.743,3.354)--(8.745,3.339)--(8.746,3.322)--(8.747,3.305)--(8.749,3.287)%
  --(8.750,3.270)--(8.751,3.256)--(8.752,3.243)--(8.754,3.230)--(8.755,3.218)--(8.756,3.207)%
  --(8.757,3.197)--(8.758,3.188)--(8.759,3.179)--(8.760,3.172)--(8.761,3.166)--(8.763,3.161)%
  --(8.764,3.158)--(8.765,3.155)--(8.766,3.155)--(8.767,3.156)--(8.768,3.158)--(8.770,3.164)%
  --(8.771,3.173)--(8.773,3.184)--(8.774,3.197)--(8.775,3.210)--(8.777,3.233)--(8.778,3.265)%
  --(8.779,3.298)--(8.780,3.331)--(8.782,3.365)--(8.783,3.400)--(8.784,3.434)--(8.785,3.465)%
  --(8.786,3.495)--(8.787,3.525)--(8.789,3.554)--(8.790,3.581)--(8.791,3.607)--(8.792,3.631)%
  --(8.793,3.653)--(8.794,3.674)--(8.795,3.693)--(8.797,3.709)--(8.798,3.722)--(8.799,3.734)%
  --(8.800,3.742)--(8.801,3.748)--(8.803,3.751)--(8.804,3.751)--(8.805,3.747)--(8.807,3.740)%
  --(8.808,3.730)--(8.810,3.720)--(8.811,3.709)--(8.812,3.696)--(8.813,3.683)--(8.815,3.670)%
  --(8.816,3.657)--(8.817,3.644)--(8.818,3.632)--(8.819,3.621)--(8.821,3.612)--(8.822,3.603)%
  --(8.823,3.596)--(8.824,3.590)--(8.825,3.585)--(8.826,3.582)--(8.827,3.581)--(8.829,3.581)%
  --(8.830,3.583)--(8.831,3.587)--(8.832,3.592)--(8.833,3.601)--(8.834,3.611)--(8.836,3.624)%
  --(8.837,3.638)--(8.838,3.659)--(8.840,3.682)--(8.841,3.708)--(8.843,3.736)--(8.844,3.761)%
  --(8.845,3.789)--(8.846,3.817)--(8.848,3.846)--(8.849,3.876)--(8.850,3.906)--(8.851,3.937)%
  --(8.853,3.967)--(8.854,3.995)--(8.855,4.022)--(8.856,4.048)--(8.857,4.072)--(8.858,4.095)%
  --(8.860,4.117)--(8.861,4.136)--(8.862,4.153)--(8.863,4.168)--(8.864,4.181)--(8.865,4.190)%
  --(8.867,4.197)--(8.868,4.201)--(8.869,4.202)--(8.870,4.199)--(8.872,4.194)--(8.873,4.183)%
  --(8.875,4.168)--(8.877,4.149)--(8.878,4.128)--(8.879,4.111)--(8.881,4.093)--(8.882,4.075)%
  --(8.883,4.057)--(8.884,4.039)--(8.885,4.021)--(8.887,4.004)--(8.888,3.988)--(8.889,3.975)%
  --(8.890,3.962)--(8.891,3.951)--(8.892,3.941)--(8.894,3.933)--(8.895,3.927)--(8.896,3.923)%
  --(8.897,3.921)--(8.898,3.921)--(8.900,3.923)--(8.901,3.927)--(8.902,3.934)--(8.903,3.944)%
  --(8.905,3.957)--(8.906,3.972)--(8.907,3.990)--(8.909,4.012)--(8.910,4.037)--(8.911,4.064)%
  --(8.913,4.093)--(8.914,4.124)--(8.916,4.156)--(8.917,4.189)--(8.919,4.223)--(8.920,4.251)%
  --(8.921,4.278)--(8.922,4.305)--(8.924,4.331)--(8.925,4.354)--(8.926,4.376)--(8.927,4.396)%
  --(8.928,4.414)--(8.929,4.430)--(8.931,4.443)--(8.932,4.454)--(8.933,4.462)--(8.934,4.467)%
  --(8.935,4.469)--(8.937,4.467)--(8.938,4.463)--(8.939,4.454)--(8.941,4.442)--(8.942,4.426)%
  --(8.944,4.407)--(8.945,4.389)--(8.947,4.369)--(8.948,4.348)--(8.949,4.326)--(8.950,4.305)%
  --(8.952,4.283)--(8.953,4.262)--(8.954,4.242)--(8.955,4.225)--(8.957,4.209)--(8.958,4.194)%
  --(8.959,4.181)--(8.960,4.169)--(8.962,4.160)--(8.963,4.152)--(8.964,4.146)--(8.965,4.142)%
  --(8.966,4.141)--(8.967,4.142)--(8.969,4.145)--(8.970,4.151)--(8.971,4.160)--(8.973,4.171)%
  --(8.974,4.185)--(8.976,4.207)--(8.977,4.232)--(8.979,4.260)--(8.981,4.291)--(8.982,4.322)%
  --(8.984,4.354)--(8.985,4.386)--(8.987,4.419)--(8.988,4.445)--(8.989,4.471)--(8.991,4.496)%
  --(8.992,4.519)--(8.993,4.541)--(8.994,4.560)--(8.995,4.577)--(8.997,4.593)--(8.998,4.606)%
  --(8.999,4.616)--(9.000,4.624)--(9.001,4.628)--(9.003,4.630)--(9.004,4.629)--(9.005,4.624)%
  --(9.007,4.617)--(9.008,4.604)--(9.010,4.587)--(9.011,4.567)--(9.013,4.545)--(9.014,4.525)%
  --(9.016,4.504)--(9.017,4.482)--(9.018,4.461)--(9.019,4.439)--(9.021,4.418)--(9.022,4.398)%
  --(9.023,4.379)--(9.025,4.362)--(9.026,4.347)--(9.027,4.333)--(9.028,4.321)--(9.029,4.310)%
  --(9.031,4.302)--(9.032,4.295)--(9.033,4.291)--(9.034,4.289)--(9.035,4.289)--(9.037,4.292)%
  --(9.038,4.297)--(9.039,4.306)--(9.041,4.317)--(9.042,4.331)--(9.044,4.348)--(9.045,4.369)%
  --(9.047,4.391)--(9.048,4.416)--(9.050,4.442)--(9.051,4.469)--(9.053,4.498)--(9.054,4.526)%
  --(9.056,4.555)--(9.057,4.578)--(9.058,4.601)--(9.060,4.622)--(9.061,4.642)--(9.062,4.659)%
  --(9.063,4.675)--(9.064,4.689)--(9.066,4.700)--(9.067,4.709)--(9.068,4.715)--(9.069,4.718)%
  --(9.071,4.719)--(9.072,4.717)--(9.073,4.712)--(9.075,4.703)--(9.076,4.692)--(9.078,4.674)%
  --(9.080,4.652)--(9.081,4.627)--(9.083,4.600)--(9.085,4.575)--(9.086,4.550)--(9.088,4.525)%
  --(9.090,4.501)--(9.091,4.483)--(9.092,4.466)--(9.093,4.450)--(9.095,4.436)--(9.096,4.423)%
  --(9.097,4.413)--(9.098,4.404)--(9.100,4.398)--(9.101,4.393)--(9.102,4.390)--(9.103,4.390)%
  --(9.105,4.392)--(9.106,4.397)--(9.108,4.404)--(9.109,4.414)--(9.110,4.427)--(9.112,4.446)%
  --(9.114,4.469)--(9.116,4.495)--(9.118,4.524)--(9.119,4.550)--(9.121,4.577)--(9.122,4.604)%
  --(9.124,4.630)--(9.125,4.651)--(9.127,4.670)--(9.128,4.689)--(9.129,4.706)--(9.130,4.720)%
  --(9.132,4.732)--(9.133,4.743)--(9.134,4.751)--(9.135,4.757)--(9.137,4.761)--(9.138,4.762)%
  --(9.139,4.761)--(9.141,4.756)--(9.142,4.749)--(9.144,4.740)--(9.145,4.727)--(9.147,4.711)%
  --(9.148,4.693)--(9.150,4.673)--(9.151,4.651)--(9.153,4.629)--(9.154,4.606)--(9.156,4.584)%
  --(9.157,4.562)--(9.159,4.545)--(9.160,4.528)--(9.161,4.513)--(9.163,4.500)--(9.164,4.488)%
  --(9.165,4.478)--(9.167,4.470)--(9.168,4.464)--(9.169,4.460)--(9.170,4.458)--(9.172,4.459)%
  --(9.173,4.461)--(9.175,4.466)--(9.176,4.473)--(9.178,4.484)--(9.179,4.496)--(9.180,4.512)%
  --(9.182,4.530)--(9.184,4.550)--(9.185,4.572)--(9.187,4.595)--(9.188,4.618)--(9.190,4.641)%
  --(9.192,4.664)--(9.193,4.683)--(9.194,4.701)--(9.195,4.717)--(9.197,4.731)--(9.198,4.744)%
  --(9.199,4.755)--(9.201,4.764)--(9.202,4.771)--(9.203,4.776)--(9.205,4.778)--(9.206,4.779)%
  --(9.207,4.777)--(9.209,4.773)--(9.210,4.766)--(9.212,4.756)--(9.213,4.745)--(9.215,4.730)%
  --(9.217,4.713)--(9.218,4.694)--(9.220,4.674)--(9.221,4.654)--(9.223,4.633)--(9.225,4.613)%
  --(9.226,4.593)--(9.227,4.578)--(9.229,4.563)--(9.230,4.550)--(9.231,4.538)--(9.233,4.528)%
  --(9.234,4.520)--(9.235,4.514)--(9.237,4.509)--(9.238,4.507)--(9.239,4.506)--(9.241,4.508)%
  --(9.242,4.511)--(9.244,4.517)--(9.245,4.526)--(9.247,4.537)--(9.248,4.550)--(9.250,4.566)%
  --(9.251,4.583)--(9.253,4.601)--(9.254,4.621)--(9.256,4.641)--(9.258,4.661)--(9.259,4.681)%
  --(9.261,4.700)--(9.262,4.715)--(9.263,4.729)--(9.265,4.742)--(9.266,4.753)--(9.268,4.762)%
  --(9.269,4.770)--(9.270,4.776)--(9.272,4.780)--(9.273,4.782)--(9.274,4.782)--(9.276,4.780)%
  --(9.277,4.776)--(9.279,4.768)--(9.280,4.759)--(9.282,4.747)--(9.283,4.733)--(9.285,4.719)%
  --(9.286,4.703)--(9.288,4.686)--(9.289,4.669)--(9.291,4.651)--(9.292,4.634)--(9.294,4.618)%
  --(9.296,4.603)--(9.297,4.590)--(9.298,4.578)--(9.299,4.568)--(9.301,4.559)--(9.302,4.552)%
  --(9.304,4.546)--(9.305,4.542)--(9.306,4.540)--(9.308,4.540)--(9.309,4.542)--(9.311,4.546)%
  --(9.312,4.552)--(9.314,4.561)--(9.315,4.573)--(9.317,4.586)--(9.319,4.602)--(9.320,4.616)%
  --(9.322,4.632)--(9.323,4.648)--(9.325,4.664)--(9.326,4.681)--(9.327,4.696)--(9.329,4.712)%
  --(9.331,4.726)--(9.332,4.737)--(9.333,4.748)--(9.335,4.757)--(9.336,4.765)--(9.337,4.771)%
  --(9.339,4.775)--(9.340,4.778)--(9.341,4.779)--(9.343,4.777)--(9.344,4.774)--(9.346,4.769)%
  --(9.348,4.762)--(9.349,4.750)--(9.351,4.737)--(9.353,4.721)--(9.355,4.705)--(9.356,4.690)%
  --(9.358,4.676)--(9.359,4.661)--(9.360,4.647)--(9.362,4.634)--(9.363,4.621)--(9.365,4.609)%
  --(9.366,4.598)--(9.368,4.589)--(9.369,4.581)--(9.370,4.575)--(9.372,4.570)--(9.373,4.567)%
  --(9.375,4.565)--(9.376,4.565)--(9.377,4.568)--(9.379,4.572)--(9.381,4.578)--(9.382,4.586)%
  --(9.384,4.596)--(9.385,4.608)--(9.387,4.622)--(9.389,4.637)--(9.390,4.653)--(9.392,4.669)%
  --(9.394,4.685)--(9.396,4.701)--(9.397,4.716)--(9.399,4.727)--(9.400,4.738)--(9.401,4.747)%
  --(9.403,4.755)--(9.404,4.761)--(9.406,4.766)--(9.407,4.770)--(9.409,4.771)--(9.410,4.771)%
  --(9.411,4.769)--(9.413,4.766)--(9.415,4.760)--(9.416,4.752)--(9.418,4.741)--(9.420,4.729)%
  --(9.422,4.715)--(9.423,4.703)--(9.425,4.690)--(9.426,4.676)--(9.428,4.663)--(9.429,4.650)%
  --(9.431,4.638)--(9.432,4.627)--(9.434,4.616)--(9.435,4.608)--(9.437,4.600)--(9.438,4.594)%
  --(9.439,4.589)--(9.441,4.586)--(9.442,4.584)--(9.444,4.584)--(9.445,4.586)--(9.447,4.589)%
  --(9.448,4.595)--(9.450,4.602)--(9.452,4.610)--(9.453,4.621)--(9.455,4.634)--(9.457,4.647)%
  --(9.458,4.661)--(9.460,4.675)--(9.462,4.690)--(9.464,4.704)--(9.465,4.717)--(9.467,4.727)%
  --(9.468,4.736)--(9.470,4.744)--(9.471,4.751)--(9.473,4.756)--(9.474,4.760)--(9.476,4.762)%
  --(9.477,4.763)--(9.479,4.762)--(9.480,4.760)--(9.482,4.755)--(9.483,4.750)--(9.485,4.741)%
  --(9.487,4.730)--(9.489,4.718)--(9.491,4.705)--(9.492,4.693)--(9.494,4.682)--(9.495,4.670)%
  --(9.497,4.659)--(9.498,4.648)--(9.500,4.638)--(9.501,4.628)--(9.503,4.620)--(9.504,4.613)%
  --(9.506,4.608)--(9.507,4.604)--(9.509,4.601)--(9.510,4.599)--(9.512,4.599)--(9.513,4.601)%
  --(9.515,4.604)--(9.516,4.609)--(9.518,4.616)--(9.520,4.625)--(9.522,4.634)--(9.523,4.645)%
  --(9.525,4.656)--(9.527,4.668)--(9.528,4.680)--(9.530,4.692)--(9.532,4.703)--(9.533,4.714)%
  --(9.535,4.724)--(9.536,4.732)--(9.538,4.739)--(9.539,4.744)--(9.541,4.749)--(9.542,4.752)%
  --(9.544,4.754)--(9.545,4.754)--(9.547,4.753)--(9.548,4.750)--(9.550,4.746)--(9.552,4.740)%
  --(9.553,4.733)--(9.555,4.724)--(9.557,4.713)--(9.559,4.702)--(9.561,4.690)--(9.562,4.678)%
  --(9.564,4.666)--(9.566,4.655)--(9.568,4.644)--(9.569,4.636)--(9.571,4.629)--(9.572,4.623)%
  --(9.574,4.618)--(9.576,4.615)--(9.577,4.612)--(9.579,4.611)--(9.580,4.612)--(9.582,4.614)%
  --(9.583,4.617)--(9.585,4.622)--(9.587,4.628)--(9.589,4.638)--(9.591,4.649)--(9.593,4.661)%
  --(9.595,4.674)--(9.597,4.686)--(9.599,4.698)--(9.601,4.708)--(9.603,4.718)--(9.604,4.726)%
  --(9.606,4.732)--(9.607,4.737)--(9.609,4.741)--(9.610,4.744)--(9.612,4.746)--(9.613,4.746)%
  --(9.615,4.745)--(9.617,4.742)--(9.618,4.738)--(9.620,4.733)--(9.622,4.726)--(9.624,4.717)%
  --(9.625,4.708)--(9.627,4.698)--(9.629,4.687)--(9.631,4.677)--(9.633,4.666)--(9.635,4.656)%
  --(9.636,4.647)--(9.638,4.640)--(9.640,4.634)--(9.641,4.629)--(9.643,4.626)--(9.644,4.623)%
  --(9.646,4.622)--(9.647,4.621)--(9.649,4.622)--(9.651,4.625)--(9.652,4.629)--(9.654,4.634)%
  --(9.656,4.641)--(9.658,4.649)--(9.660,4.658)--(9.661,4.667)--(9.663,4.677)--(9.665,4.687)%
  --(9.667,4.697)--(9.669,4.707)--(9.671,4.715)--(9.672,4.721)--(9.674,4.727)--(9.675,4.731)%
  --(9.677,4.735)--(9.679,4.737)--(9.680,4.738)--(9.682,4.738)--(9.683,4.737)--(9.685,4.734)%
  --(9.687,4.730)--(9.689,4.724)--(9.690,4.718)--(9.692,4.710)--(9.694,4.701)--(9.696,4.692)%
  --(9.698,4.683)--(9.700,4.673)--(9.702,4.664)--(9.703,4.656)--(9.705,4.648)--(9.707,4.643)%
  --(9.709,4.638)--(9.710,4.634)--(9.712,4.632)--(9.713,4.630)--(9.715,4.630)--(9.717,4.631)%
  --(9.718,4.632)--(9.720,4.636)--(9.722,4.641)--(9.724,4.646)--(9.726,4.653)--(9.727,4.661)%
  --(9.729,4.669)--(9.731,4.678)--(9.733,4.686)--(9.735,4.695)--(9.736,4.703)--(9.738,4.710)%
  --(9.740,4.716)--(9.742,4.721)--(9.743,4.725)--(9.745,4.728)--(9.746,4.730)--(9.748,4.731)%
  --(9.750,4.731)--(9.751,4.730)--(9.753,4.727)--(9.755,4.723)--(9.757,4.718)--(9.759,4.712)%
  --(9.761,4.705)--(9.763,4.698)--(9.764,4.690)--(9.766,4.682)--(9.768,4.674)--(9.770,4.667)%
  --(9.771,4.660)--(9.773,4.654)--(9.775,4.648)--(9.777,4.644)--(9.778,4.641)--(9.780,4.639)%
  --(9.782,4.637)--(9.783,4.637)--(9.785,4.638)--(9.787,4.640)--(9.788,4.643)--(9.790,4.648)%
  --(9.792,4.653)--(9.794,4.660)--(9.796,4.668)--(9.798,4.674)--(9.800,4.682)--(9.801,4.689)%
  --(9.803,4.696)--(9.805,4.702)--(9.807,4.708)--(9.808,4.713)--(9.810,4.718)--(9.812,4.721)%
  --(9.813,4.723)--(9.815,4.725)--(9.817,4.725)--(9.819,4.725)--(9.820,4.723)--(9.822,4.720)%
  --(9.824,4.716)--(9.826,4.711)--(9.828,4.704)--(9.830,4.697)--(9.832,4.689)--(9.834,4.681)%
  --(9.837,4.673)--(9.839,4.665)--(9.841,4.658)--(9.843,4.654)--(9.844,4.650)--(9.846,4.647)%
  --(9.848,4.645)--(9.849,4.643)--(9.851,4.643)--(9.853,4.644)--(9.855,4.646)--(9.857,4.649)%
  --(9.858,4.653)--(9.860,4.658)--(9.862,4.664)--(9.864,4.670)--(9.866,4.676)--(9.868,4.683)%
  --(9.870,4.690)--(9.872,4.696)--(9.873,4.702)--(9.876,4.707)--(9.877,4.712)--(9.879,4.715)%
  --(9.881,4.717)--(9.882,4.719)--(9.884,4.720)--(9.886,4.719)--(9.888,4.718)--(9.890,4.716)%
  --(9.891,4.713)--(9.893,4.708)--(9.896,4.703)--(9.898,4.696)--(9.900,4.690)--(9.902,4.684)%
  --(9.903,4.678)--(9.905,4.672)--(9.907,4.667)--(9.909,4.662)--(9.910,4.658)--(9.912,4.654)%
  --(9.914,4.651)--(9.916,4.650)--(9.918,4.649)--(9.919,4.649)--(9.921,4.649)--(9.923,4.651)%
  --(9.925,4.654)--(9.927,4.658)--(9.929,4.662)--(9.930,4.667)--(9.933,4.673)--(9.935,4.679)%
  --(9.937,4.686)--(9.939,4.692)--(9.941,4.698)--(9.943,4.703)--(9.945,4.707)--(9.947,4.710)%
  --(9.948,4.713)--(9.950,4.714)--(9.952,4.715)--(9.954,4.715)--(9.956,4.714)--(9.957,4.712)%
  --(9.959,4.709)--(9.961,4.705)--(9.963,4.700)--(9.966,4.695)--(9.968,4.689)--(9.970,4.683)%
  --(9.971,4.678)--(9.973,4.673)--(9.975,4.668)--(9.977,4.664)--(9.979,4.660)--(9.981,4.657)%
  --(9.982,4.655)--(9.984,4.654)--(9.986,4.653)--(9.988,4.653)--(9.990,4.654)--(9.991,4.656)%
  --(9.994,4.659)--(9.995,4.663)--(9.997,4.667)--(9.999,4.672)--(10.001,4.677)--(10.004,4.683)%
  --(10.006,4.688)--(10.008,4.694)--(10.010,4.698)--(10.012,4.703)--(10.014,4.706)--(10.016,4.708)%
  --(10.018,4.710)--(10.019,4.711)--(10.021,4.711)--(10.023,4.710)--(10.025,4.708)--(10.027,4.706)%
  --(10.029,4.703)--(10.031,4.698)--(10.034,4.693)--(10.036,4.687)--(10.038,4.681)--(10.041,4.675)%
  --(10.043,4.670)--(10.045,4.666)--(10.047,4.662)--(10.049,4.660)--(10.051,4.658)--(10.053,4.657)%
  --(10.055,4.657)--(10.057,4.658)--(10.058,4.659)--(10.060,4.661)--(10.062,4.664)--(10.065,4.668)%
  --(10.067,4.673)--(10.069,4.678)--(10.072,4.683)--(10.074,4.689)--(10.076,4.694)--(10.079,4.698)%
  --(10.081,4.702)--(10.083,4.704)--(10.085,4.706)--(10.086,4.707)--(10.088,4.707)--(10.090,4.707)%
  --(10.092,4.706)--(10.094,4.704)--(10.096,4.701)--(10.098,4.697)--(10.101,4.693)--(10.103,4.688)%
  --(10.105,4.683)--(10.108,4.678)--(10.110,4.673)--(10.113,4.669)--(10.115,4.666)--(10.117,4.663)%
  --(10.119,4.662)--(10.121,4.661)--(10.123,4.660)--(10.124,4.661)--(10.126,4.662)--(10.128,4.664)%
  --(10.130,4.666)--(10.133,4.670)--(10.135,4.674)--(10.137,4.679)--(10.140,4.684)--(10.142,4.689)%
  --(10.145,4.693)--(10.147,4.697)--(10.149,4.700)--(10.151,4.702)--(10.153,4.703)--(10.155,4.704)%
  --(10.157,4.704)--(10.159,4.703)--(10.161,4.702)--(10.163,4.700)--(10.165,4.697)--(10.168,4.694)%
  --(10.170,4.689)--(10.173,4.685)--(10.175,4.680)--(10.177,4.676)--(10.180,4.672)--(10.182,4.669)%
  --(10.184,4.666)--(10.186,4.665)--(10.188,4.664)--(10.190,4.663)--(10.192,4.664)--(10.194,4.665)%
  --(10.196,4.666)--(10.198,4.669)--(10.200,4.671)--(10.203,4.676)--(10.206,4.680)--(10.209,4.685)%
  --(10.212,4.690)--(10.214,4.693)--(10.216,4.696)--(10.218,4.699)--(10.221,4.700)--(10.223,4.701)%
  --(10.225,4.701)--(10.226,4.701)--(10.228,4.700)--(10.231,4.698)--(10.233,4.696)--(10.235,4.693)%
  --(10.237,4.690)--(10.239,4.686)--(10.241,4.683)--(10.244,4.679)--(10.246,4.676)--(10.248,4.673)%
  --(10.250,4.670)--(10.252,4.668)--(10.254,4.667)--(10.256,4.666)--(10.259,4.666)--(10.260,4.666)%
  --(10.263,4.667)--(10.265,4.669)--(10.267,4.671)--(10.269,4.674)--(10.271,4.677)--(10.273,4.680)%
  --(10.276,4.683)--(10.278,4.687)--(10.280,4.690)--(10.282,4.693)--(10.284,4.695)--(10.287,4.697)%
  --(10.289,4.698)--(10.291,4.699)--(10.293,4.699)--(10.295,4.698)--(10.297,4.697)--(10.299,4.696)%
  --(10.301,4.693)--(10.304,4.691)--(10.306,4.688)--(10.308,4.685)--(10.310,4.682)--(10.312,4.679)%
  --(10.315,4.676)--(10.317,4.673)--(10.319,4.671)--(10.321,4.670)--(10.323,4.669)--(10.325,4.668)%
  --(10.327,4.668)--(10.330,4.669)--(10.332,4.670)--(10.334,4.672)--(10.336,4.674)--(10.339,4.677)%
  --(10.341,4.680)--(10.343,4.683)--(10.345,4.686)--(10.347,4.688)--(10.350,4.691)--(10.352,4.693)%
  --(10.354,4.695)--(10.356,4.696)--(10.358,4.697)--(10.360,4.697)--(10.363,4.696)--(10.365,4.695)%
  --(10.367,4.694)--(10.369,4.692)--(10.372,4.690)--(10.374,4.687)--(10.376,4.684)--(10.378,4.681)%
  --(10.381,4.679)--(10.383,4.676)--(10.385,4.674)--(10.387,4.673)--(10.389,4.671)--(10.391,4.671)%
  --(10.394,4.670)--(10.396,4.670)--(10.398,4.671)--(10.400,4.672)--(10.402,4.674)--(10.405,4.676)%
  --(10.407,4.679)--(10.410,4.682)--(10.413,4.684)--(10.415,4.687)--(10.418,4.690)--(10.420,4.692)%
  --(10.423,4.694)--(10.425,4.695)--(10.427,4.695)--(10.429,4.695)--(10.432,4.694)--(10.434,4.693)%
  --(10.436,4.692)--(10.439,4.690)--(10.441,4.687)--(10.443,4.685)--(10.446,4.682)--(10.448,4.680)%
  --(10.451,4.678)--(10.453,4.676)--(10.455,4.674)--(10.458,4.673)--(10.460,4.672)--(10.462,4.672)%
  --(10.465,4.672)--(10.467,4.673)--(10.469,4.674)--(10.471,4.676)--(10.474,4.678)--(10.476,4.680)%
  --(10.479,4.683)--(10.481,4.685)--(10.483,4.687)--(10.486,4.689)--(10.488,4.691)--(10.490,4.692)%
  --(10.492,4.693)--(10.495,4.693)--(10.497,4.693)--(10.499,4.693)--(10.501,4.692)--(10.504,4.691)%
  --(10.506,4.689)--(10.509,4.687)--(10.512,4.684)--(10.514,4.682)--(10.517,4.679)--(10.520,4.677)%
  --(10.522,4.676)--(10.525,4.674)--(10.528,4.674)--(10.530,4.673)--(10.532,4.674)--(10.534,4.674)%
  --(10.537,4.675)--(10.539,4.677)--(10.542,4.679)--(10.544,4.681)--(10.547,4.683)--(10.549,4.685)%
  --(10.551,4.687)--(10.554,4.688)--(10.556,4.690)--(10.558,4.691)--(10.561,4.692)--(10.563,4.692)%
  --(10.565,4.692)--(10.567,4.691)--(10.570,4.691)--(10.572,4.689)--(10.575,4.688)--(10.578,4.686)%
  --(10.580,4.683)--(10.584,4.681)--(10.586,4.679)--(10.589,4.677)--(10.592,4.676)--(10.594,4.675)%
  --(10.597,4.675)--(10.599,4.675)--(10.602,4.675)--(10.604,4.676)--(10.606,4.677)--(10.609,4.679)%
  --(10.612,4.681)--(10.614,4.683)--(10.617,4.685)--(10.620,4.687)--(10.623,4.688)--(10.626,4.690)%
  --(10.629,4.690)--(10.631,4.691)--(10.633,4.691)--(10.636,4.690)--(10.638,4.689)--(10.641,4.688)%
  --(10.643,4.687)--(10.646,4.685)--(10.648,4.683)--(10.651,4.682)--(10.653,4.680)--(10.656,4.679)%
  --(10.658,4.677)--(10.660,4.677)--(10.663,4.676)--(10.665,4.676)--(10.668,4.676)--(10.670,4.676)%
  --(10.673,4.677)--(10.675,4.678)--(10.677,4.680)--(10.681,4.682)--(10.684,4.684)--(10.687,4.686)%
  --(10.690,4.687)--(10.693,4.689)--(10.696,4.689)--(10.699,4.690)--(10.701,4.690)--(10.704,4.689)%
  --(10.706,4.689)--(10.708,4.688)--(10.711,4.686)--(10.714,4.685)--(10.717,4.683)--(10.721,4.681)%
  --(10.724,4.679)--(10.726,4.678)--(10.729,4.677)--(10.732,4.677)--(10.735,4.677)--(10.737,4.677)%
  --(10.740,4.678)--(10.742,4.679)--(10.745,4.680)--(10.748,4.681)--(10.751,4.683)--(10.754,4.685)%
  --(10.757,4.686)--(10.760,4.687)--(10.763,4.688)--(10.765,4.689)--(10.768,4.689)--(10.771,4.689)%
  --(10.773,4.688)--(10.776,4.687)--(10.778,4.686)--(10.781,4.685)--(10.784,4.683)--(10.788,4.681)%
  --(10.791,4.680)--(10.794,4.679)--(10.796,4.678)--(10.799,4.678)--(10.802,4.678)--(10.805,4.678)%
  --(10.807,4.678)--(10.810,4.679)--(10.812,4.680)--(10.815,4.681)--(10.819,4.683)--(10.822,4.684)%
  --(10.825,4.686)--(10.828,4.687)--(10.831,4.688)--(10.834,4.688)--(10.836,4.688)--(10.839,4.688)%
  --(10.842,4.687)--(10.844,4.686)--(10.847,4.686)--(10.850,4.684)--(10.853,4.683)--(10.856,4.681)%
  --(10.859,4.680)--(10.862,4.679)--(10.865,4.679)--(10.868,4.678)--(10.871,4.678)--(10.874,4.679)%
  --(10.876,4.679)--(10.879,4.680)--(10.882,4.681)--(10.885,4.682)--(10.888,4.684)--(10.892,4.685)%
  --(10.895,4.686)--(10.898,4.687)--(10.901,4.687)--(10.904,4.687)--(10.907,4.687)--(10.910,4.687)%
  --(10.912,4.686)--(10.915,4.685)--(10.918,4.684)--(10.920,4.683)--(10.923,4.682)--(10.926,4.681)%
  --(10.929,4.680)--(10.932,4.679)--(10.935,4.679)--(10.938,4.679)--(10.941,4.679)--(10.943,4.680)%
  --(10.946,4.680)--(10.949,4.681)--(10.951,4.682)--(10.954,4.683)--(10.957,4.684)--(10.960,4.685)%
  --(10.963,4.686)--(10.966,4.686)--(10.969,4.687)--(10.972,4.687)--(10.975,4.687)--(10.977,4.686)%
  --(10.980,4.686)--(10.983,4.685)--(10.985,4.684)--(10.988,4.683)--(10.991,4.682)--(10.994,4.681)%
  --(10.997,4.680)--(11.000,4.680)--(11.003,4.680)--(11.006,4.679)--(11.009,4.680)--(11.012,4.680)%
  --(11.015,4.681)--(11.017,4.681)--(11.020,4.682)--(11.023,4.683)--(11.026,4.684)--(11.029,4.685)%
  --(11.032,4.686)--(11.035,4.686)--(11.038,4.686)--(11.041,4.686)--(11.044,4.686)--(11.047,4.685)%
  --(11.050,4.685)--(11.052,4.684)--(11.055,4.683)--(11.058,4.682)--(11.061,4.682)--(11.064,4.681)%
  --(11.067,4.680)--(11.070,4.680)--(11.073,4.680)--(11.076,4.680)--(11.079,4.680)--(11.082,4.681)%
  --(11.085,4.682)--(11.088,4.682)--(11.090,4.683)--(11.093,4.684)--(11.096,4.685)--(11.099,4.685)%
  --(11.102,4.686)--(11.105,4.686)--(11.108,4.686)--(11.111,4.686)--(11.114,4.685)--(11.117,4.685)%
  --(11.120,4.684)--(11.123,4.683)--(11.126,4.683)--(11.129,4.682)--(11.132,4.681)--(11.135,4.681)%
  --(11.137,4.680)--(11.140,4.680)--(11.143,4.680)--(11.146,4.681)--(11.149,4.681)--(11.152,4.682)%
  --(11.155,4.682)--(11.158,4.683)--(11.161,4.684)--(11.164,4.684)--(11.167,4.685)--(11.170,4.685)%
  --(11.173,4.685)--(11.176,4.685)--(11.179,4.685)--(11.182,4.685)--(11.185,4.684)--(11.188,4.684)%
  --(11.191,4.683)--(11.195,4.682)--(11.198,4.682)--(11.201,4.681)--(11.205,4.681)--(11.208,4.681)%
  --(11.212,4.681)--(11.215,4.681)--(11.218,4.681)--(11.221,4.682)--(11.224,4.683)--(11.227,4.683)%
  --(11.230,4.684)--(11.234,4.684)--(11.237,4.685)--(11.240,4.685)--(11.243,4.685)--(11.246,4.685)%
  --(11.250,4.685)--(11.253,4.684)--(11.256,4.684)--(11.259,4.683)--(11.262,4.683)--(11.265,4.682)%
  --(11.268,4.682)--(11.271,4.681)--(11.274,4.681)--(11.277,4.681)--(11.280,4.681)--(11.283,4.681)%
  --(11.287,4.682)--(11.290,4.682)--(11.293,4.683)--(11.296,4.683)--(11.300,4.684)--(11.304,4.685)%
  --(11.307,4.685)--(11.311,4.685)--(11.315,4.685)--(11.318,4.684)--(11.321,4.684)--(11.324,4.684)%
  --(11.327,4.683)--(11.330,4.683)--(11.334,4.682)--(11.337,4.682)--(11.340,4.681)--(11.344,4.681)%
  --(11.347,4.681)--(11.350,4.681)--(11.353,4.682)--(11.357,4.682)--(11.360,4.683)--(11.363,4.683)%
  --(11.367,4.684)--(11.370,4.684)--(11.373,4.684)--(11.376,4.685)--(11.379,4.685)--(11.382,4.685)%
  --(11.386,4.684)--(11.389,4.684)--(11.392,4.684)--(11.396,4.683)--(11.399,4.682)--(11.403,4.682)%
  --(11.407,4.682)--(11.411,4.681)--(11.414,4.682)--(11.418,4.682)--(11.421,4.682)--(11.424,4.682)%
  --(11.428,4.683)--(11.431,4.683)--(11.434,4.684)--(11.438,4.684)--(11.441,4.684)--(11.444,4.684)%
  --(11.447,4.684)--(11.451,4.684)--(11.454,4.684)--(11.457,4.684)--(11.460,4.683)--(11.464,4.683)%
  --(11.468,4.682)--(11.472,4.682)--(11.476,4.682)--(11.480,4.682)--(11.483,4.682)--(11.487,4.682)%
  --(11.491,4.682)--(11.494,4.683)--(11.498,4.683)--(11.501,4.683)--(11.504,4.684)--(11.508,4.684)%
  --(11.512,4.684)--(11.516,4.684)--(11.520,4.684)--(11.524,4.684)--(11.527,4.683)--(11.530,4.683)%
  --(11.534,4.683)--(11.537,4.682)--(11.541,4.682)--(11.544,4.682)--(11.547,4.682)--(11.551,4.682)%
  --(11.554,4.682)--(11.558,4.682)--(11.561,4.683)--(11.565,4.683)--(11.569,4.683)--(11.573,4.684)%
  --(11.577,4.684)--(11.580,4.684)--(11.584,4.684)--(11.588,4.684)--(11.592,4.684)--(11.595,4.683)%
  --(11.599,4.683)--(11.602,4.683)--(11.606,4.682)--(11.610,4.682)--(11.614,4.682)--(11.619,4.682)%
  --(11.623,4.682)--(11.626,4.682)--(11.630,4.683)--(11.633,4.683)--(11.637,4.683)--(11.640,4.684)%
  --(11.644,4.684)--(11.647,4.684)--(11.651,4.684)--(11.655,4.684)--(11.658,4.684)--(11.662,4.683)%
  --(11.665,4.683)--(11.669,4.683)--(11.674,4.682)--(11.678,4.682)--(11.682,4.682)--(11.686,4.682)%
  --(11.690,4.682)--(11.694,4.682)--(11.697,4.683)--(11.701,4.683)--(11.705,4.683)--(11.709,4.684)%
  --(11.713,4.684)--(11.717,4.684)--(11.721,4.684)--(11.725,4.684)--(11.729,4.683)--(11.733,4.683)%
  --(11.736,4.683)--(11.740,4.683)--(11.744,4.682)--(11.748,4.682)--(11.752,4.682)--(11.757,4.682)%
  --(11.761,4.683)--(11.765,4.683)--(11.768,4.683)--(11.772,4.683)--(11.776,4.683)--(11.779,4.684)%
  --(11.783,4.684)--(11.787,4.684)--(11.791,4.684)--(11.794,4.683)--(11.798,4.683)--(11.802,4.683)%
  --(11.805,4.683)--(11.809,4.683)--(11.813,4.682)--(11.817,4.682)--(11.821,4.682)--(11.825,4.682)%
  --(11.828,4.683)--(11.832,4.683)--(11.836,4.683)--(11.840,4.683)--(11.845,4.683)--(11.849,4.684)%
  --(11.854,4.684)--(11.858,4.684)--(11.862,4.683)--(11.866,4.683)--(11.870,4.683)--(11.874,4.683)%
  --(11.879,4.683)--(11.883,4.682)--(11.888,4.682)--(11.892,4.682)--(11.896,4.683)--(11.900,4.683)%
  --(11.904,4.683)--(11.908,4.683)--(11.913,4.683)--(11.917,4.684)--(11.922,4.684)--(11.926,4.683)%
  --(11.930,4.683)--(11.934,4.683)--(11.938,4.683)--(11.942,4.683);
\gpsetlinewidth{1.00}
\draw[gp path] (1.504,8.381)--(1.504,0.985)--(11.947,0.985)--(11.947,8.381)--cycle;
%% coordinates of the plot area
\gpdefrectangularnode{gp plot 1}{\pgfpoint{1.504cm}{0.985cm}}{\pgfpoint{11.947cm}{8.381cm}}
\end{tikzpicture}
%% gnuplot variables

    \caption{System control force over time.}\label{f:forceplot}
\end{figure}

\begin{figure}
    \centering
    \begin{tikzpicture}[gnuplot]
%% generated with GNUPLOT 5.0p3 (Lua 5.1; terminal rev. 99, script rev. 100)
%% Thu 29 Mar 2018 12:41:19 AM EDT
\gpmonochromelines
\path (0.000,0.000) rectangle (12.500,8.750);
\gpcolor{color=gp lt color border}
\gpsetlinetype{gp lt border}
\gpsetdashtype{gp dt solid}
\gpsetlinewidth{1.00}
\draw[gp path] (1.320,0.985)--(1.500,0.985);
\draw[gp path] (11.947,0.985)--(11.767,0.985);
\node[gp node right] at (1.136,0.985) {$0$};
\draw[gp path] (1.320,2.464)--(1.500,2.464);
\draw[gp path] (11.947,2.464)--(11.767,2.464);
\node[gp node right] at (1.136,2.464) {$20$};
\draw[gp path] (1.320,3.943)--(1.500,3.943);
\draw[gp path] (11.947,3.943)--(11.767,3.943);
\node[gp node right] at (1.136,3.943) {$40$};
\draw[gp path] (1.320,5.423)--(1.500,5.423);
\draw[gp path] (11.947,5.423)--(11.767,5.423);
\node[gp node right] at (1.136,5.423) {$60$};
\draw[gp path] (1.320,6.902)--(1.500,6.902);
\draw[gp path] (11.947,6.902)--(11.767,6.902);
\node[gp node right] at (1.136,6.902) {$80$};
\draw[gp path] (1.320,8.381)--(1.500,8.381);
\draw[gp path] (11.947,8.381)--(11.767,8.381);
\node[gp node right] at (1.136,8.381) {$100$};
\draw[gp path] (1.320,0.985)--(1.320,1.165);
\draw[gp path] (1.320,8.381)--(1.320,8.201);
\node[gp node center] at (1.320,0.677) {$0$};
\draw[gp path] (2.383,0.985)--(2.383,1.165);
\draw[gp path] (2.383,8.381)--(2.383,8.201);
\node[gp node center] at (2.383,0.677) {$5$};
\draw[gp path] (3.445,0.985)--(3.445,1.165);
\draw[gp path] (3.445,8.381)--(3.445,8.201);
\node[gp node center] at (3.445,0.677) {$10$};
\draw[gp path] (4.508,0.985)--(4.508,1.165);
\draw[gp path] (4.508,8.381)--(4.508,8.201);
\node[gp node center] at (4.508,0.677) {$15$};
\draw[gp path] (5.571,0.985)--(5.571,1.165);
\draw[gp path] (5.571,8.381)--(5.571,8.201);
\node[gp node center] at (5.571,0.677) {$20$};
\draw[gp path] (6.634,0.985)--(6.634,1.165);
\draw[gp path] (6.634,8.381)--(6.634,8.201);
\node[gp node center] at (6.634,0.677) {$25$};
\draw[gp path] (7.696,0.985)--(7.696,1.165);
\draw[gp path] (7.696,8.381)--(7.696,8.201);
\node[gp node center] at (7.696,0.677) {$30$};
\draw[gp path] (8.759,0.985)--(8.759,1.165);
\draw[gp path] (8.759,8.381)--(8.759,8.201);
\node[gp node center] at (8.759,0.677) {$35$};
\draw[gp path] (9.822,0.985)--(9.822,1.165);
\draw[gp path] (9.822,8.381)--(9.822,8.201);
\node[gp node center] at (9.822,0.677) {$40$};
\draw[gp path] (10.884,0.985)--(10.884,1.165);
\draw[gp path] (10.884,8.381)--(10.884,8.201);
\node[gp node center] at (10.884,0.677) {$45$};
\draw[gp path] (11.947,0.985)--(11.947,1.165);
\draw[gp path] (11.947,8.381)--(11.947,8.201);
\node[gp node center] at (11.947,0.677) {$50$};
\draw[gp path] (1.320,8.381)--(1.320,0.985)--(11.947,0.985)--(11.947,8.381)--cycle;
\node[gp node center,rotate=-270] at (0.246,4.683) {Distance, m};
\node[gp node center] at (6.633,0.215) {Time, s};
\node[gp node right] at (10.479,8.047) {$x_1$};
\gpsetlinewidth{2.00}
\draw[gp path] (10.663,8.047)--(11.579,8.047);
\draw[gp path] (1.320,0.985)--(1.321,0.985)--(1.322,0.985)--(1.323,0.985)--(1.324,0.985)%
  --(1.325,0.985)--(1.326,0.985)--(1.327,0.985)--(1.328,0.985)--(1.329,0.985)--(1.331,0.985)%
  --(1.332,0.985)--(1.333,0.985)--(1.335,0.985)--(1.336,0.985)--(1.337,0.985)--(1.339,0.985)%
  --(1.340,0.985)--(1.342,0.985)--(1.343,0.985)--(1.345,0.985)--(1.346,0.985)--(1.348,0.985)%
  --(1.350,0.985)--(1.351,0.985)--(1.353,0.985)--(1.354,0.985)--(1.355,0.985)--(1.357,0.985)%
  --(1.358,0.985)--(1.360,0.985)--(1.361,0.985)--(1.362,0.985)--(1.363,0.985)--(1.365,0.985)%
  --(1.366,0.985)--(1.367,0.986)--(1.369,0.986)--(1.370,0.986)--(1.371,0.986)--(1.373,0.986)%
  --(1.374,0.986)--(1.376,0.986)--(1.377,0.986)--(1.379,0.986)--(1.380,0.986)--(1.382,0.986)%
  --(1.384,0.986)--(1.385,0.986)--(1.387,0.986)--(1.388,0.986)--(1.389,0.986)--(1.391,0.986)%
  --(1.392,0.986)--(1.393,0.986)--(1.395,0.987)--(1.396,0.987)--(1.397,0.987)--(1.399,0.987)%
  --(1.400,0.987)--(1.401,0.987)--(1.403,0.987)--(1.404,0.987)--(1.405,0.987)--(1.407,0.987)%
  --(1.408,0.987)--(1.409,0.987)--(1.411,0.987)--(1.412,0.987)--(1.414,0.987)--(1.416,0.987)%
  --(1.417,0.987)--(1.419,0.988)--(1.420,0.988)--(1.422,0.988)--(1.423,0.988)--(1.424,0.988)%
  --(1.426,0.988)--(1.427,0.988)--(1.429,0.988)--(1.430,0.988)--(1.431,0.988)--(1.433,0.988)%
  --(1.434,0.988)--(1.435,0.989)--(1.436,0.989)--(1.438,0.989)--(1.439,0.989)--(1.440,0.989)%
  --(1.442,0.989)--(1.443,0.989)--(1.445,0.989)--(1.446,0.989)--(1.448,0.989)--(1.449,0.990)%
  --(1.451,0.990)--(1.452,0.990)--(1.454,0.990)--(1.455,0.990)--(1.457,0.990)--(1.458,0.990)%
  --(1.460,0.990)--(1.461,0.990)--(1.462,0.991)--(1.464,0.991)--(1.465,0.991)--(1.466,0.991)%
  --(1.468,0.991)--(1.469,0.991)--(1.470,0.991)--(1.472,0.991)--(1.473,0.991)--(1.474,0.991)%
  --(1.476,0.992)--(1.477,0.992)--(1.478,0.992)--(1.480,0.992)--(1.481,0.992)--(1.483,0.992)%
  --(1.485,0.992)--(1.486,0.992)--(1.488,0.993)--(1.489,0.993)--(1.490,0.993)--(1.492,0.993)%
  --(1.493,0.993)--(1.495,0.993)--(1.496,0.993)--(1.498,0.993)--(1.499,0.994)--(1.500,0.994)%
  --(1.502,0.994)--(1.503,0.994)--(1.504,0.994)--(1.506,0.994)--(1.507,0.994)--(1.508,0.995)%
  --(1.510,0.995)--(1.511,0.995)--(1.512,0.995)--(1.514,0.995)--(1.515,0.995)--(1.517,0.996)%
  --(1.518,0.996)--(1.520,0.996)--(1.521,0.996)--(1.523,0.996)--(1.524,0.996)--(1.526,0.997)%
  --(1.528,0.997)--(1.529,0.997)--(1.531,0.997)--(1.532,0.997)--(1.533,0.997)--(1.535,0.998)%
  --(1.536,0.998)--(1.537,0.998)--(1.538,0.998)--(1.540,0.998)--(1.541,0.998)--(1.542,0.998)%
  --(1.544,0.999)--(1.545,0.999)--(1.546,0.999)--(1.548,0.999)--(1.549,0.999)--(1.551,0.999)%
  --(1.552,1.000)--(1.554,1.000)--(1.556,1.000)--(1.557,1.000)--(1.559,1.000)--(1.560,1.001)%
  --(1.562,1.001)--(1.563,1.001)--(1.565,1.001)--(1.566,1.001)--(1.568,1.002)--(1.569,1.002)%
  --(1.570,1.002)--(1.571,1.002)--(1.573,1.002)--(1.574,1.003)--(1.575,1.003)--(1.577,1.003)%
  --(1.578,1.003)--(1.579,1.003)--(1.581,1.003)--(1.582,1.004)--(1.584,1.004)--(1.585,1.004)%
  --(1.587,1.004)--(1.588,1.005)--(1.590,1.005)--(1.591,1.005)--(1.593,1.005)--(1.595,1.006)%
  --(1.596,1.006)--(1.598,1.006)--(1.599,1.006)--(1.601,1.006)--(1.602,1.007)--(1.603,1.007)%
  --(1.605,1.007)--(1.606,1.007)--(1.607,1.007)--(1.608,1.008)--(1.610,1.008)--(1.611,1.008)%
  --(1.612,1.008)--(1.614,1.008)--(1.615,1.009)--(1.617,1.009)--(1.618,1.009)--(1.620,1.009)%
  --(1.621,1.010)--(1.623,1.010)--(1.624,1.010)--(1.626,1.010)--(1.627,1.011)--(1.629,1.011)%
  --(1.630,1.011)--(1.632,1.011)--(1.634,1.012)--(1.635,1.012)--(1.636,1.012)--(1.638,1.012)%
  --(1.639,1.013)--(1.640,1.013)--(1.642,1.013)--(1.643,1.013)--(1.644,1.014)--(1.645,1.014)%
  --(1.647,1.014)--(1.648,1.014)--(1.650,1.015)--(1.651,1.015)--(1.652,1.015)--(1.654,1.015)%
  --(1.655,1.016)--(1.657,1.016)--(1.659,1.016)--(1.660,1.017)--(1.662,1.017)--(1.663,1.017)%
  --(1.665,1.017)--(1.667,1.018)--(1.668,1.018)--(1.669,1.018)--(1.671,1.019)--(1.672,1.019)%
  --(1.673,1.019)--(1.675,1.019)--(1.676,1.020)--(1.677,1.020)--(1.679,1.020)--(1.680,1.020)%
  --(1.681,1.021)--(1.683,1.021)--(1.684,1.021)--(1.685,1.021)--(1.687,1.022)--(1.688,1.022)%
  --(1.690,1.022)--(1.692,1.023)--(1.693,1.023)--(1.695,1.023)--(1.696,1.023)--(1.698,1.024)%
  --(1.699,1.024)--(1.701,1.024)--(1.703,1.025)--(1.704,1.025)--(1.705,1.025)--(1.707,1.026)%
  --(1.708,1.026)--(1.709,1.026)--(1.710,1.027)--(1.712,1.027)--(1.713,1.027)--(1.714,1.027)%
  --(1.716,1.028)--(1.717,1.028)--(1.718,1.028)--(1.720,1.029)--(1.721,1.029)--(1.723,1.029)%
  --(1.724,1.030)--(1.726,1.030)--(1.728,1.030)--(1.729,1.031)--(1.731,1.031)--(1.732,1.031)%
  --(1.734,1.032)--(1.735,1.032)--(1.737,1.032)--(1.738,1.033)--(1.740,1.033)--(1.741,1.033)%
  --(1.742,1.034)--(1.744,1.034)--(1.745,1.034)--(1.746,1.035)--(1.748,1.035)--(1.749,1.035)%
  --(1.750,1.035)--(1.752,1.036)--(1.753,1.036)--(1.754,1.036)--(1.756,1.037)--(1.757,1.037)%
  --(1.759,1.037)--(1.760,1.038)--(1.762,1.038)--(1.764,1.039)--(1.765,1.039)--(1.767,1.039)%
  --(1.768,1.040)--(1.770,1.040)--(1.771,1.040)--(1.773,1.041)--(1.774,1.041)--(1.775,1.041)%
  --(1.777,1.042)--(1.778,1.042)--(1.779,1.042)--(1.781,1.043)--(1.782,1.043)--(1.783,1.044)%
  --(1.785,1.044)--(1.786,1.044)--(1.787,1.045)--(1.789,1.045)--(1.790,1.045)--(1.792,1.046)%
  --(1.793,1.046)--(1.795,1.047)--(1.797,1.047)--(1.798,1.047)--(1.800,1.048)--(1.801,1.048)%
  --(1.803,1.049)--(1.804,1.049)--(1.806,1.049)--(1.807,1.050)--(1.809,1.050)--(1.810,1.050)%
  --(1.811,1.051)--(1.813,1.051)--(1.814,1.052)--(1.815,1.052)--(1.816,1.052)--(1.818,1.053)%
  --(1.819,1.053)--(1.821,1.053)--(1.822,1.054)--(1.823,1.054)--(1.825,1.054)--(1.826,1.055)%
  --(1.828,1.055)--(1.829,1.056)--(1.831,1.056)--(1.833,1.057)--(1.834,1.057)--(1.836,1.057)%
  --(1.837,1.058)--(1.839,1.058)--(1.840,1.059)--(1.842,1.059)--(1.843,1.060)--(1.844,1.060)%
  --(1.846,1.060)--(1.847,1.061)--(1.848,1.061)--(1.850,1.061)--(1.851,1.062)--(1.852,1.062)%
  --(1.854,1.063)--(1.855,1.063)--(1.856,1.063)--(1.858,1.064)--(1.859,1.064)--(1.861,1.065)%
  --(1.862,1.065)--(1.864,1.066)--(1.865,1.066)--(1.867,1.067)--(1.869,1.067)--(1.870,1.068)%
  --(1.872,1.068)--(1.873,1.069)--(1.875,1.069)--(1.876,1.069)--(1.878,1.070)--(1.879,1.070)%
  --(1.880,1.071)--(1.882,1.071)--(1.883,1.071)--(1.884,1.072)--(1.885,1.072)--(1.887,1.073)%
  --(1.888,1.073)--(1.889,1.073)--(1.891,1.074)--(1.892,1.074)--(1.894,1.075)--(1.895,1.075)%
  --(1.897,1.076)--(1.898,1.076)--(1.900,1.077)--(1.902,1.077)--(1.903,1.078)--(1.905,1.078)%
  --(1.906,1.079)--(1.908,1.079)--(1.909,1.080)--(1.911,1.080)--(1.912,1.081)--(1.913,1.081)%
  --(1.915,1.081)--(1.916,1.082)--(1.917,1.082)--(1.919,1.083)--(1.920,1.083)--(1.921,1.084)%
  --(1.923,1.084)--(1.924,1.084)--(1.925,1.085)--(1.927,1.085)--(1.928,1.086)--(1.930,1.086)%
  --(1.931,1.087)--(1.933,1.087)--(1.934,1.088)--(1.936,1.089)--(1.938,1.089)--(1.939,1.090)%
  --(1.941,1.090)--(1.942,1.091)--(1.944,1.091)--(1.945,1.092)--(1.947,1.092)--(1.948,1.093)%
  --(1.949,1.093)--(1.951,1.093)--(1.952,1.094)--(1.953,1.094)--(1.954,1.095)--(1.956,1.095)%
  --(1.957,1.096)--(1.958,1.096)--(1.960,1.097)--(1.961,1.097)--(1.963,1.098)--(1.964,1.098)%
  --(1.966,1.099)--(1.967,1.099)--(1.969,1.100)--(1.970,1.100)--(1.972,1.101)--(1.974,1.101)%
  --(1.975,1.102)--(1.977,1.103)--(1.978,1.103)--(1.980,1.104)--(1.981,1.104)--(1.982,1.105)%
  --(1.984,1.105)--(1.985,1.106)--(1.986,1.106)--(1.988,1.107)--(1.989,1.107)--(1.990,1.107)%
  --(1.992,1.108)--(1.993,1.108)--(1.994,1.109)--(1.996,1.110)--(1.997,1.110)--(1.999,1.111)%
  --(2.000,1.111)--(2.002,1.112)--(2.003,1.112)--(2.005,1.113)--(2.007,1.114)--(2.008,1.114)%
  --(2.010,1.115)--(2.011,1.115)--(2.013,1.116)--(2.014,1.116)--(2.016,1.117)--(2.017,1.117)%
  --(2.018,1.118)--(2.019,1.118)--(2.021,1.119)--(2.022,1.119)--(2.023,1.120)--(2.025,1.120)%
  --(2.026,1.121)--(2.027,1.121)--(2.029,1.122)--(2.030,1.123)--(2.032,1.123)--(2.033,1.124)%
  --(2.035,1.124)--(2.036,1.125)--(2.038,1.126)--(2.039,1.126)--(2.041,1.127)--(2.043,1.127)%
  --(2.044,1.128)--(2.046,1.129)--(2.047,1.129)--(2.049,1.130)--(2.050,1.130)--(2.051,1.131)%
  --(2.053,1.131)--(2.054,1.132)--(2.055,1.132)--(2.057,1.133)--(2.058,1.133)--(2.059,1.134)%
  --(2.061,1.135)--(2.062,1.135)--(2.063,1.136)--(2.065,1.136)--(2.066,1.137)--(2.068,1.138)%
  --(2.069,1.138)--(2.071,1.139)--(2.072,1.139)--(2.074,1.140)--(2.076,1.141)--(2.077,1.141)%
  --(2.079,1.142)--(2.080,1.143)--(2.082,1.143)--(2.083,1.144)--(2.084,1.144)--(2.086,1.145)%
  --(2.087,1.146)--(2.088,1.146)--(2.090,1.147)--(2.091,1.147)--(2.092,1.148)--(2.094,1.148)%
  --(2.095,1.149)--(2.096,1.149)--(2.098,1.150)--(2.099,1.151)--(2.101,1.151)--(2.102,1.152)%
  --(2.104,1.152)--(2.105,1.153)--(2.107,1.154)--(2.108,1.154)--(2.110,1.155)--(2.112,1.156)%
  --(2.113,1.157)--(2.115,1.157)--(2.116,1.158)--(2.118,1.158)--(2.119,1.159)--(2.120,1.160)%
  --(2.122,1.160)--(2.123,1.161)--(2.124,1.161)--(2.126,1.162)--(2.127,1.163)--(2.128,1.163)%
  --(2.130,1.164)--(2.131,1.164)--(2.132,1.165)--(2.134,1.166)--(2.135,1.166)--(2.137,1.167)%
  --(2.138,1.168)--(2.140,1.168)--(2.141,1.169)--(2.143,1.170)--(2.144,1.170)--(2.146,1.171)%
  --(2.148,1.172)--(2.149,1.173)--(2.151,1.173)--(2.152,1.174)--(2.153,1.175)--(2.155,1.175)%
  --(2.156,1.176)--(2.157,1.176)--(2.159,1.177)--(2.160,1.177)--(2.161,1.178)--(2.163,1.179)%
  --(2.164,1.179)--(2.165,1.180)--(2.167,1.181)--(2.168,1.181)--(2.170,1.182)--(2.171,1.183)%
  --(2.173,1.183)--(2.174,1.184)--(2.176,1.185)--(2.177,1.185)--(2.179,1.186)--(2.181,1.187)%
  --(2.182,1.188)--(2.184,1.188)--(2.185,1.189)--(2.187,1.190)--(2.188,1.190)--(2.189,1.191)%
  --(2.191,1.192)--(2.192,1.192)--(2.193,1.193)--(2.194,1.194)--(2.196,1.194)--(2.197,1.195)%
  --(2.198,1.196)--(2.200,1.196)--(2.201,1.197)--(2.203,1.198)--(2.204,1.198)--(2.206,1.199)%
  --(2.207,1.200)--(2.209,1.200)--(2.210,1.201)--(2.212,1.202)--(2.213,1.203)--(2.215,1.204)%
  --(2.217,1.204)--(2.218,1.205)--(2.220,1.206)--(2.221,1.207)--(2.222,1.207)--(2.224,1.208)%
  --(2.225,1.209)--(2.226,1.209)--(2.228,1.210)--(2.229,1.210)--(2.230,1.211)--(2.232,1.212)%
  --(2.233,1.212)--(2.234,1.213)--(2.236,1.214)--(2.237,1.214)--(2.239,1.215)--(2.240,1.216)%
  --(2.242,1.217)--(2.243,1.217)--(2.245,1.218)--(2.246,1.219)--(2.248,1.220)--(2.249,1.221)%
  --(2.251,1.221)--(2.253,1.222)--(2.254,1.223)--(2.256,1.224)--(2.257,1.224)--(2.258,1.225)%
  --(2.260,1.226)--(2.261,1.226)--(2.262,1.227)--(2.263,1.228)--(2.265,1.228)--(2.266,1.229)%
  --(2.267,1.230)--(2.269,1.231)--(2.270,1.231)--(2.272,1.232)--(2.273,1.233)--(2.275,1.234)%
  --(2.276,1.234)--(2.278,1.235)--(2.279,1.236)--(2.281,1.237)--(2.282,1.238)--(2.284,1.239)%
  --(2.286,1.239)--(2.287,1.240)--(2.289,1.241)--(2.290,1.242)--(2.291,1.242)--(2.293,1.243)%
  --(2.294,1.244)--(2.295,1.245)--(2.297,1.245)--(2.298,1.246)--(2.299,1.247)--(2.301,1.247)%
  --(2.302,1.248)--(2.303,1.249)--(2.305,1.249)--(2.306,1.250)--(2.308,1.251)--(2.309,1.252)%
  --(2.311,1.253)--(2.312,1.253)--(2.314,1.254)--(2.315,1.255)--(2.317,1.256)--(2.318,1.257)%
  --(2.320,1.258)--(2.322,1.259)--(2.323,1.259)--(2.325,1.260)--(2.326,1.261)--(2.327,1.262)%
  --(2.329,1.262)--(2.330,1.263)--(2.331,1.264)--(2.332,1.265)--(2.334,1.265)--(2.335,1.266)%
  --(2.336,1.267)--(2.338,1.268)--(2.339,1.268)--(2.341,1.269)--(2.342,1.270)--(2.344,1.271)%
  --(2.345,1.272)--(2.347,1.273)--(2.348,1.273)--(2.350,1.274)--(2.351,1.275)--(2.353,1.276)%
  --(2.355,1.277)--(2.356,1.278)--(2.358,1.279)--(2.359,1.280)--(2.360,1.280)--(2.362,1.281)%
  --(2.363,1.282)--(2.364,1.283)--(2.366,1.283)--(2.367,1.284)--(2.368,1.285)--(2.370,1.286)%
  --(2.371,1.286)--(2.372,1.287)--(2.374,1.288)--(2.375,1.289)--(2.377,1.290)--(2.378,1.290)%
  --(2.380,1.291)--(2.381,1.292)--(2.383,1.293)--(2.384,1.294)--(2.386,1.295)--(2.387,1.296)%
  --(2.389,1.297)--(2.391,1.298)--(2.392,1.299)--(2.393,1.299)--(2.395,1.300)--(2.396,1.301)%
  --(2.397,1.302)--(2.399,1.302)--(2.400,1.303)--(2.401,1.304)--(2.403,1.305)--(2.404,1.306)%
  --(2.405,1.306)--(2.407,1.307)--(2.408,1.308)--(2.410,1.309)--(2.411,1.310)--(2.412,1.311)%
  --(2.414,1.312)--(2.416,1.313)--(2.417,1.313)--(2.419,1.314)--(2.420,1.315)--(2.422,1.316)%
  --(2.423,1.317)--(2.425,1.318)--(2.427,1.319)--(2.428,1.320)--(2.429,1.321)--(2.431,1.322)%
  --(2.432,1.322)--(2.433,1.323)--(2.435,1.324)--(2.436,1.325)--(2.437,1.325)--(2.438,1.326)%
  --(2.440,1.327)--(2.441,1.328)--(2.443,1.329)--(2.444,1.330)--(2.445,1.331)--(2.447,1.331)%
  --(2.448,1.332)--(2.450,1.333)--(2.452,1.334)--(2.453,1.335)--(2.455,1.336)--(2.456,1.337)%
  --(2.458,1.338)--(2.460,1.339)--(2.461,1.340)--(2.462,1.341)--(2.464,1.342)--(2.465,1.343)%
  --(2.466,1.344)--(2.468,1.344)--(2.469,1.345)--(2.470,1.346)--(2.472,1.347)--(2.473,1.348)%
  --(2.474,1.349)--(2.476,1.349)--(2.477,1.350)--(2.478,1.351)--(2.480,1.352)--(2.481,1.353)%
  --(2.483,1.354)--(2.485,1.355)--(2.486,1.356)--(2.488,1.357)--(2.489,1.358)--(2.491,1.359)%
  --(2.492,1.360)--(2.494,1.361)--(2.496,1.362)--(2.497,1.363)--(2.498,1.364)--(2.500,1.365)%
  --(2.501,1.366)--(2.502,1.366)--(2.504,1.367)--(2.505,1.368)--(2.506,1.369)--(2.507,1.370)%
  --(2.509,1.371)--(2.510,1.371)--(2.511,1.372)--(2.513,1.373)--(2.514,1.374)--(2.516,1.375)%
  --(2.517,1.376)--(2.519,1.377)--(2.521,1.378)--(2.522,1.379)--(2.524,1.380)--(2.525,1.381)%
  --(2.527,1.382)--(2.528,1.383)--(2.530,1.384)--(2.531,1.385)--(2.533,1.386)--(2.534,1.387)%
  --(2.535,1.388)--(2.537,1.389)--(2.538,1.390)--(2.539,1.391)--(2.541,1.391)--(2.542,1.392)%
  --(2.543,1.393)--(2.545,1.394)--(2.546,1.395)--(2.547,1.396)--(2.549,1.397)--(2.550,1.398)%
  --(2.552,1.399)--(2.553,1.400)--(2.555,1.401)--(2.557,1.402)--(2.558,1.403)--(2.560,1.404)%
  --(2.561,1.406)--(2.563,1.407)--(2.565,1.408)--(2.566,1.409)--(2.567,1.409)--(2.569,1.410)%
  --(2.570,1.411)--(2.571,1.412)--(2.572,1.413)--(2.574,1.414)--(2.575,1.415)--(2.576,1.416)%
  --(2.578,1.417)--(2.579,1.418)--(2.580,1.418)--(2.582,1.419)--(2.583,1.420)--(2.585,1.421)%
  --(2.586,1.423)--(2.588,1.424)--(2.590,1.425)--(2.591,1.426)--(2.593,1.427)--(2.594,1.428)%
  --(2.596,1.429)--(2.597,1.430)--(2.599,1.431)--(2.600,1.432)--(2.602,1.433)--(2.603,1.434)%
  --(2.604,1.435)--(2.606,1.436)--(2.607,1.437)--(2.608,1.438)--(2.610,1.439)--(2.611,1.440)%
  --(2.612,1.441)--(2.614,1.442)--(2.615,1.443)--(2.616,1.444)--(2.618,1.445)--(2.619,1.446)%
  --(2.621,1.447)--(2.622,1.448)--(2.624,1.449)--(2.626,1.450)--(2.627,1.451)--(2.629,1.452)%
  --(2.630,1.454)--(2.632,1.455)--(2.634,1.456)--(2.635,1.457)--(2.636,1.458)--(2.637,1.459)%
  --(2.639,1.460)--(2.640,1.461)--(2.641,1.461)--(2.643,1.462)--(2.644,1.463)--(2.645,1.464)%
  --(2.647,1.465)--(2.648,1.466)--(2.649,1.467)--(2.651,1.468)--(2.652,1.469)--(2.654,1.470)%
  --(2.655,1.471)--(2.657,1.473)--(2.659,1.474)--(2.660,1.475)--(2.662,1.476)--(2.663,1.477)%
  --(2.665,1.478)--(2.666,1.480)--(2.668,1.481)--(2.669,1.482)--(2.671,1.483)--(2.672,1.484)%
  --(2.673,1.485)--(2.675,1.486)--(2.676,1.487)--(2.677,1.488)--(2.678,1.489)--(2.680,1.490)%
  --(2.681,1.491)--(2.683,1.492)--(2.684,1.493)--(2.685,1.494)--(2.687,1.495)--(2.688,1.496)%
  --(2.690,1.497)--(2.691,1.498)--(2.693,1.499)--(2.695,1.501)--(2.696,1.502)--(2.698,1.503)%
  --(2.699,1.504)--(2.701,1.505)--(2.702,1.507)--(2.704,1.508)--(2.705,1.509)--(2.706,1.510)%
  --(2.708,1.511)--(2.709,1.512)--(2.710,1.512)--(2.712,1.513)--(2.713,1.514)--(2.714,1.515)%
  --(2.716,1.516)--(2.717,1.517)--(2.718,1.519)--(2.720,1.520)--(2.721,1.521)--(2.723,1.522)%
  --(2.724,1.523)--(2.726,1.524)--(2.727,1.525)--(2.729,1.527)--(2.731,1.528)--(2.732,1.529)%
  --(2.734,1.530)--(2.735,1.532)--(2.737,1.533)--(2.738,1.534)--(2.740,1.535)--(2.741,1.536)%
  --(2.742,1.537)--(2.744,1.538)--(2.745,1.539)--(2.746,1.540)--(2.747,1.541)--(2.749,1.542)%
  --(2.750,1.543)--(2.751,1.544)--(2.753,1.545)--(2.754,1.546)--(2.756,1.548)--(2.757,1.549)%
  --(2.759,1.550)--(2.760,1.551)--(2.762,1.552)--(2.764,1.554)--(2.765,1.555)--(2.767,1.556)%
  --(2.768,1.557)--(2.770,1.559)--(2.771,1.560)--(2.773,1.561)--(2.774,1.562)--(2.775,1.563)%
  --(2.777,1.564)--(2.778,1.565)--(2.779,1.566)--(2.781,1.567)--(2.782,1.568)--(2.783,1.569)%
  --(2.785,1.570)--(2.786,1.571)--(2.787,1.572)--(2.789,1.574)--(2.790,1.575)--(2.792,1.576)%
  --(2.793,1.577)--(2.795,1.578)--(2.796,1.580)--(2.798,1.581)--(2.800,1.582)--(2.801,1.584)%
  --(2.803,1.585)--(2.804,1.586)--(2.806,1.587)--(2.807,1.588)--(2.809,1.590)--(2.810,1.591)%
  --(2.811,1.592)--(2.813,1.593)--(2.814,1.594)--(2.815,1.595)--(2.816,1.596)--(2.818,1.597)%
  --(2.819,1.598)--(2.820,1.599)--(2.822,1.600)--(2.823,1.602)--(2.825,1.603)--(2.826,1.604)%
  --(2.828,1.605)--(2.829,1.607)--(2.831,1.608)--(2.832,1.609)--(2.834,1.611)--(2.836,1.612)%
  --(2.837,1.613)--(2.839,1.614)--(2.840,1.616)--(2.842,1.617)--(2.843,1.618)--(2.844,1.619)%
  --(2.846,1.620)--(2.847,1.621)--(2.848,1.622)--(2.850,1.623)--(2.851,1.624)--(2.852,1.626)%
  --(2.854,1.627)--(2.855,1.628)--(2.856,1.629)--(2.858,1.630)--(2.859,1.631)--(2.861,1.633)%
  --(2.862,1.634)--(2.864,1.635)--(2.865,1.637)--(2.867,1.638)--(2.869,1.639)--(2.870,1.641)%
  --(2.872,1.642)--(2.873,1.643)--(2.875,1.645)--(2.876,1.646)--(2.878,1.647)--(2.879,1.648)%
  --(2.880,1.649)--(2.881,1.650)--(2.883,1.651)--(2.884,1.652)--(2.885,1.654)--(2.887,1.655)%
  --(2.888,1.656)--(2.889,1.657)--(2.891,1.658)--(2.892,1.659)--(2.894,1.661)--(2.895,1.662)%
  --(2.897,1.663)--(2.898,1.665)--(2.900,1.666)--(2.901,1.667)--(2.903,1.669)--(2.905,1.670)%
  --(2.906,1.672)--(2.908,1.673)--(2.909,1.674)--(2.911,1.675)--(2.912,1.677)--(2.913,1.678)%
  --(2.915,1.679)--(2.916,1.680)--(2.917,1.681)--(2.919,1.682)--(2.920,1.683)--(2.921,1.685)%
  --(2.923,1.686)--(2.924,1.687)--(2.925,1.688)--(2.927,1.689)--(2.928,1.691)--(2.930,1.692)%
  --(2.931,1.693)--(2.933,1.695)--(2.934,1.696)--(2.936,1.697)--(2.938,1.699)--(2.939,1.700)%
  --(2.941,1.702)--(2.942,1.703)--(2.944,1.704)--(2.945,1.706)--(2.947,1.707)--(2.948,1.708)%
  --(2.949,1.709)--(2.950,1.710)--(2.952,1.711)--(2.953,1.713)--(2.954,1.714)--(2.956,1.715)%
  --(2.957,1.716)--(2.958,1.717)--(2.960,1.719)--(2.961,1.720)--(2.963,1.721)--(2.964,1.723)%
  --(2.966,1.724)--(2.967,1.725)--(2.969,1.727)--(2.970,1.728)--(2.972,1.730)--(2.974,1.731)%
  --(2.975,1.733)--(2.977,1.734)--(2.978,1.735)--(2.980,1.737)--(2.981,1.738)--(2.982,1.739)%
  --(2.984,1.740)--(2.985,1.741)--(2.986,1.743)--(2.988,1.744)--(2.989,1.745)--(2.990,1.746)%
  --(2.991,1.747)--(2.993,1.749)--(2.994,1.750)--(2.996,1.751)--(2.997,1.752)--(2.999,1.754)%
  --(3.000,1.755)--(3.002,1.757)--(3.003,1.758)--(3.005,1.760)--(3.006,1.761)--(3.008,1.762)%
  --(3.010,1.764)--(3.011,1.765)--(3.013,1.767)--(3.014,1.768)--(3.015,1.769)--(3.017,1.771)%
  --(3.018,1.772)--(3.019,1.773)--(3.021,1.774)--(3.022,1.775)--(3.023,1.777)--(3.025,1.778)%
  --(3.026,1.779)--(3.027,1.780)--(3.029,1.782)--(3.030,1.783)--(3.032,1.784)--(3.033,1.786)%
  --(3.035,1.787)--(3.036,1.789)--(3.038,1.790)--(3.039,1.792)--(3.041,1.793)--(3.043,1.795)%
  --(3.044,1.796)--(3.046,1.798)--(3.047,1.799)--(3.049,1.800)--(3.050,1.802)--(3.051,1.803)%
  --(3.053,1.804)--(3.054,1.805)--(3.055,1.807)--(3.057,1.808)--(3.058,1.809)--(3.059,1.810)%
  --(3.061,1.812)--(3.062,1.813)--(3.063,1.814)--(3.065,1.816)--(3.066,1.817)--(3.068,1.818)%
  --(3.069,1.820)--(3.071,1.821)--(3.072,1.823)--(3.074,1.824)--(3.076,1.826)--(3.077,1.827)%
  --(3.079,1.829)--(3.080,1.830)--(3.082,1.832)--(3.083,1.833)--(3.084,1.834)--(3.086,1.836)%
  --(3.087,1.837)--(3.088,1.838)--(3.090,1.840)--(3.091,1.841)--(3.092,1.842)--(3.094,1.843)%
  --(3.095,1.845)--(3.096,1.846)--(3.098,1.847)--(3.099,1.849)--(3.101,1.850)--(3.102,1.852)%
  --(3.104,1.853)--(3.105,1.855)--(3.107,1.856)--(3.108,1.858)--(3.110,1.859)--(3.112,1.861)%
  --(3.113,1.862)--(3.115,1.864)--(3.116,1.865)--(3.118,1.867)--(3.119,1.868)--(3.120,1.869)%
  --(3.122,1.871)--(3.123,1.872)--(3.124,1.873)--(3.126,1.875)--(3.127,1.876)--(3.128,1.877)%
  --(3.130,1.878)--(3.131,1.880)--(3.132,1.881)--(3.134,1.883)--(3.135,1.884)--(3.137,1.886)%
  --(3.138,1.887)--(3.140,1.889)--(3.141,1.890)--(3.143,1.892)--(3.145,1.893)--(3.146,1.895)%
  --(3.148,1.896)--(3.149,1.898)--(3.151,1.900)--(3.152,1.901)--(3.154,1.902)--(3.155,1.904)%
  --(3.156,1.905)--(3.157,1.906)--(3.159,1.908)--(3.160,1.909)--(3.161,1.910)--(3.163,1.911)%
  --(3.164,1.913)--(3.165,1.914)--(3.167,1.916)--(3.168,1.917)--(3.170,1.919)--(3.171,1.920)%
  --(3.173,1.922)--(3.174,1.923)--(3.176,1.925)--(3.177,1.926)--(3.179,1.928)--(3.181,1.930)%
  --(3.182,1.931)--(3.184,1.933)--(3.185,1.934)--(3.187,1.936)--(3.188,1.937)--(3.189,1.939)%
  --(3.191,1.940)--(3.192,1.941)--(3.193,1.943)--(3.195,1.944)--(3.196,1.945)--(3.197,1.947)%
  --(3.199,1.948)--(3.200,1.949)--(3.201,1.951)--(3.203,1.952)--(3.204,1.954)--(3.206,1.955)%
  --(3.207,1.957)--(3.209,1.958)--(3.210,1.960)--(3.212,1.962)--(3.214,1.963)--(3.215,1.965)%
  --(3.217,1.967)--(3.218,1.968)--(3.220,1.970)--(3.221,1.971)--(3.223,1.973)--(3.224,1.974)%
  --(3.225,1.975)--(3.226,1.977)--(3.228,1.978)--(3.229,1.979)--(3.230,1.981)--(3.232,1.982)%
  --(3.233,1.984)--(3.234,1.985)--(3.236,1.986)--(3.237,1.988)--(3.239,1.990)--(3.240,1.991)%
  --(3.242,1.993)--(3.243,1.994)--(3.245,1.996)--(3.247,1.998)--(3.248,1.999)--(3.250,2.001)%
  --(3.251,2.003)--(3.253,2.004)--(3.254,2.006)--(3.256,2.007)--(3.257,2.009)--(3.258,2.010)%
  --(3.260,2.012)--(3.261,2.013)--(3.262,2.014)--(3.264,2.016)--(3.265,2.017)--(3.266,2.019)%
  --(3.268,2.020)--(3.269,2.021)--(3.270,2.023)--(3.272,2.024)--(3.273,2.026)--(3.275,2.028)%
  --(3.276,2.029)--(3.278,2.031)--(3.279,2.033)--(3.281,2.034)--(3.283,2.036)--(3.284,2.038)%
  --(3.286,2.039)--(3.287,2.041)--(3.289,2.043)--(3.290,2.044)--(3.292,2.046)--(3.293,2.047)%
  --(3.294,2.048)--(3.296,2.050)--(3.297,2.051)--(3.298,2.053)--(3.299,2.054)--(3.301,2.055)%
  --(3.302,2.057)--(3.303,2.058)--(3.305,2.060)--(3.306,2.062)--(3.308,2.063)--(3.309,2.065)%
  --(3.311,2.066)--(3.312,2.068)--(3.314,2.070)--(3.316,2.072)--(3.317,2.073)--(3.319,2.075)%
  --(3.320,2.077)--(3.322,2.078)--(3.323,2.080)--(3.325,2.082)--(3.326,2.083)--(3.327,2.085)%
  --(3.329,2.086)--(3.330,2.087)--(3.331,2.089)--(3.333,2.090)--(3.334,2.092)--(3.335,2.093)%
  --(3.337,2.095)--(3.338,2.096)--(3.339,2.098)--(3.341,2.099)--(3.342,2.101)--(3.344,2.103)%
  --(3.345,2.104)--(3.347,2.106)--(3.348,2.108)--(3.350,2.109)--(3.352,2.111)--(3.353,2.113)%
  --(3.355,2.115)--(3.356,2.116)--(3.358,2.118)--(3.359,2.120)--(3.361,2.121)--(3.362,2.123)%
  --(3.363,2.124)--(3.365,2.126)--(3.366,2.127)--(3.367,2.128)--(3.368,2.130)--(3.370,2.131)%
  --(3.371,2.133)--(3.372,2.134)--(3.374,2.136)--(3.375,2.138)--(3.377,2.139)--(3.378,2.141)%
  --(3.380,2.143)--(3.381,2.145)--(3.383,2.146)--(3.385,2.148)--(3.386,2.150)--(3.388,2.152)%
  --(3.389,2.153)--(3.391,2.155)--(3.392,2.157)--(3.394,2.159)--(3.395,2.160)--(3.396,2.161)%
  --(3.398,2.163)--(3.399,2.164)--(3.400,2.166)--(3.402,2.167)--(3.403,2.169)--(3.404,2.170)%
  --(3.406,2.172)--(3.407,2.173)--(3.408,2.175)--(3.410,2.177)--(3.411,2.178)--(3.413,2.180)%
  --(3.414,2.182)--(3.416,2.184)--(3.418,2.185)--(3.419,2.187)--(3.421,2.189)--(3.422,2.191)%
  --(3.424,2.193)--(3.425,2.194)--(3.427,2.196)--(3.428,2.198)--(3.430,2.199)--(3.431,2.201)%
  --(3.432,2.202)--(3.434,2.204)--(3.435,2.205)--(3.436,2.207)--(3.437,2.208)--(3.439,2.210)%
  --(3.440,2.211)--(3.441,2.213)--(3.443,2.215)--(3.444,2.216)--(3.446,2.218)--(3.447,2.220)%
  --(3.449,2.222)--(3.451,2.224)--(3.452,2.225)--(3.454,2.227)--(3.455,2.229)--(3.457,2.231)%
  --(3.458,2.233)--(3.460,2.235)--(3.462,2.236)--(3.463,2.238)--(3.464,2.239)--(3.465,2.241)%
  --(3.467,2.243)--(3.468,2.244)--(3.469,2.246)--(3.471,2.247)--(3.472,2.249)--(3.473,2.250)%
  --(3.475,2.252)--(3.476,2.253)--(3.477,2.255)--(3.479,2.257)--(3.480,2.258)--(3.482,2.260)%
  --(3.483,2.262)--(3.485,2.264)--(3.487,2.266)--(3.488,2.268)--(3.490,2.269)--(3.491,2.271)%
  --(3.493,2.273)--(3.494,2.275)--(3.496,2.277)--(3.497,2.278)--(3.499,2.280)--(3.500,2.282)%
  --(3.501,2.283)--(3.503,2.285)--(3.504,2.286)--(3.505,2.288)--(3.506,2.289)--(3.508,2.291)%
  --(3.509,2.293)--(3.510,2.294)--(3.512,2.296)--(3.513,2.298)--(3.515,2.300)--(3.516,2.301)%
  --(3.518,2.303)--(3.520,2.305)--(3.521,2.307)--(3.523,2.309)--(3.524,2.311)--(3.526,2.313)%
  --(3.527,2.315)--(3.529,2.316)--(3.530,2.318)--(3.532,2.320)--(3.533,2.322)--(3.534,2.323)%
  --(3.536,2.325)--(3.537,2.326)--(3.538,2.328)--(3.540,2.329)--(3.541,2.331)--(3.542,2.333)%
  --(3.544,2.334)--(3.545,2.336)--(3.546,2.338)--(3.548,2.339)--(3.549,2.341)--(3.551,2.343)%
  --(3.553,2.345)--(3.554,2.347)--(3.555,2.349)--(3.557,2.351)--(3.559,2.352)--(3.560,2.354)%
  --(3.562,2.356)--(3.563,2.358)--(3.565,2.360)--(3.566,2.362)--(3.568,2.363)--(3.569,2.365)%
  --(3.570,2.367)--(3.572,2.368)--(3.573,2.370)--(3.574,2.371)--(3.575,2.373)--(3.577,2.375)%
  --(3.578,2.376)--(3.580,2.378)--(3.581,2.380)--(3.582,2.382)--(3.584,2.384)--(3.585,2.385)%
  --(3.587,2.387)--(3.588,2.389)--(3.590,2.391)--(3.592,2.393)--(3.593,2.395)--(3.595,2.397)%
  --(3.596,2.399)--(3.598,2.401)--(3.599,2.403)--(3.601,2.405)--(3.602,2.406)--(3.604,2.408)%
  --(3.605,2.409)--(3.606,2.411)--(3.607,2.413)--(3.609,2.414)--(3.610,2.416)--(3.611,2.418)%
  --(3.613,2.419)--(3.614,2.421)--(3.615,2.423)--(3.617,2.425)--(3.618,2.426)--(3.620,2.428)%
  --(3.621,2.430)--(3.623,2.432)--(3.625,2.434)--(3.626,2.436)--(3.628,2.438)--(3.629,2.440)%
  --(3.631,2.442)--(3.632,2.444)--(3.634,2.446)--(3.635,2.448)--(3.637,2.449)--(3.638,2.451)%
  --(3.639,2.453)--(3.641,2.454)--(3.642,2.456)--(3.643,2.458)--(3.645,2.459)--(3.646,2.461)%
  --(3.647,2.463)--(3.649,2.464)--(3.650,2.466)--(3.651,2.468)--(3.653,2.470)--(3.655,2.472)%
  --(3.656,2.474)--(3.658,2.476)--(3.659,2.478)--(3.661,2.480)--(3.662,2.482)--(3.664,2.484)%
  --(3.665,2.486)--(3.667,2.488)--(3.669,2.490)--(3.670,2.492)--(3.671,2.493)--(3.672,2.495)%
  --(3.674,2.497)--(3.675,2.499)--(3.676,2.500)--(3.678,2.502)--(3.679,2.503)--(3.680,2.505)%
  --(3.682,2.507)--(3.683,2.509)--(3.685,2.510)--(3.686,2.512)--(3.687,2.514)--(3.689,2.516)%
  --(3.690,2.518)--(3.692,2.520)--(3.694,2.522)--(3.695,2.524)--(3.697,2.526)--(3.698,2.528)%
  --(3.700,2.530)--(3.702,2.532)--(3.703,2.534)--(3.704,2.536)--(3.706,2.538)--(3.707,2.540)%
  --(3.708,2.541)--(3.710,2.543)--(3.711,2.545)--(3.712,2.547)--(3.713,2.548)--(3.715,2.550)%
  --(3.716,2.552)--(3.718,2.554)--(3.719,2.555)--(3.720,2.557)--(3.722,2.559)--(3.723,2.561)%
  --(3.725,2.563)--(3.727,2.565)--(3.728,2.567)--(3.730,2.570)--(3.731,2.572)--(3.733,2.574)%
  --(3.734,2.576)--(3.736,2.578)--(3.738,2.580)--(3.739,2.582)--(3.740,2.583)--(3.741,2.585)%
  --(3.743,2.587)--(3.744,2.589)--(3.746,2.590)--(3.747,2.592)--(3.748,2.594)--(3.749,2.595)%
  --(3.751,2.597)--(3.752,2.599)--(3.753,2.601)--(3.755,2.603)--(3.757,2.605)--(3.758,2.607)%
  --(3.760,2.609)--(3.761,2.611)--(3.763,2.613)--(3.764,2.615)--(3.766,2.617)--(3.767,2.619)%
  --(3.769,2.621)--(3.771,2.623)--(3.772,2.626)--(3.773,2.627)--(3.775,2.629)--(3.776,2.631)%
  --(3.777,2.633)--(3.779,2.634)--(3.780,2.636)--(3.781,2.638)--(3.782,2.640)--(3.784,2.641)%
  --(3.785,2.643)--(3.787,2.645)--(3.788,2.647)--(3.790,2.649)--(3.791,2.651)--(3.792,2.653)%
  --(3.794,2.655)--(3.796,2.657)--(3.797,2.659)--(3.799,2.662)--(3.800,2.664)--(3.802,2.666)%
  --(3.804,2.668)--(3.805,2.670)--(3.807,2.672)--(3.808,2.674)--(3.809,2.676)--(3.811,2.678)%
  --(3.812,2.679)--(3.813,2.681)--(3.814,2.683)--(3.816,2.685)--(3.817,2.686)--(3.818,2.688)%
  --(3.820,2.690)--(3.821,2.692)--(3.822,2.694)--(3.824,2.696)--(3.825,2.698)--(3.827,2.700)%
  --(3.829,2.702)--(3.830,2.704)--(3.832,2.706)--(3.833,2.708)--(3.835,2.711)--(3.836,2.713)%
  --(3.838,2.715)--(3.839,2.717)--(3.841,2.719)--(3.842,2.721)--(3.844,2.723)--(3.845,2.725)%
  --(3.846,2.727)--(3.848,2.728)--(3.849,2.730)--(3.850,2.732)--(3.852,2.734)--(3.853,2.736)%
  --(3.854,2.737)--(3.856,2.739)--(3.857,2.741)--(3.859,2.743)--(3.860,2.745)--(3.862,2.748)%
  --(3.863,2.750)--(3.865,2.752)--(3.866,2.754)--(3.868,2.756)--(3.869,2.758)--(3.871,2.761)%
  --(3.872,2.763)--(3.874,2.765)--(3.876,2.767)--(3.877,2.769)--(3.878,2.771)--(3.880,2.773)%
  --(3.881,2.774)--(3.882,2.776)--(3.883,2.778)--(3.885,2.780)--(3.886,2.782)--(3.887,2.784)%
  --(3.889,2.786)--(3.890,2.787)--(3.892,2.789)--(3.893,2.791)--(3.894,2.794)--(3.896,2.796)%
  --(3.898,2.798)--(3.899,2.800)--(3.901,2.802)--(3.902,2.804)--(3.904,2.807)--(3.905,2.809)%
  --(3.907,2.811)--(3.909,2.813)--(3.910,2.815)--(3.911,2.817)--(3.913,2.819)--(3.914,2.821)%
  --(3.915,2.823)--(3.917,2.825)--(3.918,2.827)--(3.919,2.828)--(3.921,2.830)--(3.922,2.832)%
  --(3.923,2.834)--(3.925,2.836)--(3.926,2.838)--(3.927,2.840)--(3.929,2.842)--(3.931,2.845)%
  --(3.932,2.847)--(3.934,2.849)--(3.935,2.851)--(3.937,2.854)--(3.938,2.856)--(3.940,2.858)%
  --(3.941,2.860)--(3.943,2.862)--(3.944,2.865)--(3.946,2.867)--(3.947,2.868)--(3.948,2.870)%
  --(3.950,2.872)--(3.951,2.874)--(3.953,2.876)--(3.954,2.878)--(3.955,2.880)--(3.956,2.882)%
  --(3.958,2.883)--(3.959,2.885)--(3.960,2.887)--(3.962,2.890)--(3.964,2.892)--(3.965,2.894)%
  --(3.967,2.896)--(3.968,2.898)--(3.970,2.901)--(3.971,2.903)--(3.973,2.905)--(3.974,2.907)%
  --(3.976,2.910)--(3.977,2.912)--(3.979,2.914)--(3.980,2.916)--(3.982,2.918)--(3.983,2.920)%
  --(3.984,2.922)--(3.986,2.924)--(3.987,2.926)--(3.988,2.928)--(3.990,2.929)--(3.991,2.931)%
  --(3.992,2.933)--(3.994,2.935)--(3.995,2.937)--(3.997,2.940)--(3.998,2.942)--(3.999,2.944)%
  --(4.001,2.946)--(4.003,2.949)--(4.004,2.951)--(4.006,2.953)--(4.007,2.956)--(4.009,2.958)%
  --(4.010,2.960)--(4.012,2.962)--(4.014,2.965)--(4.015,2.967)--(4.016,2.969)--(4.018,2.971)%
  --(4.019,2.972)--(4.020,2.974)--(4.021,2.976)--(4.023,2.978)--(4.024,2.980)--(4.025,2.982)%
  --(4.027,2.984)--(4.028,2.986)--(4.029,2.988)--(4.031,2.990)--(4.032,2.993)--(4.034,2.995)%
  --(4.036,2.997)--(4.037,2.999)--(4.039,3.002)--(4.040,3.004)--(4.042,3.006)--(4.043,3.009)%
  --(4.045,3.011)--(4.046,3.013)--(4.048,3.016)--(4.049,3.018)--(4.051,3.019)--(4.052,3.021)%
  --(4.053,3.023)--(4.055,3.025)--(4.056,3.027)--(4.057,3.029)--(4.058,3.031)--(4.060,3.033)%
  --(4.061,3.035)--(4.063,3.037)--(4.064,3.039)--(4.065,3.042)--(4.067,3.044)--(4.069,3.046)%
  --(4.070,3.049)--(4.072,3.051)--(4.073,3.053)--(4.075,3.056)--(4.076,3.058)--(4.078,3.060)%
  --(4.079,3.063)--(4.081,3.065)--(4.082,3.067)--(4.084,3.069)--(4.085,3.071)--(4.086,3.073)%
  --(4.088,3.075)--(4.089,3.077)--(4.090,3.079)--(4.092,3.081)--(4.093,3.083)--(4.094,3.085)%
  --(4.096,3.087)--(4.097,3.089)--(4.098,3.091)--(4.100,3.094)--(4.102,3.096)--(4.103,3.098)%
  --(4.104,3.101)--(4.106,3.103)--(4.108,3.105)--(4.109,3.108)--(4.111,3.110)--(4.112,3.112)%
  --(4.114,3.115)--(4.115,3.117)--(4.117,3.120)--(4.118,3.122)--(4.120,3.124)--(4.121,3.126)%
  --(4.122,3.128)--(4.124,3.130)--(4.125,3.132)--(4.126,3.134)--(4.127,3.136)--(4.129,3.138)%
  --(4.130,3.140)--(4.131,3.142)--(4.133,3.144)--(4.134,3.146)--(4.136,3.149)--(4.137,3.151)%
  --(4.139,3.153)--(4.141,3.156)--(4.142,3.158)--(4.144,3.161)--(4.145,3.163)--(4.147,3.165)%
  --(4.148,3.168)--(4.150,3.170)--(4.151,3.173)--(4.153,3.175)--(4.154,3.177)--(4.155,3.179)%
  --(4.157,3.181)--(4.158,3.183)--(4.159,3.185)--(4.161,3.187)--(4.162,3.189)--(4.163,3.191)%
  --(4.165,3.193)--(4.166,3.195)--(4.167,3.197)--(4.169,3.200)--(4.170,3.202)--(4.172,3.204)%
  --(4.174,3.207)--(4.175,3.209)--(4.177,3.212)--(4.178,3.214)--(4.180,3.216)--(4.181,3.219)%
  --(4.183,3.221)--(4.184,3.224)--(4.186,3.226)--(4.187,3.228)--(4.188,3.230)--(4.190,3.232)%
  --(4.191,3.234)--(4.192,3.236)--(4.194,3.238)--(4.195,3.240)--(4.196,3.243)--(4.198,3.245)%
  --(4.199,3.247)--(4.200,3.249)--(4.202,3.251)--(4.203,3.254)--(4.205,3.256)--(4.207,3.258)%
  --(4.208,3.261)--(4.209,3.263)--(4.211,3.266)--(4.213,3.268)--(4.214,3.271)--(4.216,3.273)%
  --(4.217,3.276)--(4.219,3.278)--(4.220,3.280)--(4.222,3.283)--(4.223,3.285)--(4.224,3.287)%
  --(4.226,3.289)--(4.227,3.291)--(4.228,3.293)--(4.229,3.295)--(4.231,3.297)--(4.232,3.299)%
  --(4.233,3.301)--(4.235,3.303)--(4.236,3.306)--(4.238,3.308)--(4.239,3.311)--(4.241,3.313)%
  --(4.242,3.315)--(4.244,3.318)--(4.246,3.320)--(4.247,3.323)--(4.249,3.325)--(4.250,3.328)%
  --(4.252,3.330)--(4.253,3.333)--(4.255,3.335)--(4.256,3.337)--(4.258,3.340)--(4.259,3.342)%
  --(4.260,3.344)--(4.261,3.346)--(4.263,3.348)--(4.264,3.350)--(4.265,3.352)--(4.267,3.354)%
  --(4.268,3.357)--(4.269,3.359)--(4.271,3.361)--(4.272,3.363)--(4.274,3.366)--(4.275,3.368)%
  --(4.277,3.371)--(4.278,3.373)--(4.280,3.376)--(4.282,3.378)--(4.283,3.381)--(4.285,3.383)%
  --(4.286,3.386)--(4.288,3.388)--(4.289,3.391)--(4.291,3.393)--(4.292,3.395)--(4.293,3.397)%
  --(4.295,3.399)--(4.296,3.402)--(4.297,3.404)--(4.299,3.406)--(4.300,3.408)--(4.301,3.410)%
  --(4.303,3.412)--(4.304,3.415)--(4.305,3.417)--(4.307,3.419)--(4.308,3.422)--(4.310,3.424)%
  --(4.311,3.427)--(4.313,3.429)--(4.314,3.432)--(4.316,3.434)--(4.318,3.437)--(4.319,3.439)%
  --(4.321,3.442)--(4.322,3.444)--(4.324,3.447)--(4.325,3.449)--(4.326,3.451)--(4.328,3.453)%
  --(4.329,3.456)--(4.330,3.458)--(4.332,3.460)--(4.333,3.462)--(4.334,3.464)--(4.336,3.467)%
  --(4.337,3.469)--(4.338,3.471)--(4.340,3.473)--(4.341,3.476)--(4.343,3.478)--(4.344,3.481)%
  --(4.346,3.483)--(4.347,3.486)--(4.349,3.489)--(4.351,3.491)--(4.352,3.494)--(4.354,3.496)%
  --(4.355,3.499)--(4.357,3.501)--(4.358,3.504)--(4.360,3.506)--(4.361,3.508)--(4.362,3.511)%
  --(4.364,3.513)--(4.365,3.515)--(4.366,3.517)--(4.367,3.519)--(4.369,3.521)--(4.370,3.524)%
  --(4.371,3.526)--(4.373,3.528)--(4.374,3.530)--(4.376,3.533)--(4.377,3.535)--(4.379,3.538)%
  --(4.380,3.541)--(4.382,3.543)--(4.384,3.546)--(4.385,3.548)--(4.387,3.551)--(4.388,3.554)%
  --(4.390,3.556)--(4.391,3.559)--(4.393,3.561)--(4.394,3.564)--(4.395,3.566)--(4.397,3.568)%
  --(4.398,3.570)--(4.399,3.572)--(4.401,3.575)--(4.402,3.577)--(4.403,3.579)--(4.405,3.581)%
  --(4.406,3.584)--(4.407,3.586)--(4.409,3.588)--(4.410,3.591)--(4.412,3.593)--(4.413,3.596)%
  --(4.415,3.599)--(4.416,3.601)--(4.418,3.604)--(4.419,3.606)--(4.421,3.609)--(4.423,3.612)%
  --(4.424,3.614)--(4.426,3.617)--(4.427,3.620)--(4.429,3.622)--(4.430,3.624)--(4.431,3.626)%
  --(4.433,3.629)--(4.434,3.631)--(4.435,3.633)--(4.436,3.635)--(4.438,3.637)--(4.439,3.640)%
  --(4.441,3.642)--(4.442,3.644)--(4.443,3.647)--(4.445,3.649)--(4.446,3.652)--(4.448,3.654)%
  --(4.449,3.657)--(4.451,3.660)--(4.452,3.662)--(4.454,3.665)--(4.456,3.668)--(4.457,3.670)%
  --(4.459,3.673)--(4.460,3.676)--(4.462,3.678)--(4.463,3.681)--(4.464,3.683)--(4.466,3.685)%
  --(4.467,3.687)--(4.468,3.690)--(4.470,3.692)--(4.471,3.694)--(4.472,3.696)--(4.474,3.699)%
  --(4.475,3.701)--(4.476,3.703)--(4.478,3.706)--(4.479,3.708)--(4.481,3.711)--(4.482,3.714)%
  --(4.484,3.716)--(4.485,3.719)--(4.487,3.722)--(4.489,3.724)--(4.490,3.727)--(4.492,3.730)%
  --(4.493,3.733)--(4.495,3.735)--(4.496,3.738)--(4.497,3.740)--(4.499,3.742)--(4.500,3.745)%
  --(4.502,3.747)--(4.503,3.749)--(4.504,3.752)--(4.505,3.754)--(4.507,3.756)--(4.508,3.758)%
  --(4.509,3.761)--(4.511,3.763)--(4.512,3.765)--(4.514,3.768)--(4.515,3.771)--(4.517,3.773)%
  --(4.518,3.776)--(4.520,3.779)--(4.521,3.782)--(4.523,3.784)--(4.524,3.787)--(4.526,3.790)%
  --(4.528,3.792)--(4.529,3.795)--(4.531,3.798)--(4.532,3.800)--(4.533,3.802)--(4.535,3.805)%
  --(4.536,3.807)--(4.537,3.809)--(4.538,3.812)--(4.540,3.814)--(4.541,3.816)--(4.543,3.819)%
  --(4.544,3.821)--(4.545,3.823)--(4.547,3.826)--(4.548,3.829)--(4.550,3.831)--(4.551,3.834)%
  --(4.553,3.837)--(4.554,3.839)--(4.556,3.842)--(4.557,3.845)--(4.559,3.848)--(4.561,3.850)%
  --(4.562,3.853)--(4.564,3.856)--(4.565,3.859)--(4.567,3.861)--(4.568,3.863)--(4.569,3.866)%
  --(4.571,3.868)--(4.572,3.870)--(4.573,3.873)--(4.574,3.875)--(4.576,3.877)--(4.577,3.880)%
  --(4.578,3.882)--(4.580,3.885)--(4.581,3.887)--(4.583,3.890)--(4.584,3.892)--(4.586,3.895)%
  --(4.587,3.898)--(4.589,3.901)--(4.590,3.903)--(4.592,3.906)--(4.594,3.909)--(4.595,3.912)%
  --(4.597,3.914)--(4.598,3.917)--(4.600,3.920)--(4.601,3.922)--(4.602,3.925)--(4.604,3.927)%
  --(4.605,3.929)--(4.606,3.932)--(4.608,3.934)--(4.609,3.936)--(4.610,3.939)--(4.611,3.941)%
  --(4.613,3.944)--(4.614,3.946)--(4.616,3.949)--(4.617,3.951)--(4.619,3.954)--(4.620,3.957)%
  --(4.622,3.960)--(4.623,3.962)--(4.625,3.965)--(4.626,3.968)--(4.628,3.971)--(4.629,3.974)%
  --(4.631,3.977)--(4.633,3.979)--(4.634,3.982)--(4.635,3.985)--(4.637,3.987)--(4.638,3.989)%
  --(4.639,3.992)--(4.641,3.994)--(4.642,3.996)--(4.643,3.999)--(4.645,4.001)--(4.646,4.004)%
  --(4.647,4.006)--(4.649,4.008)--(4.650,4.011)--(4.652,4.014)--(4.653,4.017)--(4.655,4.019)%
  --(4.656,4.022)--(4.658,4.025)--(4.659,4.028)--(4.661,4.031)--(4.662,4.033)--(4.664,4.036)%
  --(4.666,4.039)--(4.667,4.042)--(4.669,4.045)--(4.670,4.047)--(4.671,4.049)--(4.673,4.052)%
  --(4.674,4.054)--(4.675,4.057)--(4.676,4.059)--(4.678,4.061)--(4.679,4.064)--(4.680,4.066)%
  --(4.682,4.069)--(4.683,4.071)--(4.685,4.074)--(4.686,4.077)--(4.687,4.079)--(4.689,4.082)%
  --(4.691,4.085)--(4.692,4.088)--(4.694,4.091)--(4.695,4.094)--(4.697,4.097)--(4.699,4.100)%
  --(4.700,4.102)--(4.702,4.105)--(4.703,4.108)--(4.704,4.111)--(4.706,4.113)--(4.707,4.116)%
  --(4.709,4.118)--(4.710,4.120)--(4.711,4.123)--(4.712,4.125)--(4.714,4.128)--(4.715,4.130)%
  --(4.716,4.133)--(4.718,4.135)--(4.719,4.138)--(4.721,4.141)--(4.722,4.143)--(4.724,4.146)%
  --(4.725,4.149)--(4.727,4.152)--(4.728,4.155)--(4.730,4.158)--(4.731,4.160)--(4.733,4.163)%
  --(4.734,4.166)--(4.736,4.169)--(4.738,4.172)--(4.739,4.174)--(4.740,4.177)--(4.741,4.179)%
  --(4.743,4.182)--(4.744,4.184)--(4.746,4.187)--(4.747,4.189)--(4.748,4.192)--(4.749,4.194)%
  --(4.751,4.197)--(4.752,4.199)--(4.753,4.202)--(4.755,4.205)--(4.757,4.208)--(4.758,4.210)%
  --(4.760,4.213)--(4.761,4.216)--(4.763,4.219)--(4.764,4.222)--(4.766,4.225)--(4.767,4.228)%
  --(4.769,4.231)--(4.771,4.234)--(4.772,4.237)--(4.774,4.239)--(4.775,4.242)--(4.776,4.244)%
  --(4.777,4.247)--(4.779,4.249)--(4.780,4.252)--(4.781,4.254)--(4.783,4.257)--(4.784,4.259)%
  --(4.785,4.262)--(4.787,4.264)--(4.788,4.267)--(4.790,4.270)--(4.791,4.273)--(4.793,4.276)%
  --(4.794,4.279)--(4.796,4.281)--(4.797,4.284)--(4.799,4.287)--(4.800,4.290)--(4.802,4.293)%
  --(4.803,4.296)--(4.805,4.299)--(4.807,4.302)--(4.808,4.304)--(4.809,4.307)--(4.811,4.309)%
  --(4.812,4.312)--(4.813,4.314)--(4.814,4.317)--(4.816,4.319)--(4.817,4.322)--(4.818,4.324)%
  --(4.820,4.327)--(4.821,4.330)--(4.822,4.332)--(4.824,4.335)--(4.825,4.338)--(4.827,4.341)%
  --(4.829,4.344)--(4.830,4.347)--(4.832,4.350)--(4.833,4.353)--(4.835,4.356)--(4.836,4.359)%
  --(4.838,4.362)--(4.839,4.365)--(4.841,4.368)--(4.842,4.371)--(4.844,4.373)--(4.845,4.376)%
  --(4.846,4.378)--(4.848,4.381)--(4.849,4.383)--(4.850,4.386)--(4.852,4.388)--(4.853,4.391)%
  --(4.854,4.394)--(4.856,4.396)--(4.857,4.399)--(4.859,4.402)--(4.860,4.405)--(4.862,4.408)%
  --(4.863,4.411)--(4.865,4.414)--(4.866,4.417)--(4.868,4.419)--(4.869,4.422)--(4.871,4.425)%
  --(4.872,4.428)--(4.874,4.431)--(4.875,4.434)--(4.877,4.437)--(4.878,4.439)--(4.879,4.442)%
  --(4.881,4.445)--(4.882,4.447)--(4.883,4.450)--(4.885,4.452)--(4.886,4.455)--(4.887,4.457)%
  --(4.889,4.460)--(4.890,4.463)--(4.891,4.465)--(4.893,4.468)--(4.894,4.471)--(4.896,4.474)%
  --(4.897,4.477)--(4.899,4.480)--(4.901,4.483)--(4.902,4.487)--(4.904,4.490)--(4.905,4.493)%
  --(4.907,4.496)--(4.909,4.499)--(4.910,4.502)--(4.912,4.505)--(4.913,4.507)--(4.914,4.510)%
  --(4.915,4.512)--(4.917,4.515)--(4.918,4.517)--(4.919,4.520)--(4.921,4.523)--(4.922,4.525)%
  --(4.923,4.528)--(4.925,4.531)--(4.926,4.533)--(4.928,4.536)--(4.929,4.539)--(4.931,4.542)%
  --(4.932,4.545)--(4.934,4.548)--(4.935,4.551)--(4.937,4.554)--(4.938,4.557)--(4.940,4.560)%
  --(4.941,4.563)--(4.943,4.566)--(4.944,4.569)--(4.946,4.572)--(4.947,4.575)--(4.948,4.577)%
  --(4.950,4.580)--(4.951,4.582)--(4.952,4.585)--(4.954,4.588)--(4.955,4.590)--(4.956,4.593)%
  --(4.958,4.596)--(4.959,4.598)--(4.960,4.601)--(4.962,4.604)--(4.963,4.607)--(4.965,4.610)%
  --(4.966,4.613)--(4.968,4.616)--(4.970,4.620)--(4.971,4.623)--(4.973,4.626)--(4.974,4.629)%
  --(4.976,4.632)--(4.978,4.635)--(4.979,4.639)--(4.980,4.641)--(4.982,4.644)--(4.983,4.646)%
  --(4.984,4.649)--(4.986,4.652)--(4.987,4.654)--(4.988,4.657)--(4.990,4.659)--(4.990,4.661)%
  --(4.991,4.662)--(4.991,4.663)--(4.992,4.664)--(4.992,4.665)--(4.993,4.666)--(4.993,4.667)%
  --(4.994,4.668)--(4.995,4.671)--(4.996,4.673)--(4.998,4.675)--(4.999,4.678)--(5.000,4.680)%
  --(5.001,4.683)--(5.002,4.685)--(5.004,4.688)--(5.004,4.689)--(5.005,4.689)--(5.005,4.690)%
  --(5.005,4.691)--(5.006,4.692)--(5.007,4.693)--(5.007,4.695)--(5.008,4.696)--(5.009,4.698)%
  --(5.010,4.700)--(5.011,4.702)--(5.012,4.704)--(5.013,4.707)--(5.014,4.709)--(5.015,4.711)%
  --(5.016,4.713)--(5.017,4.715)--(5.019,4.718)--(5.020,4.720)--(5.021,4.722)--(5.022,4.725)%
  --(5.023,4.727)--(5.025,4.730)--(5.026,4.732)--(5.027,4.735)--(5.029,4.738)--(5.030,4.740)%
  --(5.031,4.743)--(5.033,4.746)--(5.034,4.748)--(5.035,4.750)--(5.036,4.753)--(5.037,4.755)%
  --(5.038,4.757)--(5.039,4.759)--(5.041,4.762)--(5.042,4.764)--(5.043,4.766)--(5.044,4.768)%
  --(5.045,4.770)--(5.046,4.772)--(5.047,4.774)--(5.048,4.776)--(5.049,4.779)--(5.050,4.781)%
  --(5.051,4.783)--(5.052,4.785)--(5.053,4.787)--(5.055,4.789)--(5.056,4.792)--(5.057,4.794)%
  --(5.058,4.797)--(5.059,4.799)--(5.061,4.802)--(5.063,4.805)--(5.064,4.808)--(5.066,4.811)%
  --(5.067,4.813)--(5.068,4.815)--(5.069,4.818)--(5.070,4.820)--(5.071,4.822)--(5.073,4.824)%
  --(5.074,4.827)--(5.075,4.829)--(5.076,4.831)--(5.077,4.833)--(5.078,4.835)--(5.079,4.837)%
  --(5.080,4.840)--(5.081,4.842)--(5.082,4.844)--(5.083,4.846)--(5.085,4.848)--(5.086,4.850)%
  --(5.087,4.852)--(5.088,4.855)--(5.089,4.857)--(5.090,4.859)--(5.091,4.862)--(5.092,4.864)%
  --(5.094,4.867)--(5.095,4.869)--(5.097,4.872)--(5.098,4.875)--(5.099,4.877)--(5.100,4.879)%
  --(5.102,4.882)--(5.103,4.884)--(5.104,4.886)--(5.105,4.889)--(5.106,4.891)--(5.108,4.893)%
  --(5.109,4.896)--(5.110,4.898)--(5.111,4.900)--(5.112,4.902)--(5.113,4.904)--(5.114,4.906)%
  --(5.115,4.908)--(5.116,4.910)--(5.117,4.912)--(5.119,4.914)--(5.120,4.917)--(5.121,4.919)%
  --(5.122,4.921)--(5.123,4.923)--(5.124,4.925)--(5.125,4.927)--(5.126,4.930)--(5.128,4.932)%
  --(5.129,4.935)--(5.130,4.937)--(5.132,4.940)--(5.133,4.942)--(5.134,4.945)--(5.136,4.948)%
  --(5.137,4.950)--(5.138,4.953)--(5.140,4.955)--(5.141,4.958)--(5.142,4.960)--(5.143,4.962)%
  --(5.144,4.965)--(5.146,4.967)--(5.147,4.969)--(5.148,4.971)--(5.149,4.973)--(5.150,4.975)%
  --(5.151,4.977)--(5.152,4.979)--(5.153,4.981)--(5.154,4.983)--(5.155,4.985)--(5.156,4.988)%
  --(5.157,4.990)--(5.159,4.992)--(5.160,4.994)--(5.161,4.997)--(5.162,4.999)--(5.164,5.002)%
  --(5.165,5.005)--(5.167,5.008)--(5.168,5.011)--(5.170,5.014)--(5.171,5.017)--(5.173,5.020)%
  --(5.174,5.022)--(5.176,5.025)--(5.177,5.027)--(5.178,5.029)--(5.179,5.031)--(5.180,5.033)%
  --(5.181,5.035)--(5.182,5.037)--(5.183,5.039)--(5.184,5.041)--(5.185,5.044)--(5.187,5.046)%
  --(5.188,5.048)--(5.189,5.050)--(5.190,5.052)--(5.191,5.054)--(5.192,5.056)--(5.193,5.058)%
  --(5.194,5.060)--(5.196,5.063)--(5.197,5.065)--(5.198,5.067)--(5.200,5.071)--(5.201,5.074)%
  --(5.203,5.077)--(5.205,5.080)--(5.206,5.083)--(5.208,5.085)--(5.209,5.088)--(5.210,5.090)%
  --(5.211,5.093)--(5.212,5.095)--(5.214,5.097)--(5.215,5.099)--(5.216,5.101)--(5.217,5.103)%
  --(5.218,5.105)--(5.219,5.107)--(5.220,5.109)--(5.221,5.111)--(5.222,5.113)--(5.223,5.115)%
  --(5.225,5.117)--(5.226,5.120)--(5.227,5.122)--(5.228,5.124)--(5.229,5.126)--(5.230,5.128)%
  --(5.232,5.131)--(5.233,5.133)--(5.234,5.136)--(5.236,5.138)--(5.237,5.141)--(5.238,5.144)%
  --(5.240,5.146)--(5.241,5.149)--(5.242,5.151)--(5.244,5.154)--(5.245,5.156)--(5.246,5.158)%
  --(5.247,5.160)--(5.248,5.162)--(5.249,5.164)--(5.251,5.166)--(5.252,5.168)--(5.253,5.170)%
  --(5.254,5.172)--(5.255,5.174)--(5.256,5.176)--(5.257,5.178)--(5.258,5.180)--(5.259,5.182)%
  --(5.260,5.184)--(5.261,5.187)--(5.263,5.189)--(5.264,5.191)--(5.265,5.193)--(5.266,5.195)%
  --(5.268,5.198)--(5.269,5.201)--(5.271,5.204)--(5.272,5.207)--(5.273,5.209)--(5.275,5.211)%
  --(5.276,5.213)--(5.277,5.215)--(5.278,5.217)--(5.279,5.219)--(5.280,5.222)--(5.281,5.224)%
  --(5.282,5.226)--(5.283,5.228)--(5.285,5.230)--(5.286,5.232)--(5.287,5.234)--(5.288,5.236)%
  --(5.289,5.238)--(5.290,5.240)--(5.291,5.242)--(5.292,5.244)--(5.293,5.246)--(5.294,5.248)%
  --(5.296,5.250)--(5.297,5.252)--(5.298,5.254)--(5.299,5.256)--(5.300,5.259)--(5.302,5.262)%
  --(5.303,5.264)--(5.304,5.267)--(5.306,5.269)--(5.307,5.271)--(5.308,5.273)--(5.309,5.276)%
  --(5.311,5.278)--(5.312,5.280)--(5.313,5.282)--(5.314,5.285)--(5.315,5.287)--(5.316,5.289)%
  --(5.317,5.291)--(5.319,5.293)--(5.320,5.295)--(5.321,5.296)--(5.322,5.298)--(5.323,5.300)%
  --(5.324,5.302)--(5.325,5.304)--(5.326,5.306)--(5.327,5.308)--(5.328,5.310)--(5.330,5.312)%
  --(5.331,5.314)--(5.332,5.316)--(5.333,5.319)--(5.334,5.321)--(5.336,5.323)--(5.337,5.326)%
  --(5.338,5.328)--(5.340,5.331)--(5.341,5.333)--(5.342,5.335)--(5.344,5.338)--(5.345,5.340)%
  --(5.346,5.343)--(5.348,5.345)--(5.349,5.347)--(5.350,5.349)--(5.351,5.351)--(5.352,5.353)%
  --(5.353,5.355)--(5.354,5.357)--(5.355,5.359)--(5.356,5.361)--(5.357,5.363)--(5.358,5.365)%
  --(5.360,5.367)--(5.361,5.369)--(5.362,5.371)--(5.363,5.373)--(5.364,5.375)--(5.365,5.377)%
  --(5.366,5.379)--(5.368,5.382)--(5.369,5.384)--(5.370,5.386)--(5.372,5.389)--(5.373,5.392)%
  --(5.375,5.395)--(5.376,5.397)--(5.378,5.400)--(5.379,5.403)--(5.381,5.405)--(5.382,5.408)%
  --(5.383,5.410)--(5.384,5.412)--(5.385,5.414)--(5.387,5.416)--(5.388,5.418)--(5.389,5.420)%
  --(5.390,5.422)--(5.391,5.423)--(5.392,5.425)--(5.393,5.427)--(5.394,5.429)--(5.395,5.431)%
  --(5.396,5.433)--(5.397,5.435)--(5.398,5.437)--(5.399,5.439)--(5.401,5.441)--(5.402,5.443)%
  --(5.403,5.446)--(5.404,5.448)--(5.406,5.450)--(5.408,5.453)--(5.409,5.456)--(5.411,5.459)%
  --(5.412,5.461)--(5.413,5.463)--(5.414,5.465)--(5.415,5.467)--(5.416,5.469)--(5.417,5.471)%
  --(5.419,5.473)--(5.420,5.475)--(5.421,5.477)--(5.422,5.479)--(5.423,5.481)--(5.424,5.483)%
  --(5.425,5.484)--(5.426,5.486)--(5.427,5.488)--(5.428,5.490)--(5.429,5.492)--(5.431,5.494)%
  --(5.432,5.496)--(5.433,5.498)--(5.434,5.500)--(5.435,5.502)--(5.436,5.504)--(5.437,5.506)%
  --(5.439,5.509)--(5.440,5.511)--(5.442,5.514)--(5.443,5.516)--(5.444,5.518)--(5.445,5.521)%
  --(5.446,5.523)--(5.448,5.525)--(5.449,5.527)--(5.450,5.529)--(5.451,5.531)--(5.452,5.533)%
  --(5.453,5.535)--(5.455,5.537)--(5.456,5.539)--(5.457,5.541)--(5.458,5.543)--(5.459,5.545)%
  --(5.460,5.546)--(5.461,5.548)--(5.462,5.550)--(5.463,5.552)--(5.464,5.554)--(5.465,5.556)%
  --(5.467,5.558)--(5.468,5.560)--(5.469,5.562)--(5.470,5.564)--(5.471,5.566)--(5.473,5.568)%
  --(5.474,5.570)--(5.475,5.573)--(5.476,5.575)--(5.478,5.577)--(5.479,5.580)--(5.480,5.582)%
  --(5.482,5.584)--(5.483,5.587)--(5.485,5.589)--(5.486,5.592)--(5.487,5.594)--(5.488,5.595)%
  --(5.489,5.597)--(5.490,5.599)--(5.492,5.601)--(5.493,5.603)--(5.494,5.605)--(5.495,5.607)%
  --(5.496,5.609)--(5.497,5.611)--(5.498,5.613)--(5.499,5.614)--(5.500,5.616)--(5.501,5.618)%
  --(5.502,5.620)--(5.504,5.622)--(5.505,5.624)--(5.506,5.627)--(5.507,5.629)--(5.509,5.631)%
  --(5.510,5.634)--(5.512,5.637)--(5.513,5.640)--(5.515,5.642)--(5.517,5.645)--(5.518,5.647)%
  --(5.519,5.650)--(5.521,5.652)--(5.522,5.654)--(5.523,5.656)--(5.524,5.658)--(5.525,5.660)%
  --(5.526,5.662)--(5.527,5.663)--(5.528,5.665)--(5.529,5.667)--(5.530,5.669)--(5.531,5.671)%
  --(5.533,5.673)--(5.534,5.674)--(5.535,5.676)--(5.536,5.678)--(5.537,5.680)--(5.538,5.682)%
  --(5.540,5.684)--(5.541,5.686)--(5.542,5.688)--(5.543,5.690)--(5.545,5.693)--(5.546,5.695)%
  --(5.547,5.697)--(5.549,5.700)--(5.550,5.702)--(5.551,5.704)--(5.553,5.707)--(5.554,5.709)%
  --(5.555,5.711)--(5.556,5.713)--(5.558,5.715)--(5.559,5.717)--(5.560,5.719)--(5.561,5.720)%
  --(5.562,5.722)--(5.563,5.724)--(5.564,5.726)--(5.565,5.728)--(5.566,5.730)--(5.567,5.731)%
  --(5.568,5.733)--(5.570,5.735)--(5.571,5.737)--(5.572,5.739)--(5.573,5.741)--(5.574,5.743)%
  --(5.575,5.745)--(5.577,5.747)--(5.578,5.750)--(5.580,5.753)--(5.581,5.755)--(5.583,5.758)%
  --(5.584,5.760)--(5.585,5.762)--(5.586,5.763)--(5.587,5.765)--(5.588,5.767)--(5.590,5.769)%
  --(5.591,5.771)--(5.592,5.773)--(5.593,5.775)--(5.594,5.777)--(5.595,5.779)--(5.596,5.780)%
  --(5.597,5.782)--(5.598,5.784)--(5.599,5.786)--(5.600,5.788)--(5.601,5.789)--(5.602,5.791)%
  --(5.604,5.793)--(5.605,5.795)--(5.606,5.797)--(5.607,5.799)--(5.608,5.801)--(5.609,5.803)%
  --(5.611,5.805)--(5.612,5.807)--(5.614,5.809)--(5.615,5.812)--(5.616,5.814)--(5.617,5.816)%
  --(5.618,5.818)--(5.620,5.820)--(5.621,5.822)--(5.622,5.824)--(5.623,5.826)--(5.625,5.828)%
  --(5.626,5.830)--(5.627,5.832)--(5.628,5.833)--(5.629,5.835)--(5.630,5.837)--(5.631,5.839)%
  --(5.632,5.841)--(5.633,5.842)--(5.634,5.844)--(5.635,5.846)--(5.636,5.848)--(5.638,5.850)%
  --(5.639,5.851)--(5.640,5.853)--(5.641,5.855)--(5.642,5.857)--(5.643,5.859)--(5.645,5.861)%
  --(5.646,5.863)--(5.647,5.866)--(5.649,5.868)--(5.650,5.870)--(5.651,5.872)--(5.653,5.875)%
  --(5.654,5.877)--(5.655,5.879)--(5.657,5.881)--(5.658,5.884)--(5.659,5.885)--(5.660,5.887)%
  --(5.661,5.889)--(5.662,5.891)--(5.663,5.893)--(5.665,5.894)--(5.666,5.896)--(5.667,5.898)%
  --(5.668,5.900)--(5.669,5.901)--(5.670,5.903)--(5.671,5.905)--(5.672,5.907)--(5.673,5.909)%
  --(5.674,5.910)--(5.676,5.912)--(5.677,5.914)--(5.678,5.916)--(5.679,5.918)--(5.680,5.920)%
  --(5.682,5.923)--(5.684,5.925)--(5.685,5.928)--(5.687,5.930)--(5.688,5.933)--(5.690,5.935)%
  --(5.691,5.937)--(5.692,5.940)--(5.693,5.941)--(5.695,5.943)--(5.696,5.945)--(5.697,5.947)%
  --(5.698,5.949)--(5.699,5.950)--(5.700,5.952)--(5.701,5.954)--(5.702,5.956)--(5.703,5.957)%
  --(5.704,5.959)--(5.706,5.961)--(5.707,5.963)--(5.708,5.965)--(5.709,5.966)--(5.710,5.968)%
  --(5.711,5.970)--(5.712,5.972)--(5.714,5.974)--(5.715,5.976)--(5.716,5.979)--(5.718,5.981)%
  --(5.720,5.984)--(5.721,5.986)--(5.722,5.988)--(5.723,5.990)--(5.724,5.992)--(5.726,5.994)%
  --(5.727,5.996)--(5.728,5.997)--(5.729,5.999)--(5.730,6.001)--(5.731,6.003)--(5.732,6.004)%
  --(5.733,6.006)--(5.734,6.008)--(5.736,6.010)--(5.737,6.011)--(5.738,6.013)--(5.739,6.015)%
  --(5.740,6.016)--(5.741,6.018)--(5.742,6.020)--(5.743,6.022)--(5.744,6.024)--(5.746,6.025)%
  --(5.747,6.027)--(5.748,6.029)--(5.749,6.031)--(5.751,6.034)--(5.752,6.036)--(5.753,6.038)%
  --(5.755,6.040)--(5.756,6.042)--(5.757,6.044)--(5.758,6.046)--(5.759,6.048)--(5.761,6.049)%
  --(5.762,6.051)--(5.763,6.053)--(5.764,6.055)--(5.765,6.057)--(5.766,6.058)--(5.767,6.060)%
  --(5.768,6.062)--(5.770,6.064)--(5.771,6.065)--(5.772,6.067)--(5.773,6.069)--(5.774,6.070)%
  --(5.775,6.072)--(5.776,6.074)--(5.777,6.076)--(5.778,6.078)--(5.779,6.079)--(5.781,6.081)%
  --(5.782,6.083)--(5.783,6.085)--(5.784,6.087)--(5.786,6.089)--(5.787,6.092)--(5.788,6.094)%
  --(5.790,6.096)--(5.791,6.098)--(5.792,6.100)--(5.794,6.103)--(5.795,6.105)--(5.797,6.107)%
  --(5.798,6.109)--(5.799,6.110)--(5.800,6.112)--(5.801,6.114)--(5.802,6.116)--(5.803,6.117)%
  --(5.804,6.119)--(5.805,6.120)--(5.807,6.122)--(5.808,6.124)--(5.809,6.126)--(5.810,6.127)%
  --(5.811,6.129)--(5.812,6.131)--(5.813,6.132)--(5.814,6.134)--(5.815,6.136)--(5.817,6.138)%
  --(5.818,6.140)--(5.819,6.142)--(5.821,6.144)--(5.822,6.146)--(5.823,6.148)--(5.825,6.150)%
  --(5.826,6.153)--(5.827,6.155)--(5.829,6.157)--(5.830,6.159)--(5.831,6.161)--(5.832,6.162)%
  --(5.833,6.164)--(5.835,6.166)--(5.836,6.168)--(5.837,6.169)--(5.838,6.171)--(5.839,6.173)%
  --(5.840,6.174)--(5.841,6.176)--(5.842,6.178)--(5.843,6.179)--(5.844,6.181)--(5.845,6.183)%
  --(5.846,6.184)--(5.848,6.186)--(5.849,6.188)--(5.850,6.190)--(5.851,6.192)--(5.852,6.194)%
  --(5.854,6.196)--(5.856,6.199)--(5.857,6.201)--(5.859,6.203)--(5.860,6.205)--(5.861,6.207)%
  --(5.862,6.208)--(5.863,6.210)--(5.864,6.212)--(5.865,6.214)--(5.867,6.216)--(5.868,6.217)%
  --(5.869,6.219)--(5.870,6.221)--(5.871,6.222)--(5.872,6.224)--(5.873,6.226)--(5.874,6.227)%
  --(5.875,6.229)--(5.876,6.230)--(5.877,6.232)--(5.879,6.234)--(5.880,6.235)--(5.881,6.237)%
  --(5.882,6.239)--(5.883,6.241)--(5.884,6.242)--(5.885,6.244)--(5.887,6.246)--(5.888,6.248)%
  --(5.889,6.250)--(5.891,6.252)--(5.892,6.254)--(5.893,6.256)--(5.894,6.258)--(5.896,6.260)%
  --(5.897,6.262)--(5.898,6.263)--(5.899,6.265)--(5.900,6.267)--(5.902,6.269)--(5.903,6.270)%
  --(5.904,6.272)--(5.905,6.274)--(5.906,6.275)--(5.907,6.277)--(5.908,6.279)--(5.909,6.280)%
  --(5.910,6.282)--(5.911,6.283)--(5.912,6.285)--(5.913,6.287)--(5.915,6.289)--(5.916,6.290)%
  --(5.917,6.292)--(5.918,6.294)--(5.919,6.296)--(5.921,6.298)--(5.922,6.299)--(5.923,6.301)%
  --(5.924,6.303)--(5.926,6.306)--(5.927,6.308)--(5.929,6.310)--(5.930,6.312)--(5.931,6.314)%
  --(5.933,6.316)--(5.934,6.318)--(5.935,6.319)--(5.936,6.321)--(5.937,6.323)--(5.938,6.324)%
  --(5.940,6.326)--(5.941,6.328)--(5.942,6.329)--(5.943,6.331)--(5.944,6.332)--(5.945,6.334)%
  --(5.946,6.336)--(5.947,6.337)--(5.948,6.339)--(5.949,6.340)--(5.950,6.342)--(5.951,6.344)%
  --(5.953,6.346)--(5.954,6.347)--(5.955,6.349)--(5.956,6.351)--(5.958,6.353)--(5.960,6.356)%
  --(5.961,6.358)--(5.963,6.360)--(5.964,6.362)--(5.965,6.364)--(5.967,6.367)--(5.968,6.369)%
  --(5.970,6.370)--(5.971,6.372)--(5.972,6.374)--(5.973,6.375)--(5.974,6.377)--(5.975,6.378)%
  --(5.976,6.380)--(5.977,6.382)--(5.978,6.383)--(5.979,6.385)--(5.980,6.386)--(5.981,6.388)%
  --(5.982,6.390)--(5.984,6.391)--(5.985,6.393)--(5.986,6.395)--(5.987,6.397)--(5.988,6.398)%
  --(5.990,6.400)--(5.991,6.402)--(5.992,6.404)--(5.994,6.407)--(5.996,6.409)--(5.997,6.411)%
  --(5.999,6.413)--(6.000,6.416)--(6.001,6.418)--(6.003,6.420)--(6.004,6.421)--(6.005,6.423)%
  --(6.006,6.425)--(6.007,6.426)--(6.008,6.428)--(6.009,6.429)--(6.011,6.431)--(6.012,6.432)%
  --(6.013,6.434)--(6.014,6.435)--(6.015,6.437)--(6.016,6.439)--(6.017,6.440)--(6.018,6.442)%
  --(6.019,6.443)--(6.020,6.445)--(6.022,6.447)--(6.023,6.448)--(6.024,6.450)--(6.025,6.452)%
  --(6.027,6.454)--(6.028,6.457)--(6.030,6.459)--(6.032,6.461)--(6.033,6.463)--(6.034,6.465)%
  --(6.036,6.467)--(6.037,6.469)--(6.038,6.471)--(6.039,6.472)--(6.041,6.474)--(6.042,6.476)%
  --(6.043,6.477)--(6.044,6.479)--(6.045,6.480)--(6.046,6.482)--(6.047,6.483)--(6.048,6.485)%
  --(6.049,6.487)--(6.050,6.488)--(6.051,6.490)--(6.053,6.491)--(6.054,6.493)--(6.055,6.495)%
  --(6.056,6.496)--(6.057,6.498)--(6.058,6.500)--(6.060,6.502)--(6.061,6.504)--(6.063,6.506)%
  --(6.064,6.508)--(6.066,6.511)--(6.067,6.513)--(6.069,6.515)--(6.070,6.517)--(6.072,6.519)%
  --(6.073,6.520)--(6.074,6.522)--(6.075,6.524)--(6.076,6.525)--(6.077,6.527)--(6.078,6.528)%
  --(6.079,6.530)--(6.080,6.531)--(6.082,6.533)--(6.083,6.534)--(6.084,6.536)--(6.085,6.537)%
  --(6.086,6.539)--(6.087,6.540)--(6.088,6.542)--(6.089,6.544)--(6.090,6.545)--(6.092,6.547)%
  --(6.093,6.549)--(6.094,6.550)--(6.096,6.552)--(6.097,6.555)--(6.099,6.557)--(6.100,6.559)%
  --(6.102,6.561)--(6.103,6.562)--(6.104,6.564)--(6.105,6.565)--(6.106,6.567)--(6.107,6.569)%
  --(6.108,6.570)--(6.109,6.572)--(6.110,6.573)--(6.112,6.575)--(6.113,6.576)--(6.114,6.578)%
  --(6.115,6.579)--(6.116,6.581)--(6.117,6.582)--(6.118,6.584)--(6.119,6.585)--(6.120,6.587)%
  --(6.121,6.589)--(6.122,6.590)--(6.124,6.592)--(6.125,6.593)--(6.126,6.595)--(6.127,6.597)%
  --(6.129,6.599)--(6.130,6.601)--(6.131,6.603)--(6.133,6.605)--(6.134,6.606)--(6.135,6.608)%
  --(6.136,6.610)--(6.137,6.611)--(6.139,6.613)--(6.140,6.615)--(6.141,6.616)--(6.142,6.618)%
  --(6.143,6.619)--(6.144,6.621)--(6.146,6.622)--(6.147,6.624)--(6.148,6.625)--(6.149,6.627)%
  --(6.150,6.628)--(6.151,6.630)--(6.152,6.631)--(6.153,6.633)--(6.154,6.634)--(6.155,6.636)%
  --(6.156,6.637)--(6.157,6.639)--(6.159,6.640)--(6.160,6.642)--(6.161,6.644)--(6.162,6.645)%
  --(6.164,6.647)--(6.165,6.649)--(6.166,6.651)--(6.168,6.653)--(6.169,6.654)--(6.170,6.656)%
  --(6.172,6.658)--(6.173,6.660)--(6.174,6.662)--(6.176,6.664)--(6.177,6.665)--(6.178,6.667)%
  --(6.179,6.668)--(6.180,6.670)--(6.181,6.671)--(6.182,6.673)--(6.184,6.674)--(6.185,6.676)%
  --(6.186,6.677)--(6.187,6.679)--(6.188,6.680)--(6.189,6.682)--(6.190,6.683)--(6.191,6.685)%
  --(6.192,6.686)--(6.193,6.688)--(6.195,6.690)--(6.196,6.691)--(6.197,6.693)--(6.198,6.695)%
  --(6.200,6.697)--(6.201,6.698)--(6.202,6.700)--(6.204,6.702)--(6.205,6.704)--(6.207,6.706)%
  --(6.208,6.708)--(6.209,6.710)--(6.210,6.711)--(6.212,6.713)--(6.213,6.714)--(6.214,6.716)%
  --(6.215,6.717)--(6.216,6.718)--(6.217,6.720)--(6.218,6.721)--(6.219,6.723)--(6.220,6.724)%
  --(6.221,6.726)--(6.222,6.727)--(6.224,6.729)--(6.225,6.730)--(6.226,6.731)--(6.227,6.733)%
  --(6.228,6.735)--(6.229,6.736)--(6.231,6.738)--(6.232,6.739)--(6.233,6.741)--(6.235,6.743)%
  --(6.236,6.745)--(6.238,6.747)--(6.239,6.749)--(6.240,6.750)--(6.241,6.752)--(6.242,6.753)%
  --(6.243,6.755)--(6.245,6.756)--(6.246,6.758)--(6.247,6.760)--(6.248,6.761)--(6.249,6.762)%
  --(6.250,6.764)--(6.251,6.765)--(6.252,6.767)--(6.253,6.768)--(6.254,6.769)--(6.255,6.771)%
  --(6.257,6.772)--(6.258,6.774)--(6.259,6.775)--(6.260,6.777)--(6.261,6.778)--(6.262,6.780)%
  --(6.263,6.781)--(6.265,6.783)--(6.266,6.785)--(6.267,6.787)--(6.269,6.788)--(6.270,6.790)%
  --(6.271,6.792)--(6.272,6.794)--(6.274,6.795)--(6.275,6.797)--(6.276,6.798)--(6.277,6.800)%
  --(6.279,6.802)--(6.280,6.803)--(6.281,6.805)--(6.282,6.806)--(6.283,6.807)--(6.284,6.809)%
  --(6.285,6.810)--(6.286,6.812)--(6.287,6.813)--(6.288,6.814)--(6.289,6.816)--(6.290,6.817)%
  --(6.292,6.819)--(6.293,6.820)--(6.294,6.821)--(6.295,6.823)--(6.296,6.824)--(6.297,6.826)%
  --(6.299,6.827)--(6.300,6.829)--(6.301,6.831)--(6.302,6.832)--(6.304,6.834)--(6.305,6.836)%
  --(6.306,6.837)--(6.308,6.839)--(6.309,6.841)--(6.310,6.843)--(6.312,6.844)--(6.313,6.846)%
  --(6.314,6.848)--(6.315,6.849)--(6.316,6.851)--(6.318,6.852)--(6.319,6.853)--(6.320,6.855)%
  --(6.321,6.856)--(6.322,6.857)--(6.323,6.859)--(6.324,6.860)--(6.325,6.862)--(6.326,6.863)%
  --(6.327,6.865)--(6.329,6.866)--(6.330,6.867)--(6.331,6.869)--(6.332,6.871)--(6.333,6.872)%
  --(6.334,6.874)--(6.336,6.875)--(6.337,6.877)--(6.339,6.879)--(6.340,6.881)--(6.342,6.883)%
  --(6.343,6.885)--(6.345,6.887)--(6.346,6.889)--(6.347,6.891)--(6.349,6.892)--(6.350,6.893)%
  --(6.351,6.895)--(6.352,6.896)--(6.353,6.898)--(6.354,6.899)--(6.355,6.900)--(6.356,6.902)%
  --(6.357,6.903)--(6.358,6.904)--(6.360,6.906)--(6.361,6.907)--(6.362,6.909)--(6.363,6.910)%
  --(6.364,6.911)--(6.365,6.913)--(6.366,6.914)--(6.368,6.916)--(6.369,6.917)--(6.370,6.919)%
  --(6.371,6.921)--(6.373,6.923)--(6.375,6.925)--(6.376,6.927)--(6.377,6.928)--(6.378,6.929)%
  --(6.380,6.931)--(6.381,6.932)--(6.382,6.934)--(6.383,6.935)--(6.384,6.936)--(6.385,6.938)%
  --(6.386,6.939)--(6.387,6.941)--(6.388,6.942)--(6.390,6.943)--(6.391,6.945)--(6.392,6.946)%
  --(6.393,6.947)--(6.394,6.949)--(6.395,6.950)--(6.396,6.952)--(6.397,6.953)--(6.398,6.954)%
  --(6.399,6.956)--(6.401,6.957)--(6.402,6.959)--(6.403,6.960)--(6.404,6.962)--(6.406,6.964)%
  --(6.407,6.965)--(6.409,6.967)--(6.410,6.969)--(6.411,6.970)--(6.412,6.972)--(6.413,6.973)%
  --(6.415,6.975)--(6.416,6.976)--(6.417,6.978)--(6.418,6.979)--(6.419,6.980)--(6.420,6.982)%
  --(6.421,6.983)--(6.422,6.984)--(6.424,6.986)--(6.425,6.987)--(6.426,6.988)--(6.427,6.990)%
  --(6.428,6.991)--(6.429,6.992)--(6.430,6.994)--(6.431,6.995)--(6.432,6.996)--(6.433,6.998)%
  --(6.435,6.999)--(6.436,7.000)--(6.437,7.002)--(6.438,7.003)--(6.439,7.005)--(6.441,7.007)%
  --(6.442,7.008)--(6.443,7.010)--(6.445,7.012)--(6.446,7.013)--(6.448,7.015)--(6.449,7.016)%
  --(6.450,7.017)--(6.451,7.019)--(6.452,7.020)--(6.453,7.021)--(6.454,7.023)--(6.455,7.024)%
  --(6.456,7.025)--(6.457,7.027)--(6.458,7.028)--(6.460,7.029)--(6.461,7.031)--(6.462,7.032)%
  --(6.463,7.033)--(6.464,7.035)--(6.465,7.036)--(6.466,7.037)--(6.467,7.039)--(6.468,7.040)%
  --(6.469,7.042)--(6.471,7.043)--(6.472,7.045)--(6.473,7.046)--(6.475,7.048)--(6.476,7.050)%
  --(6.478,7.052)--(6.479,7.053)--(6.480,7.055)--(6.482,7.056)--(6.483,7.058)--(6.484,7.059)%
  --(6.485,7.060)--(6.486,7.062)--(6.487,7.063)--(6.488,7.064)--(6.489,7.066)--(6.490,7.067)%
  --(6.492,7.068)--(6.493,7.070)--(6.494,7.071)--(6.495,7.072)--(6.496,7.073)--(6.497,7.075)%
  --(6.498,7.076)--(6.499,7.077)--(6.500,7.078)--(6.501,7.080)--(6.503,7.081)--(6.504,7.082)%
  --(6.505,7.084)--(6.506,7.085)--(6.507,7.087)--(6.509,7.088)--(6.510,7.090)--(6.512,7.092)%
  --(6.513,7.093)--(6.514,7.094)--(6.515,7.096)--(6.516,7.097)--(6.518,7.099)--(6.519,7.100)%
  --(6.520,7.102)--(6.521,7.103)--(6.522,7.104)--(6.523,7.106)--(6.524,7.107)--(6.526,7.108)%
  --(6.527,7.109)--(6.528,7.111)--(6.529,7.112)--(6.530,7.113)--(6.531,7.115)--(6.532,7.116)%
  --(6.533,7.117)--(6.534,7.118)--(6.535,7.120)--(6.536,7.121)--(6.538,7.122)--(6.539,7.124)%
  --(6.540,7.125)--(6.541,7.127)--(6.543,7.128)--(6.544,7.130)--(6.545,7.131)--(6.547,7.133)%
  --(6.548,7.135)--(6.549,7.136)--(6.551,7.138)--(6.552,7.139)--(6.553,7.141)--(6.555,7.142)%
  --(6.556,7.144)--(6.557,7.145)--(6.558,7.146)--(6.559,7.148)--(6.560,7.149)--(6.561,7.150)%
  --(6.562,7.151)--(6.563,7.152)--(6.564,7.154)--(6.565,7.155)--(6.567,7.156)--(6.568,7.157)%
  --(6.569,7.159)--(6.570,7.160)--(6.571,7.161)--(6.572,7.162)--(6.573,7.164)--(6.575,7.165)%
  --(6.576,7.167)--(6.577,7.168)--(6.579,7.170)--(6.580,7.172)--(6.582,7.173)--(6.584,7.175)%
  --(6.585,7.177)--(6.586,7.178)--(6.588,7.180)--(6.589,7.182)--(6.590,7.183)--(6.591,7.184)%
  --(6.592,7.186)--(6.594,7.187)--(6.595,7.188)--(6.596,7.189)--(6.597,7.190)--(6.598,7.192)%
  --(6.599,7.193)--(6.600,7.194)--(6.601,7.195)--(6.602,7.197)--(6.603,7.198)--(6.604,7.199)%
  --(6.605,7.201)--(6.607,7.202)--(6.608,7.203)--(6.609,7.205)--(6.610,7.206)--(6.612,7.207)%
  --(6.613,7.209)--(6.615,7.211)--(6.616,7.213)--(6.618,7.215)--(6.619,7.216)--(6.621,7.218)%
  --(6.622,7.219)--(6.624,7.221)--(6.625,7.222)--(6.626,7.224)--(6.627,7.225)--(6.628,7.226)%
  --(6.629,7.227)--(6.630,7.228)--(6.631,7.230)--(6.632,7.231)--(6.634,7.232)--(6.635,7.233)%
  --(6.636,7.234)--(6.637,7.236)--(6.638,7.237)--(6.639,7.238)--(6.640,7.239)--(6.641,7.240)%
  --(6.642,7.242)--(6.643,7.243)--(6.645,7.245)--(6.646,7.246)--(6.648,7.248)--(6.649,7.249)%
  --(6.651,7.251)--(6.652,7.253)--(6.654,7.254)--(6.655,7.256)--(6.657,7.257)--(6.658,7.259)%
  --(6.659,7.260)--(6.660,7.262)--(6.661,7.263)--(6.663,7.264)--(6.664,7.265)--(6.665,7.266)%
  --(6.666,7.268)--(6.667,7.269)--(6.668,7.270)--(6.669,7.271)--(6.670,7.272)--(6.671,7.274)%
  --(6.672,7.275)--(6.673,7.276)--(6.675,7.277)--(6.676,7.278)--(6.677,7.280)--(6.678,7.281)%
  --(6.679,7.283)--(6.680,7.284)--(6.682,7.286)--(6.684,7.288)--(6.685,7.289)--(6.687,7.291)%
  --(6.688,7.293)--(6.690,7.294)--(6.691,7.296)--(6.693,7.297)--(6.694,7.298)--(6.695,7.299)%
  --(6.696,7.301)--(6.697,7.302)--(6.698,7.303)--(6.699,7.304)--(6.700,7.305)--(6.701,7.307)%
  --(6.702,7.308)--(6.703,7.309)--(6.704,7.310)--(6.706,7.311)--(6.707,7.312)--(6.708,7.313)%
  --(6.709,7.315)--(6.710,7.316)--(6.711,7.317)--(6.713,7.318)--(6.714,7.320)--(6.715,7.321)%
  --(6.717,7.323)--(6.718,7.324)--(6.720,7.326)--(6.721,7.328)--(6.723,7.329)--(6.724,7.331)%
  --(6.726,7.332)--(6.727,7.334)--(6.728,7.335)--(6.729,7.336)--(6.730,7.337)--(6.731,7.339)%
  --(6.733,7.340)--(6.734,7.341)--(6.735,7.342)--(6.736,7.343)--(6.737,7.344)--(6.738,7.345)%
  --(6.739,7.347)--(6.740,7.348)--(6.741,7.349)--(6.742,7.350)--(6.743,7.351)--(6.745,7.353)%
  --(6.746,7.354)--(6.747,7.355)--(6.748,7.357)--(6.750,7.358)--(6.751,7.360)--(6.753,7.361)%
  --(6.754,7.363)--(6.756,7.365)--(6.757,7.366)--(6.759,7.368)--(6.760,7.369)--(6.761,7.371)%
  --(6.763,7.372)--(6.764,7.373)--(6.765,7.374)--(6.766,7.375)--(6.767,7.376)--(6.768,7.377)%
  --(6.769,7.379)--(6.770,7.380)--(6.771,7.381)--(6.773,7.382)--(6.774,7.383)--(6.775,7.384)%
  --(6.776,7.385)--(6.777,7.386)--(6.778,7.387)--(6.779,7.389)--(6.780,7.390)--(6.781,7.391)%
  --(6.783,7.392)--(6.784,7.394)--(6.786,7.395)--(6.787,7.397)--(6.789,7.399)--(6.790,7.400)%
  --(6.792,7.402)--(6.793,7.403)--(6.795,7.405)--(6.796,7.406)--(6.797,7.407)--(6.798,7.408)%
  --(6.799,7.409)--(6.801,7.411)--(6.802,7.412)--(6.803,7.413)--(6.804,7.414)--(6.805,7.415)%
  --(6.806,7.416)--(6.807,7.417)--(6.808,7.418)--(6.809,7.420)--(6.810,7.421)--(6.811,7.422)%
  --(6.812,7.423)--(6.814,7.424)--(6.815,7.425)--(6.816,7.427)--(6.817,7.428)--(6.818,7.429)%
  --(6.820,7.431)--(6.822,7.433)--(6.823,7.434)--(6.825,7.436)--(6.826,7.437)--(6.828,7.439)%
  --(6.829,7.440)--(6.831,7.441)--(6.832,7.443)--(6.833,7.444)--(6.834,7.445)--(6.835,7.446)%
  --(6.836,7.447)--(6.837,7.448)--(6.838,7.449)--(6.839,7.450)--(6.840,7.451)--(6.841,7.452)%
  --(6.842,7.453)--(6.843,7.455)--(6.845,7.456)--(6.846,7.457)--(6.847,7.458)--(6.848,7.459)%
  --(6.849,7.460)--(6.851,7.461)--(6.852,7.463)--(6.853,7.464)--(6.855,7.465)--(6.856,7.467)%
  --(6.858,7.468)--(6.859,7.470)--(6.861,7.472)--(6.862,7.473)--(6.864,7.474)--(6.865,7.476)%
  --(6.866,7.477)--(6.867,7.478)--(6.868,7.479)--(6.869,7.480)--(6.870,7.481)--(6.872,7.482)%
  --(6.873,7.483)--(6.874,7.484)--(6.875,7.485)--(6.876,7.486)--(6.877,7.488)--(6.878,7.489)%
  --(6.879,7.490)--(6.880,7.491)--(6.882,7.492)--(6.883,7.493)--(6.884,7.494)--(6.885,7.496)%
  --(6.886,7.497)--(6.887,7.498)--(6.889,7.500)--(6.891,7.501)--(6.892,7.503)--(6.894,7.504)%
  --(6.895,7.506)--(6.897,7.507)--(6.898,7.508)--(6.900,7.510)--(6.901,7.511)--(6.902,7.512)%
  --(6.903,7.513)--(6.904,7.514)--(6.905,7.515)--(6.906,7.516)--(6.907,7.517)--(6.908,7.518)%
  --(6.909,7.519)--(6.910,7.520)--(6.912,7.521)--(6.913,7.522)--(6.914,7.523)--(6.915,7.524)%
  --(6.916,7.526)--(6.917,7.527)--(6.918,7.528)--(6.920,7.529)--(6.921,7.530)--(6.922,7.531)%
  --(6.924,7.533)--(6.925,7.534)--(6.927,7.536)--(6.929,7.537)--(6.930,7.539)--(6.931,7.540)%
  --(6.933,7.541)--(6.934,7.543)--(6.935,7.544)--(6.936,7.545)--(6.938,7.546)--(6.939,7.547)%
  --(6.940,7.548)--(6.941,7.549)--(6.942,7.550)--(6.943,7.551)--(6.944,7.552)--(6.945,7.553)%
  --(6.946,7.554)--(6.947,7.555)--(6.948,7.556)--(6.950,7.557)--(6.951,7.558)--(6.952,7.560)%
  --(6.953,7.561)--(6.954,7.562)--(6.955,7.563)--(6.957,7.564)--(6.958,7.566)--(6.960,7.568)%
  --(6.962,7.569)--(6.963,7.571)--(6.965,7.572)--(6.966,7.573)--(6.968,7.575)--(6.969,7.576)%
  --(6.970,7.577)--(6.971,7.578)--(6.972,7.579)--(6.973,7.580)--(6.974,7.581)--(6.975,7.582)%
  --(6.977,7.583)--(6.978,7.584)--(6.979,7.585)--(6.980,7.586)--(6.981,7.587)--(6.982,7.588)%
  --(6.983,7.589)--(6.984,7.590)--(6.985,7.591)--(6.987,7.592)--(6.988,7.593)--(6.989,7.594)%
  --(6.990,7.595)--(6.991,7.597)--(6.993,7.598)--(6.994,7.599)--(6.995,7.600)--(6.997,7.601)%
  --(6.998,7.603)--(7.000,7.604)--(7.001,7.605)--(7.002,7.606)--(7.004,7.607)--(7.005,7.608)%
  --(7.006,7.610)--(7.007,7.611)--(7.008,7.612)--(7.009,7.613)--(7.010,7.614)--(7.011,7.615)%
  --(7.012,7.616)--(7.013,7.616)--(7.014,7.618)--(7.015,7.618)--(7.016,7.620)--(7.018,7.621)%
  --(7.019,7.622)--(7.020,7.623)--(7.021,7.624)--(7.022,7.625)--(7.024,7.626)--(7.025,7.627)%
  --(7.026,7.628)--(7.028,7.630)--(7.029,7.631)--(7.031,7.633)--(7.032,7.634)--(7.033,7.635)%
  --(7.034,7.636)--(7.035,7.637)--(7.036,7.638)--(7.038,7.639)--(7.039,7.640)--(7.040,7.641)%
  --(7.041,7.642)--(7.042,7.643)--(7.043,7.644)--(7.044,7.645)--(7.045,7.646)--(7.046,7.647)%
  --(7.048,7.647)--(7.049,7.648)--(7.050,7.649)--(7.051,7.650)--(7.052,7.651)--(7.053,7.652)%
  --(7.054,7.653)--(7.055,7.654)--(7.056,7.655)--(7.058,7.656)--(7.059,7.658)--(7.060,7.659)%
  --(7.062,7.660)--(7.063,7.661)--(7.064,7.662)--(7.066,7.663)--(7.067,7.664)--(7.068,7.665)%
  --(7.069,7.666)--(7.070,7.667)--(7.072,7.668)--(7.073,7.670)--(7.074,7.671)--(7.075,7.672)%
  --(7.076,7.672)--(7.077,7.673)--(7.078,7.674)--(7.079,7.675)--(7.080,7.676)--(7.082,7.677)%
  --(7.083,7.678)--(7.084,7.679)--(7.085,7.680)--(7.086,7.681)--(7.087,7.682)--(7.088,7.683)%
  --(7.089,7.684)--(7.090,7.685)--(7.092,7.686)--(7.093,7.687)--(7.094,7.688)--(7.095,7.689)%
  --(7.097,7.691)--(7.098,7.692)--(7.099,7.693)--(7.101,7.694)--(7.102,7.695)--(7.103,7.697)%
  --(7.105,7.698)--(7.106,7.699)--(7.107,7.700)--(7.109,7.701)--(7.110,7.702)--(7.111,7.703)%
  --(7.112,7.704)--(7.113,7.704)--(7.114,7.705)--(7.115,7.706)--(7.116,7.707)--(7.117,7.708)%
  --(7.118,7.709)--(7.119,7.710)--(7.120,7.711)--(7.121,7.712)--(7.123,7.713)--(7.124,7.714)%
  --(7.125,7.715)--(7.126,7.716)--(7.127,7.717)--(7.129,7.718)--(7.130,7.719)--(7.132,7.720)%
  --(7.133,7.722)--(7.135,7.723)--(7.136,7.724)--(7.138,7.725)--(7.139,7.727)--(7.141,7.728)%
  --(7.142,7.729)--(7.143,7.730)--(7.144,7.731)--(7.145,7.732)--(7.146,7.732)--(7.147,7.733)%
  --(7.148,7.734)--(7.149,7.735)--(7.151,7.736)--(7.152,7.737)--(7.153,7.738)--(7.154,7.739)%
  --(7.155,7.740)--(7.156,7.741)--(7.157,7.742)--(7.158,7.743)--(7.159,7.744)--(7.161,7.745)%
  --(7.162,7.746)--(7.163,7.747)--(7.165,7.748)--(7.166,7.749)--(7.168,7.751)--(7.169,7.752)%
  --(7.171,7.753)--(7.172,7.754)--(7.174,7.755)--(7.175,7.757)--(7.176,7.758)--(7.177,7.758)%
  --(7.178,7.759)--(7.180,7.760)--(7.181,7.761)--(7.182,7.762)--(7.183,7.763)--(7.184,7.764)%
  --(7.185,7.765)--(7.186,7.766)--(7.187,7.766)--(7.188,7.767)--(7.189,7.768)--(7.190,7.769)%
  --(7.191,7.770)--(7.193,7.771)--(7.194,7.772)--(7.195,7.773)--(7.196,7.774)--(7.198,7.775)%
  --(7.199,7.776)--(7.201,7.777)--(7.202,7.778)--(7.204,7.780)--(7.205,7.781)--(7.207,7.782)%
  --(7.208,7.783)--(7.209,7.784)--(7.211,7.785)--(7.212,7.786)--(7.213,7.787)--(7.214,7.788)%
  --(7.215,7.789)--(7.216,7.790)--(7.217,7.790)--(7.218,7.791)--(7.219,7.792)--(7.220,7.793)%
  --(7.221,7.794)--(7.222,7.795)--(7.224,7.796)--(7.225,7.797)--(7.226,7.797)--(7.227,7.798)%
  --(7.228,7.799)--(7.229,7.800)--(7.231,7.801)--(7.232,7.802)--(7.234,7.804)--(7.235,7.805)%
  --(7.236,7.806)--(7.238,7.807)--(7.239,7.808)--(7.240,7.809)--(7.242,7.810)--(7.243,7.811)%
  --(7.244,7.812)--(7.245,7.813)--(7.246,7.814)--(7.247,7.814)--(7.248,7.815)--(7.249,7.816)%
  --(7.250,7.817)--(7.252,7.818)--(7.253,7.819)--(7.254,7.819)--(7.255,7.820)--(7.256,7.821)%
  --(7.257,7.822)--(7.258,7.823)--(7.259,7.823)--(7.260,7.824)--(7.261,7.825)--(7.262,7.826)%
  --(7.264,7.827)--(7.265,7.828)--(7.266,7.829)--(7.268,7.830)--(7.269,7.831)--(7.270,7.832)%
  --(7.272,7.833)--(7.273,7.834)--(7.274,7.835)--(7.275,7.836)--(7.276,7.837)--(7.277,7.838)%
  --(7.279,7.839)--(7.280,7.839)--(7.281,7.840)--(7.282,7.841)--(7.283,7.842)--(7.284,7.843)%
  --(7.285,7.844)--(7.286,7.844)--(7.287,7.845)--(7.289,7.846)--(7.290,7.847)--(7.291,7.848)%
  --(7.292,7.849)--(7.293,7.849)--(7.294,7.850)--(7.295,7.851)--(7.296,7.852)--(7.297,7.853)%
  --(7.299,7.854)--(7.300,7.855)--(7.301,7.856)--(7.303,7.857)--(7.304,7.858)--(7.305,7.859)%
  --(7.307,7.860)--(7.308,7.861)--(7.309,7.862)--(7.311,7.863)--(7.312,7.864)--(7.314,7.865)%
  --(7.315,7.866)--(7.316,7.867)--(7.317,7.868)--(7.318,7.868)--(7.319,7.869)--(7.320,7.870)%
  --(7.321,7.871)--(7.322,7.871)--(7.323,7.872)--(7.324,7.873)--(7.326,7.874)--(7.327,7.875)%
  --(7.328,7.875)--(7.329,7.876)--(7.330,7.877)--(7.331,7.878)--(7.332,7.879)--(7.334,7.880)%
  --(7.335,7.881)--(7.336,7.881)--(7.338,7.883)--(7.339,7.884)--(7.341,7.885)--(7.343,7.886)%
  --(7.344,7.887)--(7.346,7.888)--(7.347,7.889)--(7.348,7.890)--(7.349,7.891)--(7.351,7.892)%
  --(7.352,7.893)--(7.353,7.894)--(7.354,7.894)--(7.355,7.895)--(7.356,7.896)--(7.357,7.897)%
  --(7.358,7.897)--(7.359,7.898)--(7.360,7.899)--(7.361,7.900)--(7.363,7.901)--(7.364,7.902)%
  --(7.365,7.902)--(7.366,7.903)--(7.367,7.904)--(7.368,7.905)--(7.370,7.906)--(7.371,7.907)%
  --(7.372,7.908)--(7.374,7.909)--(7.375,7.910)--(7.376,7.911)--(7.378,7.912)--(7.379,7.913)%
  --(7.380,7.914)--(7.382,7.915)--(7.383,7.915)--(7.384,7.916)--(7.385,7.917)--(7.386,7.918)%
  --(7.387,7.919)--(7.388,7.919)--(7.390,7.920)--(7.391,7.921)--(7.392,7.921)--(7.393,7.922)%
  --(7.394,7.923)--(7.395,7.924)--(7.396,7.925)--(7.397,7.925)--(7.398,7.926)--(7.399,7.927)%
  --(7.401,7.928)--(7.402,7.928)--(7.403,7.929)--(7.404,7.930)--(7.406,7.931)--(7.407,7.932)%
  --(7.409,7.933)--(7.410,7.934)--(7.411,7.935)--(7.413,7.936)--(7.414,7.937)--(7.415,7.937)%
  --(7.416,7.938)--(7.417,7.939)--(7.418,7.940)--(7.419,7.941)--(7.421,7.941)--(7.422,7.942)%
  --(7.423,7.943)--(7.424,7.944)--(7.425,7.944)--(7.426,7.945)--(7.427,7.946)--(7.428,7.947)%
  --(7.429,7.947)--(7.430,7.948)--(7.431,7.949)--(7.432,7.950)--(7.434,7.950)--(7.435,7.951)%
  --(7.436,7.952)--(7.437,7.953)--(7.438,7.954)--(7.440,7.955)--(7.441,7.956)--(7.443,7.957)%
  --(7.444,7.957)--(7.445,7.958)--(7.446,7.959)--(7.447,7.960)--(7.449,7.961)--(7.450,7.962)%
  --(7.451,7.962)--(7.452,7.963)--(7.453,7.964)--(7.454,7.965)--(7.456,7.965)--(7.457,7.966)%
  --(7.458,7.967)--(7.459,7.967)--(7.460,7.968)--(7.461,7.969)--(7.462,7.970)--(7.463,7.970)%
  --(7.464,7.971)--(7.465,7.972)--(7.466,7.972)--(7.468,7.973)--(7.469,7.974)--(7.470,7.975)%
  --(7.471,7.975)--(7.472,7.976)--(7.474,7.977)--(7.475,7.978)--(7.476,7.979)--(7.477,7.980)%
  --(7.479,7.981)--(7.480,7.981)--(7.482,7.982)--(7.483,7.983)--(7.484,7.984)--(7.486,7.985)%
  --(7.487,7.986)--(7.488,7.986)--(7.489,7.987)--(7.490,7.988)--(7.491,7.988)--(7.492,7.989)%
  --(7.493,7.990)--(7.494,7.991)--(7.496,7.991)--(7.497,7.992)--(7.498,7.993)--(7.499,7.993)%
  --(7.500,7.994)--(7.501,7.995)--(7.502,7.996)--(7.503,7.996)--(7.504,7.997)--(7.506,7.998)%
  --(7.507,7.999)--(7.508,8.000)--(7.510,8.001)--(7.511,8.002)--(7.513,8.003)--(7.515,8.004)%
  --(7.516,8.005)--(7.517,8.006)--(7.519,8.007)--(7.520,8.007)--(7.521,8.008)--(7.522,8.009)%
  --(7.524,8.009)--(7.525,8.010)--(7.526,8.011)--(7.527,8.012)--(7.528,8.012)--(7.529,8.013)%
  --(7.530,8.013)--(7.531,8.014)--(7.532,8.015)--(7.533,8.015)--(7.534,8.016)--(7.536,8.017)%
  --(7.537,8.018)--(7.538,8.018)--(7.539,8.019)--(7.540,8.020)--(7.541,8.020)--(7.543,8.021)%
  --(7.544,8.022)--(7.546,8.023)--(7.548,8.024)--(7.549,8.025)--(7.551,8.026)--(7.552,8.027)%
  --(7.553,8.028)--(7.555,8.029)--(7.556,8.029)--(7.557,8.030)--(7.558,8.031)--(7.559,8.031)%
  --(7.560,8.032)--(7.561,8.033)--(7.563,8.033)--(7.564,8.034)--(7.565,8.035)--(7.566,8.035)%
  --(7.567,8.036)--(7.568,8.037)--(7.569,8.037)--(7.570,8.038)--(7.571,8.039)--(7.572,8.039)%
  --(7.574,8.040)--(7.575,8.041)--(7.576,8.042)--(7.577,8.043)--(7.579,8.044)--(7.581,8.045)%
  --(7.582,8.046)--(7.584,8.047)--(7.585,8.047)--(7.587,8.048)--(7.588,8.049)--(7.590,8.050)%
  --(7.591,8.051)--(7.592,8.051)--(7.593,8.052)--(7.594,8.053)--(7.595,8.053)--(7.596,8.054)%
  --(7.597,8.054)--(7.598,8.055)--(7.599,8.056)--(7.600,8.056)--(7.601,8.057)--(7.602,8.057)%
  --(7.604,8.058)--(7.605,8.059)--(7.606,8.059)--(7.607,8.060)--(7.608,8.061)--(7.609,8.061)%
  --(7.611,8.062)--(7.612,8.063)--(7.613,8.064)--(7.615,8.064)--(7.616,8.065)--(7.618,8.066)%
  --(7.619,8.067)--(7.620,8.068)--(7.622,8.068)--(7.623,8.069)--(7.624,8.070)--(7.625,8.070)%
  --(7.626,8.071)--(7.628,8.072)--(7.629,8.072)--(7.630,8.073)--(7.631,8.074)--(7.632,8.074)%
  --(7.633,8.075)--(7.634,8.076)--(7.635,8.076)--(7.636,8.077)--(7.637,8.077)--(7.638,8.078)%
  --(7.639,8.079)--(7.641,8.079)--(7.642,8.080)--(7.643,8.081)--(7.644,8.081)--(7.645,8.082)%
  --(7.647,8.083)--(7.648,8.084)--(7.650,8.085)--(7.651,8.086)--(7.653,8.086)--(7.654,8.087)%
  --(7.655,8.088)--(7.656,8.088)--(7.657,8.089)--(7.658,8.090)--(7.659,8.090)--(7.660,8.091)%
  --(7.662,8.091)--(7.663,8.092)--(7.664,8.093)--(7.665,8.093)--(7.666,8.094)--(7.667,8.094)%
  --(7.668,8.095)--(7.669,8.095)--(7.670,8.096)--(7.671,8.097)--(7.672,8.097)--(7.673,8.098)%
  --(7.675,8.098)--(7.676,8.099)--(7.677,8.100)--(7.678,8.100)--(7.680,8.101)--(7.681,8.102)%
  --(7.682,8.102)--(7.684,8.103)--(7.685,8.104)--(7.686,8.104)--(7.687,8.105)--(7.689,8.106)%
  --(7.690,8.106)--(7.691,8.107)--(7.692,8.108)--(7.693,8.108)--(7.694,8.109)--(7.696,8.110)%
  --(7.697,8.110)--(7.698,8.111)--(7.699,8.111)--(7.700,8.112)--(7.701,8.113)--(7.702,8.113)%
  --(7.703,8.114)--(7.704,8.114)--(7.705,8.115)--(7.706,8.115)--(7.707,8.116)--(7.709,8.117)%
  --(7.710,8.117)--(7.711,8.118)--(7.712,8.119)--(7.713,8.119)--(7.715,8.120)--(7.716,8.121)%
  --(7.717,8.121)--(7.719,8.122)--(7.720,8.123)--(7.721,8.124)--(7.723,8.124)--(7.724,8.125)%
  --(7.726,8.126)--(7.727,8.126)--(7.728,8.127)--(7.729,8.128)--(7.730,8.128)--(7.731,8.129)%
  --(7.732,8.129)--(7.733,8.130)--(7.734,8.130)--(7.736,8.131)--(7.737,8.131)--(7.738,8.132)%
  --(7.739,8.132)--(7.740,8.133)--(7.741,8.134)--(7.742,8.134)--(7.743,8.135)--(7.744,8.135)%
  --(7.746,8.136)--(7.747,8.136)--(7.748,8.137)--(7.749,8.138)--(7.751,8.138)--(7.752,8.139)%
  --(7.754,8.140)--(7.755,8.141)--(7.757,8.141)--(7.758,8.142)--(7.760,8.143)--(7.761,8.144)%
  --(7.762,8.144)--(7.763,8.145)--(7.765,8.145)--(7.766,8.146)--(7.767,8.146)--(7.768,8.147)%
  --(7.769,8.148)--(7.770,8.148)--(7.771,8.149)--(7.772,8.149)--(7.773,8.150)--(7.774,8.150)%
  --(7.775,8.151)--(7.777,8.151)--(7.778,8.152)--(7.779,8.152)--(7.780,8.153)--(7.781,8.154)%
  --(7.782,8.154)--(7.784,8.155)--(7.785,8.156)--(7.787,8.157)--(7.788,8.157)--(7.790,8.158)%
  --(7.791,8.159)--(7.792,8.159)--(7.793,8.160)--(7.794,8.160)--(7.795,8.161)--(7.797,8.161)%
  --(7.798,8.162)--(7.799,8.162)--(7.800,8.163)--(7.801,8.163)--(7.802,8.164)--(7.803,8.164)%
  --(7.804,8.165)--(7.805,8.165)--(7.807,8.166)--(7.808,8.166)--(7.809,8.167)--(7.810,8.167)%
  --(7.811,8.168)--(7.812,8.169)--(7.813,8.169)--(7.814,8.170)--(7.815,8.170)--(7.817,8.171)%
  --(7.818,8.171)--(7.819,8.172)--(7.821,8.173)--(7.822,8.173)--(7.823,8.174)--(7.825,8.174)%
  --(7.826,8.175)--(7.827,8.175)--(7.828,8.176)--(7.829,8.177)--(7.831,8.177)--(7.832,8.178)%
  --(7.833,8.178)--(7.834,8.179)--(7.835,8.179)--(7.836,8.180)--(7.837,8.180)--(7.838,8.181)%
  --(7.839,8.181)--(7.840,8.182)--(7.841,8.182)--(7.843,8.183)--(7.844,8.183)--(7.845,8.184)%
  --(7.846,8.184)--(7.847,8.185)--(7.848,8.185)--(7.849,8.186)--(7.851,8.186)--(7.852,8.187)%
  --(7.853,8.188)--(7.854,8.188)--(7.856,8.189)--(7.857,8.189)--(7.859,8.190)--(7.860,8.191)%
  --(7.861,8.191)--(7.863,8.192)--(7.864,8.193)--(7.865,8.193)--(7.866,8.194)--(7.868,8.194)%
  --(7.869,8.195)--(7.870,8.195)--(7.871,8.196)--(7.872,8.196)--(7.873,8.197)--(7.874,8.197)%
  --(7.875,8.197)--(7.876,8.198)--(7.877,8.198)--(7.879,8.199)--(7.880,8.199)--(7.881,8.200)%
  --(7.882,8.200)--(7.883,8.201)--(7.884,8.201)--(7.886,8.202)--(7.887,8.202)--(7.888,8.203)%
  --(7.889,8.203)--(7.891,8.204)--(7.892,8.205)--(7.893,8.205)--(7.895,8.206)--(7.896,8.206)%
  --(7.897,8.207)--(7.899,8.207)--(7.900,8.208)--(7.901,8.208)--(7.902,8.209)--(7.903,8.209)%
  --(7.904,8.210)--(7.906,8.210)--(7.907,8.211)--(7.908,8.211)--(7.909,8.212)--(7.910,8.212)%
  --(7.911,8.213)--(7.912,8.213)--(7.913,8.213)--(7.914,8.214)--(7.915,8.214)--(7.916,8.215)%
  --(7.918,8.215)--(7.919,8.216)--(7.920,8.217)--(7.921,8.217)--(7.923,8.218)--(7.924,8.218)%
  --(7.926,8.219)--(7.927,8.220)--(7.929,8.220)--(7.930,8.221)--(7.931,8.221)--(7.932,8.221)%
  --(7.933,8.222)--(7.934,8.222)--(7.935,8.223)--(7.937,8.223)--(7.938,8.224)--(7.939,8.224)%
  --(7.940,8.225)--(7.941,8.225)--(7.942,8.225)--(7.943,8.226)--(7.944,8.226)--(7.945,8.227)%
  --(7.946,8.227)--(7.947,8.228)--(7.948,8.228)--(7.950,8.228)--(7.951,8.229)--(7.952,8.229)%
  --(7.953,8.230)--(7.954,8.230)--(7.955,8.231)--(7.957,8.231)--(7.958,8.232)--(7.960,8.232)%
  --(7.961,8.233)--(7.962,8.233)--(7.963,8.234)--(7.964,8.234)--(7.966,8.235)--(7.967,8.235)%
  --(7.968,8.236)--(7.969,8.236)--(7.970,8.236)--(7.971,8.237)--(7.973,8.237)--(7.974,8.238)%
  --(7.975,8.238)--(7.976,8.239)--(7.977,8.239)--(7.978,8.239)--(7.979,8.240)--(7.980,8.240)%
  --(7.981,8.241)--(7.982,8.241)--(7.984,8.242)--(7.985,8.242)--(7.986,8.242)--(7.987,8.243)%
  --(7.988,8.243)--(7.990,8.244)--(7.991,8.244)--(7.992,8.245)--(7.993,8.245)--(7.995,8.246)%
  --(7.996,8.246)--(7.997,8.247)--(7.999,8.248)--(8.000,8.248)--(8.001,8.248)--(8.003,8.249)%
  --(8.004,8.249)--(8.005,8.250)--(8.006,8.250)--(8.007,8.251)--(8.008,8.251)--(8.009,8.251)%
  --(8.010,8.252)--(8.012,8.252)--(8.013,8.253)--(8.014,8.253)--(8.015,8.253)--(8.016,8.254)%
  --(8.017,8.254)--(8.018,8.254)--(8.019,8.255)--(8.020,8.255)--(8.022,8.256)--(8.023,8.256)%
  --(8.024,8.256)--(8.025,8.257)--(8.027,8.257)--(8.028,8.258)--(8.030,8.259)--(8.032,8.259)%
  --(8.033,8.260)--(8.034,8.260)--(8.036,8.261)--(8.037,8.261)--(8.038,8.261)--(8.039,8.262)%
  --(8.041,8.262)--(8.042,8.263)--(8.043,8.263)--(8.044,8.263)--(8.045,8.264)--(8.046,8.264)%
  --(8.047,8.265)--(8.048,8.265)--(8.049,8.265)--(8.050,8.266)--(8.051,8.266)--(8.052,8.267)%
  --(8.054,8.267)--(8.055,8.267)--(8.056,8.268)--(8.057,8.268)--(8.058,8.269)--(8.060,8.269)%
  --(8.061,8.270)--(8.063,8.270)--(8.064,8.271)--(8.066,8.271)--(8.067,8.272)--(8.069,8.272)%
  --(8.070,8.273)--(8.072,8.273)--(8.073,8.273)--(8.074,8.274)--(8.075,8.274)--(8.076,8.275)%
  --(8.077,8.275)--(8.078,8.275)--(8.079,8.276)--(8.080,8.276)--(8.082,8.276)--(8.083,8.277)%
  --(8.084,8.277)--(8.085,8.277)--(8.086,8.278)--(8.087,8.278)--(8.088,8.278)--(8.089,8.279)%
  --(8.090,8.279)--(8.092,8.279)--(8.093,8.280)--(8.094,8.280)--(8.096,8.281)--(8.097,8.281)%
  --(8.099,8.282)--(8.100,8.282)--(8.102,8.283)--(8.103,8.283)--(8.105,8.283)--(8.106,8.284)%
  --(8.107,8.284)--(8.108,8.285)--(8.109,8.285)--(8.111,8.285)--(8.112,8.286)--(8.113,8.286)%
  --(8.114,8.286)--(8.115,8.287)--(8.116,8.287)--(8.117,8.287)--(8.118,8.288)--(8.119,8.288)%
  --(8.120,8.288)--(8.121,8.289)--(8.123,8.289)--(8.124,8.289)--(8.125,8.290)--(8.126,8.290)%
  --(8.127,8.291)--(8.129,8.291)--(8.130,8.292)--(8.132,8.292)--(8.133,8.292)--(8.135,8.293)%
  --(8.136,8.293)--(8.138,8.294)--(8.139,8.294)--(8.140,8.295)--(8.141,8.295)--(8.143,8.295)%
  --(8.144,8.296)--(8.145,8.296)--(8.146,8.296)--(8.147,8.297)--(8.148,8.297)--(8.149,8.297)%
  --(8.150,8.297)--(8.151,8.298)--(8.153,8.298)--(8.154,8.298)--(8.155,8.299)--(8.156,8.299)%
  --(8.157,8.299)--(8.158,8.299)--(8.159,8.300)--(8.160,8.300)--(8.162,8.301)--(8.163,8.301)%
  --(8.164,8.301)--(8.166,8.302)--(8.168,8.302)--(8.169,8.303)--(8.170,8.303)--(8.171,8.303)%
  --(8.173,8.303)--(8.174,8.304)--(8.175,8.304)--(8.176,8.304)--(8.177,8.305)--(8.178,8.305)%
  --(8.179,8.305)--(8.180,8.306)--(8.181,8.306)--(8.182,8.306)--(8.184,8.307)--(8.185,8.307)%
  --(8.186,8.307)--(8.187,8.307)--(8.188,8.308)--(8.189,8.308)--(8.190,8.308)--(8.191,8.309)%
  --(8.192,8.309)--(8.194,8.309)--(8.195,8.310)--(8.196,8.310)--(8.197,8.310)--(8.199,8.311)%
  --(8.200,8.311)--(8.201,8.311)--(8.203,8.312)--(8.204,8.312)--(8.205,8.312)--(8.206,8.313)%
  --(8.207,8.313)--(8.209,8.313)--(8.210,8.314)--(8.211,8.314)--(8.212,8.314)--(8.213,8.315)%
  --(8.214,8.315)--(8.215,8.315)--(8.216,8.315)--(8.218,8.316)--(8.219,8.316)--(8.220,8.316)%
  --(8.221,8.316)--(8.222,8.317)--(8.223,8.317)--(8.224,8.317)--(8.225,8.317)--(8.226,8.318)%
  --(8.227,8.318)--(8.229,8.318)--(8.230,8.319)--(8.231,8.319)--(8.232,8.319)--(8.234,8.319)%
  --(8.235,8.320)--(8.236,8.320)--(8.238,8.320)--(8.239,8.321)--(8.241,8.321)--(8.242,8.321)%
  --(8.243,8.322)--(8.245,8.322)--(8.246,8.322)--(8.247,8.322)--(8.248,8.323)--(8.249,8.323)%
  --(8.250,8.323)--(8.251,8.324)--(8.252,8.324)--(8.253,8.324)--(8.254,8.324)--(8.255,8.325)%
  --(8.256,8.325)--(8.258,8.325)--(8.259,8.325)--(8.260,8.326)--(8.261,8.326)--(8.262,8.326)%
  --(8.263,8.326)--(8.265,8.327)--(8.266,8.327)--(8.267,8.327)--(8.269,8.328)--(8.270,8.328)%
  --(8.272,8.329)--(8.274,8.329)--(8.275,8.329)--(8.277,8.330)--(8.278,8.330)--(8.279,8.330)%
  --(8.280,8.330)--(8.282,8.331)--(8.283,8.331)--(8.284,8.331)--(8.285,8.331)--(8.286,8.332)%
  --(8.287,8.332)--(8.288,8.332)--(8.289,8.332)--(8.290,8.332)--(8.291,8.333)--(8.292,8.333)%
  --(8.294,8.333)--(8.295,8.333)--(8.296,8.334)--(8.297,8.334)--(8.298,8.334)--(8.299,8.334)%
  --(8.301,8.334)--(8.302,8.335)--(8.303,8.335)--(8.305,8.335)--(8.306,8.336)--(8.307,8.336)%
  --(8.309,8.336)--(8.310,8.336)--(8.312,8.337)--(8.313,8.337)--(8.314,8.337)--(8.315,8.337)%
  --(8.316,8.338)--(8.317,8.338)--(8.319,8.338)--(8.320,8.338)--(8.321,8.338)--(8.322,8.339)%
  --(8.323,8.339)--(8.324,8.339)--(8.325,8.339)--(8.326,8.340)--(8.327,8.340)--(8.328,8.340)%
  --(8.329,8.340)--(8.330,8.340)--(8.332,8.341)--(8.333,8.341)--(8.334,8.341)--(8.335,8.341)%
  --(8.337,8.342)--(8.338,8.342)--(8.340,8.342)--(8.341,8.343)--(8.343,8.343)--(8.344,8.343)%
  --(8.345,8.343)--(8.346,8.344)--(8.347,8.344)--(8.348,8.344)--(8.349,8.344)--(8.351,8.344)%
  --(8.352,8.345)--(8.353,8.345)--(8.354,8.345)--(8.355,8.345)--(8.356,8.345)--(8.357,8.345)%
  --(8.358,8.346)--(8.359,8.346)--(8.360,8.346)--(8.361,8.346)--(8.363,8.346)--(8.364,8.346)%
  --(8.365,8.347)--(8.366,8.347)--(8.367,8.347)--(8.368,8.347)--(8.370,8.347)--(8.371,8.348)%
  --(8.373,8.348)--(8.374,8.348)--(8.375,8.348)--(8.376,8.348)--(8.377,8.349)--(8.379,8.349)%
  --(8.380,8.349)--(8.381,8.349)--(8.382,8.349)--(8.383,8.350)--(8.384,8.350)--(8.385,8.350)%
  --(8.387,8.350)--(8.388,8.350)--(8.389,8.350)--(8.390,8.351)--(8.391,8.351)--(8.392,8.351)%
  --(8.393,8.351)--(8.394,8.351)--(8.395,8.352)--(8.396,8.352)--(8.398,8.352)--(8.399,8.352)%
  --(8.400,8.352)--(8.401,8.353)--(8.402,8.353)--(8.404,8.353)--(8.405,8.353)--(8.406,8.353)%
  --(8.407,8.354)--(8.409,8.354)--(8.410,8.354)--(8.411,8.354)--(8.413,8.354)--(8.414,8.355)%
  --(8.416,8.355)--(8.417,8.355)--(8.418,8.355)--(8.419,8.355)--(8.420,8.355)--(8.421,8.356)%
  --(8.422,8.356)--(8.424,8.356)--(8.425,8.356)--(8.426,8.356)--(8.427,8.356)--(8.428,8.356)%
  --(8.429,8.357)--(8.430,8.357)--(8.431,8.357)--(8.432,8.357)--(8.433,8.357)--(8.435,8.357)%
  --(8.436,8.357)--(8.437,8.358)--(8.438,8.358)--(8.439,8.358)--(8.441,8.358)--(8.443,8.358)%
  --(8.445,8.358)--(8.446,8.359)--(8.448,8.359)--(8.449,8.359)--(8.451,8.359)--(8.452,8.359)%
  --(8.453,8.359)--(8.454,8.360)--(8.455,8.360)--(8.456,8.360)--(8.457,8.360)--(8.458,8.360)%
  --(8.459,8.360)--(8.460,8.360)--(8.462,8.361)--(8.463,8.361)--(8.464,8.361)--(8.465,8.361)%
  --(8.466,8.361)--(8.467,8.361)--(8.468,8.361)--(8.469,8.362)--(8.471,8.362)--(8.472,8.362)%
  --(8.473,8.362)--(8.474,8.362)--(8.476,8.362)--(8.477,8.363)--(8.478,8.363)--(8.480,8.363)%
  --(8.481,8.363)--(8.483,8.363)--(8.484,8.363)--(8.485,8.363)--(8.486,8.364)--(8.487,8.364)%
  --(8.489,8.364)--(8.490,8.364)--(8.491,8.364)--(8.492,8.364)--(8.493,8.364)--(8.494,8.364)%
  --(8.495,8.364)--(8.496,8.365)--(8.497,8.365)--(8.498,8.365)--(8.499,8.365)--(8.500,8.365)%
  --(8.502,8.365)--(8.503,8.365)--(8.504,8.365)--(8.505,8.365)--(8.506,8.365)--(8.508,8.365)%
  --(8.509,8.366)--(8.511,8.366)--(8.512,8.366)--(8.514,8.366)--(8.515,8.366)--(8.516,8.366)%
  --(8.517,8.366)--(8.518,8.366)--(8.520,8.366)--(8.521,8.367)--(8.522,8.367)--(8.523,8.367)%
  --(8.524,8.367)--(8.525,8.367)--(8.526,8.367)--(8.527,8.367)--(8.528,8.367)--(8.529,8.367)%
  --(8.530,8.367)--(8.531,8.368)--(8.533,8.368)--(8.534,8.368)--(8.535,8.368)--(8.536,8.368)%
  --(8.537,8.368)--(8.538,8.368)--(8.540,8.368)--(8.541,8.368)--(8.542,8.369)--(8.544,8.369)%
  --(8.545,8.369)--(8.547,8.369)--(8.548,8.369)--(8.549,8.369)--(8.550,8.369)--(8.551,8.369)%
  --(8.553,8.369)--(8.554,8.370)--(8.555,8.370)--(8.556,8.370)--(8.557,8.370)--(8.558,8.370)%
  --(8.560,8.370)--(8.561,8.370)--(8.562,8.370)--(8.563,8.370)--(8.564,8.370)--(8.565,8.370)%
  --(8.566,8.370)--(8.567,8.370)--(8.568,8.370)--(8.570,8.370)--(8.571,8.371)--(8.572,8.371)%
  --(8.573,8.371)--(8.574,8.371)--(8.576,8.371)--(8.577,8.371)--(8.578,8.371)--(8.580,8.371)%
  --(8.581,8.371)--(8.582,8.371)--(8.584,8.371)--(8.585,8.371)--(8.586,8.371)--(8.587,8.371)%
  --(8.588,8.372)--(8.590,8.372)--(8.591,8.372)--(8.592,8.372)--(8.593,8.372)--(8.594,8.372)%
  --(8.595,8.372)--(8.597,8.372)--(8.598,8.372)--(8.599,8.372)--(8.600,8.372)--(8.601,8.372)%
  --(8.602,8.372)--(8.603,8.372)--(8.604,8.373)--(8.605,8.373)--(8.607,8.373)--(8.608,8.373)%
  --(8.609,8.373)--(8.611,8.373)--(8.612,8.373)--(8.613,8.373)--(8.615,8.373)--(8.616,8.373)%
  --(8.617,8.373)--(8.619,8.373)--(8.620,8.374)--(8.621,8.374)--(8.623,8.374)--(8.624,8.374)%
  --(8.625,8.374)--(8.626,8.374)--(8.627,8.374)--(8.629,8.374)--(8.630,8.374)--(8.631,8.374)%
  --(8.632,8.374)--(8.633,8.374)--(8.634,8.374)--(8.635,8.374)--(8.636,8.374)--(8.638,8.374)%
  --(8.639,8.374)--(8.640,8.374)--(8.641,8.374)--(8.642,8.374)--(8.643,8.374)--(8.645,8.375)%
  --(8.646,8.375)--(8.648,8.375)--(8.649,8.375)--(8.650,8.375)--(8.652,8.375)--(8.653,8.375)%
  --(8.654,8.375)--(8.656,8.375)--(8.657,8.375)--(8.659,8.375)--(8.660,8.375)--(8.661,8.375)%
  --(8.662,8.375)--(8.663,8.375)--(8.664,8.375)--(8.665,8.375)--(8.666,8.375)--(8.668,8.375)%
  --(8.669,8.376)--(8.670,8.376)--(8.671,8.376)--(8.672,8.376)--(8.673,8.376)--(8.675,8.376)%
  --(8.676,8.376)--(8.677,8.376)--(8.678,8.376)--(8.680,8.376)--(8.681,8.376)--(8.682,8.376)%
  --(8.684,8.376)--(8.685,8.376)--(8.687,8.376)--(8.688,8.376)--(8.690,8.376)--(8.691,8.376)%
  --(8.692,8.377)--(8.694,8.377)--(8.695,8.377)--(8.696,8.377)--(8.697,8.377)--(8.699,8.377)%
  --(8.700,8.377)--(8.701,8.377)--(8.702,8.377)--(8.703,8.377)--(8.704,8.377)--(8.705,8.377)%
  --(8.706,8.377)--(8.707,8.377)--(8.709,8.377)--(8.710,8.377)--(8.711,8.377)--(8.712,8.377)%
  --(8.714,8.377)--(8.715,8.377)--(8.717,8.377)--(8.718,8.377)--(8.719,8.377)--(8.721,8.377)%
  --(8.722,8.377)--(8.723,8.377)--(8.724,8.377)--(8.726,8.377)--(8.727,8.377)--(8.728,8.377)%
  --(8.730,8.377)--(8.731,8.377)--(8.732,8.377)--(8.733,8.378)--(8.734,8.378)--(8.735,8.378)%
  --(8.736,8.378)--(8.737,8.378)--(8.739,8.378)--(8.740,8.378)--(8.741,8.378)--(8.742,8.378)%
  --(8.743,8.378)--(8.745,8.378)--(8.746,8.378)--(8.747,8.378)--(8.748,8.378)--(8.750,8.378)%
  --(8.751,8.378)--(8.753,8.378)--(8.754,8.378)--(8.755,8.378)--(8.757,8.378)--(8.758,8.378)%
  --(8.759,8.378)--(8.761,8.378)--(8.762,8.378)--(8.763,8.378)--(8.764,8.378)--(8.765,8.378)%
  --(8.767,8.378)--(8.768,8.378)--(8.769,8.378)--(8.770,8.379)--(8.771,8.379)--(8.772,8.379)%
  --(8.774,8.379)--(8.775,8.379)--(8.776,8.379)--(8.777,8.379)--(8.778,8.379)--(8.780,8.379)%
  --(8.781,8.379)--(8.782,8.379)--(8.784,8.379)--(8.785,8.379)--(8.787,8.379)--(8.788,8.379)%
  --(8.790,8.379)--(8.791,8.379)--(8.792,8.379)--(8.793,8.379)--(8.794,8.379)--(8.796,8.379)%
  --(8.797,8.379)--(8.798,8.379)--(8.799,8.379)--(8.801,8.379)--(8.802,8.379)--(8.803,8.379)%
  --(8.804,8.379)--(8.805,8.379)--(8.806,8.379)--(8.808,8.379)--(8.809,8.379)--(8.810,8.379)%
  --(8.811,8.379)--(8.812,8.379)--(8.814,8.379)--(8.815,8.379)--(8.816,8.379)--(8.818,8.379)%
  --(8.819,8.379)--(8.821,8.379)--(8.822,8.379)--(8.824,8.379)--(8.825,8.379)--(8.826,8.379)%
  --(8.828,8.379)--(8.829,8.379)--(8.830,8.380)--(8.831,8.380)--(8.833,8.380)--(8.834,8.380)%
  --(8.835,8.380)--(8.836,8.380)--(8.837,8.380)--(8.839,8.380)--(8.840,8.380)--(8.841,8.380)%
  --(8.842,8.380)--(8.843,8.380)--(8.845,8.380)--(8.846,8.380)--(8.847,8.380)--(8.848,8.380)%
  --(8.850,8.380)--(8.851,8.380)--(8.852,8.380)--(8.854,8.380)--(8.855,8.380)--(8.857,8.380)%
  --(8.858,8.380)--(8.859,8.380)--(8.861,8.380)--(8.862,8.380)--(8.864,8.380)--(8.865,8.380)%
  --(8.867,8.380)--(8.868,8.380)--(8.869,8.380)--(8.870,8.380)--(8.872,8.380)--(8.873,8.380)%
  --(8.874,8.380)--(8.875,8.380)--(8.876,8.380)--(8.877,8.380)--(8.879,8.380)--(8.880,8.380)%
  --(8.881,8.380)--(8.882,8.380)--(8.884,8.380)--(8.885,8.380)--(8.886,8.380)--(8.888,8.380)%
  --(8.889,8.380)--(8.891,8.380)--(8.892,8.380)--(8.894,8.380)--(8.895,8.380)--(8.896,8.380)%
  --(8.898,8.380)--(8.899,8.380)--(8.900,8.380)--(8.902,8.380)--(8.903,8.380)--(8.904,8.380)%
  --(8.905,8.380)--(8.906,8.380)--(8.908,8.380)--(8.909,8.380)--(8.910,8.380)--(8.911,8.380)%
  --(8.913,8.380)--(8.914,8.380)--(8.915,8.380)--(8.916,8.380)--(8.917,8.380)--(8.919,8.380)%
  --(8.920,8.380)--(8.921,8.380)--(8.923,8.380)--(8.925,8.380)--(8.927,8.380)--(8.929,8.380)%
  --(8.930,8.380)--(8.931,8.380)--(8.933,8.380)--(8.935,8.380)--(8.936,8.380)--(8.937,8.380)%
  --(8.938,8.380)--(8.940,8.380)--(8.941,8.380)--(8.942,8.380)--(8.943,8.380)--(8.945,8.380)%
  --(8.946,8.380)--(8.947,8.380)--(8.948,8.380)--(8.950,8.380)--(8.951,8.380)--(8.952,8.380)%
  --(8.954,8.380)--(8.955,8.380)--(8.957,8.380)--(8.958,8.380)--(8.960,8.380)--(8.961,8.380)%
  --(8.963,8.380)--(8.964,8.380)--(8.965,8.380)--(8.967,8.381)--(8.968,8.381)--(8.969,8.381)%
  --(8.971,8.381)--(8.972,8.381)--(8.973,8.381)--(8.974,8.381)--(8.975,8.381)--(8.977,8.381)%
  --(8.978,8.381)--(8.979,8.381)--(8.980,8.381)--(8.982,8.381)--(8.983,8.381)--(8.984,8.381)%
  --(8.985,8.381)--(8.987,8.380)--(8.988,8.380)--(8.990,8.380)--(8.991,8.380)--(8.992,8.380)%
  --(8.994,8.380)--(8.996,8.380)--(8.997,8.380)--(8.999,8.380)--(9.000,8.380)--(9.002,8.380)%
  --(9.003,8.380)--(9.005,8.380)--(9.006,8.380)--(9.007,8.380)--(9.009,8.381)--(9.010,8.381)%
  --(9.011,8.381)--(9.012,8.381)--(9.014,8.381)--(9.015,8.381)--(9.016,8.381)--(9.017,8.381)%
  --(9.019,8.381)--(9.020,8.381)--(9.021,8.381)--(9.023,8.381)--(9.024,8.381)--(9.025,8.381)%
  --(9.027,8.381)--(9.029,8.381)--(9.031,8.381)--(9.033,8.381)--(9.034,8.381)--(9.036,8.381)%
  --(9.038,8.381)--(9.039,8.381)--(9.041,8.381)--(9.042,8.381)--(9.043,8.381)--(9.044,8.381)%
  --(9.046,8.381)--(9.047,8.381)--(9.048,8.381)--(9.049,8.381)--(9.051,8.381)--(9.052,8.381)%
  --(9.053,8.381)--(9.055,8.381)--(9.056,8.381)--(9.058,8.381)--(9.059,8.381)--(9.060,8.381)%
  --(9.062,8.381)--(9.064,8.381)--(9.066,8.381)--(9.068,8.381)--(9.069,8.381)--(9.071,8.381)%
  --(9.073,8.381)--(9.074,8.381)--(9.076,8.381)--(9.077,8.381)--(9.078,8.381)--(9.079,8.381)%
  --(9.081,8.381)--(9.082,8.381)--(9.083,8.381)--(9.085,8.381)--(9.086,8.381)--(9.087,8.381)%
  --(9.089,8.381)--(9.090,8.381)--(9.091,8.381)--(9.093,8.381)--(9.094,8.381)--(9.096,8.381)%
  --(9.097,8.381)--(9.099,8.381)--(9.100,8.381)--(9.102,8.381)--(9.104,8.381)--(9.105,8.381)%
  --(9.107,8.381)--(9.108,8.381)--(9.110,8.381)--(9.111,8.381)--(9.112,8.381)--(9.114,8.381)%
  --(9.115,8.381)--(9.116,8.381)--(9.118,8.381)--(9.119,8.381)--(9.120,8.381)--(9.121,8.381)%
  --(9.123,8.381)--(9.124,8.381)--(9.126,8.381)--(9.127,8.381)--(9.129,8.381)--(9.130,8.381)%
  --(9.132,8.381)--(9.133,8.381)--(9.135,8.381)--(9.137,8.381)--(9.138,8.381)--(9.140,8.381)%
  --(9.141,8.381)--(9.143,8.381)--(9.144,8.381)--(9.146,8.381)--(9.147,8.381)--(9.148,8.381)%
  --(9.150,8.381)--(9.151,8.381)--(9.152,8.381)--(9.154,8.381)--(9.155,8.381)--(9.156,8.381)%
  --(9.158,8.381)--(9.159,8.381)--(9.161,8.381)--(9.162,8.381)--(9.164,8.381)--(9.165,8.381)%
  --(9.167,8.381)--(9.168,8.381)--(9.170,8.381)--(9.172,8.381)--(9.173,8.381)--(9.175,8.381)%
  --(9.177,8.381)--(9.178,8.381)--(9.180,8.381)--(9.181,8.381)--(9.182,8.381)--(9.184,8.381)%
  --(9.185,8.381)--(9.186,8.381)--(9.188,8.381)--(9.189,8.381)--(9.190,8.381)--(9.192,8.381)%
  --(9.193,8.381)--(9.194,8.381)--(9.196,8.381)--(9.198,8.381)--(9.199,8.381)--(9.201,8.381)%
  --(9.202,8.381)--(9.204,8.381)--(9.205,8.381)--(9.207,8.381)--(9.209,8.381)--(9.210,8.381)%
  --(9.212,8.381)--(9.213,8.381)--(9.215,8.381)--(9.216,8.381)--(9.218,8.381)--(9.219,8.381)%
  --(9.220,8.381)--(9.222,8.381)--(9.223,8.381)--(9.224,8.381)--(9.226,8.381)--(9.227,8.381)%
  --(9.229,8.381)--(9.230,8.381)--(9.232,8.381)--(9.233,8.381)--(9.235,8.381)--(9.236,8.381)%
  --(9.238,8.381)--(9.239,8.381)--(9.241,8.381)--(9.243,8.381)--(9.244,8.381)--(9.246,8.381)%
  --(9.247,8.381)--(9.249,8.381)--(9.250,8.381)--(9.252,8.381)--(9.253,8.381)--(9.254,8.381)%
  --(9.256,8.381)--(9.257,8.381)--(9.258,8.381)--(9.260,8.381)--(9.261,8.381)--(9.263,8.381)%
  --(9.264,8.381)--(9.266,8.381)--(9.267,8.381)--(9.269,8.381)--(9.271,8.381)--(9.272,8.381)%
  --(9.274,8.381)--(9.275,8.381)--(9.277,8.381)--(9.278,8.381)--(9.280,8.381)--(9.281,8.381)%
  --(9.283,8.381)--(9.285,8.381)--(9.286,8.381)--(9.287,8.381)--(9.289,8.381)--(9.290,8.381)%
  --(9.291,8.381)--(9.293,8.381)--(9.294,8.381)--(9.296,8.381)--(9.297,8.381)--(9.299,8.381)%
  --(9.300,8.381)--(9.302,8.381)--(9.303,8.381)--(9.305,8.381)--(9.307,8.381)--(9.309,8.381)%
  --(9.310,8.381)--(9.312,8.381)--(9.313,8.381)--(9.315,8.381)--(9.316,8.381)--(9.318,8.381)%
  --(9.319,8.381)--(9.321,8.381)--(9.322,8.381)--(9.324,8.381)--(9.325,8.381)--(9.326,8.381)%
  --(9.328,8.381)--(9.329,8.381)--(9.331,8.381)--(9.332,8.381)--(9.334,8.381)--(9.335,8.381)%
  --(9.337,8.381)--(9.338,8.381)--(9.340,8.381)--(9.342,8.381)--(9.344,8.381)--(9.345,8.381)%
  --(9.347,8.381)--(9.349,8.381)--(9.351,8.381)--(9.352,8.381)--(9.354,8.381)--(9.355,8.381)%
  --(9.357,8.381)--(9.358,8.381)--(9.360,8.381)--(9.361,8.381)--(9.362,8.381)--(9.364,8.381)%
  --(9.365,8.381)--(9.367,8.381)--(9.368,8.381)--(9.370,8.381)--(9.372,8.381)--(9.374,8.381)%
  --(9.375,8.381)--(9.377,8.381)--(9.379,8.381)--(9.380,8.381)--(9.382,8.381)--(9.383,8.381)%
  --(9.385,8.381)--(9.386,8.381)--(9.388,8.381)--(9.389,8.381)--(9.391,8.381)--(9.392,8.381)%
  --(9.394,8.381)--(9.395,8.381)--(9.397,8.381)--(9.398,8.381)--(9.399,8.381)--(9.401,8.381)%
  --(9.403,8.381)--(9.404,8.381)--(9.406,8.381)--(9.408,8.381)--(9.409,8.381)--(9.411,8.381)%
  --(9.413,8.381)--(9.415,8.381)--(9.416,8.381)--(9.418,8.381)--(9.420,8.381)--(9.422,8.381)%
  --(9.423,8.381)--(9.425,8.381)--(9.426,8.381)--(9.428,8.381)--(9.429,8.381)--(9.431,8.381)%
  --(9.432,8.381)--(9.434,8.381)--(9.435,8.381)--(9.437,8.381)--(9.438,8.381)--(9.440,8.381)%
  --(9.442,8.381)--(9.444,8.381)--(9.446,8.381)--(9.448,8.381)--(9.449,8.381)--(9.451,8.381)%
  --(9.452,8.381)--(9.454,8.381)--(9.455,8.381)--(9.457,8.381)--(9.458,8.381)--(9.460,8.381)%
  --(9.461,8.381)--(9.463,8.381)--(9.464,8.381)--(9.466,8.381)--(9.467,8.381)--(9.469,8.381)%
  --(9.470,8.381)--(9.472,8.381)--(9.474,8.381)--(9.475,8.381)--(9.477,8.381)--(9.479,8.381)%
  --(9.481,8.381)--(9.482,8.381)--(9.484,8.381)--(9.486,8.381)--(9.487,8.381)--(9.489,8.381)%
  --(9.491,8.381)--(9.492,8.381)--(9.494,8.381)--(9.495,8.381)--(9.497,8.381)--(9.499,8.381)%
  --(9.500,8.381)--(9.502,8.381)--(9.503,8.381)--(9.505,8.381)--(9.506,8.381)--(9.508,8.381)%
  --(9.510,8.381)--(9.511,8.381)--(9.513,8.381)--(9.515,8.381)--(9.517,8.381)--(9.519,8.381)%
  --(9.520,8.381)--(9.522,8.381)--(9.524,8.381)--(9.526,8.381)--(9.527,8.381)--(9.529,8.381)%
  --(9.531,8.381)--(9.532,8.381)--(9.534,8.381)--(9.535,8.381)--(9.537,8.381)--(9.538,8.381)%
  --(9.540,8.381)--(9.542,8.381)--(9.543,8.381)--(9.545,8.381)--(9.547,8.381)--(9.549,8.381)%
  --(9.551,8.381)--(9.554,8.381)--(9.555,8.381)--(9.557,8.381)--(9.559,8.381)--(9.561,8.381)%
  --(9.563,8.381)--(9.564,8.381)--(9.566,8.381)--(9.568,8.381)--(9.569,8.381)--(9.571,8.381)%
  --(9.572,8.381)--(9.574,8.381)--(9.576,8.381)--(9.577,8.381)--(9.579,8.381)--(9.581,8.381)%
  --(9.583,8.381)--(9.584,8.381)--(9.586,8.381)--(9.588,8.381)--(9.590,8.381)--(9.592,8.381)%
  --(9.594,8.381)--(9.596,8.381)--(9.597,8.381)--(9.599,8.381)--(9.601,8.381)--(9.602,8.381)%
  --(9.604,8.381)--(9.605,8.381)--(9.607,8.381)--(9.608,8.381)--(9.610,8.381)--(9.612,8.381)%
  --(9.614,8.381)--(9.615,8.381)--(9.617,8.381)--(9.619,8.381)--(9.621,8.381)--(9.623,8.381)%
  --(9.625,8.381)--(9.627,8.381)--(9.629,8.381)--(9.631,8.381)--(9.632,8.381)--(9.634,8.381)%
  --(9.635,8.381)--(9.637,8.381)--(9.639,8.381)--(9.640,8.381)--(9.642,8.381)--(9.643,8.381)%
  --(9.645,8.381)--(9.647,8.381)--(9.649,8.381)--(9.651,8.381)--(9.653,8.381)--(9.655,8.381)%
  --(9.656,8.381)--(9.658,8.381)--(9.660,8.381)--(9.662,8.381)--(9.664,8.381)--(9.666,8.381)%
  --(9.668,8.381)--(9.669,8.381)--(9.671,8.381)--(9.672,8.381)--(9.674,8.381)--(9.676,8.381)%
  --(9.677,8.381)--(9.679,8.381)--(9.681,8.381)--(9.683,8.381)--(9.685,8.381)--(9.686,8.381)%
  --(9.688,8.381)--(9.690,8.381)--(9.692,8.381)--(9.694,8.381)--(9.696,8.381)--(9.697,8.381)%
  --(9.699,8.381)--(9.701,8.381)--(9.703,8.381)--(9.704,8.381)--(9.706,8.381)--(9.708,8.381)%
  --(9.709,8.381)--(9.711,8.381)--(9.713,8.381)--(9.714,8.381)--(9.716,8.381)--(9.718,8.381)%
  --(9.720,8.381)--(9.722,8.381)--(9.724,8.381)--(9.726,8.381)--(9.728,8.381)--(9.729,8.381)%
  --(9.731,8.381)--(9.733,8.381)--(9.735,8.381)--(9.737,8.381)--(9.738,8.381)--(9.740,8.381)%
  --(9.742,8.381)--(9.743,8.381)--(9.745,8.381)--(9.747,8.381)--(9.748,8.381)--(9.750,8.381)%
  --(9.752,8.381)--(9.754,8.381)--(9.756,8.381)--(9.758,8.381)--(9.760,8.381)--(9.762,8.381)%
  --(9.764,8.381)--(9.765,8.381)--(9.767,8.381)--(9.769,8.381)--(9.771,8.381)--(9.773,8.381)%
  --(9.774,8.381)--(9.776,8.381)--(9.778,8.381)--(9.779,8.381)--(9.781,8.381)--(9.783,8.381)%
  --(9.785,8.381)--(9.786,8.381)--(9.788,8.381)--(9.791,8.381)--(9.793,8.381)--(9.795,8.381)%
  --(9.797,8.381)--(9.799,8.381)--(9.802,8.381)--(9.804,8.381)--(9.805,8.381)--(9.807,8.381)%
  --(9.809,8.381)--(9.811,8.381)--(9.812,8.381)--(9.814,8.381)--(9.816,8.381)--(9.818,8.381)%
  --(9.820,8.381)--(9.822,8.381)--(9.824,8.381)--(9.825,8.381)--(9.827,8.381)--(9.829,8.381)%
  --(9.831,8.381)--(9.833,8.381)--(9.835,8.381)--(9.837,8.381)--(9.839,8.381)--(9.841,8.381)%
  --(9.843,8.381)--(9.844,8.381)--(9.846,8.381)--(9.848,8.381)--(9.850,8.381)--(9.851,8.381)%
  --(9.853,8.381)--(9.855,8.381)--(9.857,8.381)--(9.859,8.381)--(9.862,8.381)--(9.864,8.381)%
  --(9.866,8.381)--(9.867,8.381)--(9.869,8.381)--(9.871,8.381)--(9.873,8.381)--(9.875,8.381)%
  --(9.876,8.381)--(9.878,8.381)--(9.880,8.381)--(9.882,8.381)--(9.883,8.381)--(9.885,8.381)%
  --(9.887,8.381)--(9.889,8.381)--(9.891,8.381)--(9.893,8.381)--(9.895,8.381)--(9.897,8.381)%
  --(9.899,8.381)--(9.901,8.381)--(9.903,8.381)--(9.905,8.381)--(9.907,8.381)--(9.910,8.381)%
  --(9.911,8.381)--(9.913,8.381)--(9.915,8.381)--(9.917,8.381)--(9.919,8.381)--(9.920,8.381)%
  --(9.922,8.381)--(9.924,8.381)--(9.926,8.381)--(9.929,8.381)--(9.931,8.381)--(9.933,8.381)%
  --(9.935,8.381)--(9.937,8.381)--(9.938,8.381)--(9.940,8.381)--(9.942,8.381)--(9.944,8.381)%
  --(9.946,8.381)--(9.948,8.381)--(9.950,8.381)--(9.951,8.381)--(9.953,8.381)--(9.955,8.381)%
  --(9.957,8.381)--(9.959,8.381)--(9.961,8.381)--(9.963,8.381)--(9.965,8.381)--(9.967,8.381)%
  --(9.969,8.381)--(9.971,8.381)--(9.974,8.381)--(9.976,8.381)--(9.978,8.381)--(9.980,8.381)%
  --(9.982,8.381)--(9.984,8.381)--(9.985,8.381)--(9.987,8.381)--(9.989,8.381)--(9.991,8.381)%
  --(9.993,8.381)--(9.995,8.381)--(9.998,8.381)--(10.000,8.381)--(10.002,8.381)--(10.005,8.381)%
  --(10.007,8.381)--(10.009,8.381)--(10.012,8.381)--(10.014,8.381)--(10.016,8.381)--(10.018,8.381)%
  --(10.019,8.381)--(10.021,8.381)--(10.023,8.381)--(10.025,8.381)--(10.027,8.381)--(10.029,8.381)%
  --(10.032,8.381)--(10.034,8.381)--(10.036,8.381)--(10.039,8.381)--(10.041,8.381)--(10.043,8.381)%
  --(10.046,8.381)--(10.048,8.381)--(10.050,8.381)--(10.052,8.381)--(10.054,8.381)--(10.056,8.381)%
  --(10.058,8.381)--(10.060,8.381)--(10.062,8.381)--(10.063,8.381)--(10.066,8.381)--(10.068,8.381)%
  --(10.071,8.381)--(10.073,8.381)--(10.075,8.381)--(10.078,8.381)--(10.080,8.381)--(10.083,8.381)%
  --(10.085,8.381)--(10.086,8.381)--(10.088,8.381)--(10.090,8.381)--(10.092,8.381)--(10.094,8.381)%
  --(10.096,8.381)--(10.098,8.381)--(10.101,8.381)--(10.103,8.381)--(10.106,8.381)--(10.108,8.381)%
  --(10.110,8.381)--(10.113,8.381)--(10.115,8.381)--(10.118,8.381)--(10.120,8.381)--(10.121,8.381)%
  --(10.124,8.381)--(10.126,8.381)--(10.127,8.381)--(10.130,8.381)--(10.132,8.381)--(10.134,8.381)%
  --(10.136,8.381)--(10.139,8.381)--(10.141,8.381)--(10.144,8.381)--(10.146,8.381)--(10.148,8.381)%
  --(10.151,8.381)--(10.153,8.381)--(10.155,8.381)--(10.157,8.381)--(10.159,8.381)--(10.161,8.381)%
  --(10.163,8.381)--(10.165,8.381)--(10.167,8.381)--(10.170,8.381)--(10.173,8.381)--(10.175,8.381)%
  --(10.178,8.381)--(10.181,8.381)--(10.183,8.381)--(10.185,8.381)--(10.188,8.381)--(10.190,8.381)%
  --(10.192,8.381)--(10.194,8.381)--(10.196,8.381)--(10.198,8.381)--(10.200,8.381)--(10.203,8.381)%
  --(10.205,8.381)--(10.207,8.381)--(10.209,8.381)--(10.211,8.381)--(10.214,8.381)--(10.216,8.381)%
  --(10.218,8.381)--(10.220,8.381)--(10.222,8.381)--(10.225,8.381)--(10.227,8.381)--(10.229,8.381)%
  --(10.231,8.381)--(10.233,8.381)--(10.235,8.381)--(10.237,8.381)--(10.239,8.381)--(10.242,8.381)%
  --(10.244,8.381)--(10.246,8.381)--(10.248,8.381)--(10.251,8.381)--(10.253,8.381)--(10.255,8.381)%
  --(10.257,8.381)--(10.260,8.381)--(10.262,8.381)--(10.264,8.381)--(10.266,8.381)--(10.268,8.381)%
  --(10.270,8.381)--(10.272,8.381)--(10.275,8.381)--(10.277,8.381)--(10.279,8.381)--(10.282,8.381)%
  --(10.284,8.381)--(10.286,8.381)--(10.288,8.381)--(10.290,8.381)--(10.293,8.381)--(10.295,8.381)%
  --(10.297,8.381)--(10.299,8.381)--(10.301,8.381)--(10.303,8.381)--(10.306,8.381)--(10.308,8.381)%
  --(10.310,8.381)--(10.313,8.381)--(10.315,8.381)--(10.317,8.381)--(10.319,8.381)--(10.321,8.381)%
  --(10.324,8.381)--(10.326,8.381)--(10.328,8.381)--(10.330,8.381)--(10.333,8.381)--(10.335,8.381)%
  --(10.337,8.381)--(10.339,8.381)--(10.341,8.381)--(10.344,8.381)--(10.346,8.381)--(10.349,8.381)%
  --(10.351,8.381)--(10.353,8.381)--(10.355,8.381)--(10.357,8.381)--(10.360,8.381)--(10.362,8.381)%
  --(10.364,8.381)--(10.366,8.381)--(10.368,8.381)--(10.371,8.381)--(10.373,8.381)--(10.375,8.381)%
  --(10.378,8.381)--(10.380,8.381)--(10.383,8.381)--(10.385,8.381)--(10.388,8.381)--(10.391,8.381)%
  --(10.394,8.381)--(10.396,8.381)--(10.398,8.381)--(10.401,8.381)--(10.403,8.381)--(10.405,8.381)%
  --(10.407,8.381)--(10.410,8.381)--(10.412,8.381)--(10.414,8.381)--(10.417,8.381)--(10.419,8.381)%
  --(10.422,8.381)--(10.424,8.381)--(10.427,8.381)--(10.429,8.381)--(10.432,8.381)--(10.434,8.381)%
  --(10.436,8.381)--(10.438,8.381)--(10.441,8.381)--(10.443,8.381)--(10.445,8.381)--(10.448,8.381)%
  --(10.451,8.381)--(10.453,8.381)--(10.455,8.381)--(10.458,8.381)--(10.460,8.381)--(10.462,8.381)%
  --(10.464,8.381)--(10.467,8.381)--(10.469,8.381)--(10.471,8.381)--(10.473,8.381)--(10.476,8.381)%
  --(10.478,8.381)--(10.480,8.381)--(10.483,8.381)--(10.486,8.381)--(10.489,8.381)--(10.492,8.381)%
  --(10.495,8.381)--(10.497,8.381)--(10.500,8.381)--(10.503,8.381)--(10.505,8.381)--(10.507,8.381)%
  --(10.509,8.381)--(10.512,8.381)--(10.514,8.381)--(10.517,8.381)--(10.519,8.381)--(10.522,8.381)%
  --(10.524,8.381)--(10.527,8.381)--(10.529,8.381)--(10.531,8.381)--(10.534,8.381)--(10.536,8.381)%
  --(10.538,8.381)--(10.541,8.381)--(10.543,8.381)--(10.546,8.381)--(10.548,8.381)--(10.550,8.381)%
  --(10.553,8.381)--(10.556,8.381)--(10.560,8.381)--(10.563,8.381)--(10.565,8.381)--(10.568,8.381)%
  --(10.571,8.381)--(10.573,8.381)--(10.576,8.381)--(10.578,8.381)--(10.580,8.381)--(10.583,8.381)%
  --(10.585,8.381)--(10.588,8.381)--(10.591,8.381)--(10.594,8.381)--(10.597,8.381)--(10.599,8.381)%
  --(10.602,8.381)--(10.605,8.381)--(10.608,8.381)--(10.610,8.381)--(10.613,8.381)--(10.615,8.381)%
  --(10.618,8.381)--(10.620,8.381)--(10.623,8.381)--(10.625,8.381)--(10.628,8.381)--(10.630,8.381)%
  --(10.633,8.381)--(10.635,8.381)--(10.638,8.381)--(10.640,8.381)--(10.643,8.381)--(10.645,8.381)%
  --(10.648,8.381)--(10.650,8.381)--(10.653,8.381)--(10.655,8.381)--(10.658,8.381)--(10.662,8.381)%
  --(10.665,8.381)--(10.668,8.381)--(10.671,8.381)--(10.674,8.381)--(10.677,8.381)--(10.679,8.381)%
  --(10.682,8.381)--(10.684,8.381)--(10.687,8.381)--(10.689,8.381)--(10.692,8.381)--(10.696,8.381)%
  --(10.699,8.381)--(10.702,8.381)--(10.705,8.381)--(10.708,8.381)--(10.711,8.381)--(10.713,8.381)%
  --(10.716,8.381)--(10.718,8.381)--(10.721,8.381)--(10.723,8.381)--(10.727,8.381)--(10.730,8.381)%
  --(10.733,8.381)--(10.736,8.381)--(10.739,8.381)--(10.742,8.381)--(10.745,8.381)--(10.748,8.381)%
  --(10.750,8.381)--(10.753,8.381)--(10.755,8.381)--(10.758,8.381)--(10.761,8.381)--(10.764,8.381)%
  --(10.767,8.381)--(10.770,8.381)--(10.773,8.381)--(10.776,8.381)--(10.779,8.381)--(10.782,8.381)%
  --(10.785,8.381)--(10.787,8.381)--(10.790,8.381)--(10.792,8.381)--(10.795,8.381)--(10.799,8.381)%
  --(10.802,8.381)--(10.805,8.381)--(10.808,8.381)--(10.811,8.381)--(10.814,8.381)--(10.817,8.381)%
  --(10.819,8.381)--(10.822,8.381)--(10.825,8.381)--(10.827,8.381)--(10.831,8.381)--(10.834,8.381)%
  --(10.837,8.381)--(10.840,8.381)--(10.843,8.381)--(10.846,8.381)--(10.849,8.381)--(10.852,8.381)%
  --(10.855,8.381)--(10.858,8.381)--(10.860,8.381)--(10.863,8.381)--(10.866,8.381)--(10.870,8.381)%
  --(10.873,8.381)--(10.877,8.381)--(10.880,8.381)--(10.883,8.381)--(10.886,8.381)--(10.889,8.381)%
  --(10.891,8.381)--(10.894,8.381)--(10.897,8.381)--(10.899,8.381)--(10.902,8.381)--(10.905,8.381)%
  --(10.908,8.381)--(10.911,8.381)--(10.914,8.381)--(10.917,8.381)--(10.920,8.381)--(10.923,8.381)%
  --(10.926,8.381)--(10.928,8.381)--(10.931,8.381)--(10.934,8.381)--(10.937,8.381)--(10.940,8.381)%
  --(10.943,8.381)--(10.946,8.381)--(10.949,8.381)--(10.951,8.381)--(10.954,8.381)--(10.957,8.381)%
  --(10.960,8.381)--(10.963,8.381)--(10.966,8.381)--(10.968,8.381)--(10.971,8.381)--(10.975,8.381)%
  --(10.978,8.381)--(10.981,8.381)--(10.984,8.381)--(10.987,8.381)--(10.990,8.381)--(10.992,8.381)%
  --(10.995,8.381)--(10.998,8.381)--(11.001,8.381)--(11.004,8.381)--(11.007,8.381)--(11.010,8.381)%
  --(11.013,8.381)--(11.016,8.381)--(11.019,8.381)--(11.022,8.381)--(11.025,8.381)--(11.028,8.381)%
  --(11.031,8.381)--(11.034,8.381)--(11.037,8.381)--(11.039,8.381)--(11.042,8.381)--(11.046,8.381)%
  --(11.049,8.381)--(11.052,8.381)--(11.055,8.381)--(11.058,8.381)--(11.060,8.381)--(11.064,8.381)%
  --(11.067,8.381)--(11.069,8.381)--(11.072,8.381)--(11.075,8.381)--(11.078,8.381)--(11.081,8.381)%
  --(11.084,8.381)--(11.087,8.381)--(11.090,8.381)--(11.093,8.381)--(11.096,8.381)--(11.099,8.381)%
  --(11.102,8.381)--(11.105,8.381)--(11.108,8.381)--(11.111,8.381)--(11.114,8.381)--(11.117,8.381)%
  --(11.120,8.381)--(11.123,8.381)--(11.126,8.381)--(11.129,8.381)--(11.132,8.381)--(11.135,8.381)%
  --(11.138,8.381)--(11.141,8.381)--(11.144,8.381)--(11.148,8.381)--(11.151,8.381)--(11.154,8.381)%
  --(11.157,8.381)--(11.160,8.381)--(11.163,8.381)--(11.165,8.381)--(11.168,8.381)--(11.171,8.381)%
  --(11.175,8.381)--(11.178,8.381)--(11.181,8.381)--(11.185,8.381)--(11.188,8.381)--(11.192,8.381)%
  --(11.195,8.381)--(11.199,8.381)--(11.202,8.381)--(11.205,8.381)--(11.208,8.381)--(11.211,8.381)%
  --(11.214,8.381)--(11.218,8.381)--(11.221,8.381)--(11.224,8.381)--(11.227,8.381)--(11.231,8.381)%
  --(11.234,8.381)--(11.237,8.381)--(11.240,8.381)--(11.243,8.381)--(11.247,8.381)--(11.250,8.381)%
  --(11.253,8.381)--(11.256,8.381)--(11.259,8.381)--(11.262,8.381)--(11.265,8.381)--(11.269,8.381)%
  --(11.272,8.381)--(11.275,8.381)--(11.278,8.381)--(11.282,8.381)--(11.285,8.381)--(11.288,8.381)%
  --(11.292,8.381)--(11.296,8.381)--(11.300,8.381)--(11.304,8.381)--(11.307,8.381)--(11.310,8.381)%
  --(11.313,8.381)--(11.316,8.381)--(11.319,8.381)--(11.323,8.381)--(11.326,8.381)--(11.330,8.381)%
  --(11.333,8.381)--(11.336,8.381)--(11.340,8.381)--(11.343,8.381)--(11.346,8.381)--(11.350,8.381)%
  --(11.353,8.381)--(11.356,8.381)--(11.360,8.381)--(11.363,8.381)--(11.366,8.381)--(11.369,8.381)%
  --(11.373,8.381)--(11.376,8.381)--(11.379,8.381)--(11.382,8.381)--(11.386,8.381)--(11.390,8.381)%
  --(11.394,8.381)--(11.398,8.381)--(11.401,8.381)--(11.405,8.381)--(11.408,8.381)--(11.412,8.381)%
  --(11.415,8.381)--(11.419,8.381)--(11.422,8.381)--(11.425,8.381)--(11.429,8.381)--(11.432,8.381)%
  --(11.435,8.381)--(11.439,8.381)--(11.442,8.381)--(11.445,8.381)--(11.449,8.381)--(11.452,8.381)%
  --(11.456,8.381)--(11.460,8.381)--(11.464,8.381)--(11.468,8.381)--(11.472,8.381)--(11.475,8.381)%
  --(11.479,8.381)--(11.483,8.381)--(11.486,8.381)--(11.490,8.381)--(11.493,8.381)--(11.497,8.381)%
  --(11.501,8.381)--(11.505,8.381)--(11.509,8.381)--(11.513,8.381)--(11.516,8.381)--(11.520,8.381)%
  --(11.523,8.381)--(11.526,8.381)--(11.530,8.381)--(11.533,8.381)--(11.537,8.381)--(11.540,8.381)%
  --(11.544,8.381)--(11.547,8.381)--(11.551,8.381)--(11.554,8.381)--(11.558,8.381)--(11.562,8.381)%
  --(11.566,8.381)--(11.570,8.381)--(11.574,8.381)--(11.578,8.381)--(11.582,8.381)--(11.586,8.381)%
  --(11.589,8.381)--(11.593,8.381)--(11.596,8.381)--(11.600,8.381)--(11.604,8.381)--(11.609,8.381)%
  --(11.613,8.381)--(11.617,8.381)--(11.621,8.381)--(11.624,8.381)--(11.628,8.381)--(11.631,8.381)%
  --(11.635,8.381)--(11.639,8.381)--(11.642,8.381)--(11.646,8.381)--(11.649,8.381)--(11.653,8.381)%
  --(11.657,8.381)--(11.660,8.381)--(11.665,8.381)--(11.669,8.381)--(11.673,8.381)--(11.677,8.381)%
  --(11.681,8.381)--(11.685,8.381)--(11.689,8.381)--(11.693,8.381)--(11.697,8.381)--(11.701,8.381)%
  --(11.705,8.381)--(11.709,8.381)--(11.713,8.381)--(11.717,8.381)--(11.721,8.381)--(11.725,8.381)%
  --(11.729,8.381)--(11.733,8.381)--(11.736,8.381)--(11.740,8.381)--(11.745,8.381)--(11.749,8.381)%
  --(11.753,8.381)--(11.758,8.381)--(11.761,8.381)--(11.765,8.381)--(11.769,8.381)--(11.773,8.381)%
  --(11.776,8.381)--(11.780,8.381)--(11.784,8.381)--(11.788,8.381)--(11.791,8.381)--(11.795,8.381)%
  --(11.799,8.381)--(11.803,8.381)--(11.807,8.381)--(11.811,8.381)--(11.815,8.381)--(11.818,8.381)%
  --(11.822,8.381)--(11.826,8.381)--(11.830,8.381)--(11.834,8.381)--(11.839,8.381)--(11.843,8.381)%
  --(11.848,8.381)--(11.852,8.381)--(11.856,8.381)--(11.860,8.381)--(11.865,8.381)--(11.869,8.381)%
  --(11.873,8.381)--(11.878,8.381)--(11.882,8.381)--(11.887,8.381)--(11.891,8.381)--(11.895,8.381)%
  --(11.899,8.381)--(11.903,8.381)--(11.908,8.381)--(11.912,8.381)--(11.917,8.381)--(11.921,8.381)%
  --(11.925,8.381)--(11.930,8.381)--(11.934,8.381)--(11.938,8.381)--(11.942,8.381);
\node[gp node right] at (10.479,7.739) {$x_2$};
\gpsetdashtype{gp dt 2}
\draw[gp path] (10.663,7.739)--(11.579,7.739);
\draw[gp path] (1.320,0.985)--(1.321,0.985)--(1.322,0.985)--(1.323,0.985)--(1.324,0.985)%
  --(1.325,0.985)--(1.326,0.985)--(1.327,0.985)--(1.328,0.985)--(1.329,0.985)--(1.331,0.985)%
  --(1.332,0.985)--(1.333,0.985)--(1.335,0.985)--(1.336,0.985)--(1.337,0.985)--(1.339,0.985)%
  --(1.340,0.985)--(1.342,0.985)--(1.343,0.985)--(1.345,0.985)--(1.346,0.985)--(1.348,0.985)%
  --(1.350,0.985)--(1.351,0.985)--(1.353,0.985)--(1.354,0.985)--(1.355,0.985)--(1.357,0.985)%
  --(1.358,0.985)--(1.360,0.985)--(1.361,0.986)--(1.362,0.986)--(1.363,0.986)--(1.365,0.986)%
  --(1.366,0.986)--(1.367,0.986)--(1.369,0.986)--(1.370,0.986)--(1.371,0.986)--(1.373,0.986)%
  --(1.374,0.986)--(1.376,0.986)--(1.377,0.986)--(1.379,0.986)--(1.380,0.986)--(1.382,0.986)%
  --(1.384,0.986)--(1.385,0.986)--(1.387,0.986)--(1.388,0.986)--(1.389,0.986)--(1.391,0.986)%
  --(1.392,0.986)--(1.393,0.986)--(1.395,0.987)--(1.396,0.987)--(1.397,0.987)--(1.399,0.987)%
  --(1.400,0.987)--(1.401,0.987)--(1.403,0.987)--(1.404,0.987)--(1.405,0.987)--(1.407,0.987)%
  --(1.408,0.987)--(1.409,0.987)--(1.411,0.987)--(1.412,0.987)--(1.414,0.987)--(1.416,0.988)%
  --(1.417,0.988)--(1.419,0.988)--(1.420,0.988)--(1.422,0.988)--(1.423,0.988)--(1.424,0.988)%
  --(1.426,0.988)--(1.427,0.988)--(1.429,0.988)--(1.430,0.988)--(1.431,0.988)--(1.433,0.989)%
  --(1.434,0.989)--(1.435,0.989)--(1.436,0.989)--(1.438,0.989)--(1.439,0.989)--(1.440,0.989)%
  --(1.442,0.989)--(1.443,0.989)--(1.445,0.989)--(1.446,0.989)--(1.448,0.989)--(1.449,0.990)%
  --(1.451,0.990)--(1.452,0.990)--(1.454,0.990)--(1.455,0.990)--(1.457,0.990)--(1.458,0.990)%
  --(1.460,0.990)--(1.461,0.990)--(1.462,0.991)--(1.464,0.991)--(1.465,0.991)--(1.466,0.991)%
  --(1.468,0.991)--(1.469,0.991)--(1.470,0.991)--(1.472,0.991)--(1.473,0.991)--(1.474,0.992)%
  --(1.476,0.992)--(1.477,0.992)--(1.478,0.992)--(1.480,0.992)--(1.481,0.992)--(1.483,0.992)%
  --(1.485,0.992)--(1.486,0.993)--(1.488,0.993)--(1.489,0.993)--(1.490,0.993)--(1.492,0.993)%
  --(1.493,0.993)--(1.495,0.993)--(1.496,0.994)--(1.498,0.994)--(1.499,0.994)--(1.500,0.994)%
  --(1.502,0.994)--(1.503,0.994)--(1.504,0.994)--(1.506,0.994)--(1.507,0.995)--(1.508,0.995)%
  --(1.510,0.995)--(1.511,0.995)--(1.512,0.995)--(1.514,0.995)--(1.515,0.995)--(1.517,0.996)%
  --(1.518,0.996)--(1.520,0.996)--(1.521,0.996)--(1.523,0.996)--(1.524,0.996)--(1.526,0.997)%
  --(1.528,0.997)--(1.529,0.997)--(1.531,0.997)--(1.532,0.997)--(1.533,0.997)--(1.535,0.998)%
  --(1.536,0.998)--(1.537,0.998)--(1.538,0.998)--(1.540,0.998)--(1.541,0.998)--(1.542,0.999)%
  --(1.544,0.999)--(1.545,0.999)--(1.546,0.999)--(1.548,0.999)--(1.549,0.999)--(1.551,1.000)%
  --(1.552,1.000)--(1.554,1.000)--(1.556,1.000)--(1.557,1.000)--(1.559,1.001)--(1.560,1.001)%
  --(1.562,1.001)--(1.563,1.001)--(1.565,1.001)--(1.566,1.002)--(1.568,1.002)--(1.569,1.002)%
  --(1.570,1.002)--(1.571,1.002)--(1.573,1.002)--(1.574,1.003)--(1.575,1.003)--(1.577,1.003)%
  --(1.578,1.003)--(1.579,1.003)--(1.581,1.004)--(1.582,1.004)--(1.584,1.004)--(1.585,1.004)%
  --(1.587,1.004)--(1.588,1.005)--(1.590,1.005)--(1.591,1.005)--(1.593,1.005)--(1.595,1.006)%
  --(1.596,1.006)--(1.598,1.006)--(1.599,1.006)--(1.601,1.006)--(1.602,1.007)--(1.603,1.007)%
  --(1.605,1.007)--(1.606,1.007)--(1.607,1.008)--(1.608,1.008)--(1.610,1.008)--(1.611,1.008)%
  --(1.612,1.008)--(1.614,1.009)--(1.615,1.009)--(1.617,1.009)--(1.618,1.009)--(1.620,1.010)%
  --(1.621,1.010)--(1.623,1.010)--(1.624,1.010)--(1.626,1.011)--(1.627,1.011)--(1.629,1.011)%
  --(1.630,1.011)--(1.632,1.012)--(1.634,1.012)--(1.635,1.012)--(1.636,1.012)--(1.638,1.013)%
  --(1.639,1.013)--(1.640,1.013)--(1.642,1.013)--(1.643,1.013)--(1.644,1.014)--(1.645,1.014)%
  --(1.647,1.014)--(1.648,1.014)--(1.650,1.015)--(1.651,1.015)--(1.652,1.015)--(1.654,1.015)%
  --(1.655,1.016)--(1.657,1.016)--(1.659,1.016)--(1.660,1.017)--(1.662,1.017)--(1.663,1.017)%
  --(1.665,1.017)--(1.667,1.018)--(1.668,1.018)--(1.669,1.018)--(1.671,1.019)--(1.672,1.019)%
  --(1.673,1.019)--(1.675,1.019)--(1.676,1.020)--(1.677,1.020)--(1.679,1.020)--(1.680,1.020)%
  --(1.681,1.021)--(1.683,1.021)--(1.684,1.021)--(1.685,1.021)--(1.687,1.022)--(1.688,1.022)%
  --(1.690,1.022)--(1.692,1.023)--(1.693,1.023)--(1.695,1.023)--(1.696,1.024)--(1.698,1.024)%
  --(1.699,1.024)--(1.701,1.025)--(1.703,1.025)--(1.704,1.025)--(1.705,1.026)--(1.707,1.026)%
  --(1.708,1.026)--(1.709,1.026)--(1.710,1.027)--(1.712,1.027)--(1.713,1.027)--(1.714,1.027)%
  --(1.716,1.028)--(1.717,1.028)--(1.718,1.028)--(1.720,1.029)--(1.721,1.029)--(1.723,1.029)%
  --(1.724,1.030)--(1.726,1.030)--(1.728,1.030)--(1.729,1.031)--(1.731,1.031)--(1.732,1.031)%
  --(1.734,1.032)--(1.735,1.032)--(1.737,1.032)--(1.738,1.033)--(1.740,1.033)--(1.741,1.033)%
  --(1.742,1.034)--(1.744,1.034)--(1.745,1.034)--(1.746,1.035)--(1.748,1.035)--(1.749,1.035)%
  --(1.750,1.036)--(1.752,1.036)--(1.753,1.036)--(1.754,1.037)--(1.756,1.037)--(1.757,1.037)%
  --(1.759,1.038)--(1.760,1.038)--(1.762,1.038)--(1.764,1.039)--(1.765,1.039)--(1.767,1.040)%
  --(1.768,1.040)--(1.770,1.040)--(1.771,1.041)--(1.773,1.041)--(1.774,1.041)--(1.775,1.042)%
  --(1.777,1.042)--(1.778,1.042)--(1.779,1.043)--(1.781,1.043)--(1.782,1.043)--(1.783,1.044)%
  --(1.785,1.044)--(1.786,1.044)--(1.787,1.045)--(1.789,1.045)--(1.790,1.045)--(1.792,1.046)%
  --(1.793,1.046)--(1.795,1.047)--(1.797,1.047)--(1.798,1.047)--(1.800,1.048)--(1.801,1.048)%
  --(1.803,1.049)--(1.804,1.049)--(1.806,1.049)--(1.807,1.050)--(1.809,1.050)--(1.810,1.051)%
  --(1.811,1.051)--(1.813,1.051)--(1.814,1.052)--(1.815,1.052)--(1.816,1.052)--(1.818,1.053)%
  --(1.819,1.053)--(1.821,1.053)--(1.822,1.054)--(1.823,1.054)--(1.825,1.055)--(1.826,1.055)%
  --(1.828,1.055)--(1.829,1.056)--(1.831,1.056)--(1.833,1.057)--(1.834,1.057)--(1.836,1.058)%
  --(1.837,1.058)--(1.839,1.059)--(1.840,1.059)--(1.842,1.059)--(1.843,1.060)--(1.844,1.060)%
  --(1.846,1.060)--(1.847,1.061)--(1.848,1.061)--(1.850,1.062)--(1.851,1.062)--(1.852,1.062)%
  --(1.854,1.063)--(1.855,1.063)--(1.856,1.064)--(1.858,1.064)--(1.859,1.064)--(1.861,1.065)%
  --(1.862,1.065)--(1.864,1.066)--(1.865,1.066)--(1.867,1.067)--(1.869,1.067)--(1.870,1.068)%
  --(1.872,1.068)--(1.873,1.069)--(1.875,1.069)--(1.876,1.069)--(1.878,1.070)--(1.879,1.070)%
  --(1.880,1.071)--(1.882,1.071)--(1.883,1.071)--(1.884,1.072)--(1.885,1.072)--(1.887,1.073)%
  --(1.888,1.073)--(1.889,1.074)--(1.891,1.074)--(1.892,1.074)--(1.894,1.075)--(1.895,1.075)%
  --(1.897,1.076)--(1.898,1.076)--(1.900,1.077)--(1.902,1.077)--(1.903,1.078)--(1.905,1.078)%
  --(1.906,1.079)--(1.908,1.079)--(1.909,1.080)--(1.911,1.080)--(1.912,1.081)--(1.913,1.081)%
  --(1.915,1.082)--(1.916,1.082)--(1.917,1.082)--(1.919,1.083)--(1.920,1.083)--(1.921,1.084)%
  --(1.923,1.084)--(1.924,1.085)--(1.925,1.085)--(1.927,1.085)--(1.928,1.086)--(1.930,1.086)%
  --(1.931,1.087)--(1.933,1.087)--(1.934,1.088)--(1.936,1.089)--(1.938,1.089)--(1.939,1.090)%
  --(1.941,1.090)--(1.942,1.091)--(1.944,1.091)--(1.945,1.092)--(1.947,1.092)--(1.948,1.093)%
  --(1.949,1.093)--(1.951,1.093)--(1.952,1.094)--(1.953,1.094)--(1.954,1.095)--(1.956,1.095)%
  --(1.957,1.096)--(1.958,1.096)--(1.960,1.097)--(1.961,1.097)--(1.963,1.098)--(1.964,1.098)%
  --(1.966,1.099)--(1.967,1.099)--(1.969,1.100)--(1.970,1.101)--(1.972,1.101)--(1.974,1.102)%
  --(1.975,1.102)--(1.977,1.103)--(1.978,1.103)--(1.980,1.104)--(1.981,1.104)--(1.982,1.105)%
  --(1.984,1.105)--(1.985,1.106)--(1.986,1.106)--(1.988,1.107)--(1.989,1.107)--(1.990,1.108)%
  --(1.992,1.108)--(1.993,1.109)--(1.994,1.109)--(1.996,1.110)--(1.997,1.110)--(1.999,1.111)%
  --(2.000,1.111)--(2.002,1.112)--(2.003,1.112)--(2.005,1.113)--(2.007,1.114)--(2.008,1.114)%
  --(2.010,1.115)--(2.011,1.115)--(2.013,1.116)--(2.014,1.117)--(2.016,1.117)--(2.017,1.118)%
  --(2.018,1.118)--(2.019,1.119)--(2.021,1.119)--(2.022,1.120)--(2.023,1.120)--(2.025,1.121)%
  --(2.026,1.121)--(2.027,1.122)--(2.029,1.122)--(2.030,1.123)--(2.032,1.123)--(2.033,1.124)%
  --(2.035,1.124)--(2.036,1.125)--(2.038,1.126)--(2.039,1.126)--(2.041,1.127)--(2.043,1.128)%
  --(2.044,1.128)--(2.046,1.129)--(2.047,1.129)--(2.049,1.130)--(2.050,1.130)--(2.051,1.131)%
  --(2.053,1.132)--(2.054,1.132)--(2.055,1.133)--(2.057,1.133)--(2.058,1.134)--(2.059,1.134)%
  --(2.061,1.135)--(2.062,1.135)--(2.063,1.136)--(2.065,1.136)--(2.066,1.137)--(2.068,1.138)%
  --(2.069,1.138)--(2.071,1.139)--(2.072,1.139)--(2.074,1.140)--(2.076,1.141)--(2.077,1.141)%
  --(2.079,1.142)--(2.080,1.143)--(2.082,1.143)--(2.083,1.144)--(2.084,1.144)--(2.086,1.145)%
  --(2.087,1.146)--(2.088,1.146)--(2.090,1.147)--(2.091,1.147)--(2.092,1.148)--(2.094,1.148)%
  --(2.095,1.149)--(2.096,1.150)--(2.098,1.150)--(2.099,1.151)--(2.101,1.151)--(2.102,1.152)%
  --(2.104,1.153)--(2.105,1.153)--(2.107,1.154)--(2.108,1.155)--(2.110,1.155)--(2.112,1.156)%
  --(2.113,1.157)--(2.115,1.157)--(2.116,1.158)--(2.118,1.159)--(2.119,1.159)--(2.120,1.160)%
  --(2.122,1.160)--(2.123,1.161)--(2.124,1.162)--(2.126,1.162)--(2.127,1.163)--(2.128,1.163)%
  --(2.130,1.164)--(2.131,1.164)--(2.132,1.165)--(2.134,1.166)--(2.135,1.166)--(2.137,1.167)%
  --(2.138,1.168)--(2.140,1.168)--(2.141,1.169)--(2.143,1.170)--(2.144,1.170)--(2.146,1.171)%
  --(2.148,1.172)--(2.149,1.173)--(2.151,1.173)--(2.152,1.174)--(2.153,1.175)--(2.155,1.175)%
  --(2.156,1.176)--(2.157,1.176)--(2.159,1.177)--(2.160,1.178)--(2.161,1.178)--(2.163,1.179)%
  --(2.164,1.179)--(2.165,1.180)--(2.167,1.181)--(2.168,1.181)--(2.170,1.182)--(2.171,1.183)%
  --(2.173,1.183)--(2.174,1.184)--(2.176,1.185)--(2.177,1.186)--(2.179,1.186)--(2.181,1.187)%
  --(2.182,1.188)--(2.184,1.189)--(2.185,1.189)--(2.187,1.190)--(2.188,1.191)--(2.189,1.191)%
  --(2.191,1.192)--(2.192,1.192)--(2.193,1.193)--(2.194,1.194)--(2.196,1.194)--(2.197,1.195)%
  --(2.198,1.196)--(2.200,1.196)--(2.201,1.197)--(2.203,1.198)--(2.204,1.198)--(2.206,1.199)%
  --(2.207,1.200)--(2.209,1.201)--(2.210,1.201)--(2.212,1.202)--(2.213,1.203)--(2.215,1.204)%
  --(2.217,1.204)--(2.218,1.205)--(2.220,1.206)--(2.221,1.207)--(2.222,1.207)--(2.224,1.208)%
  --(2.225,1.209)--(2.226,1.209)--(2.228,1.210)--(2.229,1.210)--(2.230,1.211)--(2.232,1.212)%
  --(2.233,1.212)--(2.234,1.213)--(2.236,1.214)--(2.237,1.215)--(2.239,1.215)--(2.240,1.216)%
  --(2.242,1.217)--(2.243,1.218)--(2.245,1.218)--(2.246,1.219)--(2.248,1.220)--(2.249,1.221)%
  --(2.251,1.222)--(2.253,1.222)--(2.254,1.223)--(2.256,1.224)--(2.257,1.225)--(2.258,1.225)%
  --(2.260,1.226)--(2.261,1.227)--(2.262,1.227)--(2.263,1.228)--(2.265,1.229)--(2.266,1.229)%
  --(2.267,1.230)--(2.269,1.231)--(2.270,1.231)--(2.272,1.232)--(2.273,1.233)--(2.275,1.234)%
  --(2.276,1.234)--(2.278,1.235)--(2.279,1.236)--(2.281,1.237)--(2.282,1.238)--(2.284,1.239)%
  --(2.286,1.239)--(2.287,1.240)--(2.289,1.241)--(2.290,1.242)--(2.291,1.242)--(2.293,1.243)%
  --(2.294,1.244)--(2.295,1.245)--(2.297,1.245)--(2.298,1.246)--(2.299,1.247)--(2.301,1.247)%
  --(2.302,1.248)--(2.303,1.249)--(2.305,1.250)--(2.306,1.250)--(2.308,1.251)--(2.309,1.252)%
  --(2.311,1.253)--(2.312,1.254)--(2.314,1.255)--(2.315,1.255)--(2.317,1.256)--(2.318,1.257)%
  --(2.320,1.258)--(2.322,1.259)--(2.323,1.260)--(2.325,1.260)--(2.326,1.261)--(2.327,1.262)%
  --(2.329,1.263)--(2.330,1.263)--(2.331,1.264)--(2.332,1.265)--(2.334,1.265)--(2.335,1.266)%
  --(2.336,1.267)--(2.338,1.268)--(2.339,1.268)--(2.341,1.269)--(2.342,1.270)--(2.344,1.271)%
  --(2.345,1.272)--(2.347,1.273)--(2.348,1.273)--(2.350,1.274)--(2.351,1.275)--(2.353,1.276)%
  --(2.355,1.277)--(2.356,1.278)--(2.358,1.279)--(2.359,1.280)--(2.360,1.280)--(2.362,1.281)%
  --(2.363,1.282)--(2.364,1.283)--(2.366,1.283)--(2.367,1.284)--(2.368,1.285)--(2.370,1.286)%
  --(2.371,1.286)--(2.372,1.287)--(2.374,1.288)--(2.375,1.289)--(2.377,1.290)--(2.378,1.291)%
  --(2.380,1.291)--(2.381,1.292)--(2.383,1.293)--(2.384,1.294)--(2.386,1.295)--(2.387,1.296)%
  --(2.389,1.297)--(2.391,1.298)--(2.392,1.299)--(2.393,1.300)--(2.395,1.300)--(2.396,1.301)%
  --(2.397,1.302)--(2.399,1.303)--(2.400,1.303)--(2.401,1.304)--(2.403,1.305)--(2.404,1.306)%
  --(2.405,1.306)--(2.407,1.307)--(2.408,1.308)--(2.410,1.309)--(2.411,1.310)--(2.412,1.311)%
  --(2.414,1.312)--(2.416,1.313)--(2.417,1.313)--(2.419,1.314)--(2.420,1.315)--(2.422,1.316)%
  --(2.423,1.317)--(2.425,1.318)--(2.427,1.319)--(2.428,1.320)--(2.429,1.321)--(2.431,1.322)%
  --(2.432,1.322)--(2.433,1.323)--(2.435,1.324)--(2.436,1.325)--(2.437,1.326)--(2.438,1.326)%
  --(2.440,1.327)--(2.441,1.328)--(2.443,1.329)--(2.444,1.330)--(2.445,1.331)--(2.447,1.332)%
  --(2.448,1.333)--(2.450,1.334)--(2.452,1.335)--(2.453,1.335)--(2.455,1.336)--(2.456,1.337)%
  --(2.458,1.338)--(2.460,1.339)--(2.461,1.340)--(2.462,1.341)--(2.464,1.342)--(2.465,1.343)%
  --(2.466,1.344)--(2.468,1.345)--(2.469,1.345)--(2.470,1.346)--(2.472,1.347)--(2.473,1.348)%
  --(2.474,1.349)--(2.476,1.349)--(2.477,1.350)--(2.478,1.351)--(2.480,1.352)--(2.481,1.353)%
  --(2.483,1.354)--(2.485,1.355)--(2.486,1.356)--(2.488,1.357)--(2.489,1.358)--(2.491,1.359)%
  --(2.492,1.360)--(2.494,1.361)--(2.496,1.362)--(2.497,1.363)--(2.498,1.364)--(2.500,1.365)%
  --(2.501,1.366)--(2.502,1.366)--(2.504,1.367)--(2.505,1.368)--(2.506,1.369)--(2.507,1.370)%
  --(2.509,1.371)--(2.510,1.372)--(2.511,1.372)--(2.513,1.373)--(2.514,1.374)--(2.516,1.375)%
  --(2.517,1.376)--(2.519,1.377)--(2.521,1.378)--(2.522,1.379)--(2.524,1.380)--(2.525,1.382)%
  --(2.527,1.383)--(2.528,1.384)--(2.530,1.385)--(2.531,1.386)--(2.533,1.386)--(2.534,1.387)%
  --(2.535,1.388)--(2.537,1.389)--(2.538,1.390)--(2.539,1.391)--(2.541,1.392)--(2.542,1.392)%
  --(2.543,1.393)--(2.545,1.394)--(2.546,1.395)--(2.547,1.396)--(2.549,1.397)--(2.550,1.398)%
  --(2.552,1.399)--(2.553,1.400)--(2.555,1.401)--(2.557,1.402)--(2.558,1.403)--(2.560,1.404)%
  --(2.561,1.406)--(2.563,1.407)--(2.565,1.408)--(2.566,1.409)--(2.567,1.409)--(2.569,1.410)%
  --(2.570,1.411)--(2.571,1.412)--(2.572,1.413)--(2.574,1.414)--(2.575,1.415)--(2.576,1.416)%
  --(2.578,1.417)--(2.579,1.418)--(2.580,1.419)--(2.582,1.420)--(2.583,1.421)--(2.585,1.422)%
  --(2.586,1.423)--(2.588,1.424)--(2.590,1.425)--(2.591,1.426)--(2.593,1.427)--(2.594,1.428)%
  --(2.596,1.429)--(2.597,1.430)--(2.599,1.431)--(2.600,1.432)--(2.602,1.433)--(2.603,1.434)%
  --(2.604,1.435)--(2.606,1.436)--(2.607,1.437)--(2.608,1.438)--(2.610,1.439)--(2.611,1.440)%
  --(2.612,1.441)--(2.614,1.442)--(2.615,1.443)--(2.616,1.444)--(2.618,1.445)--(2.619,1.446)%
  --(2.621,1.447)--(2.622,1.448)--(2.624,1.449)--(2.626,1.450)--(2.627,1.451)--(2.629,1.452)%
  --(2.630,1.454)--(2.632,1.455)--(2.634,1.456)--(2.635,1.457)--(2.636,1.458)--(2.637,1.459)%
  --(2.639,1.460)--(2.640,1.461)--(2.641,1.462)--(2.643,1.462)--(2.644,1.463)--(2.645,1.464)%
  --(2.647,1.465)--(2.648,1.466)--(2.649,1.467)--(2.651,1.468)--(2.652,1.469)--(2.654,1.471)%
  --(2.655,1.472)--(2.657,1.473)--(2.659,1.474)--(2.660,1.475)--(2.662,1.476)--(2.663,1.477)%
  --(2.665,1.479)--(2.666,1.480)--(2.668,1.481)--(2.669,1.482)--(2.671,1.483)--(2.672,1.484)%
  --(2.673,1.485)--(2.675,1.486)--(2.676,1.487)--(2.677,1.488)--(2.678,1.489)--(2.680,1.490)%
  --(2.681,1.491)--(2.683,1.492)--(2.684,1.493)--(2.685,1.494)--(2.687,1.495)--(2.688,1.496)%
  --(2.690,1.497)--(2.691,1.498)--(2.693,1.499)--(2.695,1.501)--(2.696,1.502)--(2.698,1.503)%
  --(2.699,1.504)--(2.701,1.505)--(2.702,1.507)--(2.704,1.508)--(2.705,1.509)--(2.706,1.510)%
  --(2.708,1.511)--(2.709,1.512)--(2.710,1.513)--(2.712,1.514)--(2.713,1.515)--(2.714,1.516)%
  --(2.716,1.517)--(2.717,1.518)--(2.718,1.519)--(2.720,1.520)--(2.721,1.521)--(2.723,1.522)%
  --(2.724,1.523)--(2.726,1.524)--(2.727,1.526)--(2.729,1.527)--(2.731,1.528)--(2.732,1.529)%
  --(2.734,1.530)--(2.735,1.532)--(2.737,1.533)--(2.738,1.534)--(2.740,1.535)--(2.741,1.536)%
  --(2.742,1.537)--(2.744,1.538)--(2.745,1.539)--(2.746,1.540)--(2.747,1.541)--(2.749,1.542)%
  --(2.750,1.543)--(2.751,1.544)--(2.753,1.545)--(2.754,1.546)--(2.756,1.548)--(2.757,1.549)%
  --(2.759,1.550)--(2.760,1.551)--(2.762,1.552)--(2.764,1.554)--(2.765,1.555)--(2.767,1.556)%
  --(2.768,1.557)--(2.770,1.559)--(2.771,1.560)--(2.773,1.561)--(2.774,1.562)--(2.775,1.563)%
  --(2.777,1.564)--(2.778,1.565)--(2.779,1.566)--(2.781,1.567)--(2.782,1.568)--(2.783,1.569)%
  --(2.785,1.570)--(2.786,1.571)--(2.787,1.573)--(2.789,1.574)--(2.790,1.575)--(2.792,1.576)%
  --(2.793,1.577)--(2.795,1.579)--(2.796,1.580)--(2.798,1.581)--(2.800,1.582)--(2.801,1.584)%
  --(2.803,1.585)--(2.804,1.586)--(2.806,1.588)--(2.807,1.589)--(2.809,1.590)--(2.810,1.591)%
  --(2.811,1.592)--(2.813,1.593)--(2.814,1.594)--(2.815,1.595)--(2.816,1.596)--(2.818,1.597)%
  --(2.819,1.598)--(2.820,1.599)--(2.822,1.600)--(2.823,1.602)--(2.825,1.603)--(2.826,1.604)%
  --(2.828,1.605)--(2.829,1.607)--(2.831,1.608)--(2.832,1.609)--(2.834,1.611)--(2.836,1.612)%
  --(2.837,1.613)--(2.839,1.614)--(2.840,1.616)--(2.842,1.617)--(2.843,1.618)--(2.844,1.619)%
  --(2.846,1.620)--(2.847,1.621)--(2.848,1.622)--(2.850,1.623)--(2.851,1.625)--(2.852,1.626)%
  --(2.854,1.627)--(2.855,1.628)--(2.856,1.629)--(2.858,1.630)--(2.859,1.632)--(2.861,1.633)%
  --(2.862,1.634)--(2.864,1.635)--(2.865,1.637)--(2.867,1.638)--(2.869,1.639)--(2.870,1.641)%
  --(2.872,1.642)--(2.873,1.643)--(2.875,1.645)--(2.876,1.646)--(2.878,1.647)--(2.879,1.648)%
  --(2.880,1.649)--(2.881,1.650)--(2.883,1.651)--(2.884,1.653)--(2.885,1.654)--(2.887,1.655)%
  --(2.888,1.656)--(2.889,1.657)--(2.891,1.658)--(2.892,1.660)--(2.894,1.661)--(2.895,1.662)%
  --(2.897,1.663)--(2.898,1.665)--(2.900,1.666)--(2.901,1.667)--(2.903,1.669)--(2.905,1.670)%
  --(2.906,1.672)--(2.908,1.673)--(2.909,1.674)--(2.911,1.675)--(2.912,1.677)--(2.913,1.678)%
  --(2.915,1.679)--(2.916,1.680)--(2.917,1.681)--(2.919,1.682)--(2.920,1.683)--(2.921,1.685)%
  --(2.923,1.686)--(2.924,1.687)--(2.925,1.688)--(2.927,1.689)--(2.928,1.691)--(2.930,1.692)%
  --(2.931,1.693)--(2.933,1.695)--(2.934,1.696)--(2.936,1.698)--(2.938,1.699)--(2.939,1.700)%
  --(2.941,1.702)--(2.942,1.703)--(2.944,1.705)--(2.945,1.706)--(2.947,1.707)--(2.948,1.708)%
  --(2.949,1.709)--(2.950,1.710)--(2.952,1.712)--(2.953,1.713)--(2.954,1.714)--(2.956,1.715)%
  --(2.957,1.716)--(2.958,1.717)--(2.960,1.719)--(2.961,1.720)--(2.963,1.721)--(2.964,1.723)%
  --(2.966,1.724)--(2.967,1.725)--(2.969,1.727)--(2.970,1.728)--(2.972,1.730)--(2.974,1.731)%
  --(2.975,1.733)--(2.977,1.734)--(2.978,1.735)--(2.980,1.737)--(2.981,1.738)--(2.982,1.739)%
  --(2.984,1.740)--(2.985,1.741)--(2.986,1.743)--(2.988,1.744)--(2.989,1.745)--(2.990,1.746)%
  --(2.991,1.747)--(2.993,1.749)--(2.994,1.750)--(2.996,1.751)--(2.997,1.753)--(2.999,1.754)%
  --(3.000,1.755)--(3.002,1.757)--(3.003,1.758)--(3.005,1.760)--(3.006,1.761)--(3.008,1.763)%
  --(3.010,1.764)--(3.011,1.766)--(3.013,1.767)--(3.014,1.768)--(3.015,1.769)--(3.017,1.771)%
  --(3.018,1.772)--(3.019,1.773)--(3.021,1.774)--(3.022,1.776)--(3.023,1.777)--(3.025,1.778)%
  --(3.026,1.779)--(3.027,1.780)--(3.029,1.782)--(3.030,1.783)--(3.032,1.785)--(3.033,1.786)%
  --(3.035,1.787)--(3.036,1.789)--(3.038,1.790)--(3.039,1.792)--(3.041,1.793)--(3.043,1.795)%
  --(3.044,1.796)--(3.046,1.798)--(3.047,1.799)--(3.049,1.800)--(3.050,1.802)--(3.051,1.803)%
  --(3.053,1.804)--(3.054,1.805)--(3.055,1.807)--(3.057,1.808)--(3.058,1.809)--(3.059,1.810)%
  --(3.061,1.812)--(3.062,1.813)--(3.063,1.814)--(3.065,1.816)--(3.066,1.817)--(3.068,1.819)%
  --(3.069,1.820)--(3.071,1.821)--(3.072,1.823)--(3.074,1.825)--(3.076,1.826)--(3.077,1.828)%
  --(3.079,1.829)--(3.080,1.831)--(3.082,1.832)--(3.083,1.833)--(3.084,1.835)--(3.086,1.836)%
  --(3.087,1.837)--(3.088,1.838)--(3.090,1.840)--(3.091,1.841)--(3.092,1.842)--(3.094,1.843)%
  --(3.095,1.845)--(3.096,1.846)--(3.098,1.847)--(3.099,1.849)--(3.101,1.850)--(3.102,1.852)%
  --(3.104,1.853)--(3.105,1.855)--(3.107,1.856)--(3.108,1.858)--(3.110,1.859)--(3.112,1.861)%
  --(3.113,1.862)--(3.115,1.864)--(3.116,1.865)--(3.118,1.867)--(3.119,1.868)--(3.120,1.869)%
  --(3.122,1.871)--(3.123,1.872)--(3.124,1.873)--(3.126,1.875)--(3.127,1.876)--(3.128,1.877)%
  --(3.130,1.879)--(3.131,1.880)--(3.132,1.881)--(3.134,1.883)--(3.135,1.884)--(3.137,1.886)%
  --(3.138,1.887)--(3.140,1.889)--(3.141,1.890)--(3.143,1.892)--(3.145,1.893)--(3.146,1.895)%
  --(3.148,1.897)--(3.149,1.898)--(3.151,1.900)--(3.152,1.901)--(3.154,1.902)--(3.155,1.904)%
  --(3.156,1.905)--(3.157,1.906)--(3.159,1.908)--(3.160,1.909)--(3.161,1.910)--(3.163,1.912)%
  --(3.164,1.913)--(3.165,1.914)--(3.167,1.916)--(3.168,1.917)--(3.170,1.919)--(3.171,1.920)%
  --(3.173,1.922)--(3.174,1.923)--(3.176,1.925)--(3.177,1.927)--(3.179,1.928)--(3.181,1.930)%
  --(3.182,1.931)--(3.184,1.933)--(3.185,1.934)--(3.187,1.936)--(3.188,1.937)--(3.189,1.939)%
  --(3.191,1.940)--(3.192,1.941)--(3.193,1.943)--(3.195,1.944)--(3.196,1.945)--(3.197,1.947)%
  --(3.199,1.948)--(3.200,1.949)--(3.201,1.951)--(3.203,1.952)--(3.204,1.954)--(3.206,1.955)%
  --(3.207,1.957)--(3.209,1.959)--(3.210,1.960)--(3.212,1.962)--(3.214,1.963)--(3.215,1.965)%
  --(3.217,1.967)--(3.218,1.968)--(3.220,1.970)--(3.221,1.971)--(3.223,1.973)--(3.224,1.974)%
  --(3.225,1.976)--(3.226,1.977)--(3.228,1.978)--(3.229,1.980)--(3.230,1.981)--(3.232,1.982)%
  --(3.233,1.984)--(3.234,1.985)--(3.236,1.986)--(3.237,1.988)--(3.239,1.990)--(3.240,1.991)%
  --(3.242,1.993)--(3.243,1.994)--(3.245,1.996)--(3.247,1.998)--(3.248,1.999)--(3.250,2.001)%
  --(3.251,2.003)--(3.253,2.004)--(3.254,2.006)--(3.256,2.007)--(3.257,2.009)--(3.258,2.010)%
  --(3.260,2.012)--(3.261,2.013)--(3.262,2.014)--(3.264,2.016)--(3.265,2.017)--(3.266,2.019)%
  --(3.268,2.020)--(3.269,2.021)--(3.270,2.023)--(3.272,2.025)--(3.273,2.026)--(3.275,2.028)%
  --(3.276,2.029)--(3.278,2.031)--(3.279,2.033)--(3.281,2.034)--(3.283,2.036)--(3.284,2.038)%
  --(3.286,2.040)--(3.287,2.041)--(3.289,2.043)--(3.290,2.044)--(3.292,2.046)--(3.293,2.047)%
  --(3.294,2.049)--(3.296,2.050)--(3.297,2.051)--(3.298,2.053)--(3.299,2.054)--(3.301,2.056)%
  --(3.302,2.057)--(3.303,2.059)--(3.305,2.060)--(3.306,2.062)--(3.308,2.063)--(3.309,2.065)%
  --(3.311,2.067)--(3.312,2.068)--(3.314,2.070)--(3.316,2.072)--(3.317,2.073)--(3.319,2.075)%
  --(3.320,2.077)--(3.322,2.078)--(3.323,2.080)--(3.325,2.082)--(3.326,2.083)--(3.327,2.085)%
  --(3.329,2.086)--(3.330,2.087)--(3.331,2.089)--(3.333,2.090)--(3.334,2.092)--(3.335,2.093)%
  --(3.337,2.095)--(3.338,2.096)--(3.339,2.098)--(3.341,2.099)--(3.342,2.101)--(3.344,2.103)%
  --(3.345,2.104)--(3.347,2.106)--(3.348,2.108)--(3.350,2.110)--(3.352,2.111)--(3.353,2.113)%
  --(3.355,2.115)--(3.356,2.117)--(3.358,2.118)--(3.359,2.120)--(3.361,2.121)--(3.362,2.123)%
  --(3.363,2.124)--(3.365,2.126)--(3.366,2.127)--(3.367,2.129)--(3.368,2.130)--(3.370,2.132)%
  --(3.371,2.133)--(3.372,2.135)--(3.374,2.136)--(3.375,2.138)--(3.377,2.139)--(3.378,2.141)%
  --(3.380,2.143)--(3.381,2.145)--(3.383,2.146)--(3.385,2.148)--(3.386,2.150)--(3.388,2.152)%
  --(3.389,2.153)--(3.391,2.155)--(3.392,2.157)--(3.394,2.159)--(3.395,2.160)--(3.396,2.161)%
  --(3.398,2.163)--(3.399,2.165)--(3.400,2.166)--(3.402,2.167)--(3.403,2.169)--(3.404,2.170)%
  --(3.406,2.172)--(3.407,2.174)--(3.408,2.175)--(3.410,2.177)--(3.411,2.179)--(3.413,2.180)%
  --(3.414,2.182)--(3.416,2.184)--(3.418,2.186)--(3.419,2.187)--(3.421,2.189)--(3.422,2.191)%
  --(3.424,2.193)--(3.425,2.195)--(3.427,2.196)--(3.428,2.198)--(3.430,2.199)--(3.431,2.201)%
  --(3.432,2.203)--(3.434,2.204)--(3.435,2.205)--(3.436,2.207)--(3.437,2.209)--(3.439,2.210)%
  --(3.440,2.212)--(3.441,2.213)--(3.443,2.215)--(3.444,2.217)--(3.446,2.218)--(3.447,2.220)%
  --(3.449,2.222)--(3.451,2.224)--(3.452,2.225)--(3.454,2.227)--(3.455,2.229)--(3.457,2.231)%
  --(3.458,2.233)--(3.460,2.235)--(3.462,2.236)--(3.463,2.238)--(3.464,2.239)--(3.465,2.241)%
  --(3.467,2.243)--(3.468,2.244)--(3.469,2.246)--(3.471,2.247)--(3.472,2.249)--(3.473,2.250)%
  --(3.475,2.252)--(3.476,2.253)--(3.477,2.255)--(3.479,2.257)--(3.480,2.259)--(3.482,2.260)%
  --(3.483,2.262)--(3.485,2.264)--(3.487,2.266)--(3.488,2.268)--(3.490,2.270)--(3.491,2.272)%
  --(3.493,2.273)--(3.494,2.275)--(3.496,2.277)--(3.497,2.279)--(3.499,2.280)--(3.500,2.282)%
  --(3.501,2.283)--(3.503,2.285)--(3.504,2.286)--(3.505,2.288)--(3.506,2.290)--(3.508,2.291)%
  --(3.509,2.293)--(3.510,2.294)--(3.512,2.296)--(3.513,2.298)--(3.515,2.300)--(3.516,2.301)%
  --(3.518,2.303)--(3.520,2.305)--(3.521,2.307)--(3.523,2.309)--(3.524,2.311)--(3.526,2.313)%
  --(3.527,2.315)--(3.529,2.316)--(3.530,2.318)--(3.532,2.320)--(3.533,2.322)--(3.534,2.323)%
  --(3.536,2.325)--(3.537,2.326)--(3.538,2.328)--(3.540,2.329)--(3.541,2.331)--(3.542,2.333)%
  --(3.544,2.334)--(3.545,2.336)--(3.546,2.338)--(3.548,2.339)--(3.549,2.341)--(3.551,2.343)%
  --(3.553,2.345)--(3.554,2.347)--(3.555,2.349)--(3.557,2.351)--(3.559,2.353)--(3.560,2.355)%
  --(3.562,2.357)--(3.563,2.358)--(3.565,2.360)--(3.566,2.362)--(3.568,2.364)--(3.569,2.365)%
  --(3.570,2.367)--(3.572,2.368)--(3.573,2.370)--(3.574,2.372)--(3.575,2.373)--(3.577,2.375)%
  --(3.578,2.377)--(3.580,2.378)--(3.581,2.380)--(3.582,2.382)--(3.584,2.384)--(3.585,2.386)%
  --(3.587,2.387)--(3.588,2.389)--(3.590,2.391)--(3.592,2.393)--(3.593,2.395)--(3.595,2.397)%
  --(3.596,2.399)--(3.598,2.401)--(3.599,2.403)--(3.601,2.405)--(3.602,2.406)--(3.604,2.408)%
  --(3.605,2.409)--(3.606,2.411)--(3.607,2.413)--(3.609,2.414)--(3.610,2.416)--(3.611,2.418)%
  --(3.613,2.419)--(3.614,2.421)--(3.615,2.423)--(3.617,2.425)--(3.618,2.427)--(3.620,2.428)%
  --(3.621,2.430)--(3.623,2.432)--(3.625,2.434)--(3.626,2.436)--(3.628,2.438)--(3.629,2.440)%
  --(3.631,2.442)--(3.632,2.444)--(3.634,2.446)--(3.635,2.448)--(3.637,2.450)--(3.638,2.451)%
  --(3.639,2.453)--(3.641,2.455)--(3.642,2.456)--(3.643,2.458)--(3.645,2.459)--(3.646,2.461)%
  --(3.647,2.463)--(3.649,2.465)--(3.650,2.466)--(3.651,2.468)--(3.653,2.470)--(3.655,2.472)%
  --(3.656,2.474)--(3.658,2.476)--(3.659,2.478)--(3.661,2.480)--(3.662,2.482)--(3.664,2.484)%
  --(3.665,2.486)--(3.667,2.488)--(3.669,2.490)--(3.670,2.492)--(3.671,2.493)--(3.672,2.495)%
  --(3.674,2.497)--(3.675,2.499)--(3.676,2.500)--(3.678,2.502)--(3.679,2.504)--(3.680,2.505)%
  --(3.682,2.507)--(3.683,2.509)--(3.685,2.511)--(3.686,2.512)--(3.687,2.514)--(3.689,2.516)%
  --(3.690,2.518)--(3.692,2.520)--(3.694,2.522)--(3.695,2.525)--(3.697,2.527)--(3.698,2.529)%
  --(3.700,2.531)--(3.702,2.533)--(3.703,2.535)--(3.704,2.536)--(3.706,2.538)--(3.707,2.540)%
  --(3.708,2.542)--(3.710,2.543)--(3.711,2.545)--(3.712,2.547)--(3.713,2.548)--(3.715,2.550)%
  --(3.716,2.552)--(3.718,2.554)--(3.719,2.555)--(3.720,2.557)--(3.722,2.559)--(3.723,2.561)%
  --(3.725,2.563)--(3.727,2.565)--(3.728,2.567)--(3.730,2.570)--(3.731,2.572)--(3.733,2.574)%
  --(3.734,2.576)--(3.736,2.578)--(3.738,2.580)--(3.739,2.582)--(3.740,2.583)--(3.741,2.585)%
  --(3.743,2.587)--(3.744,2.589)--(3.746,2.590)--(3.747,2.592)--(3.748,2.594)--(3.749,2.595)%
  --(3.751,2.597)--(3.752,2.599)--(3.753,2.601)--(3.755,2.603)--(3.757,2.605)--(3.758,2.607)%
  --(3.760,2.609)--(3.761,2.611)--(3.763,2.613)--(3.764,2.615)--(3.766,2.617)--(3.767,2.619)%
  --(3.769,2.622)--(3.771,2.624)--(3.772,2.626)--(3.773,2.628)--(3.775,2.629)--(3.776,2.631)%
  --(3.777,2.633)--(3.779,2.635)--(3.780,2.636)--(3.781,2.638)--(3.782,2.640)--(3.784,2.642)%
  --(3.785,2.643)--(3.787,2.645)--(3.788,2.647)--(3.790,2.649)--(3.791,2.651)--(3.792,2.653)%
  --(3.794,2.655)--(3.796,2.657)--(3.797,2.660)--(3.799,2.662)--(3.800,2.664)--(3.802,2.666)%
  --(3.804,2.668)--(3.805,2.670)--(3.807,2.672)--(3.808,2.674)--(3.809,2.676)--(3.811,2.678)%
  --(3.812,2.679)--(3.813,2.681)--(3.814,2.683)--(3.816,2.685)--(3.817,2.686)--(3.818,2.688)%
  --(3.820,2.690)--(3.821,2.692)--(3.822,2.694)--(3.824,2.696)--(3.825,2.698)--(3.827,2.700)%
  --(3.829,2.702)--(3.830,2.704)--(3.832,2.707)--(3.833,2.709)--(3.835,2.711)--(3.836,2.713)%
  --(3.838,2.715)--(3.839,2.717)--(3.841,2.719)--(3.842,2.721)--(3.844,2.723)--(3.845,2.725)%
  --(3.846,2.727)--(3.848,2.728)--(3.849,2.730)--(3.850,2.732)--(3.852,2.734)--(3.853,2.736)%
  --(3.854,2.738)--(3.856,2.739)--(3.857,2.741)--(3.859,2.743)--(3.860,2.746)--(3.862,2.748)%
  --(3.863,2.750)--(3.865,2.752)--(3.866,2.754)--(3.868,2.756)--(3.869,2.758)--(3.871,2.761)%
  --(3.872,2.763)--(3.874,2.765)--(3.876,2.767)--(3.877,2.769)--(3.878,2.771)--(3.880,2.773)%
  --(3.881,2.775)--(3.882,2.776)--(3.883,2.778)--(3.885,2.780)--(3.886,2.782)--(3.887,2.784)%
  --(3.889,2.786)--(3.890,2.788)--(3.892,2.789)--(3.893,2.792)--(3.894,2.794)--(3.896,2.796)%
  --(3.898,2.798)--(3.899,2.800)--(3.901,2.802)--(3.902,2.805)--(3.904,2.807)--(3.905,2.809)%
  --(3.907,2.811)--(3.909,2.813)--(3.910,2.816)--(3.911,2.818)--(3.913,2.819)--(3.914,2.821)%
  --(3.915,2.823)--(3.917,2.825)--(3.918,2.827)--(3.919,2.829)--(3.921,2.830)--(3.922,2.832)%
  --(3.923,2.834)--(3.925,2.836)--(3.926,2.838)--(3.927,2.840)--(3.929,2.843)--(3.931,2.845)%
  --(3.932,2.847)--(3.934,2.849)--(3.935,2.851)--(3.937,2.854)--(3.938,2.856)--(3.940,2.858)%
  --(3.941,2.860)--(3.943,2.862)--(3.944,2.865)--(3.946,2.867)--(3.947,2.868)--(3.948,2.870)%
  --(3.950,2.872)--(3.951,2.874)--(3.953,2.876)--(3.954,2.878)--(3.955,2.880)--(3.956,2.882)%
  --(3.958,2.884)--(3.959,2.886)--(3.960,2.888)--(3.962,2.890)--(3.964,2.892)--(3.965,2.894)%
  --(3.967,2.896)--(3.968,2.899)--(3.970,2.901)--(3.971,2.903)--(3.973,2.905)--(3.974,2.908)%
  --(3.976,2.910)--(3.977,2.912)--(3.979,2.914)--(3.980,2.916)--(3.982,2.918)--(3.983,2.920)%
  --(3.984,2.922)--(3.986,2.924)--(3.987,2.926)--(3.988,2.928)--(3.990,2.930)--(3.991,2.932)%
  --(3.992,2.934)--(3.994,2.936)--(3.995,2.938)--(3.997,2.940)--(3.998,2.942)--(3.999,2.944)%
  --(4.001,2.946)--(4.003,2.949)--(4.004,2.951)--(4.006,2.953)--(4.007,2.956)--(4.009,2.958)%
  --(4.010,2.960)--(4.012,2.962)--(4.014,2.965)--(4.015,2.967)--(4.016,2.969)--(4.018,2.971)%
  --(4.019,2.972)--(4.020,2.974)--(4.021,2.976)--(4.023,2.978)--(4.024,2.980)--(4.025,2.982)%
  --(4.027,2.984)--(4.028,2.986)--(4.029,2.988)--(4.031,2.990)--(4.032,2.993)--(4.034,2.995)%
  --(4.036,2.997)--(4.037,3.000)--(4.039,3.002)--(4.040,3.004)--(4.042,3.007)--(4.043,3.009)%
  --(4.045,3.011)--(4.046,3.013)--(4.048,3.016)--(4.049,3.018)--(4.051,3.020)--(4.052,3.022)%
  --(4.053,3.024)--(4.055,3.026)--(4.056,3.027)--(4.057,3.029)--(4.058,3.031)--(4.060,3.033)%
  --(4.061,3.035)--(4.063,3.037)--(4.064,3.039)--(4.065,3.042)--(4.067,3.044)--(4.069,3.046)%
  --(4.070,3.049)--(4.072,3.051)--(4.073,3.053)--(4.075,3.056)--(4.076,3.058)--(4.078,3.060)%
  --(4.079,3.063)--(4.081,3.065)--(4.082,3.067)--(4.084,3.069)--(4.085,3.071)--(4.086,3.073)%
  --(4.088,3.075)--(4.089,3.077)--(4.090,3.079)--(4.092,3.081)--(4.093,3.083)--(4.094,3.085)%
  --(4.096,3.087)--(4.097,3.089)--(4.098,3.091)--(4.100,3.094)--(4.102,3.096)--(4.103,3.098)%
  --(4.104,3.101)--(4.106,3.103)--(4.108,3.106)--(4.109,3.108)--(4.111,3.110)--(4.112,3.113)%
  --(4.114,3.115)--(4.115,3.117)--(4.117,3.120)--(4.118,3.122)--(4.120,3.124)--(4.121,3.126)%
  --(4.122,3.128)--(4.124,3.130)--(4.125,3.132)--(4.126,3.134)--(4.127,3.136)--(4.129,3.138)%
  --(4.130,3.140)--(4.131,3.142)--(4.133,3.144)--(4.134,3.146)--(4.136,3.149)--(4.137,3.151)%
  --(4.139,3.153)--(4.141,3.156)--(4.142,3.158)--(4.144,3.161)--(4.145,3.163)--(4.147,3.165)%
  --(4.148,3.168)--(4.150,3.170)--(4.151,3.173)--(4.153,3.175)--(4.154,3.177)--(4.155,3.179)%
  --(4.157,3.181)--(4.158,3.183)--(4.159,3.185)--(4.161,3.187)--(4.162,3.189)--(4.163,3.191)%
  --(4.165,3.193)--(4.166,3.195)--(4.167,3.197)--(4.169,3.200)--(4.170,3.202)--(4.172,3.204)%
  --(4.174,3.207)--(4.175,3.209)--(4.177,3.212)--(4.178,3.214)--(4.180,3.217)--(4.181,3.219)%
  --(4.183,3.221)--(4.184,3.224)--(4.186,3.226)--(4.187,3.228)--(4.188,3.230)--(4.190,3.232)%
  --(4.191,3.235)--(4.192,3.237)--(4.194,3.239)--(4.195,3.241)--(4.196,3.243)--(4.198,3.245)%
  --(4.199,3.247)--(4.200,3.249)--(4.202,3.251)--(4.203,3.254)--(4.205,3.256)--(4.207,3.258)%
  --(4.208,3.261)--(4.209,3.263)--(4.211,3.266)--(4.213,3.268)--(4.214,3.271)--(4.216,3.273)%
  --(4.217,3.276)--(4.219,3.278)--(4.220,3.280)--(4.222,3.283)--(4.223,3.285)--(4.224,3.287)%
  --(4.226,3.289)--(4.227,3.291)--(4.228,3.293)--(4.229,3.295)--(4.231,3.297)--(4.232,3.299)%
  --(4.233,3.301)--(4.235,3.304)--(4.236,3.306)--(4.238,3.308)--(4.239,3.311)--(4.241,3.313)%
  --(4.242,3.316)--(4.244,3.318)--(4.246,3.321)--(4.247,3.323)--(4.249,3.326)--(4.250,3.328)%
  --(4.252,3.330)--(4.253,3.333)--(4.255,3.335)--(4.256,3.338)--(4.258,3.340)--(4.259,3.342)%
  --(4.260,3.344)--(4.261,3.346)--(4.263,3.348)--(4.264,3.350)--(4.265,3.352)--(4.267,3.354)%
  --(4.268,3.357)--(4.269,3.359)--(4.271,3.361)--(4.272,3.363)--(4.274,3.366)--(4.275,3.368)%
  --(4.277,3.371)--(4.278,3.373)--(4.280,3.376)--(4.282,3.378)--(4.283,3.381)--(4.285,3.383)%
  --(4.286,3.386)--(4.288,3.388)--(4.289,3.391)--(4.291,3.393)--(4.292,3.395)--(4.293,3.397)%
  --(4.295,3.399)--(4.296,3.402)--(4.297,3.404)--(4.299,3.406)--(4.300,3.408)--(4.301,3.410)%
  --(4.303,3.412)--(4.304,3.415)--(4.305,3.417)--(4.307,3.419)--(4.308,3.422)--(4.310,3.424)%
  --(4.311,3.427)--(4.313,3.429)--(4.314,3.432)--(4.316,3.434)--(4.318,3.437)--(4.319,3.440)%
  --(4.321,3.442)--(4.322,3.445)--(4.324,3.447)--(4.325,3.449)--(4.326,3.451)--(4.328,3.454)%
  --(4.329,3.456)--(4.330,3.458)--(4.332,3.460)--(4.333,3.462)--(4.334,3.464)--(4.336,3.467)%
  --(4.337,3.469)--(4.338,3.471)--(4.340,3.473)--(4.341,3.476)--(4.343,3.478)--(4.344,3.481)%
  --(4.346,3.483)--(4.347,3.486)--(4.349,3.489)--(4.351,3.491)--(4.352,3.494)--(4.354,3.496)%
  --(4.355,3.499)--(4.357,3.501)--(4.358,3.504)--(4.360,3.506)--(4.361,3.508)--(4.362,3.511)%
  --(4.364,3.513)--(4.365,3.515)--(4.366,3.517)--(4.367,3.519)--(4.369,3.521)--(4.370,3.524)%
  --(4.371,3.526)--(4.373,3.528)--(4.374,3.530)--(4.376,3.533)--(4.377,3.536)--(4.379,3.538)%
  --(4.380,3.541)--(4.382,3.543)--(4.384,3.546)--(4.385,3.549)--(4.387,3.551)--(4.388,3.554)%
  --(4.390,3.556)--(4.391,3.559)--(4.393,3.562)--(4.394,3.564)--(4.395,3.566)--(4.397,3.568)%
  --(4.398,3.570)--(4.399,3.573)--(4.401,3.575)--(4.402,3.577)--(4.403,3.579)--(4.405,3.581)%
  --(4.406,3.584)--(4.407,3.586)--(4.409,3.588)--(4.410,3.591)--(4.412,3.593)--(4.413,3.596)%
  --(4.415,3.599)--(4.416,3.601)--(4.418,3.604)--(4.419,3.606)--(4.421,3.609)--(4.423,3.612)%
  --(4.424,3.614)--(4.426,3.617)--(4.427,3.620)--(4.429,3.622)--(4.430,3.624)--(4.431,3.626)%
  --(4.433,3.629)--(4.434,3.631)--(4.435,3.633)--(4.436,3.635)--(4.438,3.638)--(4.439,3.640)%
  --(4.441,3.642)--(4.442,3.644)--(4.443,3.647)--(4.445,3.649)--(4.446,3.652)--(4.448,3.655)%
  --(4.449,3.657)--(4.451,3.660)--(4.452,3.662)--(4.454,3.665)--(4.456,3.668)--(4.457,3.670)%
  --(4.459,3.673)--(4.460,3.676)--(4.462,3.678)--(4.463,3.681)--(4.464,3.683)--(4.466,3.685)%
  --(4.467,3.687)--(4.468,3.690)--(4.470,3.692)--(4.471,3.694)--(4.472,3.696)--(4.474,3.699)%
  --(4.475,3.701)--(4.476,3.703)--(4.478,3.706)--(4.479,3.708)--(4.481,3.711)--(4.482,3.714)%
  --(4.484,3.716)--(4.485,3.719)--(4.487,3.722)--(4.489,3.724)--(4.490,3.727)--(4.492,3.730)%
  --(4.493,3.733)--(4.495,3.735)--(4.496,3.738)--(4.497,3.740)--(4.499,3.743)--(4.500,3.745)%
  --(4.502,3.747)--(4.503,3.749)--(4.504,3.752)--(4.505,3.754)--(4.507,3.756)--(4.508,3.758)%
  --(4.509,3.761)--(4.511,3.763)--(4.512,3.766)--(4.514,3.768)--(4.515,3.771)--(4.517,3.774)%
  --(4.518,3.776)--(4.520,3.779)--(4.521,3.782)--(4.523,3.784)--(4.524,3.787)--(4.526,3.790)%
  --(4.528,3.793)--(4.529,3.795)--(4.531,3.798)--(4.532,3.800)--(4.533,3.803)--(4.535,3.805)%
  --(4.536,3.807)--(4.537,3.810)--(4.538,3.812)--(4.540,3.814)--(4.541,3.816)--(4.543,3.819)%
  --(4.544,3.821)--(4.545,3.824)--(4.547,3.826)--(4.548,3.829)--(4.550,3.831)--(4.551,3.834)%
  --(4.553,3.837)--(4.554,3.839)--(4.556,3.842)--(4.557,3.845)--(4.559,3.848)--(4.561,3.850)%
  --(4.562,3.853)--(4.564,3.856)--(4.565,3.859)--(4.567,3.861)--(4.568,3.863)--(4.569,3.866)%
  --(4.571,3.868)--(4.572,3.870)--(4.573,3.873)--(4.574,3.875)--(4.576,3.877)--(4.577,3.880)%
  --(4.578,3.882)--(4.580,3.885)--(4.581,3.887)--(4.583,3.890)--(4.584,3.893)--(4.586,3.895)%
  --(4.587,3.898)--(4.589,3.901)--(4.590,3.904)--(4.592,3.906)--(4.594,3.909)--(4.595,3.912)%
  --(4.597,3.915)--(4.598,3.917)--(4.600,3.920)--(4.601,3.922)--(4.602,3.925)--(4.604,3.927)%
  --(4.605,3.930)--(4.606,3.932)--(4.608,3.934)--(4.609,3.937)--(4.610,3.939)--(4.611,3.941)%
  --(4.613,3.944)--(4.614,3.946)--(4.616,3.949)--(4.617,3.951)--(4.619,3.954)--(4.620,3.957)%
  --(4.622,3.960)--(4.623,3.962)--(4.625,3.965)--(4.626,3.968)--(4.628,3.971)--(4.629,3.974)%
  --(4.631,3.977)--(4.633,3.979)--(4.634,3.982)--(4.635,3.985)--(4.637,3.987)--(4.638,3.989)%
  --(4.639,3.992)--(4.641,3.994)--(4.642,3.997)--(4.643,3.999)--(4.645,4.001)--(4.646,4.004)%
  --(4.647,4.006)--(4.649,4.009)--(4.650,4.011)--(4.652,4.014)--(4.653,4.017)--(4.655,4.020)%
  --(4.656,4.022)--(4.658,4.025)--(4.659,4.028)--(4.661,4.031)--(4.662,4.034)--(4.664,4.036)%
  --(4.666,4.039)--(4.667,4.042)--(4.669,4.045)--(4.670,4.047)--(4.671,4.050)--(4.673,4.052)%
  --(4.674,4.054)--(4.675,4.057)--(4.676,4.059)--(4.678,4.062)--(4.679,4.064)--(4.680,4.066)%
  --(4.682,4.069)--(4.683,4.071)--(4.685,4.074)--(4.686,4.077)--(4.687,4.080)--(4.689,4.082)%
  --(4.691,4.085)--(4.692,4.088)--(4.694,4.091)--(4.695,4.094)--(4.697,4.097)--(4.699,4.100)%
  --(4.700,4.102)--(4.702,4.105)--(4.703,4.108)--(4.704,4.111)--(4.706,4.113)--(4.707,4.116)%
  --(4.709,4.118)--(4.710,4.120)--(4.711,4.123)--(4.712,4.125)--(4.714,4.128)--(4.715,4.130)%
  --(4.716,4.133)--(4.718,4.135)--(4.719,4.138)--(4.721,4.141)--(4.722,4.143)--(4.724,4.146)%
  --(4.725,4.149)--(4.727,4.152)--(4.728,4.155)--(4.730,4.158)--(4.731,4.161)--(4.733,4.164)%
  --(4.734,4.166)--(4.736,4.169)--(4.738,4.172)--(4.739,4.175)--(4.740,4.177)--(4.741,4.180)%
  --(4.743,4.182)--(4.744,4.184)--(4.746,4.187)--(4.747,4.189)--(4.748,4.192)--(4.749,4.194)%
  --(4.751,4.197)--(4.752,4.199)--(4.753,4.202)--(4.755,4.205)--(4.757,4.208)--(4.758,4.210)%
  --(4.760,4.213)--(4.761,4.216)--(4.763,4.219)--(4.764,4.222)--(4.766,4.225)--(4.767,4.228)%
  --(4.769,4.231)--(4.771,4.234)--(4.772,4.237)--(4.774,4.239)--(4.775,4.242)--(4.776,4.244)%
  --(4.777,4.247)--(4.779,4.249)--(4.780,4.252)--(4.781,4.254)--(4.783,4.257)--(4.784,4.259)%
  --(4.785,4.262)--(4.787,4.264)--(4.788,4.267)--(4.790,4.270)--(4.791,4.273)--(4.793,4.276)%
  --(4.794,4.279)--(4.796,4.282)--(4.797,4.285)--(4.799,4.287)--(4.800,4.290)--(4.802,4.293)%
  --(4.803,4.296)--(4.805,4.299)--(4.807,4.302)--(4.808,4.305)--(4.809,4.307)--(4.811,4.310)%
  --(4.812,4.312)--(4.813,4.315)--(4.814,4.317)--(4.816,4.319)--(4.817,4.322)--(4.818,4.325)%
  --(4.820,4.327)--(4.821,4.330)--(4.822,4.332)--(4.824,4.335)--(4.825,4.338)--(4.827,4.341)%
  --(4.829,4.344)--(4.830,4.347)--(4.832,4.350)--(4.833,4.353)--(4.835,4.356)--(4.836,4.359)%
  --(4.838,4.362)--(4.839,4.365)--(4.841,4.368)--(4.842,4.371)--(4.844,4.373)--(4.845,4.376)%
  --(4.846,4.378)--(4.848,4.381)--(4.849,4.383)--(4.850,4.386)--(4.852,4.388)--(4.853,4.391)%
  --(4.854,4.394)--(4.856,4.396)--(4.857,4.399)--(4.859,4.402)--(4.860,4.405)--(4.862,4.408)%
  --(4.863,4.411)--(4.865,4.414)--(4.866,4.417)--(4.868,4.420)--(4.869,4.423)--(4.871,4.426)%
  --(4.872,4.429)--(4.874,4.432)--(4.875,4.434)--(4.877,4.437)--(4.878,4.440)--(4.879,4.442)%
  --(4.881,4.445)--(4.882,4.447)--(4.883,4.450)--(4.885,4.452)--(4.886,4.455)--(4.887,4.457)%
  --(4.889,4.460)--(4.890,4.463)--(4.891,4.465)--(4.893,4.468)--(4.894,4.471)--(4.896,4.474)%
  --(4.897,4.477)--(4.899,4.480)--(4.901,4.483)--(4.902,4.487)--(4.904,4.490)--(4.905,4.493)%
  --(4.907,4.496)--(4.909,4.499)--(4.910,4.502)--(4.912,4.505)--(4.913,4.507)--(4.914,4.510)%
  --(4.915,4.512)--(4.917,4.515)--(4.918,4.518)--(4.919,4.520)--(4.921,4.523)--(4.922,4.525)%
  --(4.923,4.528)--(4.925,4.531)--(4.926,4.533)--(4.928,4.536)--(4.929,4.539)--(4.931,4.542)%
  --(4.932,4.546)--(4.934,4.549)--(4.935,4.551)--(4.937,4.555)--(4.938,4.558)--(4.940,4.561)%
  --(4.941,4.564)--(4.943,4.567)--(4.944,4.570)--(4.946,4.572)--(4.947,4.575)--(4.948,4.577)%
  --(4.950,4.580)--(4.951,4.583)--(4.952,4.585)--(4.954,4.588)--(4.955,4.590)--(4.956,4.593)%
  --(4.958,4.596)--(4.959,4.598)--(4.960,4.601)--(4.962,4.604)--(4.963,4.607)--(4.965,4.610)%
  --(4.966,4.613)--(4.968,4.616)--(4.970,4.620)--(4.971,4.623)--(4.973,4.626)--(4.974,4.629)%
  --(4.976,4.632)--(4.978,4.635)--(4.979,4.639)--(4.980,4.641)--(4.982,4.644)--(4.983,4.646)%
  --(4.984,4.649)--(4.986,4.652)--(4.987,4.654)--(4.988,4.657)--(4.990,4.659)--(4.990,4.661)%
  --(4.991,4.662)--(4.991,4.663)--(4.992,4.664)--(4.992,4.665)--(4.993,4.666)--(4.993,4.667)%
  --(4.994,4.668)--(4.995,4.671)--(4.996,4.673)--(4.998,4.676)--(4.999,4.678)--(5.000,4.680)%
  --(5.001,4.683)--(5.002,4.685)--(5.004,4.688)--(5.004,4.689)--(5.005,4.690)--(5.005,4.691)%
  --(5.005,4.692)--(5.006,4.692)--(5.007,4.693)--(5.007,4.695)--(5.008,4.697)--(5.009,4.698)%
  --(5.010,4.700)--(5.011,4.702)--(5.012,4.705)--(5.013,4.707)--(5.014,4.709)--(5.015,4.711)%
  --(5.016,4.713)--(5.017,4.715)--(5.019,4.718)--(5.020,4.720)--(5.021,4.722)--(5.022,4.724)%
  --(5.023,4.727)--(5.025,4.730)--(5.026,4.732)--(5.027,4.735)--(5.029,4.737)--(5.030,4.740)%
  --(5.031,4.743)--(5.033,4.746)--(5.034,4.748)--(5.035,4.750)--(5.036,4.752)--(5.037,4.755)%
  --(5.038,4.757)--(5.039,4.759)--(5.041,4.761)--(5.042,4.763)--(5.043,4.766)--(5.044,4.768)%
  --(5.045,4.770)--(5.046,4.772)--(5.047,4.774)--(5.048,4.776)--(5.049,4.778)--(5.050,4.780)%
  --(5.051,4.783)--(5.052,4.785)--(5.053,4.787)--(5.055,4.789)--(5.056,4.792)--(5.057,4.794)%
  --(5.058,4.797)--(5.059,4.799)--(5.061,4.802)--(5.063,4.805)--(5.064,4.808)--(5.066,4.811)%
  --(5.067,4.813)--(5.068,4.816)--(5.069,4.818)--(5.070,4.820)--(5.071,4.822)--(5.073,4.825)%
  --(5.074,4.827)--(5.075,4.829)--(5.076,4.831)--(5.077,4.833)--(5.078,4.835)--(5.079,4.838)%
  --(5.080,4.840)--(5.081,4.842)--(5.082,4.844)--(5.083,4.846)--(5.085,4.848)--(5.086,4.850)%
  --(5.087,4.852)--(5.088,4.855)--(5.089,4.857)--(5.090,4.859)--(5.091,4.861)--(5.092,4.864)%
  --(5.094,4.866)--(5.095,4.869)--(5.097,4.872)--(5.098,4.874)--(5.099,4.877)--(5.100,4.879)%
  --(5.102,4.881)--(5.103,4.884)--(5.104,4.886)--(5.105,4.888)--(5.106,4.891)--(5.108,4.893)%
  --(5.109,4.895)--(5.110,4.897)--(5.111,4.900)--(5.112,4.902)--(5.113,4.904)--(5.114,4.906)%
  --(5.115,4.908)--(5.116,4.910)--(5.117,4.912)--(5.119,4.914)--(5.120,4.916)--(5.121,4.918)%
  --(5.122,4.921)--(5.123,4.923)--(5.124,4.925)--(5.125,4.927)--(5.126,4.930)--(5.128,4.932)%
  --(5.129,4.935)--(5.130,4.937)--(5.132,4.940)--(5.133,4.943)--(5.134,4.945)--(5.136,4.948)%
  --(5.137,4.950)--(5.138,4.953)--(5.140,4.956)--(5.141,4.958)--(5.142,4.960)--(5.143,4.963)%
  --(5.144,4.965)--(5.146,4.967)--(5.147,4.969)--(5.148,4.971)--(5.149,4.973)--(5.150,4.975)%
  --(5.151,4.977)--(5.152,4.979)--(5.153,4.981)--(5.154,4.983)--(5.155,4.985)--(5.156,4.988)%
  --(5.157,4.990)--(5.159,4.992)--(5.160,4.994)--(5.161,4.997)--(5.162,4.999)--(5.164,5.001)%
  --(5.165,5.005)--(5.167,5.008)--(5.168,5.011)--(5.170,5.014)--(5.171,5.016)--(5.173,5.019)%
  --(5.174,5.022)--(5.176,5.024)--(5.177,5.027)--(5.178,5.029)--(5.179,5.031)--(5.180,5.033)%
  --(5.181,5.035)--(5.182,5.037)--(5.183,5.039)--(5.184,5.041)--(5.185,5.043)--(5.187,5.045)%
  --(5.188,5.047)--(5.189,5.049)--(5.190,5.052)--(5.191,5.054)--(5.192,5.056)--(5.193,5.058)%
  --(5.194,5.060)--(5.196,5.063)--(5.197,5.065)--(5.198,5.067)--(5.200,5.071)--(5.201,5.074)%
  --(5.203,5.077)--(5.205,5.080)--(5.206,5.083)--(5.208,5.085)--(5.209,5.088)--(5.210,5.091)%
  --(5.211,5.093)--(5.212,5.095)--(5.214,5.097)--(5.215,5.099)--(5.216,5.101)--(5.217,5.103)%
  --(5.218,5.105)--(5.219,5.107)--(5.220,5.109)--(5.221,5.111)--(5.222,5.113)--(5.223,5.115)%
  --(5.225,5.117)--(5.226,5.119)--(5.227,5.122)--(5.228,5.124)--(5.229,5.126)--(5.230,5.128)%
  --(5.232,5.131)--(5.233,5.133)--(5.234,5.136)--(5.236,5.138)--(5.237,5.141)--(5.238,5.143)%
  --(5.240,5.146)--(5.241,5.148)--(5.242,5.151)--(5.244,5.154)--(5.245,5.156)--(5.246,5.158)%
  --(5.247,5.160)--(5.248,5.162)--(5.249,5.164)--(5.251,5.166)--(5.252,5.168)--(5.253,5.170)%
  --(5.254,5.172)--(5.255,5.174)--(5.256,5.176)--(5.257,5.178)--(5.258,5.180)--(5.259,5.182)%
  --(5.260,5.184)--(5.261,5.186)--(5.263,5.189)--(5.264,5.191)--(5.265,5.193)--(5.266,5.195)%
  --(5.268,5.198)--(5.269,5.201)--(5.271,5.204)--(5.272,5.207)--(5.273,5.209)--(5.275,5.211)%
  --(5.276,5.213)--(5.277,5.215)--(5.278,5.217)--(5.279,5.220)--(5.280,5.222)--(5.281,5.224)%
  --(5.282,5.226)--(5.283,5.228)--(5.285,5.230)--(5.286,5.232)--(5.287,5.234)--(5.288,5.236)%
  --(5.289,5.238)--(5.290,5.240)--(5.291,5.242)--(5.292,5.244)--(5.293,5.246)--(5.294,5.248)%
  --(5.296,5.250)--(5.297,5.252)--(5.298,5.254)--(5.299,5.256)--(5.300,5.259)--(5.302,5.261)%
  --(5.303,5.264)--(5.304,5.266)--(5.306,5.268)--(5.307,5.271)--(5.308,5.273)--(5.309,5.275)%
  --(5.311,5.277)--(5.312,5.280)--(5.313,5.282)--(5.314,5.284)--(5.315,5.286)--(5.316,5.288)%
  --(5.317,5.290)--(5.319,5.292)--(5.320,5.294)--(5.321,5.296)--(5.322,5.298)--(5.323,5.300)%
  --(5.324,5.302)--(5.325,5.304)--(5.326,5.306)--(5.327,5.308)--(5.328,5.310)--(5.330,5.312)%
  --(5.331,5.314)--(5.332,5.316)--(5.333,5.319)--(5.334,5.321)--(5.336,5.323)--(5.337,5.326)%
  --(5.338,5.328)--(5.340,5.331)--(5.341,5.333)--(5.342,5.336)--(5.344,5.338)--(5.345,5.340)%
  --(5.346,5.343)--(5.348,5.345)--(5.349,5.347)--(5.350,5.349)--(5.351,5.351)--(5.352,5.354)%
  --(5.353,5.355)--(5.354,5.357)--(5.355,5.359)--(5.356,5.361)--(5.357,5.363)--(5.358,5.365)%
  --(5.360,5.367)--(5.361,5.369)--(5.362,5.371)--(5.363,5.373)--(5.364,5.375)--(5.365,5.377)%
  --(5.366,5.379)--(5.368,5.381)--(5.369,5.384)--(5.370,5.386)--(5.372,5.389)--(5.373,5.392)%
  --(5.375,5.394)--(5.376,5.397)--(5.378,5.400)--(5.379,5.402)--(5.381,5.405)--(5.382,5.407)%
  --(5.383,5.409)--(5.384,5.411)--(5.385,5.413)--(5.387,5.415)--(5.388,5.417)--(5.389,5.419)%
  --(5.390,5.421)--(5.391,5.423)--(5.392,5.425)--(5.393,5.427)--(5.394,5.429)--(5.395,5.431)%
  --(5.396,5.433)--(5.397,5.435)--(5.398,5.437)--(5.399,5.439)--(5.401,5.441)--(5.402,5.443)%
  --(5.403,5.446)--(5.404,5.448)--(5.406,5.450)--(5.408,5.453)--(5.409,5.456)--(5.411,5.459)%
  --(5.412,5.461)--(5.413,5.463)--(5.414,5.465)--(5.415,5.467)--(5.416,5.469)--(5.417,5.471)%
  --(5.419,5.473)--(5.420,5.475)--(5.421,5.477)--(5.422,5.479)--(5.423,5.481)--(5.424,5.483)%
  --(5.425,5.485)--(5.426,5.486)--(5.427,5.488)--(5.428,5.490)--(5.429,5.492)--(5.431,5.494)%
  --(5.432,5.496)--(5.433,5.498)--(5.434,5.500)--(5.435,5.502)--(5.436,5.504)--(5.437,5.506)%
  --(5.439,5.509)--(5.440,5.511)--(5.442,5.514)--(5.443,5.516)--(5.444,5.518)--(5.445,5.520)%
  --(5.446,5.522)--(5.448,5.524)--(5.449,5.527)--(5.450,5.529)--(5.451,5.531)--(5.452,5.533)%
  --(5.453,5.535)--(5.455,5.537)--(5.456,5.539)--(5.457,5.541)--(5.458,5.542)--(5.459,5.544)%
  --(5.460,5.546)--(5.461,5.548)--(5.462,5.550)--(5.463,5.552)--(5.464,5.554)--(5.465,5.556)%
  --(5.467,5.558)--(5.468,5.560)--(5.469,5.562)--(5.470,5.564)--(5.471,5.566)--(5.473,5.568)%
  --(5.474,5.571)--(5.475,5.573)--(5.476,5.575)--(5.478,5.577)--(5.479,5.580)--(5.480,5.582)%
  --(5.482,5.585)--(5.483,5.587)--(5.485,5.589)--(5.486,5.592)--(5.487,5.594)--(5.488,5.596)%
  --(5.489,5.598)--(5.490,5.600)--(5.492,5.601)--(5.493,5.603)--(5.494,5.605)--(5.495,5.607)%
  --(5.496,5.609)--(5.497,5.611)--(5.498,5.613)--(5.499,5.614)--(5.500,5.616)--(5.501,5.618)%
  --(5.502,5.620)--(5.504,5.622)--(5.505,5.624)--(5.506,5.626)--(5.507,5.629)--(5.509,5.631)%
  --(5.510,5.634)--(5.512,5.636)--(5.513,5.639)--(5.515,5.642)--(5.517,5.645)--(5.518,5.647)%
  --(5.519,5.649)--(5.521,5.652)--(5.522,5.654)--(5.523,5.656)--(5.524,5.657)--(5.525,5.659)%
  --(5.526,5.661)--(5.527,5.663)--(5.528,5.665)--(5.529,5.667)--(5.530,5.669)--(5.531,5.670)%
  --(5.533,5.672)--(5.534,5.674)--(5.535,5.676)--(5.536,5.678)--(5.537,5.680)--(5.538,5.682)%
  --(5.540,5.684)--(5.541,5.686)--(5.542,5.688)--(5.543,5.690)--(5.545,5.693)--(5.546,5.695)%
  --(5.547,5.698)--(5.549,5.700)--(5.550,5.702)--(5.551,5.705)--(5.553,5.707)--(5.554,5.709)%
  --(5.555,5.711)--(5.556,5.713)--(5.558,5.715)--(5.559,5.717)--(5.560,5.719)--(5.561,5.721)%
  --(5.562,5.722)--(5.563,5.724)--(5.564,5.726)--(5.565,5.728)--(5.566,5.730)--(5.567,5.731)%
  --(5.568,5.733)--(5.570,5.735)--(5.571,5.737)--(5.572,5.739)--(5.573,5.741)--(5.574,5.743)%
  --(5.575,5.745)--(5.577,5.747)--(5.578,5.750)--(5.580,5.752)--(5.581,5.755)--(5.583,5.757)%
  --(5.584,5.759)--(5.585,5.761)--(5.586,5.763)--(5.587,5.765)--(5.588,5.767)--(5.590,5.769)%
  --(5.591,5.771)--(5.592,5.773)--(5.593,5.775)--(5.594,5.776)--(5.595,5.778)--(5.596,5.780)%
  --(5.597,5.782)--(5.598,5.784)--(5.599,5.786)--(5.600,5.787)--(5.601,5.789)--(5.602,5.791)%
  --(5.604,5.793)--(5.605,5.795)--(5.606,5.797)--(5.607,5.799)--(5.608,5.801)--(5.609,5.803)%
  --(5.611,5.805)--(5.612,5.807)--(5.614,5.809)--(5.615,5.812)--(5.616,5.814)--(5.617,5.816)%
  --(5.618,5.818)--(5.620,5.820)--(5.621,5.822)--(5.622,5.824)--(5.623,5.826)--(5.625,5.828)%
  --(5.626,5.830)--(5.627,5.832)--(5.628,5.834)--(5.629,5.835)--(5.630,5.837)--(5.631,5.839)%
  --(5.632,5.841)--(5.633,5.842)--(5.634,5.844)--(5.635,5.846)--(5.636,5.848)--(5.638,5.850)%
  --(5.639,5.851)--(5.640,5.853)--(5.641,5.855)--(5.642,5.857)--(5.643,5.859)--(5.645,5.861)%
  --(5.646,5.863)--(5.647,5.865)--(5.649,5.868)--(5.650,5.870)--(5.651,5.872)--(5.653,5.874)%
  --(5.654,5.876)--(5.655,5.879)--(5.657,5.881)--(5.658,5.883)--(5.659,5.885)--(5.660,5.887)%
  --(5.661,5.889)--(5.662,5.891)--(5.663,5.892)--(5.665,5.894)--(5.666,5.896)--(5.667,5.898)%
  --(5.668,5.899)--(5.669,5.901)--(5.670,5.903)--(5.671,5.905)--(5.672,5.907)--(5.673,5.908)%
  --(5.674,5.910)--(5.676,5.912)--(5.677,5.914)--(5.678,5.916)--(5.679,5.918)--(5.680,5.920)%
  --(5.682,5.923)--(5.684,5.925)--(5.685,5.928)--(5.687,5.930)--(5.688,5.933)--(5.690,5.935)%
  --(5.691,5.937)--(5.692,5.940)--(5.693,5.942)--(5.695,5.943)--(5.696,5.945)--(5.697,5.947)%
  --(5.698,5.949)--(5.699,5.951)--(5.700,5.952)--(5.701,5.954)--(5.702,5.956)--(5.703,5.957)%
  --(5.704,5.959)--(5.706,5.961)--(5.707,5.963)--(5.708,5.965)--(5.709,5.966)--(5.710,5.968)%
  --(5.711,5.970)--(5.712,5.972)--(5.714,5.974)--(5.715,5.976)--(5.716,5.979)--(5.718,5.981)%
  --(5.720,5.984)--(5.721,5.986)--(5.722,5.988)--(5.723,5.990)--(5.724,5.992)--(5.726,5.993)%
  --(5.727,5.995)--(5.728,5.997)--(5.729,5.999)--(5.730,6.001)--(5.731,6.002)--(5.732,6.004)%
  --(5.733,6.006)--(5.734,6.008)--(5.736,6.009)--(5.737,6.011)--(5.738,6.013)--(5.739,6.015)%
  --(5.740,6.016)--(5.741,6.018)--(5.742,6.020)--(5.743,6.022)--(5.744,6.024)--(5.746,6.025)%
  --(5.747,6.027)--(5.748,6.029)--(5.749,6.031)--(5.751,6.034)--(5.752,6.036)--(5.753,6.038)%
  --(5.755,6.040)--(5.756,6.042)--(5.757,6.044)--(5.758,6.046)--(5.759,6.048)--(5.761,6.050)%
  --(5.762,6.051)--(5.763,6.053)--(5.764,6.055)--(5.765,6.057)--(5.766,6.059)--(5.767,6.060)%
  --(5.768,6.062)--(5.770,6.064)--(5.771,6.065)--(5.772,6.067)--(5.773,6.069)--(5.774,6.070)%
  --(5.775,6.072)--(5.776,6.074)--(5.777,6.076)--(5.778,6.078)--(5.779,6.079)--(5.781,6.081)%
  --(5.782,6.083)--(5.783,6.085)--(5.784,6.087)--(5.786,6.089)--(5.787,6.091)--(5.788,6.093)%
  --(5.790,6.096)--(5.791,6.098)--(5.792,6.100)--(5.794,6.102)--(5.795,6.104)--(5.797,6.107)%
  --(5.798,6.108)--(5.799,6.110)--(5.800,6.112)--(5.801,6.114)--(5.802,6.115)--(5.803,6.117)%
  --(5.804,6.119)--(5.805,6.120)--(5.807,6.122)--(5.808,6.124)--(5.809,6.125)--(5.810,6.127)%
  --(5.811,6.129)--(5.812,6.131)--(5.813,6.132)--(5.814,6.134)--(5.815,6.136)--(5.817,6.138)%
  --(5.818,6.140)--(5.819,6.142)--(5.821,6.144)--(5.822,6.146)--(5.823,6.148)--(5.825,6.151)%
  --(5.826,6.153)--(5.827,6.155)--(5.829,6.157)--(5.830,6.159)--(5.831,6.161)--(5.832,6.162)%
  --(5.833,6.164)--(5.835,6.166)--(5.836,6.168)--(5.837,6.169)--(5.838,6.171)--(5.839,6.173)%
  --(5.840,6.174)--(5.841,6.176)--(5.842,6.178)--(5.843,6.179)--(5.844,6.181)--(5.845,6.183)%
  --(5.846,6.184)--(5.848,6.186)--(5.849,6.188)--(5.850,6.190)--(5.851,6.192)--(5.852,6.194)%
  --(5.854,6.196)--(5.856,6.198)--(5.857,6.201)--(5.859,6.203)--(5.860,6.205)--(5.861,6.206)%
  --(5.862,6.208)--(5.863,6.210)--(5.864,6.212)--(5.865,6.213)--(5.867,6.215)--(5.868,6.217)%
  --(5.869,6.219)--(5.870,6.220)--(5.871,6.222)--(5.872,6.224)--(5.873,6.225)--(5.874,6.227)%
  --(5.875,6.228)--(5.876,6.230)--(5.877,6.232)--(5.879,6.233)--(5.880,6.235)--(5.881,6.237)%
  --(5.882,6.239)--(5.883,6.240)--(5.884,6.242)--(5.885,6.244)--(5.887,6.246)--(5.888,6.248)%
  --(5.889,6.250)--(5.891,6.252)--(5.892,6.254)--(5.893,6.256)--(5.894,6.258)--(5.896,6.260)%
  --(5.897,6.262)--(5.898,6.264)--(5.899,6.265)--(5.900,6.267)--(5.902,6.269)--(5.903,6.270)%
  --(5.904,6.272)--(5.905,6.274)--(5.906,6.275)--(5.907,6.277)--(5.908,6.279)--(5.909,6.280)%
  --(5.910,6.282)--(5.911,6.283)--(5.912,6.285)--(5.913,6.287)--(5.915,6.288)--(5.916,6.290)%
  --(5.917,6.292)--(5.918,6.294)--(5.919,6.295)--(5.921,6.297)--(5.922,6.299)--(5.923,6.301)%
  --(5.924,6.303)--(5.926,6.305)--(5.927,6.307)--(5.929,6.309)--(5.930,6.311)--(5.931,6.313)%
  --(5.933,6.315)--(5.934,6.317)--(5.935,6.319)--(5.936,6.321)--(5.937,6.322)--(5.938,6.324)%
  --(5.940,6.326)--(5.941,6.327)--(5.942,6.329)--(5.943,6.330)--(5.944,6.332)--(5.945,6.334)%
  --(5.946,6.335)--(5.947,6.337)--(5.948,6.339)--(5.949,6.340)--(5.950,6.342)--(5.951,6.344)%
  --(5.953,6.345)--(5.954,6.347)--(5.955,6.349)--(5.956,6.351)--(5.958,6.353)--(5.960,6.356)%
  --(5.961,6.358)--(5.963,6.360)--(5.964,6.363)--(5.965,6.365)--(5.967,6.367)--(5.968,6.369)%
  --(5.970,6.370)--(5.971,6.372)--(5.972,6.374)--(5.973,6.375)--(5.974,6.377)--(5.975,6.379)%
  --(5.976,6.380)--(5.977,6.382)--(5.978,6.383)--(5.979,6.385)--(5.980,6.386)--(5.981,6.388)%
  --(5.982,6.390)--(5.984,6.391)--(5.985,6.393)--(5.986,6.395)--(5.987,6.396)--(5.988,6.398)%
  --(5.990,6.400)--(5.991,6.402)--(5.992,6.404)--(5.994,6.406)--(5.996,6.409)--(5.997,6.411)%
  --(5.999,6.413)--(6.000,6.415)--(6.001,6.417)--(6.003,6.419)--(6.004,6.421)--(6.005,6.423)%
  --(6.006,6.424)--(6.007,6.426)--(6.008,6.427)--(6.009,6.429)--(6.011,6.431)--(6.012,6.432)%
  --(6.013,6.434)--(6.014,6.435)--(6.015,6.437)--(6.016,6.438)--(6.017,6.440)--(6.018,6.442)%
  --(6.019,6.443)--(6.020,6.445)--(6.022,6.447)--(6.023,6.448)--(6.024,6.450)--(6.025,6.452)%
  --(6.027,6.454)--(6.028,6.457)--(6.030,6.459)--(6.032,6.461)--(6.033,6.463)--(6.034,6.465)%
  --(6.036,6.467)--(6.037,6.469)--(6.038,6.471)--(6.039,6.473)--(6.041,6.474)--(6.042,6.476)%
  --(6.043,6.477)--(6.044,6.479)--(6.045,6.480)--(6.046,6.482)--(6.047,6.484)--(6.048,6.485)%
  --(6.049,6.487)--(6.050,6.488)--(6.051,6.490)--(6.053,6.491)--(6.054,6.493)--(6.055,6.495)%
  --(6.056,6.496)--(6.057,6.498)--(6.058,6.500)--(6.060,6.501)--(6.061,6.504)--(6.063,6.506)%
  --(6.064,6.508)--(6.066,6.510)--(6.067,6.512)--(6.069,6.514)--(6.070,6.516)--(6.072,6.518)%
  --(6.073,6.520)--(6.074,6.522)--(6.075,6.523)--(6.076,6.525)--(6.077,6.526)--(6.078,6.528)%
  --(6.079,6.529)--(6.080,6.531)--(6.082,6.533)--(6.083,6.534)--(6.084,6.536)--(6.085,6.537)%
  --(6.086,6.539)--(6.087,6.540)--(6.088,6.542)--(6.089,6.543)--(6.090,6.545)--(6.092,6.547)%
  --(6.093,6.549)--(6.094,6.550)--(6.096,6.553)--(6.097,6.555)--(6.099,6.557)--(6.100,6.559)%
  --(6.102,6.561)--(6.103,6.562)--(6.104,6.564)--(6.105,6.566)--(6.106,6.567)--(6.107,6.569)%
  --(6.108,6.570)--(6.109,6.572)--(6.110,6.573)--(6.112,6.575)--(6.113,6.576)--(6.114,6.578)%
  --(6.115,6.579)--(6.116,6.581)--(6.117,6.582)--(6.118,6.584)--(6.119,6.585)--(6.120,6.587)%
  --(6.121,6.589)--(6.122,6.590)--(6.124,6.592)--(6.125,6.593)--(6.126,6.595)--(6.127,6.597)%
  --(6.129,6.598)--(6.130,6.600)--(6.131,6.602)--(6.133,6.604)--(6.134,6.606)--(6.135,6.608)%
  --(6.136,6.609)--(6.137,6.611)--(6.139,6.613)--(6.140,6.614)--(6.141,6.616)--(6.142,6.617)%
  --(6.143,6.619)--(6.144,6.621)--(6.146,6.622)--(6.147,6.624)--(6.148,6.625)--(6.149,6.627)%
  --(6.150,6.628)--(6.151,6.629)--(6.152,6.631)--(6.153,6.633)--(6.154,6.634)--(6.155,6.635)%
  --(6.156,6.637)--(6.157,6.639)--(6.159,6.640)--(6.160,6.642)--(6.161,6.644)--(6.162,6.645)%
  --(6.164,6.647)--(6.165,6.649)--(6.166,6.651)--(6.168,6.653)--(6.169,6.655)--(6.170,6.656)%
  --(6.172,6.658)--(6.173,6.660)--(6.174,6.662)--(6.176,6.664)--(6.177,6.665)--(6.178,6.667)%
  --(6.179,6.668)--(6.180,6.670)--(6.181,6.671)--(6.182,6.673)--(6.184,6.674)--(6.185,6.676)%
  --(6.186,6.677)--(6.187,6.679)--(6.188,6.680)--(6.189,6.682)--(6.190,6.683)--(6.191,6.685)%
  --(6.192,6.686)--(6.193,6.688)--(6.195,6.689)--(6.196,6.691)--(6.197,6.693)--(6.198,6.694)%
  --(6.200,6.696)--(6.201,6.698)--(6.202,6.700)--(6.204,6.702)--(6.205,6.704)--(6.207,6.706)%
  --(6.208,6.707)--(6.209,6.709)--(6.210,6.711)--(6.212,6.712)--(6.213,6.714)--(6.214,6.715)%
  --(6.215,6.717)--(6.216,6.718)--(6.217,6.720)--(6.218,6.721)--(6.219,6.723)--(6.220,6.724)%
  --(6.221,6.725)--(6.222,6.727)--(6.224,6.728)--(6.225,6.730)--(6.226,6.731)--(6.227,6.733)%
  --(6.228,6.734)--(6.229,6.736)--(6.231,6.738)--(6.232,6.739)--(6.233,6.741)--(6.235,6.743)%
  --(6.236,6.745)--(6.238,6.747)--(6.239,6.749)--(6.240,6.750)--(6.241,6.752)--(6.242,6.754)%
  --(6.243,6.755)--(6.245,6.757)--(6.246,6.758)--(6.247,6.760)--(6.248,6.761)--(6.249,6.762)%
  --(6.250,6.764)--(6.251,6.765)--(6.252,6.767)--(6.253,6.768)--(6.254,6.770)--(6.255,6.771)%
  --(6.257,6.772)--(6.258,6.774)--(6.259,6.775)--(6.260,6.777)--(6.261,6.778)--(6.262,6.780)%
  --(6.263,6.781)--(6.265,6.783)--(6.266,6.785)--(6.267,6.786)--(6.269,6.788)--(6.270,6.790)%
  --(6.271,6.792)--(6.272,6.793)--(6.274,6.795)--(6.275,6.796)--(6.276,6.798)--(6.277,6.800)%
  --(6.279,6.801)--(6.280,6.803)--(6.281,6.804)--(6.282,6.806)--(6.283,6.807)--(6.284,6.809)%
  --(6.285,6.810)--(6.286,6.811)--(6.287,6.813)--(6.288,6.814)--(6.289,6.815)--(6.290,6.817)%
  --(6.292,6.818)--(6.293,6.820)--(6.294,6.821)--(6.295,6.823)--(6.296,6.824)--(6.297,6.826)%
  --(6.299,6.827)--(6.300,6.829)--(6.301,6.831)--(6.302,6.832)--(6.304,6.834)--(6.305,6.836)%
  --(6.306,6.838)--(6.308,6.839)--(6.309,6.841)--(6.310,6.843)--(6.312,6.845)--(6.313,6.846)%
  --(6.314,6.848)--(6.315,6.849)--(6.316,6.851)--(6.318,6.852)--(6.319,6.854)--(6.320,6.855)%
  --(6.321,6.856)--(6.322,6.858)--(6.323,6.859)--(6.324,6.860)--(6.325,6.862)--(6.326,6.863)%
  --(6.327,6.865)--(6.329,6.866)--(6.330,6.867)--(6.331,6.869)--(6.332,6.870)--(6.333,6.872)%
  --(6.334,6.873)--(6.336,6.875)--(6.337,6.877)--(6.339,6.879)--(6.340,6.881)--(6.342,6.883)%
  --(6.343,6.885)--(6.345,6.887)--(6.346,6.888)--(6.347,6.890)--(6.349,6.892)--(6.350,6.893)%
  --(6.351,6.895)--(6.352,6.896)--(6.353,6.897)--(6.354,6.899)--(6.355,6.900)--(6.356,6.901)%
  --(6.357,6.903)--(6.358,6.904)--(6.360,6.906)--(6.361,6.907)--(6.362,6.908)--(6.363,6.910)%
  --(6.364,6.911)--(6.365,6.913)--(6.366,6.914)--(6.368,6.916)--(6.369,6.917)--(6.370,6.919)%
  --(6.371,6.921)--(6.373,6.923)--(6.375,6.925)--(6.376,6.927)--(6.377,6.928)--(6.378,6.930)%
  --(6.380,6.931)--(6.381,6.932)--(6.382,6.934)--(6.383,6.935)--(6.384,6.937)--(6.385,6.938)%
  --(6.386,6.939)--(6.387,6.941)--(6.388,6.942)--(6.390,6.943)--(6.391,6.945)--(6.392,6.946)%
  --(6.393,6.947)--(6.394,6.949)--(6.395,6.950)--(6.396,6.952)--(6.397,6.953)--(6.398,6.954)%
  --(6.399,6.956)--(6.401,6.957)--(6.402,6.959)--(6.403,6.960)--(6.404,6.962)--(6.406,6.963)%
  --(6.407,6.965)--(6.409,6.967)--(6.410,6.968)--(6.411,6.970)--(6.412,6.971)--(6.413,6.973)%
  --(6.415,6.974)--(6.416,6.976)--(6.417,6.977)--(6.418,6.979)--(6.419,6.980)--(6.420,6.981)%
  --(6.421,6.983)--(6.422,6.984)--(6.424,6.985)--(6.425,6.987)--(6.426,6.988)--(6.427,6.989)%
  --(6.428,6.991)--(6.429,6.992)--(6.430,6.993)--(6.431,6.995)--(6.432,6.996)--(6.433,6.998)%
  --(6.435,6.999)--(6.436,7.000)--(6.437,7.002)--(6.438,7.003)--(6.439,7.005)--(6.441,7.007)%
  --(6.442,7.008)--(6.443,7.010)--(6.445,7.012)--(6.446,7.013)--(6.448,7.015)--(6.449,7.016)%
  --(6.450,7.018)--(6.451,7.019)--(6.452,7.020)--(6.453,7.022)--(6.454,7.023)--(6.455,7.024)%
  --(6.456,7.026)--(6.457,7.027)--(6.458,7.028)--(6.460,7.030)--(6.461,7.031)--(6.462,7.032)%
  --(6.463,7.033)--(6.464,7.035)--(6.465,7.036)--(6.466,7.037)--(6.467,7.039)--(6.468,7.040)%
  --(6.469,7.041)--(6.471,7.043)--(6.472,7.044)--(6.473,7.046)--(6.475,7.048)--(6.476,7.049)%
  --(6.478,7.051)--(6.479,7.053)--(6.480,7.055)--(6.482,7.056)--(6.483,7.057)--(6.484,7.059)%
  --(6.485,7.060)--(6.486,7.061)--(6.487,7.063)--(6.488,7.064)--(6.489,7.065)--(6.490,7.067)%
  --(6.492,7.068)--(6.493,7.069)--(6.494,7.071)--(6.495,7.072)--(6.496,7.073)--(6.497,7.074)%
  --(6.498,7.076)--(6.499,7.077)--(6.500,7.078)--(6.501,7.080)--(6.503,7.081)--(6.504,7.082)%
  --(6.505,7.084)--(6.506,7.085)--(6.507,7.087)--(6.509,7.088)--(6.510,7.090)--(6.512,7.092)%
  --(6.513,7.093)--(6.514,7.095)--(6.515,7.096)--(6.516,7.097)--(6.518,7.099)--(6.519,7.100)%
  --(6.520,7.102)--(6.521,7.103)--(6.522,7.104)--(6.523,7.106)--(6.524,7.107)--(6.526,7.108)%
  --(6.527,7.110)--(6.528,7.111)--(6.529,7.112)--(6.530,7.113)--(6.531,7.115)--(6.532,7.116)%
  --(6.533,7.117)--(6.534,7.118)--(6.535,7.120)--(6.536,7.121)--(6.538,7.122)--(6.539,7.124)%
  --(6.540,7.125)--(6.541,7.127)--(6.543,7.128)--(6.544,7.130)--(6.545,7.131)--(6.547,7.133)%
  --(6.548,7.134)--(6.549,7.136)--(6.551,7.137)--(6.552,7.139)--(6.553,7.140)--(6.555,7.142)%
  --(6.556,7.143)--(6.557,7.145)--(6.558,7.146)--(6.559,7.147)--(6.560,7.148)--(6.561,7.150)%
  --(6.562,7.151)--(6.563,7.152)--(6.564,7.153)--(6.565,7.155)--(6.567,7.156)--(6.568,7.157)%
  --(6.569,7.159)--(6.570,7.160)--(6.571,7.161)--(6.572,7.162)--(6.573,7.164)--(6.575,7.165)%
  --(6.576,7.167)--(6.577,7.168)--(6.579,7.170)--(6.580,7.172)--(6.582,7.174)--(6.584,7.175)%
  --(6.585,7.177)--(6.586,7.179)--(6.588,7.180)--(6.589,7.182)--(6.590,7.183)--(6.591,7.184)%
  --(6.592,7.186)--(6.594,7.187)--(6.595,7.188)--(6.596,7.189)--(6.597,7.191)--(6.598,7.192)%
  --(6.599,7.193)--(6.600,7.194)--(6.601,7.195)--(6.602,7.197)--(6.603,7.198)--(6.604,7.199)%
  --(6.605,7.200)--(6.607,7.202)--(6.608,7.203)--(6.609,7.204)--(6.610,7.206)--(6.612,7.207)%
  --(6.613,7.209)--(6.615,7.211)--(6.616,7.213)--(6.618,7.214)--(6.619,7.216)--(6.621,7.218)%
  --(6.622,7.219)--(6.624,7.221)--(6.625,7.222)--(6.626,7.223)--(6.627,7.224)--(6.628,7.226)%
  --(6.629,7.227)--(6.630,7.228)--(6.631,7.229)--(6.632,7.231)--(6.634,7.232)--(6.635,7.233)%
  --(6.636,7.234)--(6.637,7.235)--(6.638,7.237)--(6.639,7.238)--(6.640,7.239)--(6.641,7.240)%
  --(6.642,7.242)--(6.643,7.243)--(6.645,7.245)--(6.646,7.246)--(6.648,7.248)--(6.649,7.249)%
  --(6.651,7.251)--(6.652,7.253)--(6.654,7.255)--(6.655,7.256)--(6.657,7.258)--(6.658,7.259)%
  --(6.659,7.260)--(6.660,7.262)--(6.661,7.263)--(6.663,7.264)--(6.664,7.265)--(6.665,7.266)%
  --(6.666,7.268)--(6.667,7.269)--(6.668,7.270)--(6.669,7.271)--(6.670,7.272)--(6.671,7.274)%
  --(6.672,7.275)--(6.673,7.276)--(6.675,7.277)--(6.676,7.278)--(6.677,7.280)--(6.678,7.281)%
  --(6.679,7.282)--(6.680,7.284)--(6.682,7.285)--(6.684,7.287)--(6.685,7.289)--(6.687,7.291)%
  --(6.688,7.292)--(6.690,7.294)--(6.691,7.295)--(6.693,7.297)--(6.694,7.298)--(6.695,7.299)%
  --(6.696,7.300)--(6.697,7.302)--(6.698,7.303)--(6.699,7.304)--(6.700,7.305)--(6.701,7.306)%
  --(6.702,7.307)--(6.703,7.309)--(6.704,7.310)--(6.706,7.311)--(6.707,7.312)--(6.708,7.313)%
  --(6.709,7.315)--(6.710,7.316)--(6.711,7.317)--(6.713,7.318)--(6.714,7.320)--(6.715,7.321)%
  --(6.717,7.323)--(6.718,7.325)--(6.720,7.326)--(6.721,7.328)--(6.723,7.330)--(6.724,7.331)%
  --(6.726,7.332)--(6.727,7.334)--(6.728,7.335)--(6.729,7.336)--(6.730,7.338)--(6.731,7.339)%
  --(6.733,7.340)--(6.734,7.341)--(6.735,7.342)--(6.736,7.343)--(6.737,7.344)--(6.738,7.346)%
  --(6.739,7.347)--(6.740,7.348)--(6.741,7.349)--(6.742,7.350)--(6.743,7.351)--(6.745,7.353)%
  --(6.746,7.354)--(6.747,7.355)--(6.748,7.356)--(6.750,7.358)--(6.751,7.359)--(6.753,7.361)%
  --(6.754,7.363)--(6.756,7.364)--(6.757,7.366)--(6.759,7.367)--(6.760,7.369)--(6.761,7.370)%
  --(6.763,7.371)--(6.764,7.373)--(6.765,7.374)--(6.766,7.375)--(6.767,7.376)--(6.768,7.377)%
  --(6.769,7.378)--(6.770,7.379)--(6.771,7.381)--(6.773,7.382)--(6.774,7.383)--(6.775,7.384)%
  --(6.776,7.385)--(6.777,7.386)--(6.778,7.387)--(6.779,7.389)--(6.780,7.390)--(6.781,7.391)%
  --(6.783,7.392)--(6.784,7.394)--(6.786,7.395)--(6.787,7.397)--(6.789,7.399)--(6.790,7.400)%
  --(6.792,7.402)--(6.793,7.403)--(6.795,7.405)--(6.796,7.406)--(6.797,7.407)--(6.798,7.408)%
  --(6.799,7.410)--(6.801,7.411)--(6.802,7.412)--(6.803,7.413)--(6.804,7.414)--(6.805,7.415)%
  --(6.806,7.416)--(6.807,7.417)--(6.808,7.418)--(6.809,7.420)--(6.810,7.421)--(6.811,7.422)%
  --(6.812,7.423)--(6.814,7.424)--(6.815,7.425)--(6.816,7.426)--(6.817,7.428)--(6.818,7.429)%
  --(6.820,7.431)--(6.822,7.432)--(6.823,7.434)--(6.825,7.435)--(6.826,7.437)--(6.828,7.438)%
  --(6.829,7.440)--(6.831,7.441)--(6.832,7.442)--(6.833,7.443)--(6.834,7.445)--(6.835,7.446)%
  --(6.836,7.447)--(6.837,7.448)--(6.838,7.449)--(6.839,7.450)--(6.840,7.451)--(6.841,7.452)%
  --(6.842,7.453)--(6.843,7.454)--(6.845,7.455)--(6.846,7.457)--(6.847,7.458)--(6.848,7.459)%
  --(6.849,7.460)--(6.851,7.461)--(6.852,7.463)--(6.853,7.464)--(6.855,7.465)--(6.856,7.467)%
  --(6.858,7.469)--(6.859,7.470)--(6.861,7.472)--(6.862,7.473)--(6.864,7.474)--(6.865,7.476)%
  --(6.866,7.477)--(6.867,7.478)--(6.868,7.479)--(6.869,7.480)--(6.870,7.481)--(6.872,7.482)%
  --(6.873,7.483)--(6.874,7.484)--(6.875,7.485)--(6.876,7.487)--(6.877,7.488)--(6.878,7.489)%
  --(6.879,7.490)--(6.880,7.491)--(6.882,7.492)--(6.883,7.493)--(6.884,7.494)--(6.885,7.495)%
  --(6.886,7.497)--(6.887,7.498)--(6.889,7.499)--(6.891,7.501)--(6.892,7.503)--(6.894,7.504)%
  --(6.895,7.505)--(6.897,7.507)--(6.898,7.508)--(6.900,7.509)--(6.901,7.511)--(6.902,7.512)%
  --(6.903,7.513)--(6.904,7.514)--(6.905,7.515)--(6.906,7.516)--(6.907,7.517)--(6.908,7.518)%
  --(6.909,7.519)--(6.910,7.520)--(6.912,7.521)--(6.913,7.522)--(6.914,7.523)--(6.915,7.524)%
  --(6.916,7.525)--(6.917,7.527)--(6.918,7.528)--(6.920,7.529)--(6.921,7.530)--(6.922,7.531)%
  --(6.924,7.533)--(6.925,7.534)--(6.927,7.536)--(6.929,7.538)--(6.930,7.539)--(6.931,7.540)%
  --(6.933,7.542)--(6.934,7.543)--(6.935,7.544)--(6.936,7.545)--(6.938,7.546)--(6.939,7.547)%
  --(6.940,7.548)--(6.941,7.549)--(6.942,7.550)--(6.943,7.551)--(6.944,7.552)--(6.945,7.553)%
  --(6.946,7.554)--(6.947,7.555)--(6.948,7.556)--(6.950,7.557)--(6.951,7.558)--(6.952,7.559)%
  --(6.953,7.561)--(6.954,7.562)--(6.955,7.563)--(6.957,7.564)--(6.958,7.566)--(6.960,7.567)%
  --(6.962,7.569)--(6.963,7.570)--(6.965,7.572)--(6.966,7.573)--(6.968,7.574)--(6.969,7.576)%
  --(6.970,7.577)--(6.971,7.578)--(6.972,7.579)--(6.973,7.580)--(6.974,7.581)--(6.975,7.582)%
  --(6.977,7.583)--(6.978,7.584)--(6.979,7.585)--(6.980,7.586)--(6.981,7.587)--(6.982,7.588)%
  --(6.983,7.589)--(6.984,7.590)--(6.985,7.591)--(6.987,7.592)--(6.988,7.593)--(6.989,7.594)%
  --(6.990,7.595)--(6.991,7.597)--(6.993,7.598)--(6.994,7.599)--(6.995,7.600)--(6.997,7.602)%
  --(6.998,7.603)--(7.000,7.604)--(7.001,7.605)--(7.002,7.607)--(7.004,7.608)--(7.005,7.609)%
  --(7.006,7.610)--(7.007,7.611)--(7.008,7.612)--(7.009,7.613)--(7.010,7.614)--(7.011,7.615)%
  --(7.012,7.616)--(7.013,7.617)--(7.014,7.618)--(7.015,7.618)--(7.016,7.620)--(7.018,7.620)%
  --(7.019,7.622)--(7.020,7.622)--(7.021,7.624)--(7.022,7.625)--(7.024,7.626)--(7.025,7.627)%
  --(7.026,7.628)--(7.028,7.630)--(7.029,7.631)--(7.031,7.632)--(7.032,7.633)--(7.033,7.634)%
  --(7.034,7.635)--(7.035,7.636)--(7.036,7.637)--(7.038,7.638)--(7.039,7.639)--(7.040,7.640)%
  --(7.041,7.641)--(7.042,7.642)--(7.043,7.643)--(7.044,7.644)--(7.045,7.645)--(7.046,7.646)%
  --(7.048,7.647)--(7.049,7.648)--(7.050,7.649)--(7.051,7.650)--(7.052,7.651)--(7.053,7.652)%
  --(7.054,7.653)--(7.055,7.654)--(7.056,7.655)--(7.058,7.656)--(7.059,7.658)--(7.060,7.659)%
  --(7.062,7.660)--(7.063,7.661)--(7.064,7.662)--(7.066,7.663)--(7.067,7.664)--(7.068,7.665)%
  --(7.069,7.666)--(7.070,7.668)--(7.072,7.669)--(7.073,7.670)--(7.074,7.671)--(7.075,7.672)%
  --(7.076,7.673)--(7.077,7.674)--(7.078,7.674)--(7.079,7.675)--(7.080,7.676)--(7.082,7.677)%
  --(7.083,7.678)--(7.084,7.679)--(7.085,7.680)--(7.086,7.681)--(7.087,7.682)--(7.088,7.683)%
  --(7.089,7.684)--(7.090,7.685)--(7.092,7.686)--(7.093,7.687)--(7.094,7.688)--(7.095,7.689)%
  --(7.097,7.690)--(7.098,7.692)--(7.099,7.693)--(7.101,7.694)--(7.102,7.695)--(7.103,7.696)%
  --(7.105,7.697)--(7.106,7.698)--(7.107,7.699)--(7.109,7.700)--(7.110,7.701)--(7.111,7.702)%
  --(7.112,7.703)--(7.113,7.704)--(7.114,7.705)--(7.115,7.706)--(7.116,7.707)--(7.117,7.708)%
  --(7.118,7.709)--(7.119,7.710)--(7.120,7.711)--(7.121,7.712)--(7.123,7.713)--(7.124,7.714)%
  --(7.125,7.715)--(7.126,7.716)--(7.127,7.717)--(7.129,7.718)--(7.130,7.719)--(7.132,7.720)%
  --(7.133,7.722)--(7.135,7.723)--(7.136,7.724)--(7.138,7.725)--(7.139,7.727)--(7.141,7.728)%
  --(7.142,7.729)--(7.143,7.730)--(7.144,7.731)--(7.145,7.732)--(7.146,7.733)--(7.147,7.733)%
  --(7.148,7.734)--(7.149,7.735)--(7.151,7.736)--(7.152,7.737)--(7.153,7.738)--(7.154,7.739)%
  --(7.155,7.740)--(7.156,7.741)--(7.157,7.742)--(7.158,7.742)--(7.159,7.743)--(7.161,7.745)%
  --(7.162,7.745)--(7.163,7.746)--(7.165,7.748)--(7.166,7.749)--(7.168,7.750)--(7.169,7.752)%
  --(7.171,7.753)--(7.172,7.754)--(7.174,7.755)--(7.175,7.756)--(7.176,7.757)--(7.177,7.758)%
  --(7.178,7.759)--(7.180,7.760)--(7.181,7.761)--(7.182,7.762)--(7.183,7.763)--(7.184,7.764)%
  --(7.185,7.764)--(7.186,7.765)--(7.187,7.766)--(7.188,7.767)--(7.189,7.768)--(7.190,7.769)%
  --(7.191,7.770)--(7.193,7.771)--(7.194,7.772)--(7.195,7.773)--(7.196,7.774)--(7.198,7.775)%
  --(7.199,7.776)--(7.201,7.777)--(7.202,7.779)--(7.204,7.780)--(7.205,7.781)--(7.207,7.782)%
  --(7.208,7.783)--(7.209,7.784)--(7.211,7.785)--(7.212,7.786)--(7.213,7.787)--(7.214,7.788)%
  --(7.215,7.789)--(7.216,7.790)--(7.217,7.791)--(7.218,7.791)--(7.219,7.792)--(7.220,7.793)%
  --(7.221,7.794)--(7.222,7.795)--(7.224,7.796)--(7.225,7.796)--(7.226,7.797)--(7.227,7.798)%
  --(7.228,7.799)--(7.229,7.800)--(7.231,7.801)--(7.232,7.802)--(7.234,7.803)--(7.235,7.804)%
  --(7.236,7.806)--(7.238,7.807)--(7.239,7.808)--(7.240,7.809)--(7.242,7.810)--(7.243,7.810)%
  --(7.244,7.811)--(7.245,7.812)--(7.246,7.813)--(7.247,7.814)--(7.248,7.815)--(7.249,7.816)%
  --(7.250,7.817)--(7.252,7.817)--(7.253,7.818)--(7.254,7.819)--(7.255,7.820)--(7.256,7.821)%
  --(7.257,7.822)--(7.258,7.823)--(7.259,7.823)--(7.260,7.824)--(7.261,7.825)--(7.262,7.826)%
  --(7.264,7.827)--(7.265,7.828)--(7.266,7.829)--(7.268,7.830)--(7.269,7.831)--(7.270,7.832)%
  --(7.272,7.833)--(7.273,7.834)--(7.274,7.835)--(7.275,7.836)--(7.276,7.837)--(7.277,7.838)%
  --(7.279,7.839)--(7.280,7.840)--(7.281,7.840)--(7.282,7.841)--(7.283,7.842)--(7.284,7.843)%
  --(7.285,7.844)--(7.286,7.844)--(7.287,7.845)--(7.289,7.846)--(7.290,7.847)--(7.291,7.848)%
  --(7.292,7.849)--(7.293,7.849)--(7.294,7.850)--(7.295,7.851)--(7.296,7.852)--(7.297,7.853)%
  --(7.299,7.854)--(7.300,7.855)--(7.301,7.856)--(7.303,7.857)--(7.304,7.858)--(7.305,7.859)%
  --(7.307,7.860)--(7.308,7.861)--(7.309,7.862)--(7.311,7.863)--(7.312,7.864)--(7.314,7.865)%
  --(7.315,7.865)--(7.316,7.866)--(7.317,7.867)--(7.318,7.868)--(7.319,7.869)--(7.320,7.870)%
  --(7.321,7.870)--(7.322,7.871)--(7.323,7.872)--(7.324,7.873)--(7.326,7.874)--(7.327,7.874)%
  --(7.328,7.875)--(7.329,7.876)--(7.330,7.877)--(7.331,7.878)--(7.332,7.879)--(7.334,7.880)%
  --(7.335,7.881)--(7.336,7.881)--(7.338,7.883)--(7.339,7.884)--(7.341,7.885)--(7.343,7.886)%
  --(7.344,7.887)--(7.346,7.888)--(7.347,7.889)--(7.348,7.890)--(7.349,7.891)--(7.351,7.892)%
  --(7.352,7.893)--(7.353,7.894)--(7.354,7.894)--(7.355,7.895)--(7.356,7.896)--(7.357,7.897)%
  --(7.358,7.898)--(7.359,7.898)--(7.360,7.899)--(7.361,7.900)--(7.363,7.901)--(7.364,7.901)%
  --(7.365,7.902)--(7.366,7.903)--(7.367,7.904)--(7.368,7.905)--(7.370,7.906)--(7.371,7.907)%
  --(7.372,7.908)--(7.374,7.908)--(7.375,7.909)--(7.376,7.910)--(7.378,7.911)--(7.379,7.912)%
  --(7.380,7.913)--(7.382,7.914)--(7.383,7.915)--(7.384,7.916)--(7.385,7.917)--(7.386,7.917)%
  --(7.387,7.918)--(7.388,7.919)--(7.390,7.920)--(7.391,7.921)--(7.392,7.921)--(7.393,7.922)%
  --(7.394,7.923)--(7.395,7.924)--(7.396,7.924)--(7.397,7.925)--(7.398,7.926)--(7.399,7.927)%
  --(7.401,7.928)--(7.402,7.928)--(7.403,7.929)--(7.404,7.930)--(7.406,7.931)--(7.407,7.932)%
  --(7.409,7.933)--(7.410,7.934)--(7.411,7.935)--(7.413,7.936)--(7.414,7.937)--(7.415,7.938)%
  --(7.416,7.938)--(7.417,7.939)--(7.418,7.940)--(7.419,7.941)--(7.421,7.941)--(7.422,7.942)%
  --(7.423,7.943)--(7.424,7.944)--(7.425,7.944)--(7.426,7.945)--(7.427,7.946)--(7.428,7.947)%
  --(7.429,7.947)--(7.430,7.948)--(7.431,7.949)--(7.432,7.950)--(7.434,7.950)--(7.435,7.951)%
  --(7.436,7.952)--(7.437,7.953)--(7.438,7.954)--(7.440,7.954)--(7.441,7.955)--(7.443,7.956)%
  --(7.444,7.957)--(7.445,7.958)--(7.446,7.959)--(7.447,7.960)--(7.449,7.960)--(7.450,7.961)%
  --(7.451,7.962)--(7.452,7.963)--(7.453,7.964)--(7.454,7.964)--(7.456,7.965)--(7.457,7.966)%
  --(7.458,7.966)--(7.459,7.967)--(7.460,7.968)--(7.461,7.969)--(7.462,7.969)--(7.463,7.970)%
  --(7.464,7.971)--(7.465,7.971)--(7.466,7.972)--(7.468,7.973)--(7.469,7.974)--(7.470,7.975)%
  --(7.471,7.975)--(7.472,7.976)--(7.474,7.977)--(7.475,7.978)--(7.476,7.979)--(7.477,7.980)%
  --(7.479,7.981)--(7.480,7.982)--(7.482,7.982)--(7.483,7.983)--(7.484,7.984)--(7.486,7.985)%
  --(7.487,7.986)--(7.488,7.986)--(7.489,7.987)--(7.490,7.988)--(7.491,7.989)--(7.492,7.989)%
  --(7.493,7.990)--(7.494,7.991)--(7.496,7.991)--(7.497,7.992)--(7.498,7.993)--(7.499,7.993)%
  --(7.500,7.994)--(7.501,7.995)--(7.502,7.996)--(7.503,7.996)--(7.504,7.997)--(7.506,7.998)%
  --(7.507,7.999)--(7.508,7.999)--(7.510,8.000)--(7.511,8.001)--(7.513,8.002)--(7.515,8.004)%
  --(7.516,8.004)--(7.517,8.005)--(7.519,8.006)--(7.520,8.007)--(7.521,8.008)--(7.522,8.008)%
  --(7.524,8.009)--(7.525,8.010)--(7.526,8.011)--(7.527,8.011)--(7.528,8.012)--(7.529,8.013)%
  --(7.530,8.013)--(7.531,8.014)--(7.532,8.015)--(7.533,8.015)--(7.534,8.016)--(7.536,8.017)%
  --(7.537,8.017)--(7.538,8.018)--(7.539,8.019)--(7.540,8.020)--(7.541,8.020)--(7.543,8.021)%
  --(7.544,8.022)--(7.546,8.023)--(7.548,8.024)--(7.549,8.025)--(7.551,8.026)--(7.552,8.027)%
  --(7.553,8.028)--(7.555,8.029)--(7.556,8.029)--(7.557,8.030)--(7.558,8.031)--(7.559,8.031)%
  --(7.560,8.032)--(7.561,8.033)--(7.563,8.033)--(7.564,8.034)--(7.565,8.035)--(7.566,8.035)%
  --(7.567,8.036)--(7.568,8.037)--(7.569,8.037)--(7.570,8.038)--(7.571,8.039)--(7.572,8.039)%
  --(7.574,8.040)--(7.575,8.041)--(7.576,8.042)--(7.577,8.042)--(7.579,8.043)--(7.581,8.044)%
  --(7.582,8.045)--(7.584,8.046)--(7.585,8.047)--(7.587,8.048)--(7.588,8.049)--(7.590,8.050)%
  --(7.591,8.050)--(7.592,8.051)--(7.593,8.052)--(7.594,8.052)--(7.595,8.053)--(7.596,8.053)%
  --(7.597,8.054)--(7.598,8.055)--(7.599,8.055)--(7.600,8.056)--(7.601,8.057)--(7.602,8.057)%
  --(7.604,8.058)--(7.605,8.059)--(7.606,8.059)--(7.607,8.060)--(7.608,8.061)--(7.609,8.061)%
  --(7.611,8.062)--(7.612,8.063)--(7.613,8.064)--(7.615,8.065)--(7.616,8.065)--(7.618,8.066)%
  --(7.619,8.067)--(7.620,8.068)--(7.622,8.069)--(7.623,8.069)--(7.624,8.070)--(7.625,8.071)%
  --(7.626,8.071)--(7.628,8.072)--(7.629,8.073)--(7.630,8.073)--(7.631,8.074)--(7.632,8.074)%
  --(7.633,8.075)--(7.634,8.076)--(7.635,8.076)--(7.636,8.077)--(7.637,8.077)--(7.638,8.078)%
  --(7.639,8.079)--(7.641,8.079)--(7.642,8.080)--(7.643,8.081)--(7.644,8.081)--(7.645,8.082)%
  --(7.647,8.083)--(7.648,8.084)--(7.650,8.084)--(7.651,8.085)--(7.653,8.086)--(7.654,8.087)%
  --(7.655,8.087)--(7.656,8.088)--(7.657,8.088)--(7.658,8.089)--(7.659,8.090)--(7.660,8.090)%
  --(7.662,8.091)--(7.663,8.092)--(7.664,8.092)--(7.665,8.093)--(7.666,8.093)--(7.667,8.094)%
  --(7.668,8.095)--(7.669,8.095)--(7.670,8.096)--(7.671,8.096)--(7.672,8.097)--(7.673,8.098)%
  --(7.675,8.098)--(7.676,8.099)--(7.677,8.100)--(7.678,8.100)--(7.680,8.101)--(7.681,8.102)%
  --(7.682,8.102)--(7.684,8.103)--(7.685,8.104)--(7.686,8.105)--(7.687,8.105)--(7.689,8.106)%
  --(7.690,8.107)--(7.691,8.107)--(7.692,8.108)--(7.693,8.109)--(7.694,8.109)--(7.696,8.110)%
  --(7.697,8.110)--(7.698,8.111)--(7.699,8.111)--(7.700,8.112)--(7.701,8.113)--(7.702,8.113)%
  --(7.703,8.114)--(7.704,8.114)--(7.705,8.115)--(7.706,8.115)--(7.707,8.116)--(7.709,8.117)%
  --(7.710,8.117)--(7.711,8.118)--(7.712,8.118)--(7.713,8.119)--(7.715,8.120)--(7.716,8.120)%
  --(7.717,8.121)--(7.719,8.122)--(7.720,8.123)--(7.721,8.123)--(7.723,8.124)--(7.724,8.125)%
  --(7.726,8.125)--(7.727,8.126)--(7.728,8.127)--(7.729,8.127)--(7.730,8.128)--(7.731,8.128)%
  --(7.732,8.129)--(7.733,8.129)--(7.734,8.130)--(7.736,8.131)--(7.737,8.131)--(7.738,8.132)%
  --(7.739,8.132)--(7.740,8.133)--(7.741,8.133)--(7.742,8.134)--(7.743,8.135)--(7.744,8.135)%
  --(7.746,8.136)--(7.747,8.136)--(7.748,8.137)--(7.749,8.138)--(7.751,8.138)--(7.752,8.139)%
  --(7.754,8.140)--(7.755,8.141)--(7.757,8.142)--(7.758,8.142)--(7.760,8.143)--(7.761,8.144)%
  --(7.762,8.144)--(7.763,8.145)--(7.765,8.145)--(7.766,8.146)--(7.767,8.147)--(7.768,8.147)%
  --(7.769,8.148)--(7.770,8.148)--(7.771,8.149)--(7.772,8.149)--(7.773,8.150)--(7.774,8.150)%
  --(7.775,8.151)--(7.777,8.151)--(7.778,8.152)--(7.779,8.152)--(7.780,8.153)--(7.781,8.154)%
  --(7.782,8.154)--(7.784,8.155)--(7.785,8.155)--(7.787,8.156)--(7.788,8.157)--(7.790,8.158)%
  --(7.791,8.158)--(7.792,8.159)--(7.793,8.159)--(7.794,8.160)--(7.795,8.160)--(7.797,8.161)%
  --(7.798,8.162)--(7.799,8.162)--(7.800,8.163)--(7.801,8.163)--(7.802,8.164)--(7.803,8.164)%
  --(7.804,8.165)--(7.805,8.165)--(7.807,8.166)--(7.808,8.166)--(7.809,8.167)--(7.810,8.167)%
  --(7.811,8.168)--(7.812,8.168)--(7.813,8.169)--(7.814,8.169)--(7.815,8.170)--(7.817,8.171)%
  --(7.818,8.171)--(7.819,8.172)--(7.821,8.173)--(7.822,8.173)--(7.823,8.174)--(7.825,8.174)%
  --(7.826,8.175)--(7.827,8.176)--(7.828,8.176)--(7.829,8.177)--(7.831,8.177)--(7.832,8.178)%
  --(7.833,8.178)--(7.834,8.179)--(7.835,8.179)--(7.836,8.180)--(7.837,8.180)--(7.838,8.181)%
  --(7.839,8.181)--(7.840,8.182)--(7.841,8.182)--(7.843,8.183)--(7.844,8.183)--(7.845,8.184)%
  --(7.846,8.184)--(7.847,8.185)--(7.848,8.185)--(7.849,8.186)--(7.851,8.186)--(7.852,8.187)%
  --(7.853,8.187)--(7.854,8.188)--(7.856,8.189)--(7.857,8.189)--(7.859,8.190)--(7.860,8.190)%
  --(7.861,8.191)--(7.863,8.192)--(7.864,8.192)--(7.865,8.193)--(7.866,8.193)--(7.868,8.194)%
  --(7.869,8.194)--(7.870,8.195)--(7.871,8.195)--(7.872,8.196)--(7.873,8.196)--(7.874,8.197)%
  --(7.875,8.197)--(7.876,8.198)--(7.877,8.198)--(7.879,8.199)--(7.880,8.199)--(7.881,8.200)%
  --(7.882,8.200)--(7.883,8.201)--(7.884,8.201)--(7.886,8.202)--(7.887,8.202)--(7.888,8.203)%
  --(7.889,8.203)--(7.891,8.204)--(7.892,8.205)--(7.893,8.205)--(7.895,8.206)--(7.896,8.206)%
  --(7.897,8.207)--(7.899,8.208)--(7.900,8.208)--(7.901,8.209)--(7.902,8.209)--(7.903,8.209)%
  --(7.904,8.210)--(7.906,8.210)--(7.907,8.211)--(7.908,8.211)--(7.909,8.212)--(7.910,8.212)%
  --(7.911,8.213)--(7.912,8.213)--(7.913,8.213)--(7.914,8.214)--(7.915,8.214)--(7.916,8.215)%
  --(7.918,8.215)--(7.919,8.216)--(7.920,8.216)--(7.921,8.217)--(7.923,8.217)--(7.924,8.218)%
  --(7.926,8.219)--(7.927,8.219)--(7.929,8.220)--(7.930,8.220)--(7.931,8.221)--(7.932,8.221)%
  --(7.933,8.222)--(7.934,8.222)--(7.935,8.223)--(7.937,8.223)--(7.938,8.223)--(7.939,8.224)%
  --(7.940,8.224)--(7.941,8.225)--(7.942,8.225)--(7.943,8.226)--(7.944,8.226)--(7.945,8.226)%
  --(7.946,8.227)--(7.947,8.227)--(7.948,8.228)--(7.950,8.228)--(7.951,8.229)--(7.952,8.229)%
  --(7.953,8.230)--(7.954,8.230)--(7.955,8.231)--(7.957,8.231)--(7.958,8.232)--(7.960,8.232)%
  --(7.961,8.233)--(7.962,8.233)--(7.963,8.234)--(7.964,8.234)--(7.966,8.235)--(7.967,8.235)%
  --(7.968,8.236)--(7.969,8.236)--(7.970,8.237)--(7.971,8.237)--(7.973,8.237)--(7.974,8.238)%
  --(7.975,8.238)--(7.976,8.239)--(7.977,8.239)--(7.978,8.240)--(7.979,8.240)--(7.980,8.240)%
  --(7.981,8.241)--(7.982,8.241)--(7.984,8.242)--(7.985,8.242)--(7.986,8.242)--(7.987,8.243)%
  --(7.988,8.243)--(7.990,8.244)--(7.991,8.244)--(7.992,8.245)--(7.993,8.245)--(7.995,8.246)%
  --(7.996,8.246)--(7.997,8.247)--(7.999,8.247)--(8.000,8.248)--(8.001,8.248)--(8.003,8.249)%
  --(8.004,8.249)--(8.005,8.249)--(8.006,8.250)--(8.007,8.250)--(8.008,8.251)--(8.009,8.251)%
  --(8.010,8.251)--(8.012,8.252)--(8.013,8.252)--(8.014,8.253)--(8.015,8.253)--(8.016,8.253)%
  --(8.017,8.254)--(8.018,8.254)--(8.019,8.255)--(8.020,8.255)--(8.022,8.256)--(8.023,8.256)%
  --(8.024,8.256)--(8.025,8.257)--(8.027,8.257)--(8.028,8.258)--(8.030,8.259)--(8.032,8.259)%
  --(8.033,8.260)--(8.034,8.260)--(8.036,8.261)--(8.037,8.261)--(8.038,8.262)--(8.039,8.262)%
  --(8.041,8.262)--(8.042,8.263)--(8.043,8.263)--(8.044,8.263)--(8.045,8.264)--(8.046,8.264)%
  --(8.047,8.265)--(8.048,8.265)--(8.049,8.265)--(8.050,8.266)--(8.051,8.266)--(8.052,8.266)%
  --(8.054,8.267)--(8.055,8.267)--(8.056,8.268)--(8.057,8.268)--(8.058,8.268)--(8.060,8.269)%
  --(8.061,8.269)--(8.063,8.270)--(8.064,8.270)--(8.066,8.271)--(8.067,8.271)--(8.069,8.272)%
  --(8.070,8.272)--(8.072,8.273)--(8.073,8.273)--(8.074,8.274)--(8.075,8.274)--(8.076,8.274)%
  --(8.077,8.275)--(8.078,8.275)--(8.079,8.275)--(8.080,8.276)--(8.082,8.276)--(8.083,8.276)%
  --(8.084,8.277)--(8.085,8.277)--(8.086,8.277)--(8.087,8.278)--(8.088,8.278)--(8.089,8.279)%
  --(8.090,8.279)--(8.092,8.279)--(8.093,8.280)--(8.094,8.280)--(8.096,8.281)--(8.097,8.281)%
  --(8.099,8.282)--(8.100,8.282)--(8.102,8.283)--(8.103,8.283)--(8.105,8.284)--(8.106,8.284)%
  --(8.107,8.284)--(8.108,8.285)--(8.109,8.285)--(8.111,8.285)--(8.112,8.286)--(8.113,8.286)%
  --(8.114,8.286)--(8.115,8.287)--(8.116,8.287)--(8.117,8.287)--(8.118,8.288)--(8.119,8.288)%
  --(8.120,8.288)--(8.121,8.289)--(8.123,8.289)--(8.124,8.289)--(8.125,8.290)--(8.126,8.290)%
  --(8.127,8.290)--(8.129,8.291)--(8.130,8.291)--(8.132,8.292)--(8.133,8.292)--(8.135,8.293)%
  --(8.136,8.293)--(8.138,8.293)--(8.139,8.294)--(8.140,8.294)--(8.141,8.295)--(8.143,8.295)%
  --(8.144,8.295)--(8.145,8.296)--(8.146,8.296)--(8.147,8.296)--(8.148,8.297)--(8.149,8.297)%
  --(8.150,8.297)--(8.151,8.297)--(8.153,8.298)--(8.154,8.298)--(8.155,8.298)--(8.156,8.299)%
  --(8.157,8.299)--(8.158,8.299)--(8.159,8.300)--(8.160,8.300)--(8.162,8.301)--(8.163,8.301)%
  --(8.164,8.301)--(8.166,8.302)--(8.168,8.302)--(8.169,8.303)--(8.170,8.303)--(8.171,8.303)%
  --(8.173,8.304)--(8.174,8.304)--(8.175,8.304)--(8.176,8.305)--(8.177,8.305)--(8.178,8.305)%
  --(8.179,8.305)--(8.180,8.306)--(8.181,8.306)--(8.182,8.306)--(8.184,8.307)--(8.185,8.307)%
  --(8.186,8.307)--(8.187,8.307)--(8.188,8.308)--(8.189,8.308)--(8.190,8.308)--(8.191,8.309)%
  --(8.192,8.309)--(8.194,8.309)--(8.195,8.309)--(8.196,8.310)--(8.197,8.310)--(8.199,8.311)%
  --(8.200,8.311)--(8.201,8.311)--(8.203,8.311)--(8.204,8.312)--(8.205,8.312)--(8.206,8.312)%
  --(8.207,8.313)--(8.209,8.313)--(8.210,8.313)--(8.211,8.314)--(8.212,8.314)--(8.213,8.314)%
  --(8.214,8.315)--(8.215,8.315)--(8.216,8.315)--(8.218,8.315)--(8.219,8.316)--(8.220,8.316)%
  --(8.221,8.316)--(8.222,8.316)--(8.223,8.317)--(8.224,8.317)--(8.225,8.317)--(8.226,8.318)%
  --(8.227,8.318)--(8.229,8.318)--(8.230,8.318)--(8.231,8.319)--(8.232,8.319)--(8.234,8.319)%
  --(8.235,8.320)--(8.236,8.320)--(8.238,8.320)--(8.239,8.321)--(8.241,8.321)--(8.242,8.321)%
  --(8.243,8.322)--(8.245,8.322)--(8.246,8.322)--(8.247,8.323)--(8.248,8.323)--(8.249,8.323)%
  --(8.250,8.323)--(8.251,8.324)--(8.252,8.324)--(8.253,8.324)--(8.254,8.324)--(8.255,8.325)%
  --(8.256,8.325)--(8.258,8.325)--(8.259,8.325)--(8.260,8.326)--(8.261,8.326)--(8.262,8.326)%
  --(8.263,8.326)--(8.265,8.327)--(8.266,8.327)--(8.267,8.327)--(8.269,8.328)--(8.270,8.328)%
  --(8.272,8.328)--(8.274,8.329)--(8.275,8.329)--(8.277,8.329)--(8.278,8.330)--(8.279,8.330)%
  --(8.280,8.330)--(8.282,8.330)--(8.283,8.331)--(8.284,8.331)--(8.285,8.331)--(8.286,8.331)%
  --(8.287,8.332)--(8.288,8.332)--(8.289,8.332)--(8.290,8.332)--(8.291,8.332)--(8.292,8.333)%
  --(8.294,8.333)--(8.295,8.333)--(8.296,8.333)--(8.297,8.334)--(8.298,8.334)--(8.299,8.334)%
  --(8.301,8.334)--(8.302,8.335)--(8.303,8.335)--(8.305,8.335)--(8.306,8.336)--(8.307,8.336)%
  --(8.309,8.336)--(8.310,8.336)--(8.312,8.337)--(8.313,8.337)--(8.314,8.337)--(8.315,8.337)%
  --(8.316,8.338)--(8.317,8.338)--(8.319,8.338)--(8.320,8.338)--(8.321,8.338)--(8.322,8.339)%
  --(8.323,8.339)--(8.324,8.339)--(8.325,8.339)--(8.326,8.340)--(8.327,8.340)--(8.328,8.340)%
  --(8.329,8.340)--(8.330,8.340)--(8.332,8.341)--(8.333,8.341)--(8.334,8.341)--(8.335,8.341)%
  --(8.337,8.342)--(8.338,8.342)--(8.340,8.342)--(8.341,8.342)--(8.343,8.343)--(8.344,8.343)%
  --(8.345,8.343)--(8.346,8.343)--(8.347,8.343)--(8.348,8.344)--(8.349,8.344)--(8.351,8.344)%
  --(8.352,8.344)--(8.353,8.344)--(8.354,8.345)--(8.355,8.345)--(8.356,8.345)--(8.357,8.345)%
  --(8.358,8.345)--(8.359,8.345)--(8.360,8.346)--(8.361,8.346)--(8.363,8.346)--(8.364,8.346)%
  --(8.365,8.347)--(8.366,8.347)--(8.367,8.347)--(8.368,8.347)--(8.370,8.347)--(8.371,8.348)%
  --(8.373,8.348)--(8.374,8.348)--(8.375,8.348)--(8.376,8.349)--(8.377,8.349)--(8.379,8.349)%
  --(8.380,8.349)--(8.381,8.349)--(8.382,8.350)--(8.383,8.350)--(8.384,8.350)--(8.385,8.350)%
  --(8.387,8.350)--(8.388,8.350)--(8.389,8.351)--(8.390,8.351)--(8.391,8.351)--(8.392,8.351)%
  --(8.393,8.351)--(8.394,8.351)--(8.395,8.352)--(8.396,8.352)--(8.398,8.352)--(8.399,8.352)%
  --(8.400,8.352)--(8.401,8.352)--(8.402,8.353)--(8.404,8.353)--(8.405,8.353)--(8.406,8.353)%
  --(8.407,8.353)--(8.409,8.353)--(8.410,8.354)--(8.411,8.354)--(8.413,8.354)--(8.414,8.354)%
  --(8.416,8.354)--(8.417,8.355)--(8.418,8.355)--(8.419,8.355)--(8.420,8.355)--(8.421,8.355)%
  --(8.422,8.355)--(8.424,8.356)--(8.425,8.356)--(8.426,8.356)--(8.427,8.356)--(8.428,8.356)%
  --(8.429,8.356)--(8.430,8.357)--(8.431,8.357)--(8.432,8.357)--(8.433,8.357)--(8.435,8.357)%
  --(8.436,8.357)--(8.437,8.358)--(8.438,8.358)--(8.439,8.358)--(8.441,8.358)--(8.443,8.358)%
  --(8.445,8.359)--(8.446,8.359)--(8.448,8.359)--(8.449,8.359)--(8.451,8.359)--(8.452,8.359)%
  --(8.453,8.360)--(8.454,8.360)--(8.455,8.360)--(8.456,8.360)--(8.457,8.360)--(8.458,8.360)%
  --(8.459,8.360)--(8.460,8.361)--(8.462,8.361)--(8.463,8.361)--(8.464,8.361)--(8.465,8.361)%
  --(8.466,8.361)--(8.467,8.361)--(8.468,8.361)--(8.469,8.361)--(8.471,8.362)--(8.472,8.362)%
  --(8.473,8.362)--(8.474,8.362)--(8.476,8.362)--(8.477,8.362)--(8.478,8.362)--(8.480,8.363)%
  --(8.481,8.363)--(8.483,8.363)--(8.484,8.363)--(8.485,8.363)--(8.486,8.363)--(8.487,8.363)%
  --(8.489,8.363)--(8.490,8.364)--(8.491,8.364)--(8.492,8.364)--(8.493,8.364)--(8.494,8.364)%
  --(8.495,8.364)--(8.496,8.364)--(8.497,8.364)--(8.498,8.365)--(8.499,8.365)--(8.500,8.365)%
  --(8.502,8.365)--(8.503,8.365)--(8.504,8.365)--(8.505,8.365)--(8.506,8.365)--(8.508,8.366)%
  --(8.509,8.366)--(8.511,8.366)--(8.512,8.366)--(8.514,8.366)--(8.515,8.366)--(8.516,8.366)%
  --(8.517,8.366)--(8.518,8.367)--(8.520,8.367)--(8.521,8.367)--(8.522,8.367)--(8.523,8.367)%
  --(8.524,8.367)--(8.525,8.367)--(8.526,8.367)--(8.527,8.367)--(8.528,8.367)--(8.529,8.367)%
  --(8.530,8.368)--(8.531,8.368)--(8.533,8.368)--(8.534,8.368)--(8.535,8.368)--(8.536,8.368)%
  --(8.537,8.368)--(8.538,8.368)--(8.540,8.368)--(8.541,8.368)--(8.542,8.368)--(8.544,8.369)%
  --(8.545,8.369)--(8.547,8.369)--(8.548,8.369)--(8.549,8.369)--(8.550,8.369)--(8.551,8.369)%
  --(8.553,8.369)--(8.554,8.369)--(8.555,8.369)--(8.556,8.369)--(8.557,8.369)--(8.558,8.370)%
  --(8.560,8.370)--(8.561,8.370)--(8.562,8.370)--(8.563,8.370)--(8.564,8.370)--(8.565,8.370)%
  --(8.566,8.370)--(8.567,8.370)--(8.568,8.370)--(8.570,8.370)--(8.571,8.370)--(8.572,8.371)%
  --(8.573,8.371)--(8.574,8.371)--(8.576,8.371)--(8.577,8.371)--(8.578,8.371)--(8.580,8.371)%
  --(8.581,8.371)--(8.582,8.371)--(8.584,8.371)--(8.585,8.371)--(8.586,8.372)--(8.587,8.372)%
  --(8.588,8.372)--(8.590,8.372)--(8.591,8.372)--(8.592,8.372)--(8.593,8.372)--(8.594,8.372)%
  --(8.595,8.372)--(8.597,8.372)--(8.598,8.372)--(8.599,8.372)--(8.600,8.372)--(8.601,8.372)%
  --(8.602,8.372)--(8.603,8.372)--(8.604,8.373)--(8.605,8.373)--(8.607,8.373)--(8.608,8.373)%
  --(8.609,8.373)--(8.611,8.373)--(8.612,8.373)--(8.613,8.373)--(8.615,8.373)--(8.616,8.373)%
  --(8.617,8.373)--(8.619,8.373)--(8.620,8.373)--(8.621,8.373)--(8.623,8.373)--(8.624,8.374)%
  --(8.625,8.374)--(8.626,8.374)--(8.627,8.374)--(8.629,8.374)--(8.630,8.374)--(8.631,8.374)%
  --(8.632,8.374)--(8.633,8.374)--(8.634,8.374)--(8.635,8.374)--(8.636,8.374)--(8.638,8.374)%
  --(8.639,8.374)--(8.640,8.374)--(8.641,8.374)--(8.642,8.374)--(8.643,8.374)--(8.645,8.375)%
  --(8.646,8.375)--(8.648,8.375)--(8.649,8.375)--(8.650,8.375)--(8.652,8.375)--(8.653,8.375)%
  --(8.654,8.375)--(8.656,8.375)--(8.657,8.375)--(8.659,8.375)--(8.660,8.375)--(8.661,8.375)%
  --(8.662,8.375)--(8.663,8.375)--(8.664,8.375)--(8.665,8.375)--(8.666,8.376)--(8.668,8.376)%
  --(8.669,8.376)--(8.670,8.376)--(8.671,8.376)--(8.672,8.376)--(8.673,8.376)--(8.675,8.376)%
  --(8.676,8.376)--(8.677,8.376)--(8.678,8.376)--(8.680,8.376)--(8.681,8.376)--(8.682,8.376)%
  --(8.684,8.376)--(8.685,8.376)--(8.687,8.376)--(8.688,8.376)--(8.690,8.376)--(8.691,8.376)%
  --(8.692,8.376)--(8.694,8.376)--(8.695,8.376)--(8.696,8.376)--(8.697,8.376)--(8.699,8.377)%
  --(8.700,8.377)--(8.701,8.377)--(8.702,8.377)--(8.703,8.377)--(8.704,8.377)--(8.705,8.377)%
  --(8.706,8.377)--(8.707,8.377)--(8.709,8.377)--(8.710,8.377)--(8.711,8.377)--(8.712,8.377)%
  --(8.714,8.377)--(8.715,8.377)--(8.717,8.377)--(8.718,8.377)--(8.719,8.377)--(8.721,8.377)%
  --(8.722,8.377)--(8.723,8.377)--(8.724,8.377)--(8.726,8.377)--(8.727,8.378)--(8.728,8.378)%
  --(8.730,8.378)--(8.731,8.378)--(8.732,8.378)--(8.733,8.378)--(8.734,8.378)--(8.735,8.378)%
  --(8.736,8.378)--(8.737,8.378)--(8.739,8.378)--(8.740,8.378)--(8.741,8.378)--(8.742,8.378)%
  --(8.743,8.378)--(8.745,8.378)--(8.746,8.378)--(8.747,8.378)--(8.748,8.378)--(8.750,8.378)%
  --(8.751,8.378)--(8.753,8.378)--(8.754,8.378)--(8.755,8.378)--(8.757,8.378)--(8.758,8.378)%
  --(8.759,8.378)--(8.761,8.378)--(8.762,8.378)--(8.763,8.378)--(8.764,8.378)--(8.765,8.378)%
  --(8.767,8.378)--(8.768,8.378)--(8.769,8.378)--(8.770,8.378)--(8.771,8.378)--(8.772,8.378)%
  --(8.774,8.378)--(8.775,8.378)--(8.776,8.378)--(8.777,8.379)--(8.778,8.379)--(8.780,8.379)%
  --(8.781,8.379)--(8.782,8.379)--(8.784,8.379)--(8.785,8.379)--(8.787,8.379)--(8.788,8.379)%
  --(8.790,8.379)--(8.791,8.379)--(8.792,8.379)--(8.793,8.379)--(8.794,8.379)--(8.796,8.379)%
  --(8.797,8.379)--(8.798,8.379)--(8.799,8.379)--(8.801,8.379)--(8.802,8.379)--(8.803,8.379)%
  --(8.804,8.379)--(8.805,8.379)--(8.806,8.379)--(8.808,8.379)--(8.809,8.379)--(8.810,8.379)%
  --(8.811,8.379)--(8.812,8.379)--(8.814,8.379)--(8.815,8.379)--(8.816,8.379)--(8.818,8.379)%
  --(8.819,8.379)--(8.821,8.379)--(8.822,8.379)--(8.824,8.379)--(8.825,8.379)--(8.826,8.379)%
  --(8.828,8.379)--(8.829,8.379)--(8.830,8.379)--(8.831,8.379)--(8.833,8.379)--(8.834,8.379)%
  --(8.835,8.379)--(8.836,8.379)--(8.837,8.379)--(8.839,8.379)--(8.840,8.379)--(8.841,8.379)%
  --(8.842,8.380)--(8.843,8.380)--(8.845,8.380)--(8.846,8.380)--(8.847,8.380)--(8.848,8.380)%
  --(8.850,8.380)--(8.851,8.380)--(8.852,8.380)--(8.854,8.380)--(8.855,8.380)--(8.857,8.380)%
  --(8.858,8.380)--(8.859,8.380)--(8.861,8.380)--(8.862,8.380)--(8.864,8.380)--(8.865,8.380)%
  --(8.867,8.380)--(8.868,8.380)--(8.869,8.380)--(8.870,8.380)--(8.872,8.380)--(8.873,8.380)%
  --(8.874,8.380)--(8.875,8.380)--(8.876,8.380)--(8.877,8.380)--(8.879,8.380)--(8.880,8.380)%
  --(8.881,8.380)--(8.882,8.380)--(8.884,8.380)--(8.885,8.380)--(8.886,8.380)--(8.888,8.380)%
  --(8.889,8.380)--(8.891,8.380)--(8.892,8.380)--(8.894,8.380)--(8.895,8.380)--(8.896,8.380)%
  --(8.898,8.380)--(8.899,8.380)--(8.900,8.380)--(8.902,8.380)--(8.903,8.380)--(8.904,8.380)%
  --(8.905,8.380)--(8.906,8.380)--(8.908,8.380)--(8.909,8.380)--(8.910,8.380)--(8.911,8.380)%
  --(8.913,8.380)--(8.914,8.380)--(8.915,8.380)--(8.916,8.380)--(8.917,8.380)--(8.919,8.380)%
  --(8.920,8.380)--(8.921,8.380)--(8.923,8.380)--(8.925,8.380)--(8.927,8.380)--(8.929,8.380)%
  --(8.930,8.380)--(8.931,8.380)--(8.933,8.380)--(8.935,8.380)--(8.936,8.380)--(8.937,8.380)%
  --(8.938,8.380)--(8.940,8.380)--(8.941,8.380)--(8.942,8.380)--(8.943,8.380)--(8.945,8.380)%
  --(8.946,8.380)--(8.947,8.380)--(8.948,8.380)--(8.950,8.380)--(8.951,8.380)--(8.952,8.380)%
  --(8.954,8.380)--(8.955,8.380)--(8.957,8.380)--(8.958,8.380)--(8.960,8.380)--(8.961,8.380)%
  --(8.963,8.380)--(8.964,8.380)--(8.965,8.380)--(8.967,8.380)--(8.968,8.380)--(8.969,8.380)%
  --(8.971,8.380)--(8.972,8.380)--(8.973,8.380)--(8.974,8.380)--(8.975,8.380)--(8.977,8.380)%
  --(8.978,8.380)--(8.979,8.380)--(8.980,8.380)--(8.982,8.380)--(8.983,8.380)--(8.984,8.380)%
  --(8.985,8.380)--(8.987,8.381)--(8.988,8.381)--(8.990,8.381)--(8.991,8.381)--(8.992,8.381)%
  --(8.994,8.381)--(8.996,8.381)--(8.997,8.381)--(8.999,8.381)--(9.000,8.381)--(9.002,8.381)%
  --(9.003,8.381)--(9.005,8.381)--(9.006,8.381)--(9.007,8.381)--(9.009,8.381)--(9.010,8.381)%
  --(9.011,8.381)--(9.012,8.381)--(9.014,8.381)--(9.015,8.381)--(9.016,8.381)--(9.017,8.381)%
  --(9.019,8.381)--(9.020,8.381)--(9.021,8.381)--(9.023,8.381)--(9.024,8.381)--(9.025,8.381)%
  --(9.027,8.381)--(9.029,8.381)--(9.031,8.381)--(9.033,8.381)--(9.034,8.381)--(9.036,8.381)%
  --(9.038,8.381)--(9.039,8.381)--(9.041,8.381)--(9.042,8.381)--(9.043,8.381)--(9.044,8.381)%
  --(9.046,8.381)--(9.047,8.381)--(9.048,8.381)--(9.049,8.381)--(9.051,8.381)--(9.052,8.381)%
  --(9.053,8.381)--(9.055,8.381)--(9.056,8.381)--(9.058,8.381)--(9.059,8.381)--(9.060,8.381)%
  --(9.062,8.381)--(9.064,8.381)--(9.066,8.381)--(9.068,8.381)--(9.069,8.381)--(9.071,8.381)%
  --(9.073,8.381)--(9.074,8.381)--(9.076,8.381)--(9.077,8.381)--(9.078,8.381)--(9.079,8.381)%
  --(9.081,8.381)--(9.082,8.381)--(9.083,8.381)--(9.085,8.381)--(9.086,8.381)--(9.087,8.381)%
  --(9.089,8.381)--(9.090,8.381)--(9.091,8.381)--(9.093,8.381)--(9.094,8.381)--(9.096,8.381)%
  --(9.097,8.381)--(9.099,8.381)--(9.100,8.381)--(9.102,8.381)--(9.104,8.381)--(9.105,8.381)%
  --(9.107,8.381)--(9.108,8.381)--(9.110,8.381)--(9.111,8.381)--(9.112,8.381)--(9.114,8.381)%
  --(9.115,8.381)--(9.116,8.381)--(9.118,8.381)--(9.119,8.381)--(9.120,8.381)--(9.121,8.381)%
  --(9.123,8.381)--(9.124,8.381)--(9.126,8.381)--(9.127,8.381)--(9.129,8.381)--(9.130,8.381)%
  --(9.132,8.381)--(9.133,8.381)--(9.135,8.381)--(9.137,8.381)--(9.138,8.381)--(9.140,8.381)%
  --(9.141,8.381)--(9.143,8.381)--(9.144,8.381)--(9.146,8.381)--(9.147,8.381)--(9.148,8.381)%
  --(9.150,8.381)--(9.151,8.381)--(9.152,8.381)--(9.154,8.381)--(9.155,8.381)--(9.156,8.381)%
  --(9.158,8.381)--(9.159,8.381)--(9.161,8.381)--(9.162,8.381)--(9.164,8.381)--(9.165,8.381)%
  --(9.167,8.381)--(9.168,8.381)--(9.170,8.381)--(9.172,8.381)--(9.173,8.381)--(9.175,8.381)%
  --(9.177,8.381)--(9.178,8.381)--(9.180,8.381)--(9.181,8.381)--(9.182,8.381)--(9.184,8.381)%
  --(9.185,8.381)--(9.186,8.381)--(9.188,8.381)--(9.189,8.381)--(9.190,8.381)--(9.192,8.381)%
  --(9.193,8.381)--(9.194,8.381)--(9.196,8.381)--(9.198,8.381)--(9.199,8.381)--(9.201,8.381)%
  --(9.202,8.381)--(9.204,8.381)--(9.205,8.381)--(9.207,8.381)--(9.209,8.381)--(9.210,8.381)%
  --(9.212,8.381)--(9.213,8.381)--(9.215,8.381)--(9.216,8.381)--(9.218,8.381)--(9.219,8.381)%
  --(9.220,8.381)--(9.222,8.381)--(9.223,8.381)--(9.224,8.381)--(9.226,8.381)--(9.227,8.381)%
  --(9.229,8.381)--(9.230,8.381)--(9.232,8.381)--(9.233,8.381)--(9.235,8.381)--(9.236,8.381)%
  --(9.238,8.381)--(9.239,8.381)--(9.241,8.381)--(9.243,8.381)--(9.244,8.381)--(9.246,8.381)%
  --(9.247,8.381)--(9.249,8.381)--(9.250,8.381)--(9.252,8.381)--(9.253,8.381)--(9.254,8.381)%
  --(9.256,8.381)--(9.257,8.381)--(9.258,8.381)--(9.260,8.381)--(9.261,8.381)--(9.263,8.381)%
  --(9.264,8.381)--(9.266,8.381)--(9.267,8.381)--(9.269,8.381)--(9.271,8.381)--(9.272,8.381)%
  --(9.274,8.381)--(9.275,8.381)--(9.277,8.381)--(9.278,8.381)--(9.280,8.381)--(9.281,8.381)%
  --(9.283,8.381)--(9.285,8.381)--(9.286,8.381)--(9.287,8.381)--(9.289,8.381)--(9.290,8.381)%
  --(9.291,8.381)--(9.293,8.381)--(9.294,8.381)--(9.296,8.381)--(9.297,8.381)--(9.299,8.381)%
  --(9.300,8.381)--(9.302,8.381)--(9.303,8.381)--(9.305,8.381)--(9.307,8.381)--(9.309,8.381)%
  --(9.310,8.381)--(9.312,8.381)--(9.313,8.381)--(9.315,8.381)--(9.316,8.381)--(9.318,8.381)%
  --(9.319,8.381)--(9.321,8.381)--(9.322,8.381)--(9.324,8.381)--(9.325,8.381)--(9.326,8.381)%
  --(9.328,8.381)--(9.329,8.381)--(9.331,8.381)--(9.332,8.381)--(9.334,8.381)--(9.335,8.381)%
  --(9.337,8.381)--(9.338,8.381)--(9.340,8.381)--(9.342,8.381)--(9.344,8.381)--(9.345,8.381)%
  --(9.347,8.381)--(9.349,8.381)--(9.351,8.381)--(9.352,8.381)--(9.354,8.381)--(9.355,8.381)%
  --(9.357,8.381)--(9.358,8.381)--(9.360,8.381)--(9.361,8.381)--(9.362,8.381)--(9.364,8.381)%
  --(9.365,8.381)--(9.367,8.381)--(9.368,8.381)--(9.370,8.381)--(9.372,8.381)--(9.374,8.381)%
  --(9.375,8.381)--(9.377,8.381)--(9.379,8.381)--(9.380,8.381)--(9.382,8.381)--(9.383,8.381)%
  --(9.385,8.381)--(9.386,8.381)--(9.388,8.381)--(9.389,8.381)--(9.391,8.381)--(9.392,8.381)%
  --(9.394,8.381)--(9.395,8.381)--(9.397,8.381)--(9.398,8.381)--(9.399,8.381)--(9.401,8.381)%
  --(9.403,8.381)--(9.404,8.381)--(9.406,8.381)--(9.408,8.381)--(9.409,8.381)--(9.411,8.381)%
  --(9.413,8.381)--(9.415,8.381)--(9.416,8.381)--(9.418,8.381)--(9.420,8.381)--(9.422,8.381)%
  --(9.423,8.381)--(9.425,8.381)--(9.426,8.381)--(9.428,8.381)--(9.429,8.381)--(9.431,8.381)%
  --(9.432,8.381)--(9.434,8.381)--(9.435,8.381)--(9.437,8.381)--(9.438,8.381)--(9.440,8.381)%
  --(9.442,8.381)--(9.444,8.381)--(9.446,8.381)--(9.448,8.381)--(9.449,8.381)--(9.451,8.381)%
  --(9.452,8.381)--(9.454,8.381)--(9.455,8.381)--(9.457,8.381)--(9.458,8.381)--(9.460,8.381)%
  --(9.461,8.381)--(9.463,8.381)--(9.464,8.381)--(9.466,8.381)--(9.467,8.381)--(9.469,8.381)%
  --(9.470,8.381)--(9.472,8.381)--(9.474,8.381)--(9.475,8.381)--(9.477,8.381)--(9.479,8.381)%
  --(9.481,8.381)--(9.482,8.381)--(9.484,8.381)--(9.486,8.381)--(9.487,8.381)--(9.489,8.381)%
  --(9.491,8.381)--(9.492,8.381)--(9.494,8.381)--(9.495,8.381)--(9.497,8.381)--(9.499,8.381)%
  --(9.500,8.381)--(9.502,8.381)--(9.503,8.381)--(9.505,8.381)--(9.506,8.381)--(9.508,8.381)%
  --(9.510,8.381)--(9.511,8.381)--(9.513,8.381)--(9.515,8.381)--(9.517,8.381)--(9.519,8.381)%
  --(9.520,8.381)--(9.522,8.381)--(9.524,8.381)--(9.526,8.381)--(9.527,8.381)--(9.529,8.381)%
  --(9.531,8.381)--(9.532,8.381)--(9.534,8.381)--(9.535,8.381)--(9.537,8.381)--(9.538,8.381)%
  --(9.540,8.381)--(9.542,8.381)--(9.543,8.381)--(9.545,8.381)--(9.547,8.381)--(9.549,8.381)%
  --(9.551,8.381)--(9.554,8.381)--(9.555,8.381)--(9.557,8.381)--(9.559,8.381)--(9.561,8.381)%
  --(9.563,8.381)--(9.564,8.381)--(9.566,8.381)--(9.568,8.381)--(9.569,8.381)--(9.571,8.381)%
  --(9.572,8.381)--(9.574,8.381)--(9.576,8.381)--(9.577,8.381)--(9.579,8.381)--(9.581,8.381)%
  --(9.583,8.381)--(9.584,8.381)--(9.586,8.381)--(9.588,8.381)--(9.590,8.381)--(9.592,8.381)%
  --(9.594,8.381)--(9.596,8.381)--(9.597,8.381)--(9.599,8.381)--(9.601,8.381)--(9.602,8.381)%
  --(9.604,8.381)--(9.605,8.381)--(9.607,8.381)--(9.608,8.381)--(9.610,8.381)--(9.612,8.381)%
  --(9.614,8.381)--(9.615,8.381)--(9.617,8.381)--(9.619,8.381)--(9.621,8.381)--(9.623,8.381)%
  --(9.625,8.381)--(9.627,8.381)--(9.629,8.381)--(9.631,8.381)--(9.632,8.381)--(9.634,8.381)%
  --(9.635,8.381)--(9.637,8.381)--(9.639,8.381)--(9.640,8.381)--(9.642,8.381)--(9.643,8.381)%
  --(9.645,8.381)--(9.647,8.381)--(9.649,8.381)--(9.651,8.381)--(9.653,8.381)--(9.655,8.381)%
  --(9.656,8.381)--(9.658,8.381)--(9.660,8.381)--(9.662,8.381)--(9.664,8.381)--(9.666,8.381)%
  --(9.668,8.381)--(9.669,8.381)--(9.671,8.381)--(9.672,8.381)--(9.674,8.381)--(9.676,8.381)%
  --(9.677,8.381)--(9.679,8.381)--(9.681,8.381)--(9.683,8.381)--(9.685,8.381)--(9.686,8.381)%
  --(9.688,8.381)--(9.690,8.381)--(9.692,8.381)--(9.694,8.381)--(9.696,8.381)--(9.697,8.381)%
  --(9.699,8.381)--(9.701,8.381)--(9.703,8.381)--(9.704,8.381)--(9.706,8.381)--(9.708,8.381)%
  --(9.709,8.381)--(9.711,8.381)--(9.713,8.381)--(9.714,8.381)--(9.716,8.381)--(9.718,8.381)%
  --(9.720,8.381)--(9.722,8.381)--(9.724,8.381)--(9.726,8.381)--(9.728,8.381)--(9.729,8.381)%
  --(9.731,8.381)--(9.733,8.381)--(9.735,8.381)--(9.737,8.381)--(9.738,8.381)--(9.740,8.381)%
  --(9.742,8.381)--(9.743,8.381)--(9.745,8.381)--(9.747,8.381)--(9.748,8.381)--(9.750,8.381)%
  --(9.752,8.381)--(9.754,8.381)--(9.756,8.381)--(9.758,8.381)--(9.760,8.381)--(9.762,8.381)%
  --(9.764,8.381)--(9.765,8.381)--(9.767,8.381)--(9.769,8.381)--(9.771,8.381)--(9.773,8.381)%
  --(9.774,8.381)--(9.776,8.381)--(9.778,8.381)--(9.779,8.381)--(9.781,8.381)--(9.783,8.381)%
  --(9.785,8.381)--(9.786,8.381)--(9.788,8.381)--(9.791,8.381)--(9.793,8.381)--(9.795,8.381)%
  --(9.797,8.381)--(9.799,8.381)--(9.802,8.381)--(9.804,8.381)--(9.805,8.381)--(9.807,8.381)%
  --(9.809,8.381)--(9.811,8.381)--(9.812,8.381)--(9.814,8.381)--(9.816,8.381)--(9.818,8.381)%
  --(9.820,8.381)--(9.822,8.381)--(9.824,8.381)--(9.825,8.381)--(9.827,8.381)--(9.829,8.381)%
  --(9.831,8.381)--(9.833,8.381)--(9.835,8.381)--(9.837,8.381)--(9.839,8.381)--(9.841,8.381)%
  --(9.843,8.381)--(9.844,8.381)--(9.846,8.381)--(9.848,8.381)--(9.850,8.381)--(9.851,8.381)%
  --(9.853,8.381)--(9.855,8.381)--(9.857,8.381)--(9.859,8.381)--(9.862,8.381)--(9.864,8.381)%
  --(9.866,8.381)--(9.867,8.381)--(9.869,8.381)--(9.871,8.381)--(9.873,8.381)--(9.875,8.381)%
  --(9.876,8.381)--(9.878,8.381)--(9.880,8.381)--(9.882,8.381)--(9.883,8.381)--(9.885,8.381)%
  --(9.887,8.381)--(9.889,8.381)--(9.891,8.381)--(9.893,8.381)--(9.895,8.381)--(9.897,8.381)%
  --(9.899,8.381)--(9.901,8.381)--(9.903,8.381)--(9.905,8.381)--(9.907,8.381)--(9.910,8.381)%
  --(9.911,8.381)--(9.913,8.381)--(9.915,8.381)--(9.917,8.381)--(9.919,8.381)--(9.920,8.381)%
  --(9.922,8.381)--(9.924,8.381)--(9.926,8.381)--(9.929,8.381)--(9.931,8.381)--(9.933,8.381)%
  --(9.935,8.381)--(9.937,8.381)--(9.938,8.381)--(9.940,8.381)--(9.942,8.381)--(9.944,8.381)%
  --(9.946,8.381)--(9.948,8.381)--(9.950,8.381)--(9.951,8.381)--(9.953,8.381)--(9.955,8.381)%
  --(9.957,8.381)--(9.959,8.381)--(9.961,8.381)--(9.963,8.381)--(9.965,8.381)--(9.967,8.381)%
  --(9.969,8.381)--(9.971,8.381)--(9.974,8.381)--(9.976,8.381)--(9.978,8.381)--(9.980,8.381)%
  --(9.982,8.381)--(9.984,8.381)--(9.985,8.381)--(9.987,8.381)--(9.989,8.381)--(9.991,8.381)%
  --(9.993,8.381)--(9.995,8.381)--(9.998,8.381)--(10.000,8.381)--(10.002,8.381)--(10.005,8.381)%
  --(10.007,8.381)--(10.009,8.381)--(10.012,8.381)--(10.014,8.381)--(10.016,8.381)--(10.018,8.381)%
  --(10.019,8.381)--(10.021,8.381)--(10.023,8.381)--(10.025,8.381)--(10.027,8.381)--(10.029,8.381)%
  --(10.032,8.381)--(10.034,8.381)--(10.036,8.381)--(10.039,8.381)--(10.041,8.381)--(10.043,8.381)%
  --(10.046,8.381)--(10.048,8.381)--(10.050,8.381)--(10.052,8.381)--(10.054,8.381)--(10.056,8.381)%
  --(10.058,8.381)--(10.060,8.381)--(10.062,8.381)--(10.063,8.381)--(10.066,8.381)--(10.068,8.381)%
  --(10.071,8.381)--(10.073,8.381)--(10.075,8.381)--(10.078,8.381)--(10.080,8.381)--(10.083,8.381)%
  --(10.085,8.381)--(10.086,8.381)--(10.088,8.381)--(10.090,8.381)--(10.092,8.381)--(10.094,8.381)%
  --(10.096,8.381)--(10.098,8.381)--(10.101,8.381)--(10.103,8.381)--(10.106,8.381)--(10.108,8.381)%
  --(10.110,8.381)--(10.113,8.381)--(10.115,8.381)--(10.118,8.381)--(10.120,8.381)--(10.121,8.381)%
  --(10.124,8.381)--(10.126,8.381)--(10.127,8.381)--(10.130,8.381)--(10.132,8.381)--(10.134,8.381)%
  --(10.136,8.381)--(10.139,8.381)--(10.141,8.381)--(10.144,8.381)--(10.146,8.381)--(10.148,8.381)%
  --(10.151,8.381)--(10.153,8.381)--(10.155,8.381)--(10.157,8.381)--(10.159,8.381)--(10.161,8.381)%
  --(10.163,8.381)--(10.165,8.381)--(10.167,8.381)--(10.170,8.381)--(10.173,8.381)--(10.175,8.381)%
  --(10.178,8.381)--(10.181,8.381)--(10.183,8.381)--(10.185,8.381)--(10.188,8.381)--(10.190,8.381)%
  --(10.192,8.381)--(10.194,8.381)--(10.196,8.381)--(10.198,8.381)--(10.200,8.381)--(10.203,8.381)%
  --(10.205,8.381)--(10.207,8.381)--(10.209,8.381)--(10.211,8.381)--(10.214,8.381)--(10.216,8.381)%
  --(10.218,8.381)--(10.220,8.381)--(10.222,8.381)--(10.225,8.381)--(10.227,8.381)--(10.229,8.381)%
  --(10.231,8.381)--(10.233,8.381)--(10.235,8.381)--(10.237,8.381)--(10.239,8.381)--(10.242,8.381)%
  --(10.244,8.381)--(10.246,8.381)--(10.248,8.381)--(10.251,8.381)--(10.253,8.381)--(10.255,8.381)%
  --(10.257,8.381)--(10.260,8.381)--(10.262,8.381)--(10.264,8.381)--(10.266,8.381)--(10.268,8.381)%
  --(10.270,8.381)--(10.272,8.381)--(10.275,8.381)--(10.277,8.381)--(10.279,8.381)--(10.282,8.381)%
  --(10.284,8.381)--(10.286,8.381)--(10.288,8.381)--(10.290,8.381)--(10.293,8.381)--(10.295,8.381)%
  --(10.297,8.381)--(10.299,8.381)--(10.301,8.381)--(10.303,8.381)--(10.306,8.381)--(10.308,8.381)%
  --(10.310,8.381)--(10.313,8.381)--(10.315,8.381)--(10.317,8.381)--(10.319,8.381)--(10.321,8.381)%
  --(10.324,8.381)--(10.326,8.381)--(10.328,8.381)--(10.330,8.381)--(10.333,8.381)--(10.335,8.381)%
  --(10.337,8.381)--(10.339,8.381)--(10.341,8.381)--(10.344,8.381)--(10.346,8.381)--(10.349,8.381)%
  --(10.351,8.381)--(10.353,8.381)--(10.355,8.381)--(10.357,8.381)--(10.360,8.381)--(10.362,8.381)%
  --(10.364,8.381)--(10.366,8.381)--(10.368,8.381)--(10.371,8.381)--(10.373,8.381)--(10.375,8.381)%
  --(10.378,8.381)--(10.380,8.381)--(10.383,8.381)--(10.385,8.381)--(10.388,8.381)--(10.391,8.381)%
  --(10.394,8.381)--(10.396,8.381)--(10.398,8.381)--(10.401,8.381)--(10.403,8.381)--(10.405,8.381)%
  --(10.407,8.381)--(10.410,8.381)--(10.412,8.381)--(10.414,8.381)--(10.417,8.381)--(10.419,8.381)%
  --(10.422,8.381)--(10.424,8.381)--(10.427,8.381)--(10.429,8.381)--(10.432,8.381)--(10.434,8.381)%
  --(10.436,8.381)--(10.438,8.381)--(10.441,8.381)--(10.443,8.381)--(10.445,8.381)--(10.448,8.381)%
  --(10.451,8.381)--(10.453,8.381)--(10.455,8.381)--(10.458,8.381)--(10.460,8.381)--(10.462,8.381)%
  --(10.464,8.381)--(10.467,8.381)--(10.469,8.381)--(10.471,8.381)--(10.473,8.381)--(10.476,8.381)%
  --(10.478,8.381)--(10.480,8.381)--(10.483,8.381)--(10.486,8.381)--(10.489,8.381)--(10.492,8.381)%
  --(10.495,8.381)--(10.497,8.381)--(10.500,8.381)--(10.503,8.381)--(10.505,8.381)--(10.507,8.381)%
  --(10.509,8.381)--(10.512,8.381)--(10.514,8.381)--(10.517,8.381)--(10.519,8.381)--(10.522,8.381)%
  --(10.524,8.381)--(10.527,8.381)--(10.529,8.381)--(10.531,8.381)--(10.534,8.381)--(10.536,8.381)%
  --(10.538,8.381)--(10.541,8.381)--(10.543,8.381)--(10.546,8.381)--(10.548,8.381)--(10.550,8.381)%
  --(10.553,8.381)--(10.556,8.381)--(10.560,8.381)--(10.563,8.381)--(10.565,8.381)--(10.568,8.381)%
  --(10.571,8.381)--(10.573,8.381)--(10.576,8.381)--(10.578,8.381)--(10.580,8.381)--(10.583,8.381)%
  --(10.585,8.381)--(10.588,8.381)--(10.591,8.381)--(10.594,8.381)--(10.597,8.381)--(10.599,8.381)%
  --(10.602,8.381)--(10.605,8.381)--(10.608,8.381)--(10.610,8.381)--(10.613,8.381)--(10.615,8.381)%
  --(10.618,8.381)--(10.620,8.381)--(10.623,8.381)--(10.625,8.381)--(10.628,8.381)--(10.630,8.381)%
  --(10.633,8.381)--(10.635,8.381)--(10.638,8.381)--(10.640,8.381)--(10.643,8.381)--(10.645,8.381)%
  --(10.648,8.381)--(10.650,8.381)--(10.653,8.381)--(10.655,8.381)--(10.658,8.381)--(10.662,8.381)%
  --(10.665,8.381)--(10.668,8.381)--(10.671,8.381)--(10.674,8.381)--(10.677,8.381)--(10.679,8.381)%
  --(10.682,8.381)--(10.684,8.381)--(10.687,8.381)--(10.689,8.381)--(10.692,8.381)--(10.696,8.381)%
  --(10.699,8.381)--(10.702,8.381)--(10.705,8.381)--(10.708,8.381)--(10.711,8.381)--(10.713,8.381)%
  --(10.716,8.381)--(10.718,8.381)--(10.721,8.381)--(10.723,8.381)--(10.727,8.381)--(10.730,8.381)%
  --(10.733,8.381)--(10.736,8.381)--(10.739,8.381)--(10.742,8.381)--(10.745,8.381)--(10.748,8.381)%
  --(10.750,8.381)--(10.753,8.381)--(10.755,8.381)--(10.758,8.381)--(10.761,8.381)--(10.764,8.381)%
  --(10.767,8.381)--(10.770,8.381)--(10.773,8.381)--(10.776,8.381)--(10.779,8.381)--(10.782,8.381)%
  --(10.785,8.381)--(10.787,8.381)--(10.790,8.381)--(10.792,8.381)--(10.795,8.381)--(10.799,8.381)%
  --(10.802,8.381)--(10.805,8.381)--(10.808,8.381)--(10.811,8.381)--(10.814,8.381)--(10.817,8.381)%
  --(10.819,8.381)--(10.822,8.381)--(10.825,8.381)--(10.827,8.381)--(10.831,8.381)--(10.834,8.381)%
  --(10.837,8.381)--(10.840,8.381)--(10.843,8.381)--(10.846,8.381)--(10.849,8.381)--(10.852,8.381)%
  --(10.855,8.381)--(10.858,8.381)--(10.860,8.381)--(10.863,8.381)--(10.866,8.381)--(10.870,8.381)%
  --(10.873,8.381)--(10.877,8.381)--(10.880,8.381)--(10.883,8.381)--(10.886,8.381)--(10.889,8.381)%
  --(10.891,8.381)--(10.894,8.381)--(10.897,8.381)--(10.899,8.381)--(10.902,8.381)--(10.905,8.381)%
  --(10.908,8.381)--(10.911,8.381)--(10.914,8.381)--(10.917,8.381)--(10.920,8.381)--(10.923,8.381)%
  --(10.926,8.381)--(10.928,8.381)--(10.931,8.381)--(10.934,8.381)--(10.937,8.381)--(10.940,8.381)%
  --(10.943,8.381)--(10.946,8.381)--(10.949,8.381)--(10.951,8.381)--(10.954,8.381)--(10.957,8.381)%
  --(10.960,8.381)--(10.963,8.381)--(10.966,8.381)--(10.968,8.381)--(10.971,8.381)--(10.975,8.381)%
  --(10.978,8.381)--(10.981,8.381)--(10.984,8.381)--(10.987,8.381)--(10.990,8.381)--(10.992,8.381)%
  --(10.995,8.381)--(10.998,8.381)--(11.001,8.381)--(11.004,8.381)--(11.007,8.381)--(11.010,8.381)%
  --(11.013,8.381)--(11.016,8.381)--(11.019,8.381)--(11.022,8.381)--(11.025,8.381)--(11.028,8.381)%
  --(11.031,8.381)--(11.034,8.381)--(11.037,8.381)--(11.039,8.381)--(11.042,8.381)--(11.046,8.381)%
  --(11.049,8.381)--(11.052,8.381)--(11.055,8.381)--(11.058,8.381)--(11.060,8.381)--(11.064,8.381)%
  --(11.067,8.381)--(11.069,8.381)--(11.072,8.381)--(11.075,8.381)--(11.078,8.381)--(11.081,8.381)%
  --(11.084,8.381)--(11.087,8.381)--(11.090,8.381)--(11.093,8.381)--(11.096,8.381)--(11.099,8.381)%
  --(11.102,8.381)--(11.105,8.381)--(11.108,8.381)--(11.111,8.381)--(11.114,8.381)--(11.117,8.381)%
  --(11.120,8.381)--(11.123,8.381)--(11.126,8.381)--(11.129,8.381)--(11.132,8.381)--(11.135,8.381)%
  --(11.138,8.381)--(11.141,8.381)--(11.144,8.381)--(11.148,8.381)--(11.151,8.381)--(11.154,8.381)%
  --(11.157,8.381)--(11.160,8.381)--(11.163,8.381)--(11.165,8.381)--(11.168,8.381)--(11.171,8.381)%
  --(11.175,8.381)--(11.178,8.381)--(11.181,8.381)--(11.185,8.381)--(11.188,8.381)--(11.192,8.381)%
  --(11.195,8.381)--(11.199,8.381)--(11.202,8.381)--(11.205,8.381)--(11.208,8.381)--(11.211,8.381)%
  --(11.214,8.381)--(11.218,8.381)--(11.221,8.381)--(11.224,8.381)--(11.227,8.381)--(11.231,8.381)%
  --(11.234,8.381)--(11.237,8.381)--(11.240,8.381)--(11.243,8.381)--(11.247,8.381)--(11.250,8.381)%
  --(11.253,8.381)--(11.256,8.381)--(11.259,8.381)--(11.262,8.381)--(11.265,8.381)--(11.269,8.381)%
  --(11.272,8.381)--(11.275,8.381)--(11.278,8.381)--(11.282,8.381)--(11.285,8.381)--(11.288,8.381)%
  --(11.292,8.381)--(11.296,8.381)--(11.300,8.381)--(11.304,8.381)--(11.307,8.381)--(11.310,8.381)%
  --(11.313,8.381)--(11.316,8.381)--(11.319,8.381)--(11.323,8.381)--(11.326,8.381)--(11.330,8.381)%
  --(11.333,8.381)--(11.336,8.381)--(11.340,8.381)--(11.343,8.381)--(11.346,8.381)--(11.350,8.381)%
  --(11.353,8.381)--(11.356,8.381)--(11.360,8.381)--(11.363,8.381)--(11.366,8.381)--(11.369,8.381)%
  --(11.373,8.381)--(11.376,8.381)--(11.379,8.381)--(11.382,8.381)--(11.386,8.381)--(11.390,8.381)%
  --(11.394,8.381)--(11.398,8.381)--(11.401,8.381)--(11.405,8.381)--(11.408,8.381)--(11.412,8.381)%
  --(11.415,8.381)--(11.419,8.381)--(11.422,8.381)--(11.425,8.381)--(11.429,8.381)--(11.432,8.381)%
  --(11.435,8.381)--(11.439,8.381)--(11.442,8.381)--(11.445,8.381)--(11.449,8.381)--(11.452,8.381)%
  --(11.456,8.381)--(11.460,8.381)--(11.464,8.381)--(11.468,8.381)--(11.472,8.381)--(11.475,8.381)%
  --(11.479,8.381)--(11.483,8.381)--(11.486,8.381)--(11.490,8.381)--(11.493,8.381)--(11.497,8.381)%
  --(11.501,8.381)--(11.505,8.381)--(11.509,8.381)--(11.513,8.381)--(11.516,8.381)--(11.520,8.381)%
  --(11.523,8.381)--(11.526,8.381)--(11.530,8.381)--(11.533,8.381)--(11.537,8.381)--(11.540,8.381)%
  --(11.544,8.381)--(11.547,8.381)--(11.551,8.381)--(11.554,8.381)--(11.558,8.381)--(11.562,8.381)%
  --(11.566,8.381)--(11.570,8.381)--(11.574,8.381)--(11.578,8.381)--(11.582,8.381)--(11.586,8.381)%
  --(11.589,8.381)--(11.593,8.381)--(11.596,8.381)--(11.600,8.381)--(11.604,8.381)--(11.609,8.381)%
  --(11.613,8.381)--(11.617,8.381)--(11.621,8.381)--(11.624,8.381)--(11.628,8.381)--(11.631,8.381)%
  --(11.635,8.381)--(11.639,8.381)--(11.642,8.381)--(11.646,8.381)--(11.649,8.381)--(11.653,8.381)%
  --(11.657,8.381)--(11.660,8.381)--(11.665,8.381)--(11.669,8.381)--(11.673,8.381)--(11.677,8.381)%
  --(11.681,8.381)--(11.685,8.381)--(11.689,8.381)--(11.693,8.381)--(11.697,8.381)--(11.701,8.381)%
  --(11.705,8.381)--(11.709,8.381)--(11.713,8.381)--(11.717,8.381)--(11.721,8.381)--(11.725,8.381)%
  --(11.729,8.381)--(11.733,8.381)--(11.736,8.381)--(11.740,8.381)--(11.745,8.381)--(11.749,8.381)%
  --(11.753,8.381)--(11.758,8.381)--(11.761,8.381)--(11.765,8.381)--(11.769,8.381)--(11.773,8.381)%
  --(11.776,8.381)--(11.780,8.381)--(11.784,8.381)--(11.788,8.381)--(11.791,8.381)--(11.795,8.381)%
  --(11.799,8.381)--(11.803,8.381)--(11.807,8.381)--(11.811,8.381)--(11.815,8.381)--(11.818,8.381)%
  --(11.822,8.381)--(11.826,8.381)--(11.830,8.381)--(11.834,8.381)--(11.839,8.381)--(11.843,8.381)%
  --(11.848,8.381)--(11.852,8.381)--(11.856,8.381)--(11.860,8.381)--(11.865,8.381)--(11.869,8.381)%
  --(11.873,8.381)--(11.878,8.381)--(11.882,8.381)--(11.887,8.381)--(11.891,8.381)--(11.895,8.381)%
  --(11.899,8.381)--(11.903,8.381)--(11.908,8.381)--(11.912,8.381)--(11.917,8.381)--(11.921,8.381)%
  --(11.925,8.381)--(11.930,8.381)--(11.934,8.381)--(11.938,8.381)--(11.942,8.381);
\gpsetdashtype{gp dt solid}
\gpsetlinewidth{1.00}
\draw[gp path] (1.320,8.381)--(1.320,0.985)--(11.947,0.985)--(11.947,8.381)--cycle;
%% coordinates of the plot area
\gpdefrectangularnode{gp plot 1}{\pgfpoint{1.320cm}{0.985cm}}{\pgfpoint{11.947cm}{8.381cm}}
\end{tikzpicture}
%% gnuplot variables

    \caption{Cart position time history. Note that the displacement between the carts is negligible.}%
    \label{f:accresponse}
\end{figure}

%\begin{figure}
    %\centering
    %\begin{tikzpicture}[gnuplot]
%% generated with GNUPLOT 5.0p3 (Lua 5.1; terminal rev. 99, script rev. 100)
%% Tue 27 Mar 2018 11:23:02 PM EDT
\gpmonochromelines
\path (0.000,0.000) rectangle (12.500,8.750);
\gpcolor{color=gp lt color border}
\gpsetlinetype{gp lt border}
\gpsetdashtype{gp dt solid}
\gpsetlinewidth{1.00}
\draw[gp path] (1.504,1.022)--(1.684,1.022);
\draw[gp path] (10.475,1.022)--(10.295,1.022);
\node[gp node right] at (1.320,1.022) {$-1$};
\draw[gp path] (1.504,2.852)--(1.684,2.852);
\draw[gp path] (10.475,2.852)--(10.295,2.852);
\node[gp node right] at (1.320,2.852) {$-0.5$};
\draw[gp path] (1.504,4.683)--(1.684,4.683);
\draw[gp path] (10.475,4.683)--(10.295,4.683);
\node[gp node right] at (1.320,4.683) {$0$};
\draw[gp path] (1.504,6.514)--(1.684,6.514);
\draw[gp path] (10.475,6.514)--(10.295,6.514);
\node[gp node right] at (1.320,6.514) {$0.5$};
\draw[gp path] (1.504,8.344)--(1.684,8.344);
\draw[gp path] (10.475,8.344)--(10.295,8.344);
\node[gp node right] at (1.320,8.344) {$1$};
\draw[gp path] (1.504,0.985)--(1.504,1.165);
\draw[gp path] (1.504,8.381)--(1.504,8.201);
\node[gp node center] at (1.504,0.677) {$0$};
\draw[gp path] (2.401,0.985)--(2.401,1.165);
\draw[gp path] (2.401,8.381)--(2.401,8.201);
\node[gp node center] at (2.401,0.677) {$5$};
\draw[gp path] (3.298,0.985)--(3.298,1.165);
\draw[gp path] (3.298,8.381)--(3.298,8.201);
\node[gp node center] at (3.298,0.677) {$10$};
\draw[gp path] (4.195,0.985)--(4.195,1.165);
\draw[gp path] (4.195,8.381)--(4.195,8.201);
\node[gp node center] at (4.195,0.677) {$15$};
\draw[gp path] (5.092,0.985)--(5.092,1.165);
\draw[gp path] (5.092,8.381)--(5.092,8.201);
\node[gp node center] at (5.092,0.677) {$20$};
\draw[gp path] (5.990,0.985)--(5.990,1.165);
\draw[gp path] (5.990,8.381)--(5.990,8.201);
\node[gp node center] at (5.990,0.677) {$25$};
\draw[gp path] (6.887,0.985)--(6.887,1.165);
\draw[gp path] (6.887,8.381)--(6.887,8.201);
\node[gp node center] at (6.887,0.677) {$30$};
\draw[gp path] (7.784,0.985)--(7.784,1.165);
\draw[gp path] (7.784,8.381)--(7.784,8.201);
\node[gp node center] at (7.784,0.677) {$35$};
\draw[gp path] (8.681,0.985)--(8.681,1.165);
\draw[gp path] (8.681,8.381)--(8.681,8.201);
\node[gp node center] at (8.681,0.677) {$40$};
\draw[gp path] (9.578,0.985)--(9.578,1.165);
\draw[gp path] (9.578,8.381)--(9.578,8.201);
\node[gp node center] at (9.578,0.677) {$45$};
\draw[gp path] (10.475,0.985)--(10.475,1.165);
\draw[gp path] (10.475,8.381)--(10.475,8.201);
\node[gp node center] at (10.475,0.677) {$50$};
\draw[gp path] (10.475,3.450)--(10.295,3.450);
\node[gp node left] at (10.659,3.450) {$-0.01$};
\draw[gp path] (10.475,4.683)--(10.295,4.683);
\node[gp node left] at (10.659,4.683) {$-0.005$};
\draw[gp path] (10.475,5.916)--(10.295,5.916);
\node[gp node left] at (10.659,5.916) {$0$};
\draw[gp path] (10.475,7.148)--(10.295,7.148);
\node[gp node left] at (10.659,7.148) {$0.005$};
\draw[gp path] (10.475,8.381)--(10.295,8.381);
\node[gp node left] at (10.659,8.381) {$0.01$};
\draw[gp path] (1.504,8.381)--(1.504,0.985)--(10.475,0.985)--(10.475,8.381)--cycle;
\node[gp node center,rotate=-270] at (0.246,4.683) {Relative displacement, m};
\node[gp node center] at (5.989,0.215) {Time, s};
\node[gp node right] at (9.007,8.047) {$x_1-x_2$};
\draw[gp path] (9.191,8.047)--(10.107,8.047);
\draw[gp path] (1.504,5.916)--(1.505,5.915)--(1.505,5.912)--(1.506,5.907)--(1.507,5.900)%
  --(1.508,5.892)--(1.508,5.881)--(1.509,5.869)--(1.510,5.855)--(1.511,5.831)--(1.512,5.803)%
  --(1.513,5.772)--(1.514,5.739)--(1.515,5.703)--(1.516,5.665)--(1.517,5.626)--(1.519,5.587)%
  --(1.520,5.544)--(1.521,5.502)--(1.522,5.462)--(1.523,5.424)--(1.525,5.380)--(1.526,5.342)%
  --(1.528,5.310)--(1.529,5.285)--(1.531,5.271)--(1.532,5.262)--(1.533,5.258)--(1.534,5.260)%
  --(1.535,5.266)--(1.536,5.277)--(1.537,5.292)--(1.539,5.312)--(1.540,5.336)--(1.541,5.362)%
  --(1.542,5.392)--(1.543,5.425)--(1.544,5.460)--(1.545,5.497)--(1.546,5.536)--(1.547,5.575)%
  --(1.549,5.617)--(1.550,5.659)--(1.551,5.699)--(1.552,5.738)--(1.554,5.782)--(1.555,5.821)%
  --(1.556,5.854)--(1.558,5.881)--(1.559,5.897)--(1.560,5.908)--(1.561,5.914)--(1.563,5.915)%
  --(1.564,5.911)--(1.565,5.903)--(1.566,5.889)--(1.567,5.870)--(1.568,5.848)--(1.569,5.823)%
  --(1.570,5.794)--(1.572,5.763)--(1.573,5.729)--(1.574,5.692)--(1.575,5.654)--(1.576,5.616)%
  --(1.577,5.574)--(1.578,5.532)--(1.579,5.492)--(1.581,5.452)--(1.582,5.409)--(1.583,5.369)%
  --(1.585,5.335)--(1.586,5.306)--(1.587,5.287)--(1.589,5.272)--(1.590,5.263)--(1.591,5.258)%
  --(1.592,5.260)--(1.593,5.266)--(1.595,5.278)--(1.596,5.294)--(1.597,5.314)--(1.598,5.338)%
  --(1.599,5.365)--(1.600,5.395)--(1.601,5.428)--(1.602,5.463)--(1.603,5.500)--(1.604,5.538)%
  --(1.606,5.579)--(1.607,5.621)--(1.608,5.661)--(1.609,5.701)--(1.611,5.744)--(1.612,5.785)%
  --(1.613,5.821)--(1.615,5.852)--(1.616,5.875)--(1.617,5.893)--(1.618,5.906)--(1.619,5.913)%
  --(1.621,5.916)--(1.622,5.912)--(1.623,5.903)--(1.624,5.889)--(1.625,5.871)--(1.626,5.849)%
  --(1.628,5.823)--(1.629,5.795)--(1.630,5.763)--(1.631,5.729)--(1.632,5.693)--(1.633,5.656)%
  --(1.634,5.615)--(1.635,5.574)--(1.636,5.534)--(1.638,5.494)--(1.639,5.450)--(1.640,5.410)%
  --(1.642,5.372)--(1.643,5.339)--(1.644,5.314)--(1.645,5.292)--(1.647,5.272)--(1.648,5.265)%
  --(1.649,5.260)--(1.650,5.260)--(1.652,5.267)--(1.653,5.280)--(1.654,5.294)--(1.655,5.317)%
  --(1.656,5.341)--(1.657,5.368)--(1.658,5.400)--(1.660,5.432)--(1.661,5.469)--(1.662,5.504)%
  --(1.663,5.546)--(1.664,5.585)--(1.665,5.625)--(1.666,5.664)--(1.668,5.709)--(1.669,5.748)%
  --(1.670,5.787)--(1.671,5.822)--(1.673,5.854)--(1.674,5.876)--(1.675,5.893)--(1.677,5.908)%
  --(1.678,5.916)--(1.679,5.916)--(1.680,5.908)--(1.682,5.896)--(1.683,5.881)--(1.684,5.861)%
  --(1.685,5.839)--(1.686,5.812)--(1.687,5.780)--(1.688,5.748)--(1.690,5.716)--(1.691,5.679)%
  --(1.692,5.640)--(1.693,5.598)--(1.694,5.558)--(1.695,5.516)--(1.696,5.474)--(1.698,5.432)%
  --(1.699,5.395)--(1.700,5.361)--(1.702,5.329)--(1.703,5.304)--(1.704,5.285)--(1.705,5.270)%
  --(1.707,5.260)--(1.708,5.260)--(1.709,5.262)--(1.711,5.275)--(1.712,5.289)--(1.713,5.309)%
  --(1.714,5.329)--(1.715,5.359)--(1.716,5.386)--(1.717,5.418)--(1.718,5.455)--(1.720,5.492)%
  --(1.721,5.531)--(1.722,5.571)--(1.723,5.610)--(1.724,5.649)--(1.725,5.694)--(1.727,5.733)%
  --(1.728,5.773)--(1.729,5.807)--(1.730,5.842)--(1.732,5.866)--(1.733,5.889)--(1.734,5.903)%
  --(1.736,5.913)--(1.737,5.916)--(1.738,5.913)--(1.740,5.901)--(1.741,5.886)--(1.742,5.869)%
  --(1.743,5.844)--(1.744,5.820)--(1.745,5.790)--(1.746,5.755)--(1.747,5.723)--(1.749,5.686)%
  --(1.750,5.649)--(1.751,5.608)--(1.752,5.568)--(1.753,5.526)--(1.754,5.484)--(1.756,5.442)%
  --(1.757,5.403)--(1.758,5.366)--(1.759,5.334)--(1.761,5.307)--(1.762,5.285)--(1.763,5.270)%
  --(1.765,5.262)--(1.766,5.260)--(1.767,5.262)--(1.769,5.272)--(1.770,5.287)--(1.771,5.307)%
  --(1.772,5.326)--(1.773,5.354)--(1.774,5.383)--(1.775,5.418)--(1.777,5.450)--(1.778,5.487)%
  --(1.779,5.526)--(1.780,5.563)--(1.781,5.605)--(1.782,5.644)--(1.783,5.686)--(1.785,5.728)%
  --(1.786,5.768)--(1.787,5.805)--(1.789,5.837)--(1.790,5.866)--(1.791,5.886)--(1.793,5.903)%
  --(1.794,5.913)--(1.795,5.916)--(1.797,5.911)--(1.798,5.901)--(1.799,5.886)--(1.800,5.869)%
  --(1.801,5.847)--(1.802,5.820)--(1.803,5.792)--(1.805,5.760)--(1.806,5.726)--(1.807,5.689)%
  --(1.808,5.649)--(1.809,5.610)--(1.810,5.568)--(1.811,5.531)--(1.813,5.487)--(1.814,5.445)%
  --(1.815,5.405)--(1.816,5.371)--(1.818,5.339)--(1.819,5.309)--(1.820,5.287)--(1.822,5.272)%
  --(1.823,5.262)--(1.824,5.260)--(1.826,5.262)--(1.827,5.272)--(1.828,5.287)--(1.829,5.304)%
  --(1.830,5.329)--(1.831,5.354)--(1.833,5.383)--(1.834,5.413)--(1.835,5.450)--(1.836,5.487)%
  --(1.837,5.524)--(1.838,5.563)--(1.839,5.603)--(1.840,5.644)--(1.842,5.686)--(1.843,5.728)%
  --(1.844,5.768)--(1.845,5.802)--(1.847,5.837)--(1.848,5.864)--(1.849,5.886)--(1.851,5.903)%
  --(1.852,5.911)--(1.853,5.916)--(1.855,5.911)--(1.856,5.901)--(1.857,5.889)--(1.858,5.869)%
  --(1.859,5.847)--(1.861,5.820)--(1.862,5.792)--(1.863,5.760)--(1.864,5.726)--(1.865,5.689)%
  --(1.866,5.649)--(1.867,5.612)--(1.868,5.571)--(1.869,5.531)--(1.871,5.487)--(1.872,5.447)%
  --(1.873,5.405)--(1.874,5.371)--(1.876,5.336)--(1.877,5.309)--(1.878,5.287)--(1.880,5.270)%
  --(1.881,5.260)--(1.882,5.260)--(1.884,5.262)--(1.885,5.272)--(1.886,5.287)--(1.887,5.304)%
  --(1.889,5.329)--(1.890,5.354)--(1.891,5.383)--(1.892,5.415)--(1.893,5.447)--(1.894,5.484)%
  --(1.895,5.521)--(1.896,5.563)--(1.897,5.605)--(1.899,5.644)--(1.900,5.686)--(1.901,5.728)%
  --(1.902,5.765)--(1.904,5.802)--(1.905,5.837)--(1.906,5.864)--(1.908,5.886)--(1.909,5.903)%
  --(1.910,5.911)--(1.912,5.916)--(1.913,5.911)--(1.914,5.903)--(1.915,5.886)--(1.916,5.869)%
  --(1.918,5.847)--(1.919,5.820)--(1.920,5.790)--(1.921,5.758)--(1.922,5.726)--(1.923,5.691)%
  --(1.924,5.649)--(1.925,5.612)--(1.927,5.568)--(1.928,5.531)--(1.929,5.487)--(1.930,5.445)%
  --(1.931,5.408)--(1.933,5.371)--(1.934,5.339)--(1.935,5.309)--(1.937,5.287)--(1.938,5.272)%
  --(1.939,5.262)--(1.941,5.260)--(1.942,5.262)--(1.943,5.267)--(1.944,5.275)--(1.946,5.299)%
  --(1.947,5.324)--(1.948,5.349)--(1.949,5.373)--(1.950,5.423)--(1.951,5.447)--(1.952,5.497)%
  --(1.953,5.521)--(1.954,5.571)--(1.956,5.620)--(1.957,5.644)--(1.958,5.694)--(1.959,5.743)%
  --(1.961,5.768)--(1.962,5.817)--(1.963,5.842)--(1.964,5.866)--(1.966,5.866)--(1.967,5.916)%
  --(1.968,5.916)--(1.970,5.916)--(1.971,5.891)--(1.972,5.916)--(1.974,5.891)--(1.975,5.866)%
  --(1.976,5.842)--(1.977,5.817)--(1.978,5.792)--(1.979,5.743)--(1.980,5.718)--(1.981,5.694)%
  --(1.982,5.644)--(1.984,5.595)--(1.985,5.571)--(1.986,5.521)--(1.987,5.497)--(1.988,5.447)%
  --(1.990,5.398)--(1.991,5.373)--(1.992,5.324)--(1.994,5.324)--(1.995,5.299)--(1.996,5.275)%
  --(1.998,5.250)--(1.999,5.250)--(2.000,5.275)--(2.002,5.275)--(2.003,5.299)--(2.004,5.299)%
  --(2.005,5.324)--(2.006,5.349)--(2.007,5.398)--(2.008,5.423)--(2.009,5.447)--(2.010,5.472)%
  --(2.012,5.521)--(2.013,5.571)--(2.014,5.595)--(2.015,5.644)--(2.016,5.669)--(2.018,5.718)%
  --(2.019,5.768)--(2.020,5.817)--(2.021,5.842)--(2.023,5.866)--(2.024,5.891)--(2.025,5.916)%
  --(2.027,5.916)--(2.028,5.916)--(2.029,5.916)--(2.031,5.891)--(2.032,5.891)--(2.033,5.866)%
  --(2.034,5.842)--(2.035,5.817)--(2.036,5.792)--(2.037,5.768)--(2.038,5.718)--(2.040,5.694)%
  --(2.041,5.644)--(2.042,5.620)--(2.043,5.571)--(2.044,5.521)--(2.045,5.497)--(2.047,5.447)%
  --(2.048,5.398)--(2.049,5.373)--(2.050,5.349)--(2.052,5.299)--(2.053,5.299)--(2.054,5.275)%
  --(2.056,5.275)--(2.057,5.250)--(2.058,5.250)--(2.060,5.275)--(2.061,5.275)--(2.062,5.324)%
  --(2.063,5.324)--(2.064,5.349)--(2.065,5.373)--(2.066,5.423)--(2.068,5.472)--(2.069,5.472)%
  --(2.070,5.521)--(2.071,5.571)--(2.072,5.595)--(2.073,5.644)--(2.074,5.694)--(2.076,5.718)%
  --(2.077,5.768)--(2.078,5.817)--(2.080,5.842)--(2.081,5.866)--(2.082,5.891)--(2.084,5.916)%
  --(2.085,5.916)--(2.086,5.916)--(2.088,5.916)--(2.089,5.891)--(2.090,5.891)--(2.091,5.866)%
  --(2.092,5.842)--(2.093,5.817)--(2.094,5.792)--(2.096,5.743)--(2.097,5.718)--(2.098,5.694)%
  --(2.099,5.644)--(2.100,5.595)--(2.101,5.571)--(2.102,5.521)--(2.104,5.472)--(2.105,5.447)%
  --(2.106,5.398)--(2.107,5.373)--(2.109,5.324)--(2.110,5.299)--(2.111,5.275)--(2.113,5.275)%
  --(2.114,5.250)--(2.115,5.275)--(2.117,5.250)--(2.118,5.275)--(2.119,5.299)--(2.120,5.299)%
  --(2.121,5.324)--(2.122,5.349)--(2.124,5.398)--(2.125,5.423)--(2.126,5.447)--(2.127,5.497)%
  --(2.128,5.521)--(2.129,5.546)--(2.130,5.595)--(2.131,5.644)--(2.133,5.694)--(2.134,5.718)%
  --(2.135,5.768)--(2.136,5.817)--(2.138,5.842)--(2.139,5.866)--(2.140,5.891)--(2.142,5.891)%
  --(2.143,5.891)--(2.144,5.916)--(2.146,5.916)--(2.147,5.916)--(2.148,5.891)--(2.149,5.866)%
  --(2.150,5.842)--(2.152,5.817)--(2.153,5.792)--(2.154,5.768)--(2.155,5.718)--(2.156,5.669)%
  --(2.157,5.644)--(2.158,5.620)--(2.159,5.571)--(2.161,5.521)--(2.162,5.497)--(2.163,5.447)%
  --(2.164,5.423)--(2.166,5.373)--(2.167,5.349)--(2.168,5.299)--(2.170,5.299)--(2.171,5.275)%
  --(2.172,5.275)--(2.174,5.250)--(2.175,5.275)--(2.176,5.275)--(2.177,5.275)--(2.178,5.299)%
  --(2.180,5.324)--(2.181,5.349)--(2.182,5.373)--(2.183,5.423)--(2.184,5.447)--(2.185,5.497)%
  --(2.186,5.521)--(2.187,5.571)--(2.188,5.620)--(2.190,5.644)--(2.191,5.694)--(2.192,5.743)%
  --(2.193,5.768)--(2.195,5.817)--(2.196,5.842)--(2.197,5.866)--(2.199,5.891)--(2.200,5.916)%
  --(2.201,5.891)--(2.203,5.916)--(2.204,5.916)--(2.205,5.916)--(2.206,5.891)--(2.208,5.866)%
  --(2.209,5.842)--(2.210,5.842)--(2.211,5.792)--(2.212,5.768)--(2.213,5.718)--(2.214,5.694)%
  --(2.215,5.644)--(2.216,5.620)--(2.218,5.571)--(2.219,5.521)--(2.220,5.497)--(2.221,5.447)%
  --(2.223,5.423)--(2.224,5.373)--(2.225,5.349)--(2.226,5.324)--(2.228,5.299)--(2.229,5.250)%
  --(2.230,5.250)--(2.232,5.275)--(2.233,5.250)--(2.234,5.275)--(2.236,5.299)--(2.237,5.299)%
  --(2.238,5.324)--(2.239,5.349)--(2.240,5.373)--(2.241,5.423)--(2.242,5.447)--(2.243,5.497)%
  --(2.244,5.521)--(2.246,5.571)--(2.247,5.595)--(2.248,5.644)--(2.249,5.694)--(2.250,5.718)%
  --(2.252,5.768)--(2.253,5.792)--(2.254,5.842)--(2.256,5.866)--(2.257,5.866)--(2.258,5.916)%
  --(2.260,5.916)--(2.261,5.916)--(2.262,5.916)--(2.264,5.916)--(2.265,5.891)--(2.266,5.891)%
  --(2.267,5.866)--(2.268,5.842)--(2.269,5.768)--(2.270,5.743)--(2.271,5.718)--(2.272,5.669)%
  --(2.274,5.669)--(2.275,5.620)--(2.276,5.571)--(2.277,5.521)--(2.278,5.497)--(2.279,5.447)%
  --(2.281,5.398)--(2.282,5.373)--(2.283,5.324)--(2.285,5.299)--(2.286,5.275)--(2.287,5.275)%
  --(2.289,5.275)--(2.290,5.250)--(2.291,5.275)--(2.293,5.275)--(2.294,5.299)--(2.295,5.299)%
  --(2.296,5.324)--(2.297,5.373)--(2.298,5.373)--(2.299,5.398)--(2.300,5.447)--(2.302,5.497)%
  --(2.303,5.521)--(2.304,5.546)--(2.305,5.620)--(2.306,5.644)--(2.307,5.694)--(2.309,5.743)%
  --(2.310,5.768)--(2.311,5.792)--(2.312,5.817)--(2.314,5.866)--(2.315,5.891)--(2.316,5.916)%
  --(2.318,5.916)--(2.319,5.916)--(2.320,5.916)--(2.322,5.916)--(2.323,5.891)--(2.324,5.866)%
  --(2.325,5.842)--(2.326,5.842)--(2.327,5.792)--(2.328,5.768)--(2.330,5.718)--(2.331,5.694)%
  --(2.332,5.644)--(2.333,5.620)--(2.334,5.571)--(2.335,5.521)--(2.336,5.472)--(2.338,5.447)%
  --(2.339,5.423)--(2.340,5.373)--(2.342,5.324)--(2.343,5.299)--(2.344,5.275)--(2.346,5.275)%
  --(2.347,5.250)--(2.348,5.250)--(2.350,5.250)--(2.351,5.275)--(2.352,5.299)--(2.353,5.299)%
  --(2.354,5.324)--(2.355,5.349)--(2.356,5.373)--(2.358,5.423)--(2.359,5.447)--(2.360,5.472)%
  --(2.361,5.521)--(2.362,5.571)--(2.363,5.595)--(2.364,5.644)--(2.366,5.669)--(2.367,5.718)%
  --(2.368,5.768)--(2.369,5.817)--(2.371,5.842)--(2.372,5.842)--(2.373,5.891)--(2.375,5.891)%
  --(2.376,5.916)--(2.377,5.916)--(2.379,5.916)--(2.380,5.891)--(2.381,5.891)--(2.382,5.866)%
  --(2.383,5.842)--(2.384,5.842)--(2.386,5.792)--(2.387,5.768)--(2.388,5.718)--(2.389,5.694)%
  --(2.390,5.644)--(2.391,5.620)--(2.392,5.571)--(2.393,5.546)--(2.395,5.497)--(2.396,5.447)%
  --(2.397,5.398)--(2.398,5.373)--(2.400,5.324)--(2.401,5.299)--(2.402,5.299)--(2.404,5.275)%
  --(2.405,5.250)--(2.406,5.275)--(2.408,5.250)--(2.409,5.275)--(2.410,5.275)--(2.411,5.299)%
  --(2.412,5.324)--(2.414,5.349)--(2.415,5.373)--(2.416,5.398)--(2.417,5.447)--(2.418,5.472)%
  --(2.419,5.521)--(2.420,5.546)--(2.421,5.595)--(2.422,5.644)--(2.424,5.694)--(2.425,5.718)%
  --(2.426,5.768)--(2.428,5.792)--(2.429,5.842)--(2.430,5.866)--(2.432,5.866)--(2.433,5.891)%
  --(2.434,5.916)--(2.436,5.916)--(2.437,5.916)--(2.438,5.891)--(2.439,5.866)--(2.440,5.866)%
  --(2.442,5.842)--(2.443,5.817)--(2.444,5.792)--(2.445,5.768)--(2.446,5.718)--(2.447,5.694)%
  --(2.448,5.644)--(2.449,5.620)--(2.450,5.571)--(2.452,5.521)--(2.453,5.497)--(2.454,5.447)%
  --(2.455,5.423)--(2.457,5.373)--(2.458,5.324)--(2.459,5.299)--(2.461,5.299)--(2.462,5.275)%
  --(2.463,5.250)--(2.465,5.275)--(2.466,5.275)--(2.467,5.275)--(2.468,5.275)--(2.470,5.324)%
  --(2.471,5.349)--(2.472,5.349)--(2.473,5.373)--(2.474,5.423)--(2.475,5.447)--(2.476,5.497)%
  --(2.477,5.521)--(2.478,5.571)--(2.480,5.595)--(2.481,5.644)--(2.482,5.694)--(2.483,5.743)%
  --(2.484,5.768)--(2.486,5.817)--(2.487,5.842)--(2.488,5.842)--(2.490,5.891)--(2.491,5.916)%
  --(2.492,5.916)--(2.494,5.916)--(2.495,5.916)--(2.496,5.916)--(2.498,5.891)--(2.499,5.866)%
  --(2.500,5.842)--(2.501,5.817)--(2.502,5.792)--(2.503,5.768)--(2.504,5.718)--(2.505,5.669)%
  --(2.506,5.644)--(2.508,5.620)--(2.509,5.571)--(2.510,5.546)--(2.511,5.497)--(2.512,5.447)%
  --(2.514,5.423)--(2.515,5.373)--(2.516,5.324)--(2.518,5.299)--(2.519,5.299)--(2.520,5.250)%
  --(2.522,5.250)--(2.523,5.275)--(2.524,5.275)--(2.526,5.275)--(2.527,5.299)--(2.528,5.299)%
  --(2.529,5.324)--(2.530,5.349)--(2.531,5.373)--(2.532,5.398)--(2.533,5.447)--(2.534,5.497)%
  --(2.536,5.521)--(2.537,5.571)--(2.538,5.620)--(2.539,5.644)--(2.540,5.669)--(2.541,5.743)%
  --(2.543,5.768)--(2.544,5.817)--(2.545,5.817)--(2.547,5.866)--(2.548,5.866)--(2.549,5.891)%
  --(2.551,5.916)--(2.552,5.916)--(2.553,5.916)--(2.555,5.891)--(2.556,5.891)--(2.557,5.866)%
  --(2.558,5.842)--(2.559,5.817)--(2.560,5.792)--(2.561,5.743)--(2.562,5.718)--(2.563,5.669)%
  --(2.565,5.644)--(2.566,5.620)--(2.567,5.571)--(2.568,5.521)--(2.569,5.497)--(2.571,5.447)%
  --(2.572,5.398)--(2.573,5.349)--(2.574,5.324)--(2.576,5.299)--(2.577,5.299)--(2.578,5.275)%
  --(2.580,5.250)--(2.581,5.275)--(2.582,5.275)--(2.584,5.275)--(2.585,5.299)--(2.586,5.299)%
  --(2.587,5.324)--(2.588,5.349)--(2.589,5.398)--(2.590,5.423)--(2.592,5.447)--(2.593,5.497)%
  --(2.594,5.521)--(2.595,5.546)--(2.596,5.620)--(2.597,5.644)--(2.598,5.694)--(2.600,5.718)%
  --(2.601,5.768)--(2.602,5.792)--(2.604,5.842)--(2.605,5.866)--(2.606,5.891)--(2.608,5.891)%
  --(2.609,5.916)--(2.610,5.916)--(2.611,5.916)--(2.613,5.891)--(2.614,5.891)--(2.615,5.866)%
  --(2.616,5.842)--(2.617,5.817)--(2.618,5.792)--(2.620,5.743)--(2.621,5.718)--(2.622,5.669)%
  --(2.623,5.644)--(2.624,5.595)--(2.625,5.571)--(2.626,5.521)--(2.627,5.497)--(2.629,5.447)%
  --(2.630,5.398)--(2.631,5.373)--(2.633,5.349)--(2.634,5.299)--(2.635,5.275)--(2.637,5.275)%
  --(2.638,5.250)--(2.639,5.275)--(2.641,5.275)--(2.642,5.275)--(2.643,5.299)--(2.644,5.299)%
  --(2.645,5.324)--(2.646,5.349)--(2.648,5.373)--(2.649,5.423)--(2.650,5.447)--(2.651,5.472)%
  --(2.652,5.521)--(2.653,5.546)--(2.654,5.620)--(2.655,5.644)--(2.657,5.669)--(2.658,5.718)%
  --(2.659,5.768)--(2.660,5.792)--(2.662,5.842)--(2.663,5.866)--(2.664,5.891)--(2.666,5.916)%
  --(2.667,5.891)--(2.668,5.916)--(2.670,5.916)--(2.671,5.891)--(2.672,5.891)--(2.673,5.866)%
  --(2.674,5.842)--(2.676,5.817)--(2.677,5.792)--(2.678,5.768)--(2.679,5.718)--(2.680,5.694)%
  --(2.681,5.644)--(2.682,5.620)--(2.683,5.571)--(2.684,5.546)--(2.686,5.497)--(2.687,5.447)%
  --(2.688,5.398)--(2.689,5.373)--(2.691,5.349)--(2.692,5.324)--(2.693,5.275)--(2.695,5.275)%
  --(2.696,5.250)--(2.697,5.275)--(2.699,5.250)--(2.700,5.275)--(2.701,5.275)--(2.702,5.299)%
  --(2.704,5.324)--(2.705,5.349)--(2.706,5.373)--(2.707,5.423)--(2.708,5.447)--(2.709,5.497)%
  --(2.710,5.546)--(2.711,5.546)--(2.712,5.595)--(2.714,5.644)--(2.715,5.694)--(2.716,5.743)%
  --(2.717,5.768)--(2.719,5.792)--(2.720,5.842)--(2.721,5.866)--(2.723,5.866)--(2.724,5.891)%
  --(2.725,5.916)--(2.727,5.916)--(2.728,5.916)--(2.729,5.916)--(2.730,5.891)--(2.732,5.866)%
  --(2.733,5.842)--(2.734,5.817)--(2.735,5.792)--(2.736,5.768)--(2.737,5.718)--(2.738,5.694)%
  --(2.739,5.644)--(2.740,5.595)--(2.742,5.571)--(2.743,5.521)--(2.744,5.472)--(2.745,5.447)%
  --(2.746,5.398)--(2.748,5.349)--(2.749,5.324)--(2.750,5.299)--(2.752,5.299)--(2.753,5.275)%
  --(2.754,5.275)--(2.756,5.250)--(2.757,5.250)--(2.758,5.250)--(2.759,5.299)--(2.761,5.299)%
  --(2.762,5.324)--(2.763,5.349)--(2.764,5.373)--(2.765,5.423)--(2.766,5.447)--(2.767,5.472)%
  --(2.768,5.521)--(2.770,5.546)--(2.771,5.595)--(2.772,5.644)--(2.773,5.694)--(2.774,5.743)%
  --(2.776,5.768)--(2.777,5.817)--(2.778,5.842)--(2.779,5.866)--(2.781,5.866)--(2.782,5.891)%
  --(2.783,5.916)--(2.785,5.916)--(2.786,5.916)--(2.787,5.891)--(2.789,5.891)--(2.790,5.866)%
  --(2.791,5.866)--(2.792,5.842)--(2.793,5.792)--(2.794,5.768)--(2.795,5.718)--(2.796,5.694)%
  --(2.797,5.669)--(2.799,5.620)--(2.800,5.571)--(2.801,5.521)--(2.802,5.472)--(2.803,5.447)%
  --(2.805,5.398)--(2.806,5.373)--(2.807,5.349)--(2.809,5.299)--(2.810,5.275)--(2.811,5.275)%
  --(2.813,5.250)--(2.814,5.250)--(2.815,5.275)--(2.817,5.275)--(2.818,5.275)--(2.819,5.299)%
  --(2.820,5.324)--(2.821,5.349)--(2.822,5.398)--(2.823,5.423)--(2.824,5.447)--(2.825,5.472)%
  --(2.827,5.521)--(2.828,5.571)--(2.829,5.595)--(2.830,5.644)--(2.831,5.694)--(2.833,5.718)%
  --(2.834,5.768)--(2.835,5.792)--(2.836,5.842)--(2.838,5.866)--(2.839,5.891)--(2.840,5.891)%
  --(2.842,5.916)--(2.843,5.916)--(2.844,5.891)--(2.846,5.891)--(2.847,5.891)--(2.848,5.866)%
  --(2.849,5.842)--(2.850,5.817)--(2.851,5.792)--(2.852,5.768)--(2.853,5.718)--(2.855,5.669)%
  --(2.856,5.644)--(2.857,5.620)--(2.858,5.571)--(2.859,5.521)--(2.860,5.497)--(2.862,5.447)%
  --(2.863,5.398)--(2.864,5.373)--(2.865,5.324)--(2.867,5.299)--(2.868,5.299)--(2.869,5.275)%
  --(2.871,5.250)--(2.872,5.275)--(2.873,5.275)--(2.875,5.275)--(2.876,5.299)--(2.877,5.324)%
  --(2.878,5.324)--(2.879,5.349)--(2.880,5.373)--(2.881,5.423)--(2.883,5.447)--(2.884,5.472)%
  --(2.885,5.546)--(2.886,5.571)--(2.887,5.620)--(2.888,5.644)--(2.889,5.694)--(2.891,5.718)%
  --(2.892,5.768)--(2.893,5.792)--(2.895,5.669)--(2.896,5.669)--(2.897,5.916)--(2.899,5.916)%
  --(2.900,5.916)--(2.901,5.916)--(2.903,5.916)--(2.904,5.916)--(2.905,5.916)--(2.906,5.916)%
  --(2.907,5.916)--(2.908,5.916)--(2.909,5.916)--(2.911,5.916)--(2.912,5.916)--(2.913,5.916)%
  --(2.914,5.669)--(2.915,5.423)--(2.916,5.669)--(2.917,5.423)--(2.919,5.669)--(2.920,5.423)%
  --(2.921,5.176)--(2.922,5.423)--(2.924,5.176)--(2.925,5.423)--(2.926,5.423)--(2.928,5.176)%
  --(2.929,5.423)--(2.930,5.176)--(2.932,5.176)--(2.933,5.176)--(2.934,5.423)--(2.935,5.176)%
  --(2.936,5.423)--(2.938,5.423)--(2.939,5.423)--(2.940,5.423)--(2.941,5.423)--(2.942,5.423)%
  --(2.943,5.423)--(2.944,5.423)--(2.945,5.669)--(2.946,5.669)--(2.948,5.669)--(2.949,5.669)%
  --(2.950,5.669)--(2.951,5.669)--(2.953,5.669)--(2.954,5.669)--(2.955,5.916)--(2.957,5.916)%
  --(2.958,5.916)--(2.959,5.916)--(2.961,5.916)--(2.962,5.916)--(2.963,5.916)--(2.964,5.916)%
  --(2.966,5.916)--(2.967,5.916)--(2.968,5.669)--(2.969,5.916)--(2.970,5.669)--(2.971,5.669)%
  --(2.972,5.669)--(2.973,5.669)--(2.974,5.423)--(2.976,5.423)--(2.977,5.669)--(2.978,5.423)%
  --(2.979,5.423)--(2.981,5.423)--(2.982,5.176)--(2.983,5.423)--(2.985,5.176)--(2.986,5.423)%
  --(2.987,5.176)--(2.989,5.423)--(2.990,5.176)--(2.991,5.423)--(2.992,5.176)--(2.994,5.176)%
  --(2.995,5.423)--(2.996,5.423)--(2.997,5.423)--(2.998,5.423)--(2.999,5.423)--(3.000,5.423)%
  --(3.001,5.669)--(3.002,5.669)--(3.004,5.423)--(3.005,5.669)--(3.006,5.669)--(3.007,5.916)%
  --(3.008,5.669)--(3.010,5.669)--(3.011,5.916)--(3.012,5.916)--(3.014,5.916)--(3.015,5.916)%
  --(3.016,5.916)--(3.018,5.916)--(3.019,5.669)--(3.020,5.916)--(3.022,5.916)--(3.023,5.916)%
  --(3.024,5.916)--(3.025,5.916)--(3.026,5.916)--(3.027,5.669)--(3.028,5.669)--(3.029,5.669)%
  --(3.030,5.669)--(3.032,5.669)--(3.033,5.669)--(3.034,5.669)--(3.035,5.423)--(3.036,5.423)%
  --(3.038,5.423)--(3.039,5.423)--(3.040,5.423)--(3.042,5.423)--(3.043,5.176)--(3.044,5.176)%
  --(3.046,5.423)--(3.047,5.176)--(3.048,5.176)--(3.050,5.423)--(3.051,5.423)--(3.052,5.423)%
  --(3.053,5.423)--(3.054,5.423)--(3.055,5.423)--(3.056,5.423)--(3.057,5.423)--(3.058,5.423)%
  --(3.060,5.669)--(3.061,5.423)--(3.062,5.669)--(3.063,5.669)--(3.064,5.669)--(3.065,5.669)%
  --(3.067,5.916)--(3.068,5.669)--(3.069,5.916)--(3.071,5.916)--(3.072,5.669)--(3.073,5.916)%
  --(3.075,5.916)--(3.076,5.916)--(3.077,5.916)--(3.079,5.916)--(3.080,5.916)--(3.081,5.669)%
  --(3.082,5.916)--(3.083,5.916)--(3.084,5.916)--(3.085,5.916)--(3.086,5.669)--(3.088,5.669)%
  --(3.089,5.669)--(3.090,5.423)--(3.091,5.669)--(3.092,5.669)--(3.093,5.423)--(3.095,5.423)%
  --(3.096,5.423)--(3.097,5.423)--(3.099,5.176)--(3.100,5.176)--(3.101,5.176)--(3.102,5.176)%
  --(3.104,5.176)--(3.105,5.176)--(3.106,5.423)--(3.108,5.423)--(3.109,5.176)--(3.110,5.423)%
  --(3.111,5.423)--(3.112,5.176)--(3.113,5.423)--(3.114,5.423)--(3.116,5.423)--(3.117,5.423)%
  --(3.118,5.669)--(3.119,5.669)--(3.120,5.669)--(3.121,5.669)--(3.122,5.669)--(3.124,5.669)%
  --(3.125,5.669)--(3.126,5.916)--(3.128,5.669)--(3.129,5.916)--(3.130,5.916)--(3.132,5.916)%
  --(3.133,5.916)--(3.134,5.916)--(3.136,5.916)--(3.137,5.916)--(3.138,5.669)--(3.139,5.669)%
  --(3.140,5.669)--(3.141,5.669)--(3.143,5.916)--(3.144,5.669)--(3.145,5.669)--(3.146,5.669)%
  --(3.147,5.423)--(3.148,5.669)--(3.149,5.669)--(3.150,5.423)--(3.152,5.423)--(3.153,5.423)%
  --(3.154,5.423)--(3.155,5.423)--(3.157,5.176)--(3.158,5.423)--(3.159,5.423)--(3.161,5.176)%
  --(3.162,5.176)--(3.163,5.176)--(3.165,5.176)--(3.166,5.176)--(3.167,5.176)--(3.168,5.423)%
  --(3.169,5.423)--(3.171,5.176)--(3.172,5.423)--(3.173,5.423)--(3.174,5.423)--(3.175,5.423)%
  --(3.176,5.423)--(3.177,5.669)--(3.178,5.669)--(3.179,5.669)--(3.181,5.669)--(3.182,5.669)%
  --(3.183,5.669)--(3.185,5.669)--(3.186,5.916)--(3.187,5.916)--(3.189,5.916)--(3.190,5.916)%
  --(3.191,5.916)--(3.193,5.916)--(3.194,5.916)--(3.195,5.916)--(3.196,5.916)--(3.197,5.916)%
  --(3.199,5.916)--(3.200,5.916)--(3.201,5.916)--(3.202,5.669)--(3.203,5.669)--(3.204,5.669)%
  --(3.205,5.669)--(3.206,5.423)--(3.208,5.423)--(3.209,5.423)--(3.210,5.669)--(3.211,5.423)%
  --(3.212,5.423)--(3.214,5.423)--(3.215,5.176)--(3.216,5.423)--(3.218,5.176)--(3.219,5.423)%
  --(3.220,5.176)--(3.222,5.176)--(3.223,5.423)--(3.224,5.176)--(3.225,5.176)--(3.227,5.176)%
  --(3.228,5.176)--(3.229,5.176)--(3.230,5.423)--(3.231,5.423)--(3.232,5.423)--(3.233,5.423)%
  --(3.234,5.669)--(3.235,5.423)--(3.237,5.669)--(3.238,5.669)--(3.239,5.669)--(3.240,5.669)%
  --(3.242,5.669)--(3.243,5.669)--(3.244,5.669)--(3.246,5.916)--(3.247,5.916)--(3.248,5.916)%
  --(3.250,5.916)--(3.251,5.916)--(3.252,5.916)--(3.253,5.916)--(3.255,5.916)--(3.256,5.916)%
  --(3.257,5.916)--(3.258,5.669)--(3.259,5.669)--(3.260,5.669)--(3.261,5.669)--(3.262,5.669)%
  --(3.263,5.669)--(3.265,5.669)--(3.266,5.669)--(3.267,5.423)--(3.268,5.423)--(3.269,5.423)%
  --(3.271,5.423)--(3.272,5.423)--(3.273,5.176)--(3.275,5.423)--(3.276,5.423)--(3.277,5.176)%
  --(3.279,5.423)--(3.280,5.176)--(3.281,5.176)--(3.283,5.423)--(3.284,5.176)--(3.285,5.176)%
  --(3.286,5.423)--(3.287,5.423)--(3.288,5.423)--(3.289,5.423)--(3.290,5.669)--(3.291,5.423)%
  --(3.293,5.423)--(3.294,5.423)--(3.295,5.669)--(3.296,5.669)--(3.297,5.669)--(3.299,5.669)%
  --(3.300,5.669)--(3.301,5.669)--(3.303,5.916)--(3.304,5.916)--(3.305,5.916)--(3.306,5.916)%
  --(3.308,5.916)--(3.309,5.916)--(3.310,5.916)--(3.312,5.669)--(3.313,5.916)--(3.314,5.916)%
  --(3.315,5.916)--(3.316,5.916)--(3.317,5.916)--(3.318,5.669)--(3.320,5.916)--(3.321,5.669)%
  --(3.322,5.669)--(3.323,5.669)--(3.324,5.423)--(3.325,5.669)--(3.326,5.423)--(3.328,5.423)%
  --(3.329,5.423)--(3.330,5.423)--(3.332,5.423)--(3.333,5.423)--(3.334,5.423)--(3.336,5.423)%
  --(3.337,5.176)--(3.338,5.176)--(3.340,5.176)--(3.341,5.423)--(3.342,5.423)--(3.343,5.176)%
  --(3.344,5.423)--(3.345,5.423)--(3.346,5.423)--(3.348,5.423)--(3.349,5.423)--(3.350,5.423)%
  --(3.351,5.669)--(3.352,5.669)--(3.353,5.669)--(3.354,5.669)--(3.356,5.669)--(3.357,5.669)%
  --(3.358,5.916)--(3.359,5.669)--(3.361,5.916)--(3.362,5.669)--(3.363,5.916)--(3.365,5.669)%
  --(3.366,5.916)--(3.367,5.916)--(3.369,5.916)--(3.370,5.916)--(3.371,5.916)--(3.372,5.916)%
  --(3.373,5.669)--(3.374,5.916)--(3.376,5.669)--(3.377,5.669)--(3.378,5.669)--(3.379,5.669)%
  --(3.380,5.669)--(3.381,5.669)--(3.382,5.669)--(3.383,5.669)--(3.385,5.669)--(3.386,5.423)%
  --(3.387,5.423)--(3.389,5.423)--(3.390,5.423)--(3.391,5.423)--(3.393,5.423)--(3.394,5.176)%
  --(3.395,5.176)--(3.397,5.176)--(3.398,5.176)--(3.399,5.176)--(3.400,5.176)--(3.401,5.176)%
  --(3.402,5.176)--(3.404,5.176)--(3.405,5.423)--(3.406,5.423)--(3.407,5.423)--(3.408,5.423)%
  --(3.409,5.669)--(3.410,5.423)--(3.411,5.669)--(3.412,5.669)--(3.414,5.669)--(3.415,5.916)%
  --(3.416,5.669)--(3.418,5.669)--(3.419,5.916)--(3.420,5.916)--(3.422,5.669)--(3.423,5.916)%
  --(3.424,5.916)--(3.426,5.916)--(3.427,5.916)--(3.428,5.916)--(3.429,5.916)--(3.431,5.916)%
  --(3.432,5.916)--(3.433,5.916)--(3.434,5.916)--(3.435,5.916)--(3.436,5.669)--(3.437,5.669)%
  --(3.438,5.669)--(3.439,5.669)--(3.440,5.669)--(3.442,5.423)--(3.443,5.423)--(3.444,5.423)%
  --(3.446,5.423)--(3.447,5.423)--(3.448,5.423)--(3.449,5.176)--(3.451,5.423)--(3.452,5.176)%
  --(3.453,5.176)--(3.455,5.423)--(3.456,5.423)--(3.457,5.176)--(3.459,5.423)--(3.460,5.423)%
  --(3.461,5.176)--(3.462,5.423)--(3.463,5.423)--(3.464,5.423)--(3.465,5.423)--(3.466,5.423)%
  --(3.467,5.669)--(3.468,5.669)--(3.470,5.423)--(3.471,5.669)--(3.472,5.669)--(3.473,5.669)%
  --(3.475,5.916)--(3.476,5.916)--(3.477,5.916)--(3.479,5.916)--(3.480,5.916)--(3.481,5.916)%
  --(3.483,5.916)--(3.484,5.916)--(3.485,5.916)--(3.487,5.916)--(3.488,5.916)--(3.489,5.669)%
  --(3.490,5.669)--(3.491,5.669)--(3.492,5.916)--(3.493,5.669)--(3.494,5.916)--(3.495,5.669)%
  --(3.496,5.669)--(3.498,5.669)--(3.499,5.669)--(3.500,5.423)--(3.501,5.423)--(3.503,5.423)%
  --(3.504,5.176)--(3.505,5.423)--(3.507,5.176)--(3.508,5.423)--(3.509,5.176)--(3.510,5.176)%
  --(3.512,5.176)--(3.513,5.176)--(3.514,5.423)--(3.516,5.176)--(3.517,5.423)--(3.518,5.423)%
  --(3.519,5.176)--(3.520,5.423)--(3.521,5.423)--(3.522,5.423)--(3.523,5.669)--(3.524,5.669)%
  --(3.526,5.669)--(3.527,5.669)--(3.528,5.669)--(3.529,5.669)--(3.530,5.669)--(3.532,5.669)%
  --(3.533,5.669)--(3.534,5.669)--(3.536,5.669)--(3.537,5.916)--(3.538,5.916)--(3.540,5.916)%
  --(3.541,5.916)--(3.542,5.916)--(3.543,5.916)--(3.545,5.916)--(3.546,5.916)--(3.547,5.669)%
  --(3.548,5.916)--(3.549,5.669)--(3.550,5.916)--(3.552,5.916)--(3.553,5.669)--(3.554,5.669)%
  --(3.555,5.669)--(3.556,5.423)--(3.557,5.669)--(3.558,5.669)--(3.560,5.423)--(3.561,5.423)%
  --(3.562,5.423)--(3.563,5.423)--(3.565,5.176)--(3.566,5.176)--(3.567,5.176)--(3.569,5.423)%
  --(3.570,5.423)--(3.571,5.176)--(3.573,5.176)--(3.574,5.423)--(3.575,5.423)--(3.576,5.423)%
  --(3.577,5.423)--(3.578,5.423)--(3.580,5.423)--(3.581,5.423)--(3.582,5.669)--(3.583,5.423)%
  --(3.584,5.669)--(3.585,5.669)--(3.586,5.669)--(3.587,5.669)--(3.589,5.669)--(3.590,5.669)%
  --(3.591,5.669)--(3.592,5.669)--(3.594,5.916)--(3.595,5.669)--(3.597,5.916)--(3.598,5.916)%
  --(3.599,5.916)--(3.601,5.916)--(3.602,5.669)--(3.603,5.916)--(3.604,5.916)--(3.605,5.669)%
  --(3.606,5.916)--(3.608,5.916)--(3.609,5.916)--(3.610,5.669)--(3.611,5.669)--(3.612,5.669)%
  --(3.613,5.669)--(3.614,5.669)--(3.615,5.669)--(3.616,5.669)--(3.618,5.423)--(3.619,5.423)%
  --(3.620,5.423)--(3.622,5.176)--(3.623,5.176)--(3.624,5.176)--(3.626,5.176)--(3.627,5.176)%
  --(3.628,5.176)--(3.630,5.176)--(3.631,5.176)--(3.632,5.176)--(3.633,5.423)--(3.634,5.176)%
  --(3.636,5.423)--(3.637,5.423)--(3.638,5.423)--(3.639,5.423)--(3.640,5.669)--(3.641,5.423)%
  --(3.642,5.423)--(3.643,5.669)--(3.644,5.423)--(3.646,5.669)--(3.647,5.916)--(3.648,5.669)%
  --(3.650,5.669)--(3.651,5.916)--(3.652,5.669)--(3.653,5.916)--(3.655,5.916)--(3.656,5.916)%
  --(3.657,5.916)--(3.659,5.916)--(3.660,5.916)--(3.661,5.916)--(3.662,5.916)--(3.663,5.916)%
  --(3.665,5.916)--(3.666,5.669)--(3.667,5.669)--(3.668,5.669)--(3.669,5.669)--(3.670,5.669)%
  --(3.671,5.669)--(3.672,5.669)--(3.674,5.423)--(3.675,5.423)--(3.676,5.423)--(3.677,5.423)%
  --(3.679,5.423)--(3.680,5.423)--(3.681,5.176)--(3.683,5.423)--(3.684,5.423)--(3.685,5.176)%
  --(3.686,5.423)--(3.688,5.423)--(3.689,5.176)--(3.690,5.176)--(3.691,5.176)--(3.693,5.423)%
  --(3.694,5.423)--(3.695,5.423)--(3.696,5.423)--(3.697,5.423)--(3.698,5.423)--(3.699,5.423)%
  --(3.700,5.423)--(3.702,5.669)--(3.703,5.669)--(3.704,5.669)--(3.705,5.916)--(3.706,5.669)%
  --(3.708,5.916)--(3.709,5.669)--(3.710,5.669)--(3.712,5.669)--(3.713,5.916)--(3.714,5.916)%
  --(3.716,5.916)--(3.717,5.916)--(3.718,5.916)--(3.719,5.916)--(3.721,5.916)--(3.722,5.916)%
  --(3.723,5.916)--(3.724,5.669)--(3.725,5.669)--(3.726,5.669)--(3.727,5.916)--(3.728,5.669)%
  --(3.730,5.669)--(3.731,5.423)--(3.732,5.669)--(3.733,5.423)--(3.734,5.423)--(3.736,5.423)%
  --(3.737,5.423)--(3.738,5.423)--(3.739,5.176)--(3.741,5.423)--(3.742,5.176)--(3.743,5.423)%
  --(3.745,5.176)--(3.746,5.176)--(3.747,5.176)--(3.749,5.423)--(3.750,5.176)--(3.751,5.176)%
  --(3.752,5.176)--(3.753,5.176)--(3.754,5.423)--(3.755,5.423)--(3.756,5.423)--(3.758,5.423)%
  --(3.759,5.669)--(3.760,5.669)--(3.761,5.669)--(3.762,5.669)--(3.763,5.669)--(3.765,5.669)%
  --(3.766,5.916)--(3.767,5.916)--(3.769,5.916)--(3.770,5.916)--(3.771,5.916)--(3.773,5.916)%
  --(3.774,5.916)--(3.775,5.916)--(3.777,5.916)--(3.778,5.916)--(3.779,5.916)--(3.780,5.916)%
  --(3.781,5.916)--(3.782,5.916)--(3.783,5.916)--(3.784,5.669)--(3.786,5.669)--(3.787,5.669)%
  --(3.788,5.669)--(3.789,5.669)--(3.790,5.423)--(3.791,5.423)--(3.793,5.669)--(3.794,5.423)%
  --(3.795,5.423)--(3.796,5.176)--(3.798,5.176)--(3.799,5.423)--(3.800,5.423)--(3.802,5.176)%
  --(3.803,5.176)--(3.804,5.176)--(3.806,5.423)--(3.807,5.176)--(3.808,5.423)--(3.809,5.176)%
  --(3.810,5.423)--(3.811,5.423)--(3.812,5.423)--(3.813,5.423)--(3.815,5.669)--(3.816,5.423)%
  --(3.817,5.669)--(3.818,5.669)--(3.819,5.669)--(3.820,5.669)--(3.822,5.669)--(3.823,5.916)%
  --(3.824,5.916)--(3.826,5.916)--(3.827,5.916)--(3.828,5.916)--(3.829,5.916)--(3.831,5.916)%
  --(3.832,5.916)--(3.833,5.916)--(3.835,5.916)--(3.836,5.916)--(3.837,5.916)--(3.838,5.916)%
  --(3.839,5.916)--(3.840,5.916)--(3.841,5.669)--(3.843,5.916)--(3.844,5.669)--(3.845,5.669)%
  --(3.846,5.669)--(3.847,5.669)--(3.848,5.669)--(3.849,5.423)--(3.851,5.423)--(3.852,5.423)%
  --(3.853,5.176)--(3.855,5.423)--(3.856,5.423)--(3.857,5.423)--(3.859,5.423)--(3.860,5.423)%
  --(3.861,5.423)--(3.862,5.423)--(3.864,5.423)--(3.865,5.423)--(3.866,5.176)--(3.867,5.176)%
  --(3.868,5.423)--(3.869,5.423)--(3.871,5.423)--(3.872,5.423)--(3.873,5.423)--(3.874,5.423)%
  --(3.875,5.423)--(3.876,5.423)--(3.877,5.669)--(3.878,5.669)--(3.880,5.669)--(3.881,5.669)%
  --(3.882,5.916)--(3.884,5.916)--(3.885,5.916)--(3.886,5.916)--(3.888,5.916)--(3.889,5.916)%
  --(3.890,5.916)--(3.892,5.916)--(3.893,5.916)--(3.894,5.916)--(3.895,5.916)--(3.896,5.916)%
  --(3.897,5.669)--(3.899,5.916)--(3.900,5.916)--(3.901,5.669)--(3.902,5.669)--(3.903,5.669)%
  --(3.904,5.669)--(3.905,5.669)--(3.906,5.423)--(3.908,5.669)--(3.909,5.423)--(3.910,5.423)%
  --(3.911,5.423)--(3.913,5.176)--(3.914,5.423)--(3.915,5.423)--(3.917,5.176)--(3.918,5.176)%
  --(3.919,5.176)--(3.921,5.423)--(3.922,5.423)--(3.923,5.423)--(3.924,5.423)--(3.925,5.423)%
  --(3.927,5.423)--(3.928,5.423)--(3.929,5.423)--(3.930,5.423)--(3.931,5.423)--(3.932,5.423)%
  --(3.933,5.423)--(3.934,5.669)--(3.935,5.669)--(3.937,5.669)--(3.938,5.916)--(3.939,5.669)%
  --(3.941,5.669)--(3.942,5.916)--(3.943,5.916)--(3.944,5.916)--(3.946,5.916)--(3.947,5.916)%
  --(3.948,5.916)--(3.950,5.916)--(3.951,5.916)--(3.952,5.916)--(3.954,5.916)--(3.955,5.669)%
  --(3.956,5.916)--(3.957,5.916)--(3.958,5.916)--(3.959,5.669)--(3.960,5.669)--(3.961,5.669)%
  --(3.962,5.669)--(3.963,5.669)--(3.965,5.423)--(3.966,5.423)--(3.967,5.423)--(3.969,5.423)%
  --(3.970,5.423)--(3.971,5.423)--(3.972,5.176)--(3.974,5.423)--(3.975,5.176)--(3.976,5.176)%
  --(3.978,5.176)--(3.979,5.176)--(3.980,5.423)--(3.982,5.176)--(3.983,5.176)--(3.984,5.423)%
  --(3.985,5.176)--(3.986,5.423)--(3.987,5.423)--(3.988,5.423)--(3.989,5.423)--(3.990,5.423)%
  --(3.991,5.669)--(3.993,5.669)--(3.994,5.669)--(3.995,5.669)--(3.996,5.669)--(3.998,5.916)%
  --(3.999,5.669)--(4.000,5.916)--(4.001,5.916)--(4.003,5.916)--(4.004,5.916)--(4.005,5.916)%
  --(4.007,5.916)--(4.008,5.916)--(4.009,5.916)--(4.011,5.669)--(4.012,5.916)--(4.013,5.916)%
  --(4.014,5.916)--(4.015,5.916)--(4.016,5.669)--(4.017,5.669)--(4.018,5.669)--(4.019,5.669)%
  --(4.021,5.423)--(4.022,5.423)--(4.023,5.423)--(4.024,5.669)--(4.025,5.423)--(4.027,5.423)%
  --(4.028,5.423)--(4.029,5.423)--(4.031,5.423)--(4.032,5.176)--(4.033,5.176)--(4.035,5.423)%
  --(4.036,5.176)--(4.037,5.176)--(4.038,5.176)--(4.040,5.176)--(4.041,5.423)--(4.042,5.423)%
  --(4.043,5.176)--(4.044,5.423)--(4.045,5.423)--(4.046,5.423)--(4.047,5.669)--(4.049,5.669)%
  --(4.050,5.669)--(4.051,5.423)--(4.052,5.669)--(4.053,5.669)--(4.054,5.669)--(4.056,5.669)%
  --(4.057,5.916)--(4.058,5.916)--(4.060,5.916)--(4.061,5.916)--(4.062,5.916)--(4.064,5.669)%
  --(4.065,5.916)--(4.066,5.916)--(4.068,5.916)--(4.069,5.916)--(4.070,5.916)--(4.071,5.916)%
  --(4.072,5.916)--(4.073,5.669)--(4.074,5.669)--(4.075,5.669)--(4.077,5.916)--(4.078,5.669)%
  --(4.079,5.669)--(4.080,5.669)--(4.081,5.423)--(4.082,5.669)--(4.084,5.423)--(4.085,5.423)%
  --(4.086,5.423)--(4.087,5.176)--(4.089,5.176)--(4.090,5.176)--(4.091,5.176)--(4.093,5.423)%
  --(4.094,5.423)--(4.095,5.176)--(4.097,5.176)--(4.098,5.176)--(4.099,5.176)--(4.100,5.176)%
  --(4.101,5.176)--(4.102,5.176)--(4.103,5.423)--(4.105,5.423)--(4.106,5.423)--(4.107,5.423)%
  --(4.108,5.669)--(4.109,5.669)--(4.110,5.669)--(4.111,5.669)--(4.113,5.669)--(4.114,5.669)%
  --(4.115,5.669)--(4.117,5.916)--(4.118,5.916)--(4.119,5.916)--(4.120,5.916)--(4.122,5.916)%
  --(4.123,5.916)--(4.124,5.916)--(4.126,5.916)--(4.127,5.916)--(4.128,5.916)--(4.129,5.916)%
  --(4.130,5.669)--(4.132,5.916)--(4.133,5.669)--(4.134,5.669)--(4.135,5.669)--(4.136,5.669)%
  --(4.137,5.669)--(4.138,5.669)--(4.139,5.423)--(4.141,5.423)--(4.142,5.669)--(4.143,5.423)%
  --(4.144,5.423)--(4.146,5.176)--(4.147,5.176)--(4.148,5.423)--(4.150,5.423)--(4.151,5.423)%
  --(4.152,5.423)--(4.153,5.176)--(4.155,5.176)--(4.156,5.176)--(4.157,5.176)--(4.158,5.423)%
  --(4.159,5.176)--(4.161,5.423)--(4.162,5.423)--(4.163,5.423)--(4.164,5.423)--(4.165,5.423)%
  --(4.166,5.669)--(4.167,5.669)--(4.168,5.423)--(4.170,5.669)--(4.171,5.669)--(4.172,5.916)%
  --(4.173,5.916)--(4.175,5.669)--(4.176,5.916)--(4.177,5.669)--(4.179,5.916)--(4.180,5.916)%
  --(4.181,5.916)--(4.183,5.916)--(4.184,5.916)--(4.185,5.916)--(4.186,5.669)--(4.188,5.669)%
  --(4.189,5.669)--(4.190,5.669)--(4.191,5.916)--(4.192,5.669)--(4.193,5.669)--(4.194,5.669)%
  --(4.195,5.669)--(4.196,5.669)--(4.198,5.423)--(4.199,5.423)--(4.200,5.423)--(4.201,5.423)%
  --(4.203,5.176)--(4.204,5.176)--(4.205,5.423)--(4.207,5.423)--(4.208,5.176)--(4.209,5.423)%
  --(4.210,5.176)--(4.212,5.176)--(4.213,5.423)--(4.214,5.423)--(4.215,5.423)--(4.217,5.176)%
  --(4.218,5.423)--(4.219,5.423)--(4.220,5.423)--(4.221,5.423)--(4.222,5.423)--(4.223,5.669)%
  --(4.224,5.423)--(4.225,5.423)--(4.227,5.669)--(4.228,5.669)--(4.229,5.669)--(4.230,5.669)%
  --(4.232,5.669)--(4.233,5.916)--(4.234,5.916)--(4.236,5.669)--(4.237,5.916)--(4.238,5.916)%
  --(4.240,5.916)--(4.241,5.916)--(4.242,5.916)--(4.244,5.916)--(4.245,5.916)--(4.246,5.916)%
  --(4.247,5.669)--(4.248,5.916)--(4.249,5.916)--(4.250,5.669)--(4.251,5.669)--(4.252,5.669)%
  --(4.253,5.669)--(4.255,5.423)--(4.256,5.669)--(4.257,5.423)--(4.258,5.423)--(4.260,5.423)%
  --(4.261,5.423)--(4.262,5.423)--(4.263,5.423)--(4.265,5.176)--(4.266,5.423)--(4.267,5.176)%
  --(4.269,5.176)--(4.270,5.423)--(4.271,5.176)--(4.273,5.423)--(4.274,5.423)--(4.275,5.423)%
  --(4.276,5.423)--(4.277,5.423)--(4.278,5.423)--(4.279,5.423)--(4.280,5.669)--(4.281,5.423)%
  --(4.282,5.423)--(4.284,5.423)--(4.285,5.423)--(4.286,5.669)--(4.287,5.669)--(4.289,5.669)%
  --(4.290,5.916)--(4.291,5.916)--(4.293,5.916)--(4.294,5.669)--(4.295,5.916)--(4.296,5.916)%
  --(4.298,5.916)--(4.299,5.916)--(4.300,5.916)--(4.302,5.916)--(4.303,5.916)--(4.304,5.916)%
  --(4.305,5.669)--(4.306,5.916)--(4.307,5.669)--(4.308,5.669)--(4.309,5.669)--(4.311,5.669)%
  --(4.312,5.669)--(4.313,5.423)--(4.314,5.423)--(4.315,5.669)--(4.316,5.423)--(4.318,5.423)%
  --(4.319,5.423)--(4.320,5.423)--(4.322,5.176)--(4.323,5.176)--(4.324,5.176)--(4.326,5.176)%
  --(4.327,5.176)--(4.328,5.423)--(4.330,5.423)--(4.331,5.176)--(4.332,5.423)--(4.333,5.176)%
  --(4.334,5.423)--(4.335,5.423)--(4.336,5.423)--(4.337,5.423)--(4.338,5.423)--(4.340,5.669)%
  --(4.341,5.423)--(4.342,5.423)--(4.343,5.669)--(4.344,5.669)--(4.345,5.669)--(4.347,5.669)%
  --(4.348,5.669)--(4.349,5.669)--(4.351,5.669)--(4.352,5.916)--(4.353,5.916)--(4.355,5.916)%
  --(4.356,5.916)--(4.357,5.916)--(4.359,5.916)--(4.360,5.916)--(4.361,5.916)--(4.362,5.916)%
  --(4.363,5.916)--(4.364,5.916)--(4.366,5.669)--(4.367,5.916)--(4.368,5.669)--(4.369,5.669)%
  --(4.370,5.423)--(4.371,5.423)--(4.372,5.423)--(4.373,5.423)--(4.375,5.669)--(4.376,5.423)%
  --(4.377,5.176)--(4.379,5.423)--(4.380,5.423)--(4.381,5.176)--(4.382,5.176)--(4.384,5.423)%
  --(4.385,5.423)--(4.386,5.176)--(4.388,5.176)--(4.389,5.176)--(4.390,5.176)--(4.391,5.423)%
  --(4.392,5.176)--(4.393,5.423)--(4.394,5.176)--(4.396,5.423)--(4.397,5.423)--(4.398,5.423)%
  --(4.399,5.669)--(4.400,5.423)--(4.401,5.669)--(4.402,5.669)--(4.404,5.669)--(4.405,5.669)%
  --(4.406,5.916)--(4.408,5.916)--(4.409,5.916)--(4.410,5.916)--(4.412,5.916)--(4.413,5.916)%
  --(4.414,5.916)--(4.416,5.916)--(4.417,5.916)--(4.418,5.916)--(4.419,5.916)--(4.420,5.916)%
  --(4.422,5.669)--(4.423,5.916)--(4.424,5.669)--(4.425,5.669)--(4.426,5.669)--(4.427,5.669)%
  --(4.428,5.669)--(4.429,5.669)--(4.431,5.423)--(4.432,5.423)--(4.433,5.423)--(4.434,5.423)%
  --(4.436,5.423)--(4.437,5.423)--(4.438,5.176)--(4.439,5.423)--(4.441,5.176)--(4.442,5.176)%
  --(4.443,5.176)--(4.445,5.176)--(4.446,5.176)--(4.447,5.176)--(4.448,5.176)--(4.449,5.176)%
  --(4.451,5.176)--(4.452,5.423)--(4.453,5.423)--(4.454,5.423)--(4.455,5.423)--(4.456,5.423)%
  --(4.457,5.423)--(4.458,5.669)--(4.459,5.669)--(4.461,5.669)--(4.462,5.669)--(4.463,5.916)%
  --(4.464,5.669)--(4.466,5.916)--(4.467,5.916)--(4.468,5.916)--(4.470,5.916)--(4.471,5.916)%
  --(4.473,5.916)--(4.474,5.916)--(4.475,5.916)--(4.476,5.916)--(4.478,5.916)--(4.479,5.916)%
  --(4.480,5.669)--(4.481,5.916)--(4.482,5.669)--(4.483,5.669)--(4.484,5.669)--(4.485,5.669)%
  --(4.486,5.669)--(4.488,5.423)--(4.489,5.669)--(4.490,5.423)--(4.491,5.423)--(4.492,5.423)%
  --(4.494,5.423)--(4.495,5.176)--(4.496,5.176)--(4.498,5.176)--(4.499,5.176)--(4.500,5.176)%
  --(4.502,5.176)--(4.503,5.423)--(4.504,5.423)--(4.505,5.423)--(4.507,5.423)--(4.508,5.176)%
  --(4.509,5.423)--(4.510,5.423)--(4.511,5.423)--(4.512,5.423)--(4.513,5.423)--(4.514,5.669)%
  --(4.515,5.423)--(4.516,5.423)--(4.518,5.669)--(4.519,5.669)--(4.520,5.669)--(4.521,5.916)%
  --(4.523,5.669)--(4.524,5.916)--(4.525,5.669)--(4.527,5.916)--(4.528,5.916)--(4.529,5.916)%
  --(4.531,5.916)--(4.532,5.916)--(4.533,5.916)--(4.535,5.916)--(4.536,5.916)--(4.537,5.669)%
  --(4.538,5.669)--(4.539,5.669)--(4.540,5.916)--(4.541,5.669)--(4.542,5.916)--(4.544,5.669)%
  --(4.545,5.669)--(4.546,5.669)--(4.547,5.423)--(4.548,5.423)--(4.549,5.423)--(4.551,5.423)%
  --(4.552,5.423)--(4.553,5.423)--(4.555,5.176)--(4.556,5.423)--(4.557,5.423)--(4.558,5.176)%
  --(4.560,5.176)--(4.561,5.423)--(4.562,5.423)--(4.564,5.176)--(4.565,5.423)--(4.566,5.423)%
  --(4.567,5.176)--(4.568,5.423)--(4.569,5.423)--(4.570,5.423)--(4.571,5.669)--(4.572,5.423)%
  --(4.574,5.669)--(4.575,5.423)--(4.576,5.669)--(4.577,5.423)--(4.578,5.916)--(4.580,5.916)%
  --(4.581,5.669)--(4.582,5.669)--(4.584,5.916)--(4.585,5.916)--(4.586,5.916)--(4.588,5.916)%
  --(4.589,5.916)--(4.590,5.916)--(4.592,5.916)--(4.593,5.916)--(4.594,5.916)--(4.595,5.916)%
  --(4.596,5.669)--(4.597,5.916)--(4.598,5.669)--(4.600,5.669)--(4.601,5.669)--(4.602,5.669)%
  --(4.603,5.423)--(4.603,5.669)--(4.604,5.669)--(4.605,5.669)--(4.605,5.423)--(4.605,5.669)%
  --(4.606,5.423)--(4.607,5.423)--(4.609,5.423)--(4.610,5.423)--(4.612,5.423)--(4.613,5.176)%
  --(4.614,5.176)--(4.614,5.423)--(4.614,5.176)--(4.614,5.423)--(4.615,5.423)--(4.616,5.423)%
  --(4.617,5.423)--(4.618,5.669)--(4.619,5.669)--(4.620,5.669)--(4.621,5.669)--(4.622,5.669)%
  --(4.622,5.916)--(4.623,5.916)--(4.624,5.916)--(4.625,5.916)--(4.626,6.162)--(4.627,6.409)%
  --(4.628,6.409)--(4.629,6.409)--(4.630,6.409)--(4.631,6.655)--(4.632,6.655)--(4.633,6.902)%
  --(4.635,6.902)--(4.636,6.902)--(4.637,6.902)--(4.638,7.148)--(4.639,7.148)--(4.640,6.902)%
  --(4.641,6.902)--(4.642,7.148)--(4.643,7.148)--(4.644,7.148)--(4.645,7.148)--(4.646,6.902)%
  --(4.647,6.902)--(4.647,7.148)--(4.648,6.902)--(4.649,6.902)--(4.650,6.655)--(4.651,6.655)%
  --(4.652,6.655)--(4.653,6.409)--(4.654,6.409)--(4.655,6.409)--(4.656,6.409)--(4.657,6.409)%
  --(4.658,6.162)--(4.659,5.916)--(4.660,5.916)--(4.661,5.669)--(4.662,5.669)--(4.663,5.669)%
  --(4.665,5.669)--(4.666,5.669)--(4.667,5.423)--(4.668,5.423)--(4.669,5.423)--(4.670,5.423)%
  --(4.671,5.423)--(4.672,5.176)--(4.673,5.423)--(4.674,5.423)--(4.675,5.423)--(4.677,5.423)%
  --(4.677,5.669)--(4.678,5.669)--(4.679,5.669)--(4.680,5.916)--(4.681,5.669)--(4.682,5.916)%
  --(4.683,5.916)--(4.684,5.916)--(4.685,6.162)--(4.686,6.162)--(4.687,6.162)--(4.688,6.409)%
  --(4.689,6.409)--(4.690,6.655)--(4.691,6.902)--(4.692,6.655)--(4.693,6.902)--(4.694,6.902)%
  --(4.695,7.148)--(4.696,7.148)--(4.697,6.902)--(4.698,7.148)--(4.699,7.148)--(4.700,7.148)%
  --(4.701,6.902)--(4.702,7.148)--(4.703,7.148)--(4.704,7.148)--(4.705,6.902)--(4.706,6.902)%
  --(4.707,6.902)--(4.708,6.902)--(4.709,6.655)--(4.710,6.655)--(4.711,6.655)--(4.712,6.409)%
  --(4.713,6.409)--(4.714,6.162)--(4.715,6.162)--(4.716,6.162)--(4.717,6.162)--(4.718,5.916)%
  --(4.719,5.916)--(4.720,5.669)--(4.722,5.669)--(4.723,5.423)--(4.724,5.669)--(4.725,5.423)%
  --(4.726,5.423)--(4.727,5.176)--(4.729,5.423)--(4.730,5.423)--(4.731,5.423)--(4.732,5.176)%
  --(4.732,5.423)--(4.733,5.423)--(4.734,5.423)--(4.735,5.669)--(4.736,5.669)--(4.737,5.669)%
  --(4.738,5.669)--(4.739,5.669)--(4.740,5.916)--(4.741,5.916)--(4.742,5.916)--(4.743,6.162)%
  --(4.745,6.162)--(4.745,6.409)--(4.746,6.409)--(4.748,6.655)--(4.749,6.655)--(4.750,6.655)%
  --(4.751,6.902)--(4.753,6.902)--(4.754,7.148)--(4.755,7.148)--(4.757,7.148)--(4.758,7.148)%
  --(4.759,7.148)--(4.760,7.148)--(4.761,7.148)--(4.762,6.902)--(4.763,6.902)--(4.764,7.148)%
  --(4.764,6.902)--(4.765,6.902)--(4.766,6.902)--(4.767,6.902)--(4.768,6.655)--(4.769,6.655)%
  --(4.770,6.409)--(4.771,6.409)--(4.772,6.409)--(4.773,6.162)--(4.774,6.162)--(4.775,6.162)%
  --(4.776,5.916)--(4.777,5.916)--(4.778,5.916)--(4.779,5.669)--(4.781,5.669)--(4.782,5.423)%
  --(4.783,5.423)--(4.785,5.423)--(4.786,5.423)--(4.787,5.176)--(4.788,5.423)--(4.789,5.176)%
  --(4.790,5.423)--(4.791,5.423)--(4.792,5.423)--(4.793,5.669)--(4.794,5.423)--(4.795,5.669)%
  --(4.796,5.916)--(4.797,5.916)--(4.798,5.916)--(4.799,5.916)--(4.800,6.162)--(4.801,6.162)%
  --(4.802,6.409)--(4.803,6.409)--(4.804,6.409)--(4.805,6.409)--(4.806,6.409)--(4.807,6.655)%
  --(4.808,6.655)--(4.809,6.902)--(4.811,6.902)--(4.812,6.902)--(4.813,6.902)--(4.814,7.148)%
  --(4.815,7.148)--(4.816,6.902)--(4.817,7.148)--(4.818,7.148)--(4.819,6.902)--(4.820,7.148)%
  --(4.821,6.902)--(4.822,6.902)--(4.823,6.902)--(4.824,6.902)--(4.825,6.902)--(4.826,6.655)%
  --(4.827,6.655)--(4.828,6.655)--(4.829,6.409)--(4.830,6.162)--(4.831,6.162)--(4.832,6.162)%
  --(4.833,5.916)--(4.834,6.162)--(4.835,5.916)--(4.837,5.916)--(4.838,5.669)--(4.839,5.669)%
  --(4.840,5.669)--(4.841,5.423)--(4.842,5.423)--(4.843,5.423)--(4.844,5.176)--(4.845,5.423)%
  --(4.846,5.423)--(4.847,5.423)--(4.848,5.176)--(4.849,5.423)--(4.850,5.423)--(4.851,5.423)%
  --(4.852,5.423)--(4.853,5.669)--(4.854,5.669)--(4.855,5.916)--(4.856,5.916)--(4.857,5.916)%
  --(4.858,5.916)--(4.859,6.162)--(4.860,6.162)--(4.861,6.409)--(4.862,6.409)--(4.863,6.655)%
  --(4.864,6.655)--(4.865,6.655)--(4.866,6.902)--(4.868,6.902)--(4.869,6.902)--(4.870,6.902)%
  --(4.871,6.902)--(4.872,7.148)--(4.873,6.902)--(4.874,7.148)--(4.875,7.148)--(4.876,7.148)%
  --(4.877,7.148)--(4.878,7.148)--(4.879,6.902)--(4.881,7.148)--(4.881,6.902)--(4.882,6.655)%
  --(4.883,6.902)--(4.884,6.655)--(4.885,6.655)--(4.886,6.409)--(4.887,6.655)--(4.888,6.409)%
  --(4.889,6.162)--(4.890,6.162)--(4.891,6.162)--(4.892,6.162)--(4.893,5.916)--(4.894,5.916)%
  --(4.895,5.669)--(4.896,5.669)--(4.897,5.669)--(4.898,5.669)--(4.899,5.423)--(4.901,5.423)%
  --(4.902,5.423)--(4.903,5.423)--(4.904,5.423)--(4.905,5.423)--(4.906,5.423)--(4.907,5.423)%
  --(4.908,5.423)--(4.909,5.669)--(4.910,5.669)--(4.910,5.423)--(4.911,5.669)--(4.912,5.669)%
  --(4.913,5.916)--(4.914,5.669)--(4.915,5.916)--(4.916,5.916)--(4.917,6.162)--(4.918,6.409)%
  --(4.919,6.162)--(4.920,6.409)--(4.921,6.409)--(4.922,6.409)--(4.923,6.655)--(4.924,6.902)%
  --(4.926,6.655)--(4.927,6.902)--(4.928,6.902)--(4.929,7.148)--(4.931,6.902)--(4.932,7.148)%
  --(4.933,7.148)--(4.934,7.148)--(4.935,7.148)--(4.936,7.148)--(4.937,6.902)--(4.938,6.902)%
  --(4.939,6.902)--(4.940,6.902)--(4.941,6.902)--(4.942,6.655)--(4.943,6.655)--(4.944,6.655)%
  --(4.945,6.409)--(4.946,6.409)--(4.947,6.162)--(4.948,6.162)--(4.949,6.162)--(4.950,6.162)%
  --(4.951,5.916)--(4.952,5.916)--(4.953,5.669)--(4.955,5.669)--(4.956,5.423)--(4.957,5.423)%
  --(4.958,5.669)--(4.959,5.423)--(4.960,5.423)--(4.961,5.423)--(4.962,5.423)--(4.963,5.423)%
  --(4.964,5.423)--(4.965,5.423)--(4.966,5.423)--(4.967,5.423)--(4.968,5.423)--(4.968,5.669)%
  --(4.969,5.669)--(4.970,5.669)--(4.971,5.916)--(4.972,5.669)--(4.973,5.916)--(4.974,5.916)%
  --(4.975,6.162)--(4.976,6.162)--(4.977,6.162)--(4.978,6.162)--(4.979,6.409)--(4.980,6.409)%
  --(4.981,6.655)--(4.982,6.655)--(4.983,6.655)--(4.984,6.902)--(4.985,7.148)--(4.987,6.902)%
  --(4.987,7.148)--(4.989,7.148)--(4.990,7.148)--(4.992,7.148)--(4.993,7.148)--(4.994,7.148)%
  --(4.995,7.148)--(4.996,7.148)--(4.997,6.902)--(4.998,6.902)--(4.999,6.655)--(5.000,6.655)%
  --(5.001,6.655)--(5.002,6.655)--(5.003,6.655)--(5.003,6.409)--(5.004,6.409)--(5.005,6.162)%
  --(5.006,6.162)--(5.007,6.162)--(5.008,6.162)--(5.010,5.916)--(5.011,5.669)--(5.012,5.669)%
  --(5.013,5.669)--(5.014,5.669)--(5.015,5.423)--(5.016,5.423)--(5.017,5.423)--(5.018,5.423)%
  --(5.020,5.423)--(5.021,5.423)--(5.022,5.423)--(5.023,5.423)--(5.024,5.423)--(5.025,5.423)%
  --(5.026,5.669)--(5.027,5.669)--(5.028,5.669)--(5.029,5.916)--(5.030,5.669)--(5.031,5.916)%
  --(5.032,5.916)--(5.033,6.162)--(5.034,6.162)--(5.035,6.162)--(5.036,6.409)--(5.037,6.409)%
  --(5.038,6.409)--(5.039,6.655)--(5.040,6.655)--(5.041,6.902)--(5.043,6.902)--(5.044,6.902)%
  --(5.045,6.902)--(5.047,6.902)--(5.048,7.148)--(5.049,7.148)--(5.050,7.148)--(5.051,7.148)%
  --(5.052,7.148)--(5.053,7.148)--(5.054,7.148)--(5.055,7.148)--(5.056,6.902)--(5.057,6.902)%
  --(5.058,6.902)--(5.059,6.655)--(5.060,6.655)--(5.061,6.409)--(5.062,6.409)--(5.063,6.409)%
  --(5.064,6.409)--(5.065,6.162)--(5.066,5.916)--(5.067,5.916)--(5.068,5.669)--(5.069,5.916)%
  --(5.070,5.669)--(5.071,5.669)--(5.073,5.423)--(5.074,5.669)--(5.075,5.423)--(5.076,5.423)%
  --(5.077,5.423)--(5.078,5.176)--(5.079,5.176)--(5.080,5.423)--(5.081,5.423)--(5.082,5.423)%
  --(5.083,5.669)--(5.084,5.423)--(5.085,5.423)--(5.086,5.669)--(5.087,5.669)--(5.088,5.916)%
  --(5.089,5.916)--(5.090,5.916)--(5.091,6.162)--(5.092,6.162)--(5.093,6.162)--(5.094,6.162)%
  --(5.095,6.409)--(5.096,6.409)--(5.097,6.655)--(5.099,6.655)--(5.100,6.902)--(5.101,6.902)%
  --(5.102,7.148)--(5.103,7.148)--(5.104,7.148)--(5.105,6.902)--(5.106,6.902)--(5.107,7.148)%
  --(5.108,7.148)--(5.109,7.148)--(5.110,7.148)--(5.111,6.902)--(5.112,7.148)--(5.113,6.902)%
  --(5.114,6.902)--(5.115,6.902)--(5.116,6.902)--(5.116,6.655)--(5.117,6.655)--(5.118,6.655)%
  --(5.119,6.655)--(5.120,6.409)--(5.121,6.409)--(5.122,6.162)--(5.123,6.162)--(5.124,6.162)%
  --(5.125,6.162)--(5.126,5.916)--(5.127,5.916)--(5.128,5.669)--(5.130,5.669)--(5.131,5.423)%
  --(5.132,5.669)--(5.133,5.423)--(5.134,5.423)--(5.135,5.423)--(5.136,5.423)--(5.137,5.423)%
  --(5.138,5.423)--(5.139,5.423)--(5.140,5.423)--(5.141,5.423)--(5.142,5.423)--(5.142,5.669)%
  --(5.143,5.669)--(5.144,5.669)--(5.145,5.669)--(5.146,5.916)--(5.147,5.916)--(5.148,5.916)%
  --(5.149,5.916)--(5.150,6.162)--(5.151,6.162)--(5.152,6.162)--(5.153,6.409)--(5.154,6.409)%
  --(5.155,6.655)--(5.156,6.655)--(5.157,6.655)--(5.158,6.655)--(5.159,6.902)--(5.160,6.902)%
  --(5.161,6.902)--(5.163,7.148)--(5.164,7.148)--(5.165,7.148)--(5.166,7.148)--(5.167,7.148)%
  --(5.168,7.148)--(5.169,6.902)--(5.170,6.902)--(5.171,7.148)--(5.172,7.148)--(5.172,6.902)%
  --(5.173,6.902)--(5.174,6.902)--(5.175,6.655)--(5.176,6.655)--(5.177,6.655)--(5.178,6.409)%
  --(5.179,6.409)--(5.180,6.162)--(5.181,6.162)--(5.182,6.162)--(5.183,5.916)--(5.184,5.916)%
  --(5.185,5.916)--(5.186,5.916)--(5.188,5.669)--(5.189,5.669)--(5.190,5.423)--(5.191,5.423)%
  --(5.193,5.423)--(5.194,5.176)--(5.195,5.176)--(5.196,5.176)--(5.197,5.423)--(5.198,5.423)%
  --(5.199,5.423)--(5.200,5.423)--(5.201,5.669)--(5.201,5.423)--(5.202,5.669)--(5.203,5.669)%
  --(5.204,5.916)--(5.205,5.669)--(5.206,5.916)--(5.207,6.162)--(5.208,6.162)--(5.209,6.162)%
  --(5.210,6.162)--(5.211,6.409)--(5.212,6.409)--(5.213,6.655)--(5.214,6.655)--(5.215,6.655)%
  --(5.217,6.655)--(5.218,6.902)--(5.219,6.902)--(5.220,7.148)--(5.221,7.148)--(5.222,6.902)%
  --(5.223,7.148)--(5.224,7.148)--(5.225,7.148)--(5.226,7.148)--(5.227,7.148)--(5.228,6.902)%
  --(5.229,6.902)--(5.230,6.902)--(5.231,6.902)--(5.232,6.655)--(5.233,6.655)--(5.234,6.655)%
  --(5.235,6.655)--(5.236,6.655)--(5.237,6.409)--(5.238,6.409)--(5.239,6.162)--(5.240,6.162)%
  --(5.241,6.162)--(5.242,5.916)--(5.243,5.916)--(5.244,5.916)--(5.245,5.669)--(5.247,5.669)%
  --(5.248,5.423)--(5.248,5.669)--(5.250,5.423)--(5.251,5.423)--(5.252,5.423)--(5.253,5.423)%
  --(5.254,5.423)--(5.255,5.176)--(5.255,5.423)--(5.256,5.423)--(5.257,5.423)--(5.258,5.669)%
  --(5.259,5.669)--(5.260,5.669)--(5.261,5.669)--(5.262,5.669)--(5.263,5.669)--(5.264,5.916)%
  --(5.265,5.916)--(5.266,6.162)--(5.268,6.162)--(5.268,6.409)--(5.269,6.409)--(5.271,6.409)%
  --(5.272,6.655)--(5.273,6.655)--(5.274,6.655)--(5.275,6.902)--(5.276,6.902)--(5.277,6.902)%
  --(5.278,6.902)--(5.280,7.148)--(5.281,7.148)--(5.282,6.902)--(5.283,7.148)--(5.284,6.902)%
  --(5.285,7.148)--(5.286,7.148)--(5.287,6.902)--(5.288,7.148)--(5.289,6.902)--(5.290,6.902)%
  --(5.290,6.655)--(5.291,6.655)--(5.292,6.655)--(5.293,6.409)--(5.294,6.409)--(5.295,6.409)%
  --(5.296,6.162)--(5.297,6.162)--(5.298,6.162)--(5.299,6.162)--(5.300,5.916)--(5.301,5.916)%
  --(5.302,5.669)--(5.303,5.916)--(5.304,5.669)--(5.306,5.669)--(5.307,5.423)--(5.308,5.423)%
  --(5.309,5.423)--(5.310,5.423)--(5.311,5.176)--(5.312,5.423)--(5.313,5.423)--(5.314,5.423)%
  --(5.315,5.669)--(5.316,5.423)--(5.317,5.669)--(5.318,5.423)--(5.319,5.669)--(5.320,5.916)%
  --(5.320,5.669)--(5.321,5.916)--(5.322,5.916)--(5.323,5.916)--(5.324,6.162)--(5.325,6.162)%
  --(5.326,6.409)--(5.327,6.409)--(5.328,6.409)--(5.329,6.409)--(5.330,6.655)--(5.331,6.655)%
  --(5.333,6.902)--(5.334,6.902)--(5.335,7.148)--(5.336,7.148)--(5.337,6.902)--(5.338,6.902)%
  --(5.339,7.148)--(5.340,7.148)--(5.341,7.148)--(5.342,7.148)--(5.343,7.148)--(5.344,6.902)%
  --(5.345,6.902)--(5.346,7.148)--(5.347,6.902)--(5.348,6.902)--(5.349,6.902)--(5.350,6.655)%
  --(5.351,6.655)--(5.352,6.655)--(5.353,6.409)--(5.354,6.162)--(5.355,6.162)--(5.356,6.162)%
  --(5.357,5.916)--(5.358,5.916)--(5.359,5.916)--(5.360,5.916)--(5.361,5.669)--(5.363,5.669)%
  --(5.364,5.669)--(5.365,5.423)--(5.366,5.423)--(5.367,5.423)--(5.368,5.423)--(5.369,5.423)%
  --(5.370,5.176)--(5.371,5.423)--(5.372,5.423)--(5.373,5.423)--(5.374,5.423)--(5.375,5.423)%
  --(5.376,5.423)--(5.377,5.669)--(5.378,5.669)--(5.379,5.916)--(5.380,5.916)--(5.381,5.916)%
  --(5.382,6.162)--(5.383,6.162)--(5.384,6.162)--(5.385,6.409)--(5.386,6.409)--(5.387,6.409)%
  --(5.388,6.409)--(5.389,6.655)--(5.390,6.655)--(5.391,6.902)--(5.392,6.902)--(5.393,6.902)%
  --(5.394,7.148)--(5.395,6.902)--(5.397,7.148)--(5.398,7.148)--(5.399,6.902)--(5.400,7.148)%
  --(5.401,7.148)--(5.402,6.902)--(5.403,6.902)--(5.404,7.148)--(5.405,6.902)--(5.406,6.902)%
  --(5.407,6.902)--(5.408,6.655)--(5.409,6.655)--(5.410,6.655)--(5.411,6.409)--(5.412,6.409)%
  --(5.413,6.409)--(5.414,6.162)--(5.415,6.162)--(5.416,5.916)--(5.417,5.916)--(5.418,5.916)%
  --(5.419,5.916)--(5.421,5.669)--(5.422,5.423)--(5.423,5.423)--(5.425,5.423)--(5.426,5.423)%
  --(5.427,5.423)--(5.428,5.423)--(5.429,5.176)--(5.430,5.423)--(5.431,5.423)--(5.432,5.423)%
  --(5.433,5.669)--(5.434,5.423)--(5.435,5.669)--(5.436,5.669)--(5.437,5.669)--(5.438,5.916)%
  --(5.439,5.916)--(5.440,5.916)--(5.441,6.162)--(5.442,6.162)--(5.443,6.409)--(5.444,6.409)%
  --(5.445,6.655)--(5.446,6.409)--(5.447,6.655)--(5.448,6.902)--(5.450,6.655)--(5.451,6.902)%
  --(5.452,7.148)--(5.454,7.148)--(5.455,7.148)--(5.456,6.902)--(5.457,7.148)--(5.458,6.902)%
  --(5.459,7.148)--(5.460,7.148)--(5.461,7.148)--(5.462,6.902)--(5.463,6.902)--(5.464,6.902)%
  --(5.465,6.902)--(5.466,6.902)--(5.467,6.655)--(5.468,6.655)--(5.469,6.409)--(5.470,6.409)%
  --(5.471,6.409)--(5.472,6.162)--(5.473,6.162)--(5.474,6.162)--(5.475,5.916)--(5.476,5.669)%
  --(5.477,5.669)--(5.479,5.669)--(5.480,5.423)--(5.481,5.423)--(5.483,5.423)--(5.484,5.423)%
  --(5.485,5.176)--(5.486,5.423)--(5.487,5.423)--(5.488,5.423)--(5.489,5.423)--(5.490,5.423)%
  --(5.491,5.423)--(5.492,5.669)--(5.493,5.669)--(5.494,5.669)--(5.495,5.916)--(5.496,5.916)%
  --(5.497,5.916)--(5.498,6.162)--(5.499,6.162)--(5.500,6.162)--(5.501,6.409)--(5.502,6.162)%
  --(5.503,6.409)--(5.504,6.409)--(5.505,6.655)--(5.507,6.902)--(5.508,6.655)--(5.509,6.902)%
  --(5.510,6.902)--(5.512,7.148)--(5.513,7.148)--(5.514,7.148)--(5.515,6.902)--(5.516,7.148)%
  --(5.517,7.148)--(5.518,7.148)--(5.519,6.902)--(5.520,7.148)--(5.521,6.902)--(5.522,6.902)%
  --(5.523,6.902)--(5.524,6.655)--(5.525,6.655)--(5.526,6.655)--(5.527,6.655)--(5.528,6.409)%
  --(5.529,6.409)--(5.530,6.162)--(5.531,6.162)--(5.532,5.916)--(5.533,5.916)--(5.534,5.916)%
  --(5.535,5.669)--(5.537,5.669)--(5.538,5.423)--(5.539,5.669)--(5.540,5.423)--(5.541,5.423)%
  --(5.542,5.423)--(5.543,5.423)--(5.544,5.423)--(5.545,5.423)--(5.546,5.423)--(5.547,5.423)%
  --(5.548,5.423)--(5.549,5.423)--(5.550,5.423)--(5.551,5.669)--(5.552,5.669)--(5.553,5.669)%
  --(5.554,5.916)--(5.555,5.916)--(5.556,6.162)--(5.557,6.162)--(5.558,6.162)--(5.559,6.162)%
  --(5.560,6.162)--(5.561,6.409)--(5.562,6.409)--(5.563,6.655)--(5.564,6.655)--(5.566,6.902)%
  --(5.567,6.902)--(5.568,6.902)--(5.569,6.902)--(5.570,7.148)--(5.571,6.902)--(5.572,7.148)%
  --(5.573,7.148)--(5.574,6.902)--(5.575,7.148)--(5.576,7.148)--(5.577,7.148)--(5.578,7.148)%
  --(5.578,6.902)--(5.579,6.902)--(5.580,6.902)--(5.581,6.902)--(5.582,6.655)--(5.583,6.655)%
  --(5.584,6.409)--(5.585,6.409)--(5.586,6.409)--(5.587,6.162)--(5.588,6.162)--(5.589,6.162)%
  --(5.590,5.916)--(5.591,5.916)--(5.592,5.916)--(5.593,5.916)--(5.594,5.916)--(5.595,5.669)%
  --(5.596,5.669)--(5.597,5.423)--(5.599,5.423)--(5.600,5.423)--(5.601,5.423)--(5.602,5.423)%
  --(5.603,5.176)--(5.604,5.423)--(5.605,5.423)--(5.606,5.423)--(5.607,5.423)--(5.608,5.669)%
  --(5.609,5.669)--(5.610,5.423)--(5.611,5.669)--(5.612,5.669)--(5.613,5.916)--(5.614,5.916)%
  --(5.615,5.916)--(5.616,6.162)--(5.617,6.162)--(5.618,6.162)--(5.619,6.409)--(5.620,6.409)%
  --(5.621,6.409)--(5.622,6.655)--(5.623,6.902)--(5.625,6.902)--(5.626,6.902)--(5.627,6.902)%
  --(5.628,6.902)--(5.629,6.902)--(5.630,7.148)--(5.631,7.148)--(5.632,7.148)--(5.633,6.902)%
  --(5.634,7.148)--(5.635,6.902)--(5.636,7.148)--(5.637,6.902)--(5.638,6.902)--(5.639,6.902)%
  --(5.640,6.655)--(5.641,6.655)--(5.642,6.655)--(5.643,6.655)--(5.643,6.409)--(5.644,6.409)%
  --(5.645,6.409)--(5.646,6.409)--(5.647,6.162)--(5.648,6.162)--(5.649,6.162)--(5.650,5.916)%
  --(5.652,5.669)--(5.653,5.669)--(5.654,5.423)--(5.655,5.423)--(5.656,5.423)--(5.657,5.423)%
  --(5.658,5.423)--(5.659,5.176)--(5.660,5.423)--(5.661,5.423)--(5.662,5.423)--(5.663,5.423)%
  --(5.664,5.423)--(5.665,5.669)--(5.666,5.423)--(5.667,5.669)--(5.668,5.669)--(5.669,5.669)%
  --(5.670,5.916)--(5.671,5.916)--(5.672,5.916)--(5.673,5.916)--(5.674,6.162)--(5.675,6.162)%
  --(5.676,6.409)--(5.677,6.409)--(5.678,6.409)--(5.679,6.655)--(5.680,6.655)--(5.681,6.902)%
  --(5.683,6.902)--(5.684,6.902)--(5.685,7.148)--(5.686,6.902)--(5.687,6.902)--(5.688,7.148)%
  --(5.689,6.902)--(5.690,7.148)--(5.691,7.148)--(5.692,7.148)--(5.693,6.902)--(5.694,7.148)%
  --(5.695,6.902)--(5.696,6.902)--(5.697,6.902)--(5.698,6.902)--(5.699,6.902)--(5.700,6.655)%
  --(5.701,6.409)--(5.702,6.409)--(5.703,6.409)--(5.704,6.409)--(5.705,6.162)--(5.706,6.162)%
  --(5.707,5.916)--(5.708,5.916)--(5.709,5.916)--(5.710,5.669)--(5.711,5.669)--(5.712,5.669)%
  --(5.713,5.423)--(5.714,5.423)--(5.716,5.423)--(5.717,5.423)--(5.718,5.423)--(5.719,5.423)%
  --(5.720,5.176)--(5.721,5.423)--(5.722,5.423)--(5.723,5.423)--(5.724,5.423)--(5.725,5.669)%
  --(5.726,5.669)--(5.726,5.423)--(5.727,5.669)--(5.728,5.669)--(5.729,5.916)--(5.730,5.916)%
  --(5.731,6.162)--(5.732,6.162)--(5.733,6.162)--(5.734,6.162)--(5.735,6.409)--(5.736,6.409)%
  --(5.737,6.655)--(5.738,6.655)--(5.739,6.902)--(5.741,6.902)--(5.742,7.148)--(5.743,6.902)%
  --(5.744,6.902)--(5.746,7.148)--(5.747,7.148)--(5.748,6.902)--(5.749,7.148)--(5.750,7.148)%
  --(5.751,7.148)--(5.752,7.148)--(5.753,6.902)--(5.754,6.902)--(5.755,6.902)--(5.756,6.902)%
  --(5.756,6.655)--(5.757,6.655)--(5.758,6.655)--(5.759,6.655)--(5.760,6.655)--(5.761,6.162)%
  --(5.762,6.162)--(5.763,6.162)--(5.764,6.162)--(5.765,5.916)--(5.766,5.916)--(5.767,5.669)%
  --(5.768,5.916)--(5.770,5.669)--(5.771,5.423)--(5.772,5.423)--(5.773,5.423)--(5.774,5.423)%
  --(5.775,5.423)--(5.776,5.423)--(5.777,5.423)--(5.778,5.423)--(5.779,5.423)--(5.780,5.423)%
  --(5.781,5.423)--(5.782,5.423)--(5.783,5.423)--(5.784,5.669)--(5.785,5.669)--(5.786,5.669)%
  --(5.787,5.916)--(5.788,5.916)--(5.789,5.916)--(5.790,6.162)--(5.791,6.162)--(5.792,6.409)%
  --(5.793,6.409)--(5.794,6.409)--(5.795,6.655)--(5.796,6.655)--(5.797,6.655)--(5.798,6.655)%
  --(5.800,6.902)--(5.801,6.902)--(5.802,6.902)--(5.803,7.148)--(5.804,7.148)--(5.805,7.148)%
  --(5.806,7.148)--(5.807,7.148)--(5.808,6.902)--(5.809,7.148)--(5.810,7.148)--(5.811,6.902)%
  --(5.812,7.148)--(5.813,6.902)--(5.814,6.902)--(5.815,6.655)--(5.816,6.655)--(5.817,6.655)%
  --(5.818,6.655)--(5.819,6.655)--(5.820,6.409)--(5.820,6.162)--(5.822,6.162)--(5.822,5.916)%
  --(5.824,5.916)--(5.825,5.916)--(5.826,5.669)--(5.827,5.669)--(5.828,5.669)--(5.829,5.669)%
  --(5.830,5.669)--(5.831,5.423)--(5.833,5.423)--(5.834,5.423)--(5.835,5.176)--(5.836,5.423)%
  --(5.837,5.423)--(5.838,5.423)--(5.839,5.423)--(5.840,5.423)--(5.841,5.423)--(5.842,5.669)%
  --(5.843,5.423)--(5.844,5.669)--(5.845,5.669)--(5.845,5.916)--(5.846,5.916)--(5.847,5.916)%
  --(5.848,5.916)--(5.849,6.162)--(5.850,6.162)--(5.851,6.409)--(5.852,6.409)--(5.853,6.409)%
  --(5.854,6.655)--(5.855,6.655)--(5.857,6.655)--(5.858,6.902)--(5.859,6.902)--(5.860,6.902)%
  --(5.861,7.148)--(5.862,6.902)--(5.863,7.148)--(5.864,7.148)--(5.865,7.148)--(5.866,7.148)%
  --(5.867,7.148)--(5.868,7.148)--(5.869,6.902)--(5.870,7.148)--(5.871,6.902)--(5.872,6.902)%
  --(5.873,6.902)--(5.873,6.655)--(5.874,6.655)--(5.875,6.655)--(5.876,6.409)--(5.877,6.409)%
  --(5.878,6.409)--(5.879,6.162)--(5.880,6.162)--(5.881,5.916)--(5.882,5.916)--(5.883,5.916)%
  --(5.884,5.669)--(5.885,5.669)--(5.887,5.669)--(5.888,5.669)--(5.889,5.423)--(5.890,5.423)%
  --(5.891,5.423)--(5.892,5.423)--(5.893,5.423)--(5.894,5.176)--(5.895,5.423)--(5.896,5.423)%
  --(5.897,5.423)--(5.898,5.669)--(5.899,5.669)--(5.900,5.669)--(5.901,5.669)--(5.902,5.669)%
  --(5.903,5.669)--(5.904,5.916)--(5.905,5.916)--(5.906,5.916)--(5.907,6.162)--(5.908,6.162)%
  --(5.909,6.162)--(5.909,6.409)--(5.911,6.655)--(5.912,6.655)--(5.913,6.655)--(5.914,6.655)%
  --(5.915,6.902)--(5.916,6.902)--(5.917,7.148)--(5.918,6.902)--(5.920,6.902)--(5.921,7.148)%
  --(5.922,7.148)--(5.923,7.148)--(5.924,7.148)--(5.925,7.148)--(5.926,7.148)--(5.927,6.902)%
  --(5.928,7.148)--(5.928,6.902)--(5.929,6.902)--(5.930,6.655)--(5.931,6.655)--(5.932,6.902)%
  --(5.933,6.655)--(5.934,6.655)--(5.935,6.409)--(5.936,6.409)--(5.937,6.162)--(5.938,6.162)%
  --(5.939,5.916)--(5.940,5.916)--(5.941,5.916)--(5.942,5.916)--(5.943,5.669)--(5.945,5.669)%
  --(5.946,5.423)--(5.947,5.423)--(5.948,5.669)--(5.950,5.423)--(5.951,5.423)--(5.952,5.423)%
  --(5.953,5.423)--(5.954,5.423)--(5.955,5.423)--(5.956,5.423)--(5.957,5.423)--(5.958,5.669)%
  --(5.958,5.423)--(5.959,5.669)--(5.960,5.669)--(5.961,5.669)--(5.962,5.916)--(5.963,5.916)%
  --(5.964,5.916)--(5.965,6.162)--(5.966,6.162)--(5.967,6.162)--(5.968,6.409)--(5.969,6.409)%
  --(5.970,6.409)--(5.971,6.655)--(5.972,6.655)--(5.974,6.902)--(5.975,6.902)--(5.976,7.148)%
  --(5.977,6.902)--(5.979,7.148)--(5.980,7.148)--(5.981,7.148)--(5.982,7.148)--(5.983,7.148)%
  --(5.984,6.902)--(5.985,6.902)--(5.986,6.902)--(5.987,6.902)--(5.988,6.902)--(5.989,6.655)%
  --(5.990,6.655)--(5.991,6.655)--(5.992,6.655)--(5.993,6.409)--(5.994,6.409)--(5.995,6.162)%
  --(5.996,6.162)--(5.997,6.162)--(5.998,5.916)--(5.999,5.916)--(6.000,5.916)--(6.001,5.669)%
  --(6.003,5.669)--(6.004,5.669)--(6.005,5.423)--(6.007,5.423)--(6.008,5.423)--(6.009,5.423)%
  --(6.010,5.423)--(6.011,5.423)--(6.012,5.423)--(6.013,5.669)--(6.014,5.423)--(6.015,5.423)%
  --(6.016,5.423)--(6.017,5.423)--(6.018,5.669)--(6.019,5.916)--(6.020,5.916)--(6.021,5.916)%
  --(6.022,6.162)--(6.023,6.162)--(6.024,6.409)--(6.025,6.162)--(6.026,6.409)--(6.027,6.409)%
  --(6.028,6.409)--(6.029,6.655)--(6.031,6.902)--(6.032,6.902)--(6.033,6.902)--(6.035,6.902)%
  --(6.036,7.148)--(6.037,6.902)--(6.038,7.148)--(6.039,7.148)--(6.040,7.148)--(6.041,6.902)%
  --(6.042,7.148)--(6.043,6.902)--(6.044,7.148)--(6.045,6.902)--(6.046,6.902)--(6.047,6.655)%
  --(6.048,6.902)--(6.049,6.655)--(6.050,6.409)--(6.051,6.409)--(6.052,6.162)--(6.053,6.162)%
  --(6.054,6.162)--(6.055,6.162)--(6.056,6.162)--(6.057,5.916)--(6.058,5.916)--(6.060,5.916)%
  --(6.061,5.669)--(6.062,5.669)--(6.064,5.669)--(6.065,5.423)--(6.066,5.423)--(6.067,5.423)%
  --(6.068,5.423)--(6.069,5.423)--(6.070,5.176)--(6.071,5.423)--(6.072,5.423)--(6.073,5.423)%
  --(6.074,5.669)--(6.075,5.423)--(6.076,5.669)--(6.077,5.916)--(6.078,5.669)--(6.078,5.916)%
  --(6.080,5.916)--(6.080,6.162)--(6.081,6.162)--(6.082,6.162)--(6.083,6.162)--(6.084,6.409)%
  --(6.085,6.409)--(6.086,6.655)--(6.087,6.655)--(6.089,6.655)--(6.090,6.655)--(6.091,7.148)%
  --(6.093,6.902)--(6.094,6.902)--(6.095,6.902)--(6.096,7.148)--(6.098,7.148)--(6.099,7.148)%
  --(6.100,7.148)--(6.101,6.902)--(6.102,6.902)--(6.103,6.902)--(6.104,6.902)--(6.105,6.655)%
  --(6.106,6.655)--(6.107,6.655)--(6.108,6.655)--(6.109,6.655)--(6.110,6.655)--(6.110,6.409)%
  --(6.112,6.162)--(6.113,6.162)--(6.114,6.162)--(6.115,5.916)--(6.117,5.916)--(6.118,5.669)%
  --(6.119,5.669)--(6.121,5.423)--(6.122,5.423)--(6.123,5.423)--(6.124,5.423)--(6.126,5.423)%
  --(6.127,5.423)--(6.128,5.423)--(6.129,5.423)--(6.131,5.423)--(6.131,5.669)--(6.132,5.669)%
  --(6.133,5.423)--(6.134,5.423)--(6.135,5.916)--(6.136,5.916)--(6.137,5.916)--(6.138,5.916)%
  --(6.139,5.916)--(6.140,6.162)--(6.141,6.162)--(6.143,6.409)--(6.144,6.655)--(6.145,6.655)%
  --(6.146,6.655)--(6.147,6.655)--(6.148,6.902)--(6.150,6.902)--(6.151,6.902)--(6.152,6.902)%
  --(6.153,7.148)--(6.155,7.148)--(6.156,7.148)--(6.157,7.148)--(6.158,6.902)--(6.159,7.148)%
  --(6.160,7.148)--(6.160,6.902)--(6.161,7.148)--(6.162,6.902)--(6.163,6.655)--(6.164,6.902)%
  --(6.165,6.655)--(6.166,6.655)--(6.167,6.655)--(6.168,6.409)--(6.169,6.409)--(6.170,6.409)%
  --(6.171,6.162)--(6.172,6.162)--(6.173,5.916)--(6.174,5.916)--(6.175,5.916)--(6.176,5.669)%
  --(6.178,5.669)--(6.179,5.423)--(6.180,5.423)--(6.181,5.423)--(6.183,5.423)--(6.184,5.423)%
  --(6.185,5.423)--(6.186,5.423)--(6.187,5.423)--(6.188,5.423)--(6.189,5.423)--(6.190,5.423)%
  --(6.191,5.669)--(6.192,5.669)--(6.193,5.669)--(6.194,5.669)--(6.195,5.916)--(6.196,5.916)%
  --(6.197,6.162)--(6.198,6.162)--(6.199,6.162)--(6.200,6.409)--(6.201,6.162)--(6.202,6.409)%
  --(6.203,6.655)--(6.204,6.655)--(6.205,6.902)--(6.207,6.902)--(6.208,6.902)--(6.209,7.148)%
  --(6.211,6.902)--(6.212,7.148)--(6.213,7.148)--(6.214,7.148)--(6.215,7.148)--(6.216,7.148)%
  --(6.217,6.902)--(6.218,7.148)--(6.219,7.148)--(6.220,7.148)--(6.221,6.655)--(6.222,6.655)%
  --(6.223,6.655)--(6.224,6.655)--(6.225,6.655)--(6.226,6.409)--(6.227,6.162)--(6.228,6.409)%
  --(6.229,6.162)--(6.230,6.162)--(6.231,6.162)--(6.232,5.669)--(6.233,5.669)--(6.234,5.669)%
  --(6.236,5.669)--(6.237,5.423)--(6.239,5.423)--(6.240,5.423)--(6.241,5.423)--(6.242,5.423)%
  --(6.243,5.423)--(6.244,5.423)--(6.245,5.423)--(6.246,5.423)--(6.247,5.423)--(6.248,5.423)%
  --(6.249,5.669)--(6.250,5.669)--(6.251,5.669)--(6.252,5.916)--(6.253,5.916)--(6.254,5.916)%
  --(6.255,5.916)--(6.256,6.162)--(6.257,6.162)--(6.258,6.409)--(6.259,6.409)--(6.260,6.409)%
  --(6.261,6.655)--(6.262,6.655)--(6.264,6.902)--(6.265,6.902)--(6.267,7.148)--(6.268,7.148)%
  --(6.269,6.902)--(6.270,7.148)--(6.272,6.902)--(6.273,7.148)--(6.274,7.148)--(6.275,7.148)%
  --(6.276,7.148)--(6.277,6.902)--(6.278,6.902)--(6.279,6.902)--(6.280,6.902)--(6.281,6.655)%
  --(6.282,6.655)--(6.283,6.655)--(6.284,6.409)--(6.285,6.409)--(6.286,6.409)--(6.286,6.162)%
  --(6.288,6.162)--(6.289,6.162)--(6.289,5.916)--(6.291,5.916)--(6.292,5.916)--(6.293,5.669)%
  --(6.294,5.423)--(6.295,5.423)--(6.296,5.423)--(6.297,5.423)--(6.299,5.423)--(6.300,5.176)%
  --(6.301,5.423)--(6.302,5.423)--(6.303,5.423)--(6.304,5.423)--(6.305,5.423)--(6.306,5.669)%
  --(6.307,5.423)--(6.308,5.669)--(6.309,5.669)--(6.310,5.669)--(6.311,5.916)--(6.312,5.916)%
  --(6.313,5.916)--(6.314,6.162)--(6.315,6.162)--(6.316,6.409)--(6.317,6.409)--(6.318,6.409)%
  --(6.319,6.655)--(6.320,6.655)--(6.321,6.902)--(6.322,6.655)--(6.324,6.902)--(6.325,6.902)%
  --(6.326,7.148)--(6.327,7.148)--(6.328,7.148)--(6.329,7.148)--(6.330,7.148)--(6.331,6.902)%
  --(6.332,7.148)--(6.333,7.148)--(6.334,7.148)--(6.335,7.148)--(6.335,6.902)--(6.336,6.902)%
  --(6.337,6.902)--(6.338,6.655)--(6.339,6.655)--(6.340,6.655)--(6.341,6.655)--(6.342,6.655)%
  --(6.343,6.409)--(6.344,6.162)--(6.345,6.162)--(6.346,6.162)--(6.347,6.162)--(6.348,6.162)%
  --(6.349,5.916)--(6.350,5.916)--(6.351,5.669)--(6.352,5.669)--(6.353,5.669)--(6.354,5.423)%
  --(6.355,5.423)--(6.356,5.423)--(6.357,5.423)--(6.358,5.423)--(6.359,5.423)--(6.360,5.423)%
  --(6.361,5.423)--(6.362,5.423)--(6.363,5.423)--(6.364,5.423)--(6.365,5.669)--(6.366,5.423)%
  --(6.367,5.669)--(6.368,5.669)--(6.369,5.916)--(6.370,5.916)--(6.371,6.162)--(6.372,6.162)%
  --(6.373,6.162)--(6.374,6.162)--(6.375,6.409)--(6.376,6.409)--(6.377,6.655)--(6.378,6.655)%
  --(6.379,6.655)--(6.381,6.902)--(6.382,6.902)--(6.383,6.902)--(6.384,7.148)--(6.385,7.148)%
  --(6.386,6.902)--(6.387,7.148)--(6.389,7.148)--(6.391,7.148)--(6.392,7.148)--(6.393,6.902)%
  --(6.394,6.902)--(6.395,6.902)--(6.396,6.902)--(6.397,6.902)--(6.398,6.655)--(6.399,6.655)%
  --(6.400,6.409)--(6.401,6.409)--(6.402,6.162)--(6.403,6.162)--(6.404,6.162)--(6.405,5.916)%
  --(6.406,5.916)--(6.408,5.916)--(6.409,5.669)--(6.410,5.669)--(6.411,5.669)--(6.413,5.669)%
  --(6.414,5.423)--(6.415,5.423)--(6.416,5.423)--(6.418,5.423)--(6.420,5.423)--(6.421,5.669)%
  --(6.422,5.669)--(6.423,5.423)--(6.424,5.423)--(6.425,5.423)--(6.426,5.669)--(6.427,5.669)%
  --(6.428,5.669)--(6.429,5.916)--(6.430,6.162)--(6.431,5.916)--(6.431,6.162)--(6.432,6.409)%
  --(6.433,6.409)--(6.434,6.409)--(6.436,6.655)--(6.437,6.655)--(6.438,6.902)--(6.439,6.902)%
  --(6.441,6.902)--(6.442,6.902)--(6.443,7.148)--(6.444,7.148)--(6.445,6.902)--(6.447,7.148)%
  --(6.448,7.148)--(6.449,7.148)--(6.450,7.148)--(6.451,7.148)--(6.452,6.902)--(6.453,6.902)%
  --(6.454,6.902)--(6.455,6.655)--(6.456,6.655)--(6.457,6.655)--(6.458,6.409)--(6.459,6.409)%
  --(6.460,6.409)--(6.462,6.409)--(6.462,6.162)--(6.464,6.162)--(6.465,5.916)--(6.466,5.916)%
  --(6.467,5.669)--(6.468,5.669)--(6.470,5.669)--(6.471,5.423)--(6.472,5.423)--(6.473,5.423)%
  --(6.474,5.423)--(6.476,5.423)--(6.477,5.423)--(6.478,5.423)--(6.479,5.423)--(6.480,5.669)%
  --(6.481,5.423)--(6.482,5.669)--(6.483,5.669)--(6.484,5.669)--(6.485,5.669)--(6.486,5.916)%
  --(6.487,5.916)--(6.488,6.162)--(6.489,6.162)--(6.490,6.162)--(6.492,6.409)--(6.493,6.409)%
  --(6.494,6.409)--(6.495,6.409)--(6.496,6.655)--(6.497,6.902)--(6.499,6.902)--(6.500,7.148)%
  --(6.501,7.148)--(6.502,7.148)--(6.503,7.148)--(6.504,7.148)--(6.505,7.148)--(6.506,7.148)%
  --(6.507,7.148)--(6.508,6.902)--(6.509,7.148)--(6.510,6.902)--(6.511,6.902)--(6.512,6.902)%
  --(6.513,6.655)--(6.514,6.655)--(6.515,6.655)--(6.516,6.655)--(6.517,6.409)--(6.518,6.409)%
  --(6.520,6.409)--(6.520,6.162)--(6.521,6.162)--(6.523,5.916)--(6.524,5.916)--(6.525,5.669)%
  --(6.526,5.916)--(6.527,5.669)--(6.528,5.423)--(6.529,5.423)--(6.530,5.669)--(6.531,5.423)%
  --(6.532,5.423)--(6.533,5.423)--(6.534,5.423)--(6.535,5.423)--(6.536,5.423)--(6.537,5.423)%
  --(6.538,5.423)--(6.539,5.669)--(6.540,5.669)--(6.541,5.669)--(6.542,5.669)--(6.543,5.669)%
  --(6.544,5.916)--(6.545,5.916)--(6.546,5.916)--(6.547,6.162)--(6.548,6.162)--(6.549,6.162)%
  --(6.550,6.409)--(6.551,6.409)--(6.552,6.655)--(6.553,6.655)--(6.554,6.655)--(6.556,6.902)%
  --(6.557,6.902)--(6.558,6.902)--(6.559,7.148)--(6.560,7.148)--(6.561,7.148)--(6.562,6.902)%
  --(6.564,7.148)--(6.565,7.148)--(6.565,6.902)--(6.566,7.148)--(6.567,7.148)--(6.568,6.902)%
  --(6.569,6.902)--(6.570,6.902)--(6.571,6.902)--(6.572,6.902)--(6.573,6.655)--(6.574,6.655)%
  --(6.575,6.409)--(6.577,6.162)--(6.578,6.162)--(6.579,5.916)--(6.581,5.916)--(6.582,5.669)%
  --(6.582,5.916)--(6.584,5.669)--(6.585,5.423)--(6.587,5.669)--(6.588,5.423)--(6.589,5.423)%
  --(6.591,5.423)--(6.592,5.423)--(6.593,5.423)--(6.594,5.423)--(6.595,5.423)--(6.596,5.423)%
  --(6.597,5.423)--(6.598,5.669)--(6.598,5.423)--(6.599,5.669)--(6.600,5.916)--(6.601,5.669)%
  --(6.602,5.916)--(6.603,5.916)--(6.604,5.916)--(6.605,6.162)--(6.606,6.162)--(6.607,6.162)%
  --(6.608,6.409)--(6.609,6.655)--(6.610,6.655)--(6.611,6.655)--(6.612,6.902)--(6.613,6.902)%
  --(6.614,6.902)--(6.615,7.148)--(6.617,7.148)--(6.618,7.148)--(6.619,7.148)--(6.620,7.148)%
  --(6.621,7.148)--(6.622,7.148)--(6.623,7.148)--(6.624,6.902)--(6.625,6.902)--(6.626,6.902)%
  --(6.627,6.902)--(6.628,6.902)--(6.629,6.655)--(6.630,6.655)--(6.631,6.409)--(6.632,6.409)%
  --(6.633,6.655)--(6.634,6.409)--(6.635,6.162)--(6.636,6.409)--(6.637,6.162)--(6.638,5.916)%
  --(6.639,5.916)--(6.640,5.669)--(6.641,5.916)--(6.643,5.669)--(6.644,5.669)--(6.645,5.423)%
  --(6.646,5.669)--(6.647,5.423)--(6.648,5.423)--(6.649,5.423)--(6.650,5.423)--(6.651,5.423)%
  --(6.652,5.423)--(6.653,5.423)--(6.654,5.423)--(6.655,5.423)--(6.656,5.423)--(6.657,5.423)%
  --(6.658,5.669)--(6.659,5.669)--(6.660,5.916)--(6.661,5.916)--(6.662,5.916)--(6.663,6.162)%
  --(6.664,6.162)--(6.665,6.162)--(6.666,6.409)--(6.667,6.409)--(6.668,6.409)--(6.669,6.655)%
  --(6.670,6.902)--(6.671,6.655)--(6.673,6.902)--(6.675,6.902)--(6.676,7.148)--(6.678,7.148)%
  --(6.679,7.148)--(6.680,7.148)--(6.681,6.902)--(6.682,6.902)--(6.683,7.148)--(6.684,7.148)%
  --(6.685,6.902)--(6.686,6.902)--(6.687,6.902)--(6.688,6.902)--(6.689,6.655)--(6.690,6.655)%
  --(6.691,6.409)--(6.692,6.655)--(6.693,6.409)--(6.694,6.409)--(6.694,6.162)--(6.696,5.916)%
  --(6.697,5.916)--(6.698,5.916)--(6.699,5.916)--(6.700,5.916)--(6.701,5.669)--(6.702,5.423)%
  --(6.703,5.423)--(6.704,5.423)--(6.705,5.423)--(6.707,5.423)--(6.708,5.176)--(6.709,5.423)%
  --(6.710,5.423)--(6.711,5.423)--(6.712,5.423)--(6.713,5.423)--(6.714,5.423)--(6.715,5.669)%
  --(6.715,5.423)--(6.716,5.669)--(6.717,5.669)--(6.718,5.669)--(6.719,5.916)--(6.720,5.916)%
  --(6.721,5.916)--(6.722,6.162)--(6.723,6.162)--(6.724,6.162)--(6.725,6.409)--(6.726,6.409)%
  --(6.727,6.655)--(6.728,6.655)--(6.729,6.655)--(6.731,6.902)--(6.732,6.902)--(6.733,7.148)%
  --(6.734,7.148)--(6.736,7.148)--(6.737,7.148)--(6.738,7.148)--(6.739,7.148)--(6.740,7.148)%
  --(6.741,6.902)--(6.742,7.148)--(6.743,7.148)--(6.744,7.148)--(6.744,6.902)--(6.745,6.655)%
  --(6.746,6.655)--(6.747,6.655)--(6.748,6.655)--(6.749,6.409)--(6.750,6.409)--(6.751,6.409)%
  --(6.752,6.409)--(6.753,6.162)--(6.754,6.162)--(6.755,5.916)--(6.756,5.916)--(6.757,5.916)%
  --(6.758,5.669)--(6.760,5.423)--(6.761,5.669)--(6.762,5.423)--(6.764,5.423)--(6.765,5.423)%
  --(6.766,5.423)--(6.767,5.423)--(6.768,5.423)--(6.769,5.423)--(6.770,5.423)--(6.771,5.669)%
  --(6.772,5.423)--(6.773,5.423)--(6.774,5.423)--(6.775,5.669)--(6.776,5.669)--(6.777,5.669)%
  --(6.778,6.162)--(6.779,6.162)--(6.780,5.916)--(6.781,6.162)--(6.782,6.162)--(6.783,6.409)%
  --(6.784,6.409)--(6.785,6.409)--(6.786,6.902)--(6.788,6.655)--(6.789,6.902)--(6.790,6.902)%
  --(6.792,6.902)--(6.793,7.148)--(6.794,7.148)--(6.795,7.148)--(6.797,7.148)--(6.798,7.148)%
  --(6.799,7.148)--(6.800,7.148)--(6.801,6.902)--(6.802,7.148)--(6.803,6.902)--(6.804,6.655)%
  --(6.805,6.902)--(6.806,6.655)--(6.807,6.409)--(6.807,6.655)--(6.809,6.655)--(6.809,6.162)%
  --(6.810,6.162)--(6.811,6.162)--(6.812,5.916)--(6.813,5.916)--(6.814,5.916)--(6.816,5.916)%
  --(6.817,5.669)--(6.818,5.669)--(6.819,5.423)--(6.820,5.423)--(6.821,5.423)--(6.822,5.423)%
  --(6.824,5.423)--(6.825,5.423)--(6.826,5.176)--(6.827,5.423)--(6.828,5.423)--(6.829,5.423)%
  --(6.830,5.423)--(6.830,5.669)--(6.831,5.423)--(6.832,5.669)--(6.833,5.669)--(6.834,5.916)%
  --(6.835,5.916)--(6.836,5.669)--(6.837,5.916)--(6.838,6.162)--(6.839,6.162)--(6.840,6.409)%
  --(6.841,6.409)--(6.842,6.655)--(6.843,6.409)--(6.844,6.409)--(6.845,6.655)--(6.846,6.902)%
  --(6.847,6.902)--(6.849,6.902)--(6.850,6.902)--(6.851,7.148)--(6.852,7.148)--(6.853,6.902)%
  --(6.854,7.148)--(6.855,7.148)--(6.856,6.902)--(6.856,7.148)--(6.857,7.148)--(6.858,7.148)%
  --(6.859,6.902)--(6.860,6.902)--(6.861,6.902)--(6.862,6.902)--(6.863,6.655)--(6.864,6.655)%
  --(6.865,6.655)--(6.866,6.409)--(6.867,6.409)--(6.868,6.162)--(6.869,6.162)--(6.870,6.162)%
  --(6.871,5.916)--(6.873,5.916)--(6.874,5.916)--(6.875,5.916)--(6.876,5.423)--(6.877,5.423)%
  --(6.878,5.423)--(6.879,5.423)--(6.880,5.423)--(6.881,5.423)--(6.882,5.176)--(6.883,5.423)%
  --(6.884,5.423)--(6.885,5.423)--(6.886,5.423)--(6.887,5.423)--(6.888,5.423)--(6.889,5.423)%
  --(6.890,5.669)--(6.891,5.669)--(6.892,5.669)--(6.893,5.916)--(6.894,5.916)--(6.895,6.162)%
  --(6.896,6.162)--(6.897,6.162)--(6.898,6.409)--(6.899,6.409)--(6.900,6.409)--(6.901,6.655)%
  --(6.902,6.655)--(6.903,6.655)--(6.904,6.902)--(6.906,6.902)--(6.907,6.902)--(6.908,6.902)%
  --(6.909,7.148)--(6.910,6.902)--(6.911,7.148)--(6.912,6.902)--(6.913,6.902)--(6.914,7.148)%
  --(6.915,7.148)--(6.916,7.148)--(6.917,6.902)--(6.918,6.902)--(6.919,6.902)--(6.920,6.902)%
  --(6.921,6.655)--(6.922,6.902)--(6.922,6.655)--(6.923,6.409)--(6.924,6.655)--(6.925,6.409)%
  --(6.926,6.409)--(6.927,6.162)--(6.928,5.916)--(6.929,6.162)--(6.930,5.916)--(6.931,5.916)%
  --(6.933,5.669)--(6.934,5.669)--(6.935,5.423)--(6.937,5.423)--(6.938,5.423)--(6.939,5.423)%
  --(6.940,5.176)--(6.942,5.423)--(6.943,5.423)--(6.944,5.423)--(6.945,5.423)--(6.946,5.423)%
  --(6.947,5.423)--(6.948,5.669)--(6.949,5.669)--(6.950,5.669)--(6.951,5.916)--(6.952,5.916)%
  --(6.952,6.162)--(6.954,5.916)--(6.954,6.162)--(6.955,6.409)--(6.956,6.162)--(6.957,6.409)%
  --(6.958,6.409)--(6.959,6.409)--(6.960,6.409)--(6.962,6.655)--(6.963,6.902)--(6.964,6.902)%
  --(6.966,6.902)--(6.967,7.148)--(6.968,7.148)--(6.969,7.148)--(6.969,6.902)--(6.970,7.148)%
  --(6.971,7.148)--(6.972,7.148)--(6.973,7.148)--(6.974,7.148)--(6.975,7.148)--(6.976,7.148)%
  --(6.977,6.902)--(6.978,6.902)--(6.979,6.902)--(6.980,6.902)--(6.981,6.655)--(6.982,6.655)%
  --(6.982,6.409)--(6.983,6.409)--(6.984,6.409)--(6.985,6.162)--(6.986,6.162)--(6.987,5.916)%
  --(6.988,5.916)--(6.989,5.669)--(6.991,5.669)--(6.992,5.916)--(6.993,5.669)--(6.994,5.669)%
  --(6.995,5.423)--(6.996,5.423)--(6.997,5.423)--(6.998,5.423)--(6.999,5.423)--(7.000,5.423)%
  --(7.001,5.423)--(7.002,5.423)--(7.003,5.423)--(7.004,5.423)--(7.005,5.423)--(7.006,5.669)%
  --(7.007,5.669)--(7.008,5.669)--(7.009,5.669)--(7.010,5.916)--(7.011,5.916)--(7.012,6.162)%
  --(7.013,6.162)--(7.014,6.162)--(7.015,6.409)--(7.016,6.409)--(7.017,6.409)--(7.018,6.655)%
  --(7.019,6.655)--(7.020,6.655)--(7.021,6.902)--(7.022,6.902)--(7.024,7.148)--(7.025,6.902)%
  --(7.026,7.148)--(7.027,7.148)--(7.028,7.148)--(7.029,6.902)--(7.030,7.148)--(7.031,7.148)%
  --(7.032,6.902)--(7.033,7.148)--(7.034,6.902)--(7.035,6.902)--(7.036,6.902)--(7.037,6.902)%
  --(7.038,6.655)--(7.039,6.655)--(7.039,6.409)--(7.041,6.655)--(7.041,6.409)--(7.042,6.162)%
  --(7.043,6.409)--(7.044,6.162)--(7.045,5.916)--(7.046,5.916)--(7.047,5.916)--(7.048,5.669)%
  --(7.050,5.669)--(7.051,5.669)--(7.052,5.669)--(7.053,5.423)--(7.054,5.423)--(7.055,5.423)%
  --(7.057,5.423)--(7.058,5.423)--(7.059,5.423)--(7.060,5.423)--(7.061,5.423)--(7.062,5.423)%
  --(7.063,5.423)--(7.064,5.423)--(7.065,5.669)--(7.066,5.669)--(7.067,5.669)--(7.068,5.669)%
  --(7.069,5.916)--(7.070,5.916)--(7.071,6.162)--(7.072,6.162)--(7.072,6.409)--(7.074,6.409)%
  --(7.075,6.409)--(7.076,6.655)--(7.077,6.655)--(7.078,6.902)--(7.079,6.655)--(7.080,6.902)%
  --(7.082,7.148)--(7.083,6.902)--(7.084,7.148)--(7.085,7.148)--(7.086,6.902)--(7.086,7.148)%
  --(7.088,7.148)--(7.090,6.902)--(7.091,6.902)--(7.092,6.902)--(7.093,6.902)--(7.094,6.902)%
  --(7.095,6.902)--(7.096,6.902)--(7.097,6.655)--(7.098,6.655)--(7.100,6.409)--(7.101,6.162)%
  --(7.102,6.162)--(7.103,6.162)--(7.104,5.916)--(7.105,5.916)--(7.107,5.669)--(7.108,5.669)%
  --(7.109,5.669)--(7.110,5.669)--(7.111,5.669)--(7.112,5.423)--(7.113,5.423)--(7.114,5.423)%
  --(7.115,5.423)--(7.116,5.176)--(7.117,5.423)--(7.118,5.423)--(7.119,5.423)--(7.120,5.423)%
  --(7.121,5.423)--(7.122,5.669)--(7.123,5.423)--(7.124,5.669)--(7.125,5.669)--(7.126,5.669)%
  --(7.127,5.916)--(7.128,6.162)--(7.129,6.162)--(7.130,6.162)--(7.131,6.162)--(7.132,6.409)%
  --(7.133,6.409)--(7.134,6.409)--(7.135,6.655)--(7.136,6.655)--(7.137,6.902)--(7.139,6.902)%
  --(7.140,6.902)--(7.141,7.148)--(7.142,7.148)--(7.143,7.148)--(7.144,6.902)--(7.145,7.148)%
  --(7.146,7.148)--(7.147,6.902)--(7.148,7.148)--(7.149,6.902)--(7.150,6.902)--(7.151,7.148)%
  --(7.152,6.902)--(7.153,6.902)--(7.154,6.655)--(7.155,6.655)--(7.156,6.655)--(7.156,6.409)%
  --(7.157,6.409)--(7.158,6.409)--(7.159,6.409)--(7.160,6.162)--(7.161,6.162)--(7.162,6.162)%
  --(7.163,5.916)--(7.164,5.916)--(7.166,5.669)--(7.167,5.669)--(7.168,5.423)--(7.170,5.669)%
  --(7.171,5.423)--(7.172,5.423)--(7.173,5.423)--(7.174,5.423)--(7.175,5.423)--(7.176,5.423)%
  --(7.177,5.423)--(7.178,5.423)--(7.179,5.669)--(7.180,5.669)--(7.181,5.669)--(7.182,5.669)%
  --(7.183,5.669)--(7.184,5.669)--(7.185,5.916)--(7.186,5.916)--(7.187,6.162)--(7.188,6.162)%
  --(7.189,6.162)--(7.190,6.409)--(7.191,6.409)--(7.192,6.655)--(7.193,6.655)--(7.195,6.655)%
  --(7.196,6.902)--(7.197,6.902)--(7.199,7.148)--(7.200,6.902)--(7.201,7.148)--(7.202,7.148)%
  --(7.203,7.148)--(7.205,6.902)--(7.206,6.902)--(7.207,6.902)--(7.208,6.902)--(7.209,7.148)%
  --(7.210,6.902)--(7.211,6.902)--(7.212,6.655)--(7.213,6.655)--(7.214,6.655)--(7.215,6.655)%
  --(7.216,6.409)--(7.217,6.409)--(7.218,6.162)--(7.219,6.162)--(7.220,6.162)--(7.221,5.916)%
  --(7.222,5.916)--(7.224,5.669)--(7.225,5.669)--(7.226,5.423)--(7.228,5.423)--(7.229,5.423)%
  --(7.230,5.423)--(7.231,5.423)--(7.233,5.176)--(7.234,5.423)--(7.235,5.423)--(7.236,5.423)%
  --(7.237,5.423)--(7.238,5.669)--(7.239,5.423)--(7.240,5.669)--(7.241,5.669)--(7.242,5.669)%
  --(7.243,5.916)--(7.244,5.916)--(7.245,6.162)--(7.245,5.916)--(7.247,6.162)--(7.248,6.409)%
  --(7.249,6.655)--(7.250,6.655)--(7.252,6.655)--(7.253,6.902)--(7.254,6.902)--(7.255,6.902)%
  --(7.257,6.902)--(7.258,7.148)--(7.259,7.148)--(7.260,6.902)--(7.262,7.148)--(7.264,7.148)%
  --(7.264,6.902)--(7.265,7.148)--(7.266,6.902)--(7.267,6.902)--(7.268,6.902)--(7.269,6.902)%
  --(7.270,6.655)--(7.271,6.655)--(7.272,6.655)--(7.273,6.655)--(7.274,6.409)--(7.275,6.409)%
  --(7.275,6.162)--(7.276,6.162)--(7.277,6.162)--(7.278,5.916)--(7.280,5.916)--(7.281,5.916)%
  --(7.282,5.916)--(7.283,5.669)--(7.285,5.669)--(7.286,5.669)--(7.287,5.423)--(7.288,5.423)%
  --(7.289,5.423)--(7.290,5.423)--(7.291,5.423)--(7.292,5.423)--(7.294,5.423)--(7.295,5.423)%
  --(7.296,5.423)--(7.297,5.669)--(7.298,5.669)--(7.299,5.669)--(7.300,5.916)--(7.301,5.916)%
  --(7.302,5.916)--(7.304,5.916)--(7.304,6.162)--(7.305,6.162)--(7.306,6.409)--(7.307,6.409)%
  --(7.308,6.409)--(7.309,6.655)--(7.311,6.655)--(7.312,6.902)--(7.313,6.902)--(7.314,7.148)%
  --(7.315,6.902)--(7.316,6.902)--(7.317,6.902)--(7.318,7.148)--(7.319,7.148)--(7.320,7.148)%
  --(7.321,7.148)--(7.322,7.148)--(7.323,7.148)--(7.324,6.902)--(7.325,7.148)--(7.326,6.902)%
  --(7.327,6.902)--(7.328,6.655)--(7.329,6.902)--(7.329,6.655)--(7.330,6.655)--(7.331,6.409)%
  --(7.332,6.655)--(7.333,6.409)--(7.334,6.409)--(7.335,6.162)--(7.336,6.162)--(7.337,6.162)%
  --(7.338,5.916)--(7.339,5.916)--(7.340,5.669)--(7.341,5.669)--(7.343,5.423)--(7.344,5.669)%
  --(7.345,5.423)--(7.346,5.423)--(7.347,5.423)--(7.348,5.423)--(7.350,5.423)--(7.351,5.176)%
  --(7.352,5.423)--(7.353,5.423)--(7.354,5.423)--(7.355,5.669)--(7.356,5.669)--(7.357,5.669)%
  --(7.358,5.669)--(7.359,5.916)--(7.360,5.669)--(7.360,5.916)--(7.362,6.162)--(7.362,5.916)%
  --(7.363,6.162)--(7.364,6.409)--(7.365,6.409)--(7.366,6.409)--(7.367,6.655)--(7.369,6.655)%
  --(7.370,6.902)--(7.371,6.902)--(7.373,6.902)--(7.374,7.148)--(7.376,7.148)--(7.377,7.148)%
  --(7.378,7.148)--(7.379,7.148)--(7.380,6.902)--(7.381,7.148)--(7.382,7.148)--(7.383,7.148)%
  --(7.384,6.902)--(7.385,6.902)--(7.386,6.902)--(7.387,6.902)--(7.388,6.409)--(7.389,6.409)%
  --(7.390,6.409)--(7.391,6.409)--(7.392,6.162)--(7.393,6.162)--(7.394,6.162)--(7.395,5.916)%
  --(7.396,5.916)--(7.397,5.916)--(7.398,5.669)--(7.399,5.669)--(7.400,5.423)--(7.401,5.669)%
  --(7.403,5.423)--(7.404,5.423)--(7.405,5.423)--(7.406,5.423)--(7.407,5.423)--(7.408,5.423)%
  --(7.409,5.423)--(7.410,5.423)--(7.411,5.423)--(7.412,5.423)--(7.413,5.423)--(7.414,5.669)%
  --(7.415,5.669)--(7.416,5.669)--(7.417,5.916)--(7.418,5.916)--(7.419,5.916)--(7.420,5.916)%
  --(7.421,6.162)--(7.422,6.409)--(7.423,6.409)--(7.424,6.409)--(7.425,6.655)--(7.426,6.655)%
  --(7.427,6.655)--(7.429,6.655)--(7.430,6.902)--(7.431,6.902)--(7.432,6.902)--(7.433,6.902)%
  --(7.434,7.148)--(7.435,7.148)--(7.436,7.148)--(7.437,7.148)--(7.438,6.902)--(7.439,7.148)%
  --(7.440,7.148)--(7.441,6.902)--(7.442,7.148)--(7.443,6.902)--(7.444,6.655)--(7.444,6.902)%
  --(7.445,6.655)--(7.446,6.655)--(7.447,6.655)--(7.448,6.409)--(7.449,6.409)--(7.450,6.409)%
  --(7.451,6.162)--(7.452,6.162)--(7.453,5.916)--(7.454,5.916)--(7.455,5.916)--(7.456,5.916)%
  --(7.458,5.916)--(7.459,5.423)--(7.460,5.669)--(7.461,5.423)--(7.462,5.423)--(7.463,5.423)%
  --(7.464,5.423)--(7.465,5.423)--(7.466,5.176)--(7.467,5.423)--(7.468,5.176)--(7.468,5.423)%
  --(7.470,5.423)--(7.470,5.669)--(7.471,5.423)--(7.472,5.669)--(7.473,5.669)--(7.474,5.916)%
  --(7.475,5.916)--(7.476,5.669)--(7.477,5.916)--(7.478,5.916)--(7.479,6.162)--(7.480,6.409)%
  --(7.480,6.162)--(7.482,6.409)--(7.483,6.655)--(7.484,6.655)--(7.485,6.655)--(7.486,6.655)%
  --(7.487,6.902)--(7.488,6.902)--(7.489,6.902)--(7.490,7.148)--(7.492,7.148)--(7.493,6.902)%
  --(7.494,7.148)--(7.495,7.148)--(7.496,7.148)--(7.497,7.148)--(7.498,7.148)--(7.499,6.902)%
  --(7.500,6.902)--(7.501,6.902)--(7.502,6.902)--(7.503,6.655)--(7.504,6.655)--(7.505,6.655)%
  --(7.506,6.655)--(7.507,6.655)--(7.508,6.162)--(7.509,6.162)--(7.510,6.162)--(7.511,6.162)%
  --(7.512,5.916)--(7.513,5.916)--(7.514,5.916)--(7.515,5.916)--(7.517,5.423)--(7.518,5.423)%
  --(7.520,5.423)--(7.521,5.423)--(7.522,5.423)--(7.523,5.423)--(7.524,5.423)--(7.525,5.423)%
  --(7.526,5.423)--(7.527,5.423)--(7.528,5.669)--(7.529,5.423)--(7.530,5.423)--(7.531,5.669)%
  --(7.532,5.669)--(7.533,5.669)--(7.534,5.916)--(7.535,6.162)--(7.536,6.162)--(7.537,6.162)%
  --(7.538,6.409)--(7.539,6.409)--(7.540,6.409)--(7.541,6.409)--(7.542,6.655)--(7.543,6.655)%
  --(7.545,6.902)--(7.546,6.902)--(7.547,6.902)--(7.548,6.902)--(7.549,6.902)--(7.550,6.902)%
  --(7.552,7.148)--(7.553,6.902)--(7.554,7.148)--(7.555,7.148)--(7.557,7.148)--(7.557,6.902)%
  --(7.558,6.902)--(7.559,6.902)--(7.560,6.902)--(7.561,6.655)--(7.562,6.902)--(7.563,6.655)%
  --(7.564,6.655)--(7.565,6.409)--(7.566,6.409)--(7.567,6.409)--(7.567,6.162)--(7.569,6.162)%
  --(7.571,5.916)--(7.572,5.669)--(7.573,5.669)--(7.574,5.669)--(7.576,5.423)--(7.577,5.423)%
  --(7.578,5.423)--(7.579,5.423)--(7.580,5.423)--(7.581,5.423)--(7.582,5.176)--(7.583,5.423)%
  --(7.585,5.423)--(7.586,5.423)--(7.587,5.669)--(7.588,5.669)--(7.589,5.423)--(7.590,5.669)%
  --(7.591,5.669)--(7.592,5.669)--(7.593,5.916)--(7.594,5.916)--(7.595,6.162)--(7.597,6.162)%
  --(7.598,6.162)--(7.599,6.409)--(7.600,6.409)--(7.601,6.409)--(7.602,6.655)--(7.603,6.902)%
  --(7.605,6.655)--(7.606,6.902)--(7.607,6.902)--(7.608,6.902)--(7.609,6.902)--(7.610,6.902)%
  --(7.611,6.902)--(7.612,6.902)--(7.613,6.902)--(7.614,6.902)--(7.615,6.902)--(7.616,6.655)%
  --(7.617,6.655)--(7.618,6.655)--(7.619,6.655)--(7.620,6.655)--(7.621,6.655)--(7.622,6.409)%
  --(7.623,6.409)--(7.624,6.409)--(7.625,6.162)--(7.626,6.162)--(7.627,6.162)--(7.628,5.916)%
  --(7.629,5.669)--(7.630,5.669)--(7.631,5.669)--(7.632,5.669)--(7.634,5.423)--(7.635,5.423)%
  --(7.636,5.423)--(7.637,5.423)--(7.638,5.423)--(7.639,5.423)--(7.640,5.423)--(7.641,5.423)%
  --(7.642,5.423)--(7.643,5.423)--(7.644,5.423)--(7.645,5.423)--(7.646,5.669)--(7.647,5.669)%
  --(7.648,5.669)--(7.649,5.669)--(7.650,5.916)--(7.651,5.916)--(7.652,5.916)--(7.653,5.916)%
  --(7.654,6.162)--(7.655,6.162)--(7.656,6.162)--(7.657,6.162)--(7.658,6.409)--(7.660,6.409)%
  --(7.661,6.655)--(7.662,6.655)--(7.663,6.655)--(7.664,6.655)--(7.665,6.655)--(7.666,6.655)%
  --(7.667,6.655)--(7.669,6.655)--(7.670,6.655)--(7.671,6.902)--(7.672,6.902)--(7.673,6.655)%
  --(7.674,6.655)--(7.675,6.655)--(7.676,6.655)--(7.677,6.409)--(7.678,6.409)--(7.679,6.409)%
  --(7.680,6.409)--(7.681,6.409)--(7.682,6.162)--(7.683,6.162)--(7.684,5.916)--(7.685,6.162)%
  --(7.686,5.916)--(7.687,5.916)--(7.689,5.669)--(7.690,5.669)--(7.691,5.669)--(7.692,5.423)%
  --(7.693,5.423)--(7.694,5.423)--(7.695,5.423)--(7.697,5.423)--(7.698,5.423)--(7.699,5.176)%
  --(7.700,5.423)--(7.701,5.423)--(7.702,5.176)--(7.703,5.423)--(7.704,5.423)--(7.705,5.669)%
  --(7.706,5.423)--(7.707,5.669)--(7.708,5.669)--(7.709,5.916)--(7.709,5.669)--(7.710,5.916)%
  --(7.711,5.916)--(7.712,6.162)--(7.714,5.916)--(7.714,6.162)--(7.716,6.162)--(7.717,6.409)%
  --(7.718,6.409)--(7.719,6.409)--(7.720,6.655)--(7.721,6.409)--(7.723,6.655)--(7.724,6.655)%
  --(7.725,6.655)--(7.726,6.655)--(7.728,6.655)--(7.729,6.655)--(7.730,6.655)--(7.731,6.655)%
  --(7.732,6.655)--(7.733,6.655)--(7.734,6.655)--(7.735,6.409)--(7.736,6.655)--(7.737,6.409)%
  --(7.737,6.162)--(7.738,6.409)--(7.739,6.162)--(7.740,6.162)--(7.741,6.162)--(7.742,6.162)%
  --(7.744,5.916)--(7.744,5.669)--(7.746,5.916)--(7.747,5.669)--(7.748,5.423)--(7.749,5.669)%
  --(7.750,5.423)--(7.751,5.669)--(7.752,5.423)--(7.754,5.423)--(7.755,5.423)--(7.756,5.423)%
  --(7.757,5.423)--(7.758,5.423)--(7.759,5.423)--(7.760,5.423)--(7.761,5.423)--(7.762,5.423)%
  --(7.763,5.423)--(7.764,5.669)--(7.765,5.423)--(7.766,5.669)--(7.767,5.669)--(7.768,5.669)%
  --(7.769,5.916)--(7.770,5.916)--(7.771,6.162)--(7.772,5.916)--(7.773,6.162)--(7.774,6.162)%
  --(7.775,6.162)--(7.776,6.409)--(7.777,6.409)--(7.779,6.409)--(7.780,6.409)--(7.781,6.655)%
  --(7.782,6.409)--(7.783,6.655)--(7.784,6.655)--(7.785,6.655)--(7.786,6.655)--(7.787,6.655)%
  --(7.788,6.655)--(7.789,6.655)--(7.790,6.409)--(7.791,6.409)--(7.792,6.409)--(7.793,6.409)%
  --(7.794,6.409)--(7.795,6.409)--(7.796,6.162)--(7.797,6.162)--(7.798,6.162)--(7.799,6.162)%
  --(7.800,6.162)--(7.801,5.916)--(7.802,5.916)--(7.803,5.916)--(7.805,5.669)--(7.806,5.669)%
  --(7.807,5.423)--(7.808,5.669)--(7.810,5.423)--(7.812,5.423)--(7.813,5.423)--(7.814,5.423)%
  --(7.815,5.423)--(7.816,5.423)--(7.817,5.423)--(7.818,5.423)--(7.819,5.423)--(7.820,5.423)%
  --(7.821,5.669)--(7.822,5.423)--(7.823,5.423)--(7.824,5.669)--(7.825,5.669)--(7.826,5.916)%
  --(7.827,5.916)--(7.828,5.669)--(7.829,5.916)--(7.830,5.916)--(7.831,6.162)--(7.832,6.162)%
  --(7.833,6.162)--(7.834,6.162)--(7.836,6.162)--(7.837,6.409)--(7.839,6.409)--(7.840,6.655)%
  --(7.841,6.409)--(7.842,6.409)--(7.843,6.409)--(7.844,6.655)--(7.845,6.655)--(7.846,6.409)%
  --(7.847,6.409)--(7.848,6.409)--(7.849,6.409)--(7.850,6.409)--(7.851,6.409)--(7.852,6.409)%
  --(7.853,6.409)--(7.854,6.162)--(7.855,6.162)--(7.856,6.162)--(7.857,6.162)--(7.858,5.916)%
  --(7.859,5.916)--(7.860,5.916)--(7.861,5.916)--(7.862,5.669)--(7.864,5.669)--(7.865,5.669)%
  --(7.866,5.423)--(7.867,5.423)--(7.869,5.423)--(7.870,5.423)--(7.871,5.669)--(7.872,5.423)%
  --(7.874,5.423)--(7.875,5.423)--(7.876,5.423)--(7.877,5.423)--(7.878,5.669)--(7.879,5.669)%
  --(7.880,5.423)--(7.881,5.423)--(7.882,5.669)--(7.883,5.669)--(7.884,5.669)--(7.885,5.669)%
  --(7.886,5.916)--(7.887,5.916)--(7.888,5.916)--(7.889,6.162)--(7.890,6.162)--(7.891,6.162)%
  --(7.893,6.409)--(7.894,6.162)--(7.895,6.162)--(7.896,6.409)--(7.897,6.409)--(7.899,6.409)%
  --(7.900,6.409)--(7.901,6.409)--(7.902,6.655)--(7.903,6.409)--(7.904,6.409)--(7.905,6.409)%
  --(7.906,6.409)--(7.907,6.409)--(7.908,6.409)--(7.909,6.409)--(7.910,6.162)--(7.911,6.162)%
  --(7.912,6.162)--(7.913,6.162)--(7.914,6.162)--(7.915,5.916)--(7.916,5.916)--(7.918,5.916)%
  --(7.919,5.916)--(7.920,5.916)--(7.921,5.916)--(7.922,5.669)--(7.924,5.669)--(7.925,5.669)%
  --(7.927,5.669)--(7.928,5.669)--(7.929,5.423)--(7.931,5.423)--(7.932,5.423)--(7.933,5.423)%
  --(7.934,5.423)--(7.935,5.423)--(7.936,5.669)--(7.937,5.669)--(7.938,5.669)--(7.939,5.669)%
  --(7.941,5.669)--(7.942,5.669)--(7.942,5.916)--(7.944,5.916)--(7.945,5.916)--(7.946,5.916)%
  --(7.947,6.162)--(7.948,6.162)--(7.949,6.162)--(7.951,6.162)--(7.952,6.409)--(7.953,6.409)%
  --(7.955,6.162)--(7.956,6.162)--(7.957,6.162)--(7.958,6.162)--(7.959,6.409)--(7.960,6.409)%
  --(7.961,6.409)--(7.962,6.409)--(7.963,6.409)--(7.965,6.409)--(7.967,6.409)--(7.968,6.162)%
  --(7.969,6.162)--(7.970,6.162)--(7.971,6.162)--(7.972,6.162)--(7.973,6.162)--(7.974,6.162)%
  --(7.975,6.162)--(7.976,5.669)--(7.977,5.669)--(7.979,5.669)--(7.980,5.669)--(7.981,5.669)%
  --(7.982,5.669)--(7.984,5.669)--(7.985,5.669)--(7.986,5.423)--(7.987,5.423)--(7.989,5.423)%
  --(7.990,5.423)--(7.991,5.423)--(7.992,5.423)--(7.993,5.423)--(7.995,5.669)--(7.996,5.669)%
  --(7.998,5.669)--(7.999,5.669)--(8.000,5.669)--(8.001,5.669)--(8.002,5.916)--(8.003,5.916)%
  --(8.004,5.916)--(8.005,5.916)--(8.006,5.916)--(8.008,6.162)--(8.009,6.162)--(8.010,6.162)%
  --(8.012,6.162)--(8.013,6.162)--(8.015,6.162)--(8.016,6.409)--(8.018,6.409)--(8.019,6.409)%
  --(8.020,6.409)--(8.021,6.409)--(8.023,6.409)--(8.024,6.409)--(8.025,6.409)--(8.026,6.162)%
  --(8.027,6.162)--(8.028,6.162)--(8.029,6.162)--(8.030,6.162)--(8.031,5.916)--(8.032,5.916)%
  --(8.033,5.916)--(8.035,5.916)--(8.036,5.916)--(8.037,5.669)--(8.038,5.669)--(8.040,5.669)%
  --(8.041,5.669)--(8.043,5.669)--(8.044,5.669)--(8.046,5.669)--(8.047,5.669)--(8.049,5.669)%
  --(8.050,5.669)--(8.051,5.669)--(8.052,5.669)--(8.053,5.669)--(8.054,5.669)--(8.055,5.669)%
  --(8.056,5.669)--(8.057,5.669)--(8.059,5.916)--(8.060,5.916)--(8.061,5.916)--(8.062,5.916)%
  --(8.063,5.916)--(8.064,5.916)--(8.066,5.916)--(8.067,6.162)--(8.068,6.162)--(8.069,6.162)%
  --(8.071,6.162)--(8.072,6.162)--(8.073,6.162)--(8.075,6.162)--(8.076,6.162)--(8.077,6.162)%
  --(8.079,6.162)--(8.080,6.162)--(8.081,6.162)--(8.082,6.162)--(8.083,6.162)--(8.084,6.162)%
  --(8.085,6.162)--(8.087,6.162)--(8.088,5.916)--(8.089,5.916)--(8.090,5.916)--(8.091,5.916)%
  --(8.092,5.916)--(8.093,5.916)--(8.095,5.916)--(8.096,5.916)--(8.097,5.669)--(8.098,5.669)%
  --(8.100,5.669)--(8.101,5.669)--(8.103,5.669)--(8.104,5.669)--(8.105,5.669)--(8.107,5.669)%
  --(8.108,5.669)--(8.109,5.669)--(8.110,5.669)--(8.111,5.669)--(8.113,5.669)--(8.114,5.916)%
  --(8.115,5.916)--(8.116,5.916)--(8.117,5.916)--(8.118,5.916)--(8.119,5.916)--(8.120,5.916)%
  --(8.122,5.916)--(8.123,5.916)--(8.124,6.162)--(8.125,6.162)--(8.127,6.162)--(8.128,6.162)%
  --(8.129,6.162)--(8.131,6.162)--(8.132,6.162)--(8.133,6.162)--(8.135,6.162)--(8.136,6.162)%
  --(8.138,6.162)--(8.139,6.162)--(8.140,6.162)--(8.141,6.162)--(8.142,6.162)--(8.143,6.162)%
  --(8.145,6.162)--(8.146,6.162)--(8.147,5.916)--(8.148,5.916)--(8.149,5.916)--(8.150,5.916)%
  --(8.151,5.916)--(8.153,5.916)--(8.154,5.669)--(8.155,5.669)--(8.157,5.669)--(8.158,5.669)%
  --(8.159,5.669)--(8.161,5.669)--(8.162,5.669)--(8.163,5.669)--(8.165,5.669)--(8.166,5.669)%
  --(8.167,5.669)--(8.169,5.669)--(8.170,5.669)--(8.171,5.669)--(8.172,5.669)--(8.173,5.669)%
  --(8.174,5.916)--(8.175,5.916)--(8.177,5.916)--(8.178,5.916)--(8.179,5.916)--(8.180,5.916)%
  --(8.181,5.916)--(8.183,5.916)--(8.184,5.916)--(8.185,5.916)--(8.187,6.162)--(8.188,6.162)%
  --(8.189,6.162)--(8.191,6.162)--(8.192,6.162)--(8.193,6.162)--(8.195,6.162)--(8.196,6.162)%
  --(8.197,6.162)--(8.198,6.162)--(8.200,6.162)--(8.201,6.162)--(8.202,6.162)--(8.203,6.162)%
  --(8.204,5.916)--(8.205,5.916)--(8.207,5.916)--(8.208,5.916)--(8.209,5.916)--(8.210,5.916)%
  --(8.211,5.916)--(8.213,5.916)--(8.214,5.916)--(8.216,5.916)--(8.217,5.669)--(8.218,5.669)%
  --(8.220,5.669)--(8.221,5.669)--(8.222,5.669)--(8.223,5.669)--(8.225,5.669)--(8.226,5.669)%
  --(8.227,5.669)--(8.228,5.669)--(8.230,5.669)--(8.231,5.669)--(8.232,5.916)--(8.233,5.916)%
  --(8.234,5.916)--(8.236,5.916)--(8.237,5.916)--(8.238,5.916)--(8.239,5.916)--(8.241,6.162)%
  --(8.242,6.162)--(8.243,6.162)--(8.245,6.162)--(8.247,6.162)--(8.248,6.162)--(8.249,6.162)%
  --(8.251,6.162)--(8.252,6.162)--(8.253,6.162)--(8.254,6.162)--(8.256,6.162)--(8.257,6.162)%
  --(8.258,6.162)--(8.259,6.162)--(8.260,6.162)--(8.261,6.162)--(8.263,6.162)--(8.264,6.162)%
  --(8.265,6.162)--(8.266,6.162)--(8.268,5.669)--(8.269,5.669)--(8.270,5.669)--(8.272,5.669)%
  --(8.273,5.669)--(8.274,5.669)--(8.276,5.669)--(8.277,5.669)--(8.279,5.669)--(8.280,5.669)%
  --(8.282,5.669)--(8.283,5.669)--(8.285,5.669)--(8.286,5.669)--(8.287,5.669)--(8.288,5.669)%
  --(8.289,5.669)--(8.291,5.669)--(8.292,5.669)--(8.293,5.669)--(8.294,5.669)--(8.296,5.916)%
  --(8.297,5.916)--(8.298,5.916)--(8.300,6.162)--(8.301,6.162)--(8.303,6.162)--(8.304,6.162)%
  --(8.306,6.162)--(8.307,6.162)--(8.308,6.162)--(8.309,6.162)--(8.311,6.162)--(8.312,6.162)%
  --(8.313,6.162)--(8.315,6.162)--(8.316,6.162)--(8.317,6.162)--(8.318,6.162)--(8.320,5.916)%
  --(8.321,5.916)--(8.322,5.916)--(8.323,5.916)--(8.324,5.916)--(8.326,5.916)--(8.327,5.916)%
  --(8.329,5.916)--(8.330,5.669)--(8.331,5.669)--(8.333,5.669)--(8.334,5.669)--(8.336,5.669)%
  --(8.337,5.669)--(8.339,5.669)--(8.340,5.669)--(8.342,5.669)--(8.343,5.669)--(8.344,5.669)%
  --(8.346,5.669)--(8.347,5.669)--(8.348,5.669)--(8.349,5.669)--(8.351,5.916)--(8.352,5.916)%
  --(8.353,5.916)--(8.355,5.916)--(8.356,5.916)--(8.357,5.916)--(8.359,5.916)--(8.360,5.916)%
  --(8.362,5.916)--(8.363,5.916)--(8.365,5.916)--(8.366,5.916)--(8.368,5.916)--(8.369,5.916)%
  --(8.370,5.916)--(8.371,5.916)--(8.373,5.916)--(8.374,5.916)--(8.375,5.916)--(8.377,5.916)%
  --(8.378,5.916)--(8.379,5.916)--(8.380,5.916)--(8.382,5.916)--(8.383,5.916)--(8.384,5.916)%
  --(8.386,5.916)--(8.387,5.916)--(8.389,5.916)--(8.390,5.916)--(8.392,5.669)--(8.393,5.669)%
  --(8.394,5.669)--(8.396,5.669)--(8.397,5.669)--(8.399,5.669)--(8.400,5.669)--(8.401,5.669)%
  --(8.403,5.669)--(8.404,5.669)--(8.405,5.669)--(8.407,5.916)--(8.408,5.916)--(8.409,5.916)%
  --(8.411,5.916)--(8.412,5.916)--(8.413,5.916)--(8.415,5.916)--(8.416,5.916)--(8.417,5.916)%
  --(8.419,5.916)--(8.420,5.916)--(8.422,5.916)--(8.424,5.916)--(8.425,5.916)--(8.427,5.916)%
  --(8.428,5.916)--(8.430,5.916)--(8.431,5.916)--(8.432,5.916)--(8.434,5.916)--(8.435,5.916)%
  --(8.436,5.916)--(8.438,5.916)--(8.439,5.916)--(8.440,5.916)--(8.442,5.916)--(8.443,5.916)%
  --(8.445,5.916)--(8.446,5.916)--(8.447,5.916)--(8.449,5.916)--(8.451,5.916)--(8.453,5.669)%
  --(8.455,5.669)--(8.456,5.669)--(8.458,5.669)--(8.459,5.669)--(8.461,5.669)--(8.462,5.916)%
  --(8.464,5.916)--(8.465,5.916)--(8.466,5.916)--(8.468,5.916)--(8.469,5.916)--(8.470,5.916)%
  --(8.472,5.916)--(8.473,5.916)--(8.475,5.916)--(8.476,5.916)--(8.478,5.916)--(8.479,5.916)%
  --(8.481,5.916)--(8.482,5.916)--(8.484,5.916)--(8.485,5.916)--(8.487,5.916)--(8.488,5.916)%
  --(8.490,5.916)--(8.492,5.916)--(8.493,5.916)--(8.494,5.916)--(8.495,5.916)--(8.497,5.916)%
  --(8.498,5.916)--(8.500,5.916)--(8.501,5.916)--(8.502,5.916)--(8.504,5.916)--(8.505,5.916)%
  --(8.507,5.916)--(8.508,5.916)--(8.510,5.916)--(8.512,5.916)--(8.513,5.916)--(8.515,5.916)%
  --(8.516,5.916)--(8.518,5.916)--(8.520,5.916)--(8.521,5.916)--(8.522,5.916)--(8.524,5.916)%
  --(8.525,5.916)--(8.526,5.916)--(8.528,5.916)--(8.529,5.916)--(8.530,5.916)--(8.532,5.916)%
  --(8.533,5.916)--(8.535,5.916)--(8.537,5.916)--(8.538,5.916)--(8.540,5.916)--(8.541,5.916)%
  --(8.543,5.916)--(8.544,5.916)--(8.546,5.916)--(8.548,5.916)--(8.549,5.916)--(8.551,5.916)%
  --(8.552,5.916)--(8.553,5.916)--(8.555,5.916)--(8.556,5.916)--(8.558,5.916)--(8.559,5.916)%
  --(8.560,5.916)--(8.562,5.916)--(8.563,5.916)--(8.565,5.916)--(8.567,5.916)--(8.568,5.916)%
  --(8.570,5.916)--(8.571,5.916)--(8.573,5.916)--(8.574,5.916)--(8.576,5.916)--(8.578,5.916)%
  --(8.579,5.916)--(8.581,5.916)--(8.582,5.916)--(8.583,5.916)--(8.585,5.916)--(8.586,5.916)%
  --(8.588,5.916)--(8.589,5.916)--(8.590,5.916)--(8.592,5.916)--(8.594,5.916)--(8.595,5.916)%
  --(8.597,5.916)--(8.598,5.916)--(8.600,5.916)--(8.601,5.916)--(8.603,5.916)--(8.605,5.916)%
  --(8.606,5.916)--(8.608,5.916)--(8.609,5.916)--(8.610,5.916)--(8.612,5.916)--(8.613,5.916)%
  --(8.615,5.916)--(8.616,5.916)--(8.618,5.916)--(8.619,5.916)--(8.621,5.916)--(8.622,5.916)%
  --(8.624,5.916)--(8.626,5.916)--(8.627,5.916)--(8.629,5.916)--(8.630,5.916)--(8.632,5.916)%
  --(8.633,5.916)--(8.635,5.916)--(8.636,5.916)--(8.638,5.916)--(8.639,5.916)--(8.641,5.916)%
  --(8.642,5.916)--(8.644,5.916)--(8.645,5.916)--(8.647,5.916)--(8.648,5.916)--(8.650,5.916)%
  --(8.651,5.916)--(8.653,5.916)--(8.655,5.916)--(8.657,5.916)--(8.658,5.916)--(8.660,5.916)%
  --(8.662,5.916)--(8.664,5.916)--(8.666,5.916)--(8.667,5.916)--(8.669,5.916)--(8.670,5.916)%
  --(8.672,5.916)--(8.673,5.916)--(8.675,5.916)--(8.676,5.916)--(8.678,5.916)--(8.679,5.916)%
  --(8.681,5.916)--(8.682,5.916)--(8.684,5.916)--(8.686,5.916)--(8.687,5.916)--(8.689,5.916)%
  --(8.690,5.916)--(8.692,5.916)--(8.694,5.916)--(8.696,5.916)--(8.697,5.916)--(8.699,5.916)%
  --(8.700,5.916)--(8.701,5.916)--(8.703,5.916)--(8.704,5.916)--(8.706,5.916)--(8.708,5.916)%
  --(8.709,5.916)--(8.711,5.916)--(8.713,5.916)--(8.715,5.916)--(8.716,5.916)--(8.718,5.916)%
  --(8.719,5.916)--(8.721,5.916)--(8.722,5.916)--(8.724,5.916)--(8.725,5.916)--(8.727,5.916)%
  --(8.729,5.916)--(8.730,5.916)--(8.732,5.916)--(8.733,5.916)--(8.734,5.916)--(8.736,5.916)%
  --(8.738,5.916)--(8.739,5.916)--(8.741,5.916)--(8.743,5.916)--(8.744,5.916)--(8.746,5.916)%
  --(8.748,5.916)--(8.750,5.916)--(8.751,5.916)--(8.753,5.916)--(8.755,5.916)--(8.757,5.916)%
  --(8.758,5.916)--(8.760,5.916)--(8.761,5.916)--(8.763,5.916)--(8.764,5.916)--(8.766,5.916)%
  --(8.767,5.916)--(8.769,5.916)--(8.771,5.916)--(8.773,5.916)--(8.775,5.916)--(8.776,5.916)%
  --(8.778,5.916)--(8.779,5.916)--(8.781,5.916)--(8.783,5.916)--(8.784,5.916)--(8.786,5.916)%
  --(8.787,5.916)--(8.789,5.916)--(8.790,5.916)--(8.792,5.916)--(8.793,5.916)--(8.795,5.916)%
  --(8.797,5.916)--(8.798,5.916)--(8.800,5.916)--(8.802,5.916)--(8.804,5.916)--(8.805,5.916)%
  --(8.807,5.916)--(8.809,5.916)--(8.811,5.916)--(8.813,5.916)--(8.814,5.916)--(8.816,5.916)%
  --(8.818,5.916)--(8.819,5.916)--(8.821,5.916)--(8.822,5.916)--(8.824,5.916)--(8.826,5.916)%
  --(8.827,5.916)--(8.829,5.916)--(8.831,5.916)--(8.833,5.916)--(8.835,5.916)--(8.837,5.916)%
  --(8.839,5.916)--(8.841,5.916)--(8.843,5.916)--(8.845,5.916)--(8.846,5.916)--(8.848,5.916)%
  --(8.849,5.916)--(8.851,5.916)--(8.853,5.916)--(8.854,5.916)--(8.856,5.916)--(8.858,5.916)%
  --(8.860,5.916)--(8.862,5.916)--(8.864,5.916)--(8.866,5.916)--(8.868,5.916)--(8.870,5.916)%
  --(8.872,5.916)--(8.873,5.916)--(8.875,5.916)--(8.877,5.916)--(8.878,5.916)--(8.880,5.916)%
  --(8.882,5.916)--(8.883,5.916)--(8.885,5.916)--(8.887,5.916)--(8.889,5.916)--(8.891,5.916)%
  --(8.893,5.916)--(8.895,5.916)--(8.897,5.916)--(8.899,5.916)--(8.901,5.916)--(8.903,5.916)%
  --(8.904,5.916)--(8.906,5.916)--(8.908,5.916)--(8.909,5.916)--(8.911,5.916)--(8.913,5.916)%
  --(8.914,5.916)--(8.916,5.916)--(8.919,5.916)--(8.921,5.916)--(8.923,5.916)--(8.925,5.916)%
  --(8.927,5.916)--(8.929,5.916)--(8.931,5.916)--(8.932,5.916)--(8.934,5.916)--(8.936,5.916)%
  --(8.937,5.916)--(8.939,5.916)--(8.941,5.916)--(8.943,5.916)--(8.944,5.916)--(8.946,5.916)%
  --(8.948,5.916)--(8.951,5.916)--(8.953,5.916)--(8.955,5.916)--(8.957,5.916)--(8.959,5.916)%
  --(8.961,5.916)--(8.962,5.916)--(8.964,5.916)--(8.966,5.916)--(8.968,5.916)--(8.969,5.916)%
  --(8.971,5.916)--(8.973,5.916)--(8.975,5.916)--(8.977,5.916)--(8.979,5.916)--(8.982,5.916)%
  --(8.984,5.916)--(8.986,5.916)--(8.988,5.916)--(8.990,5.916)--(8.992,5.916)--(8.994,5.916)%
  --(8.995,5.916)--(8.997,5.916)--(8.999,5.916)--(9.001,5.916)--(9.003,5.916)--(9.004,5.916)%
  --(9.006,5.916)--(9.008,5.916)--(9.010,5.916)--(9.012,5.916)--(9.014,5.916)--(9.015,5.916)%
  --(9.017,5.916)--(9.019,5.916)--(9.021,5.916)--(9.023,5.916)--(9.025,5.916)--(9.026,5.916)%
  --(9.028,5.916)--(9.030,5.916)--(9.032,5.916)--(9.034,5.916)--(9.036,5.916)--(9.037,5.916)%
  --(9.039,5.916)--(9.041,5.916)--(9.043,5.916)--(9.045,5.916)--(9.047,5.916)--(9.049,5.916)%
  --(9.051,5.916)--(9.052,5.916)--(9.054,5.916)--(9.056,5.916)--(9.058,5.916)--(9.059,5.916)%
  --(9.061,5.916)--(9.063,5.916)--(9.065,5.916)--(9.067,5.916)--(9.069,5.916)--(9.071,5.916)%
  --(9.073,5.916)--(9.075,5.916)--(9.077,5.916)--(9.078,5.916)--(9.080,5.916)--(9.082,5.916)%
  --(9.084,5.916)--(9.086,5.916)--(9.087,5.916)--(9.089,5.916)--(9.091,5.916)--(9.093,5.916)%
  --(9.095,5.916)--(9.097,5.916)--(9.099,5.916)--(9.101,5.916)--(9.103,5.916)--(9.105,5.916)%
  --(9.107,5.916)--(9.108,5.916)--(9.110,5.916)--(9.112,5.916)--(9.114,5.916)--(9.116,5.916)%
  --(9.118,5.916)--(9.119,5.916)--(9.122,5.916)--(9.124,5.916)--(9.126,5.916)--(9.128,5.916)%
  --(9.130,5.916)--(9.131,5.916)--(9.133,5.916)--(9.135,5.916)--(9.137,5.916)--(9.139,5.916)%
  --(9.141,5.916)--(9.142,5.916)--(9.144,5.916)--(9.146,5.916)--(9.148,5.916)--(9.150,5.916)%
  --(9.152,5.916)--(9.155,5.916)--(9.157,5.916)--(9.159,5.916)--(9.161,5.916)--(9.164,5.916)%
  --(9.166,5.916)--(9.168,5.916)--(9.170,5.916)--(9.171,5.916)--(9.173,5.916)--(9.175,5.916)%
  --(9.177,5.916)--(9.179,5.916)--(9.181,5.916)--(9.183,5.916)--(9.185,5.916)--(9.187,5.916)%
  --(9.189,5.916)--(9.192,5.916)--(9.194,5.916)--(9.196,5.916)--(9.198,5.916)--(9.200,5.916)%
  --(9.201,5.916)--(9.203,5.916)--(9.205,5.916)--(9.207,5.916)--(9.210,5.916)--(9.212,5.916)%
  --(9.214,5.916)--(9.216,5.916)--(9.218,5.916)--(9.220,5.916)--(9.222,5.916)--(9.223,5.916)%
  --(9.225,5.916)--(9.227,5.916)--(9.229,5.916)--(9.231,5.916)--(9.233,5.916)--(9.235,5.916)%
  --(9.237,5.916)--(9.239,5.916)--(9.242,5.916)--(9.244,5.916)--(9.247,5.916)--(9.249,5.916)%
  --(9.251,5.916)--(9.254,5.916)--(9.256,5.916)--(9.258,5.916)--(9.260,5.916)--(9.261,5.916)%
  --(9.263,5.916)--(9.266,5.916)--(9.268,5.916)--(9.270,5.916)--(9.272,5.916)--(9.274,5.916)%
  --(9.276,5.916)--(9.278,5.916)--(9.280,5.916)--(9.282,5.916)--(9.284,5.916)--(9.286,5.916)%
  --(9.288,5.916)--(9.290,5.916)--(9.292,5.916)--(9.294,5.916)--(9.296,5.916)--(9.299,5.916)%
  --(9.301,5.916)--(9.304,5.916)--(9.306,5.916)--(9.309,5.916)--(9.311,5.916)--(9.313,5.916)%
  --(9.315,5.916)--(9.317,5.916)--(9.319,5.916)--(9.321,5.916)--(9.323,5.916)--(9.326,5.916)%
  --(9.328,5.916)--(9.330,5.916)--(9.333,5.916)--(9.335,5.916)--(9.337,5.916)--(9.340,5.916)%
  --(9.343,5.916)--(9.344,5.916)--(9.346,5.916)--(9.349,5.916)--(9.351,5.916)--(9.353,5.916)%
  --(9.355,5.916)--(9.357,5.916)--(9.359,5.916)--(9.362,5.916)--(9.363,5.916)--(9.366,5.916)%
  --(9.368,5.916)--(9.370,5.916)--(9.372,5.916)--(9.374,5.916)--(9.376,5.916)--(9.378,5.916)%
  --(9.380,5.916)--(9.382,5.916)--(9.384,5.916)--(9.387,5.916)--(9.390,5.916)--(9.393,5.916)%
  --(9.396,5.916)--(9.398,5.916)--(9.400,5.916)--(9.403,5.916)--(9.405,5.916)--(9.407,5.916)%
  --(9.409,5.916)--(9.411,5.916)--(9.413,5.916)--(9.416,5.916)--(9.419,5.916)--(9.421,5.916)%
  --(9.424,5.916)--(9.426,5.916)--(9.429,5.916)--(9.431,5.916)--(9.434,5.916)--(9.436,5.916)%
  --(9.438,5.916)--(9.440,5.916)--(9.442,5.916)--(9.445,5.916)--(9.447,5.916)--(9.450,5.916)%
  --(9.453,5.916)--(9.455,5.916)--(9.458,5.916)--(9.460,5.916)--(9.463,5.916)--(9.465,5.916)%
  --(9.467,5.916)--(9.469,5.916)--(9.471,5.916)--(9.474,5.916)--(9.476,5.916)--(9.479,5.916)%
  --(9.482,5.916)--(9.484,5.916)--(9.487,5.916)--(9.489,5.916)--(9.492,5.916)--(9.494,5.916)%
  --(9.496,5.916)--(9.498,5.916)--(9.500,5.916)--(9.503,5.916)--(9.506,5.916)--(9.508,5.916)%
  --(9.511,5.916)--(9.513,5.916)--(9.516,5.916)--(9.519,5.916)--(9.521,5.916)--(9.523,5.916)%
  --(9.525,5.916)--(9.527,5.916)--(9.530,5.916)--(9.533,5.916)--(9.535,5.916)--(9.538,5.916)%
  --(9.541,5.916)--(9.543,5.916)--(9.546,5.916)--(9.548,5.916)--(9.551,5.916)--(9.553,5.916)%
  --(9.555,5.916)--(9.557,5.916)--(9.560,5.916)--(9.563,5.916)--(9.566,5.916)--(9.569,5.916)%
  --(9.572,5.916)--(9.574,5.916)--(9.577,5.916)--(9.579,5.916)--(9.581,5.916)--(9.584,5.916)%
  --(9.586,5.916)--(9.588,5.916)--(9.591,5.916)--(9.593,5.916)--(9.596,5.916)--(9.598,5.916)%
  --(9.601,5.916)--(9.603,5.916)--(9.606,5.916)--(9.608,5.916)--(9.610,5.916)--(9.613,5.916)%
  --(9.615,5.916)--(9.617,5.916)--(9.620,5.916)--(9.622,5.916)--(9.625,5.916)--(9.627,5.916)%
  --(9.630,5.916)--(9.632,5.916)--(9.635,5.916)--(9.637,5.916)--(9.640,5.916)--(9.642,5.916)%
  --(9.644,5.916)--(9.647,5.916)--(9.649,5.916)--(9.651,5.916)--(9.654,5.916)--(9.657,5.916)%
  --(9.659,5.916)--(9.662,5.916)--(9.664,5.916)--(9.667,5.916)--(9.669,5.916)--(9.672,5.916)%
  --(9.674,5.916)--(9.676,5.916)--(9.679,5.916)--(9.681,5.916)--(9.684,5.916)--(9.686,5.916)%
  --(9.689,5.916)--(9.691,5.916)--(9.694,5.916)--(9.697,5.916)--(9.699,5.916)--(9.702,5.916)%
  --(9.704,5.916)--(9.707,5.916)--(9.709,5.916)--(9.711,5.916)--(9.714,5.916)--(9.717,5.916)%
  --(9.719,5.916)--(9.722,5.916)--(9.724,5.916)--(9.727,5.916)--(9.729,5.916)--(9.732,5.916)%
  --(9.734,5.916)--(9.737,5.916)--(9.739,5.916)--(9.742,5.916)--(9.744,5.916)--(9.747,5.916)%
  --(9.749,5.916)--(9.752,5.916)--(9.754,5.916)--(9.757,5.916)--(9.759,5.916)--(9.762,5.916)%
  --(9.764,5.916)--(9.767,5.916)--(9.770,5.916)--(9.772,5.916)--(9.775,5.916)--(9.777,5.916)%
  --(9.780,5.916)--(9.782,5.916)--(9.785,5.916)--(9.787,5.916)--(9.790,5.916)--(9.792,5.916)%
  --(9.795,5.916)--(9.798,5.916)--(9.800,5.916)--(9.803,5.916)--(9.805,5.916)--(9.808,5.916)%
  --(9.810,5.916)--(9.813,5.916)--(9.815,5.916)--(9.818,5.916)--(9.820,5.916)--(9.823,5.916)%
  --(9.826,5.916)--(9.829,5.916)--(9.831,5.916)--(9.834,5.916)--(9.838,5.916)--(9.841,5.916)%
  --(9.844,5.916)--(9.846,5.916)--(9.849,5.916)--(9.851,5.916)--(9.854,5.916)--(9.857,5.916)%
  --(9.859,5.916)--(9.862,5.916)--(9.865,5.916)--(9.867,5.916)--(9.870,5.916)--(9.873,5.916)%
  --(9.876,5.916)--(9.878,5.916)--(9.881,5.916)--(9.884,5.916)--(9.887,5.916)--(9.889,5.916)%
  --(9.892,5.916)--(9.894,5.916)--(9.897,5.916)--(9.900,5.916)--(9.902,5.916)--(9.905,5.916)%
  --(9.908,5.916)--(9.911,5.916)--(9.913,5.916)--(9.916,5.916)--(9.919,5.916)--(9.922,5.916)%
  --(9.925,5.916)--(9.929,5.916)--(9.932,5.916)--(9.934,5.916)--(9.937,5.916)--(9.940,5.916)%
  --(9.942,5.916)--(9.945,5.916)--(9.948,5.916)--(9.951,5.916)--(9.954,5.916)--(9.957,5.916)%
  --(9.959,5.916)--(9.962,5.916)--(9.965,5.916)--(9.968,5.916)--(9.971,5.916)--(9.974,5.916)%
  --(9.976,5.916)--(9.979,5.916)--(9.982,5.916)--(9.984,5.916)--(9.987,5.916)--(9.990,5.916)%
  --(9.993,5.916)--(9.995,5.916)--(9.998,5.916)--(10.001,5.916)--(10.005,5.916)--(10.008,5.916)%
  --(10.011,5.916)--(10.014,5.916)--(10.017,5.916)--(10.020,5.916)--(10.023,5.916)--(10.026,5.916)%
  --(10.029,5.916)--(10.032,5.916)--(10.035,5.916)--(10.037,5.916)--(10.040,5.916)--(10.043,5.916)%
  --(10.046,5.916)--(10.049,5.916)--(10.051,5.916)--(10.054,5.916)--(10.057,5.916)--(10.060,5.916)%
  --(10.064,5.916)--(10.067,5.916)--(10.071,5.916)--(10.074,5.916)--(10.077,5.916)--(10.080,5.916)%
  --(10.083,5.916)--(10.086,5.916)--(10.089,5.916)--(10.092,5.916)--(10.095,5.916)--(10.098,5.916)%
  --(10.102,5.916)--(10.105,5.916)--(10.108,5.916)--(10.111,5.916)--(10.114,5.916)--(10.117,5.916)%
  --(10.120,5.916)--(10.123,5.916)--(10.126,5.916)--(10.129,5.916)--(10.132,5.916)--(10.135,5.916)%
  --(10.138,5.916)--(10.141,5.916)--(10.144,5.916)--(10.147,5.916)--(10.150,5.916)--(10.154,5.916)%
  --(10.157,5.916)--(10.160,5.916)--(10.163,5.916)--(10.167,5.916)--(10.170,5.916)--(10.173,5.916)%
  --(10.176,5.916)--(10.179,5.916)--(10.182,5.916)--(10.186,5.916)--(10.189,5.916)--(10.193,5.916)%
  --(10.197,5.916)--(10.200,5.916)--(10.203,5.916)--(10.206,5.916)--(10.209,5.916)--(10.212,5.916)%
  --(10.215,5.916)--(10.218,5.916)--(10.221,5.916)--(10.224,5.916)--(10.227,5.916)--(10.230,5.916)%
  --(10.233,5.916)--(10.237,5.916)--(10.240,5.916)--(10.244,5.916)--(10.247,5.916)--(10.251,5.916)%
  --(10.254,5.916)--(10.257,5.916)--(10.261,5.916)--(10.264,5.916)--(10.267,5.916)--(10.271,5.916)%
  --(10.274,5.916)--(10.277,5.916)--(10.281,5.916)--(10.284,5.916)--(10.288,5.916)--(10.291,5.916)%
  --(10.294,5.916)--(10.297,5.916)--(10.300,5.916)--(10.304,5.916)--(10.308,5.916)--(10.312,5.916)%
  --(10.315,5.916)--(10.318,5.916)--(10.321,5.916)--(10.325,5.916)--(10.328,5.916)--(10.331,5.916)%
  --(10.334,5.916)--(10.337,5.916)--(10.341,5.916)--(10.344,5.916)--(10.347,5.916)--(10.350,5.916)%
  --(10.353,5.916)--(10.357,5.916)--(10.360,5.916)--(10.363,5.916)--(10.366,5.916)--(10.370,5.916)%
  --(10.373,5.916)--(10.376,5.916)--(10.380,5.916)--(10.383,5.916)--(10.387,5.916)--(10.391,5.916)%
  --(10.395,5.916)--(10.399,5.916)--(10.402,5.916)--(10.405,5.916)--(10.409,5.916)--(10.413,5.916)%
  --(10.417,5.916)--(10.420,5.916)--(10.424,5.916)--(10.427,5.916)--(10.431,5.916)--(10.434,5.916)%
  --(10.438,5.916)--(10.442,5.916)--(10.446,5.916)--(10.449,5.916)--(10.453,5.916)--(10.457,5.916)%
  --(10.460,5.916)--(10.464,5.916)--(10.467,5.916)--(10.471,5.916);
\node[gp node right] at (9.007,7.739) {$u$};
\gpsetdashtype{gp dt 3}
\draw[gp path] (9.191,7.739)--(10.107,7.739);
\draw[gp path] (1.504,8.344)--(1.505,8.344)--(1.506,8.344)--(1.507,8.344)--(1.508,8.344)%
  --(1.509,8.344)--(1.510,8.344)--(1.511,8.344)--(1.512,8.344)--(1.513,8.344)--(1.514,8.344)%
  --(1.515,8.344)--(1.516,8.344)--(1.517,8.344)--(1.519,8.344)--(1.520,8.344)--(1.521,8.344)%
  --(1.522,8.344)--(1.523,8.344)--(1.525,8.344)--(1.526,8.344)--(1.528,8.344)--(1.529,8.344)%
  --(1.531,8.344)--(1.532,8.344)--(1.533,8.344)--(1.534,8.344)--(1.535,8.344)--(1.536,8.344)%
  --(1.537,8.344)--(1.539,8.344)--(1.540,8.344)--(1.541,8.344)--(1.542,8.344)--(1.543,8.344)%
  --(1.544,8.344)--(1.545,8.344)--(1.546,8.344)--(1.547,8.344)--(1.549,8.344)--(1.550,8.344)%
  --(1.551,8.344)--(1.552,8.344)--(1.554,8.344)--(1.555,8.344)--(1.556,8.344)--(1.558,8.344)%
  --(1.559,8.344)--(1.560,8.344)--(1.561,8.344)--(1.563,8.344)--(1.564,8.344)--(1.565,8.344)%
  --(1.566,8.344)--(1.567,8.344)--(1.568,8.344)--(1.569,8.344)--(1.570,8.344)--(1.572,8.344)%
  --(1.573,8.344)--(1.574,8.344)--(1.575,8.344)--(1.576,8.344)--(1.577,8.344)--(1.578,8.344)%
  --(1.579,8.344)--(1.581,8.344)--(1.582,8.344)--(1.583,8.344)--(1.585,8.344)--(1.586,8.344)%
  --(1.587,8.344)--(1.589,8.344)--(1.590,8.344)--(1.591,8.344)--(1.592,8.344)--(1.593,8.344)%
  --(1.595,8.344)--(1.596,8.344)--(1.597,8.344)--(1.598,8.344)--(1.599,8.344)--(1.600,8.344)%
  --(1.601,8.344)--(1.602,8.344)--(1.603,8.344)--(1.604,8.344)--(1.606,8.344)--(1.607,8.344)%
  --(1.608,8.344)--(1.609,8.344)--(1.611,8.344)--(1.612,8.344)--(1.613,8.344)--(1.615,8.344)%
  --(1.616,8.344)--(1.617,8.344)--(1.618,8.344)--(1.619,8.344)--(1.621,8.344)--(1.622,8.344)%
  --(1.623,8.344)--(1.624,8.344)--(1.625,8.344)--(1.626,8.344)--(1.628,8.344)--(1.629,8.344)%
  --(1.630,8.344)--(1.631,8.344)--(1.632,8.344)--(1.633,8.344)--(1.634,8.344)--(1.635,8.344)%
  --(1.636,8.344)--(1.638,8.344)--(1.639,8.344)--(1.640,8.344)--(1.642,8.344)--(1.643,8.344)%
  --(1.644,8.344)--(1.645,8.344)--(1.647,8.344)--(1.648,8.344)--(1.649,8.344)--(1.650,8.344)%
  --(1.652,8.344)--(1.653,8.344)--(1.654,8.344)--(1.655,8.344)--(1.656,8.344)--(1.657,8.344)%
  --(1.658,8.344)--(1.660,8.344)--(1.661,8.344)--(1.662,8.344)--(1.663,8.344)--(1.664,8.344)%
  --(1.665,8.344)--(1.666,8.344)--(1.668,8.344)--(1.669,8.344)--(1.670,8.344)--(1.671,8.344)%
  --(1.673,8.344)--(1.674,8.344)--(1.675,8.344)--(1.677,8.344)--(1.678,8.344)--(1.679,8.344)%
  --(1.680,8.344)--(1.682,8.344)--(1.683,8.344)--(1.684,8.344)--(1.685,8.344)--(1.686,8.344)%
  --(1.687,8.344)--(1.688,8.344)--(1.690,8.344)--(1.691,8.344)--(1.692,8.344)--(1.693,8.344)%
  --(1.694,8.344)--(1.695,8.344)--(1.696,8.344)--(1.698,8.344)--(1.699,8.344)--(1.700,8.344)%
  --(1.702,8.344)--(1.703,8.344)--(1.704,8.344)--(1.705,8.344)--(1.707,8.344)--(1.708,8.344)%
  --(1.709,8.344)--(1.711,8.344)--(1.712,8.344)--(1.713,8.344)--(1.714,8.344)--(1.715,8.344)%
  --(1.716,8.344)--(1.717,8.344)--(1.718,8.344)--(1.720,8.344)--(1.721,8.344)--(1.722,8.344)%
  --(1.723,8.344)--(1.724,8.344)--(1.725,8.344)--(1.727,8.344)--(1.728,8.344)--(1.729,8.344)%
  --(1.730,8.344)--(1.732,8.344)--(1.733,8.344)--(1.734,8.344)--(1.736,8.344)--(1.737,8.344)%
  --(1.738,8.344)--(1.740,8.344)--(1.741,8.344)--(1.742,8.344)--(1.743,8.344)--(1.744,8.344)%
  --(1.745,8.344)--(1.746,8.344)--(1.747,8.344)--(1.749,8.344)--(1.750,8.344)--(1.751,8.344)%
  --(1.752,8.344)--(1.753,8.344)--(1.754,8.344)--(1.756,8.344)--(1.757,8.344)--(1.758,8.344)%
  --(1.759,8.344)--(1.761,8.344)--(1.762,8.344)--(1.763,8.344)--(1.765,8.344)--(1.766,8.344)%
  --(1.767,8.344)--(1.769,8.344)--(1.770,8.344)--(1.771,8.344)--(1.772,8.344)--(1.773,8.344)%
  --(1.774,8.344)--(1.775,8.344)--(1.777,8.344)--(1.778,8.344)--(1.779,8.344)--(1.780,8.344)%
  --(1.781,8.344)--(1.782,8.344)--(1.783,8.344)--(1.785,8.344)--(1.786,8.344)--(1.787,8.344)%
  --(1.789,8.344)--(1.790,8.344)--(1.791,8.344)--(1.793,8.344)--(1.794,8.344)--(1.795,8.344)%
  --(1.797,8.344)--(1.798,8.344)--(1.799,8.344)--(1.800,8.344)--(1.801,8.344)--(1.802,8.344)%
  --(1.803,8.344)--(1.805,8.344)--(1.806,8.344)--(1.807,8.344)--(1.808,8.344)--(1.809,8.344)%
  --(1.810,8.344)--(1.811,8.344)--(1.813,8.344)--(1.814,8.344)--(1.815,8.344)--(1.816,8.344)%
  --(1.818,8.344)--(1.819,8.344)--(1.820,8.344)--(1.822,8.344)--(1.823,8.344)--(1.824,8.344)%
  --(1.826,8.344)--(1.827,8.344)--(1.828,8.344)--(1.829,8.344)--(1.830,8.344)--(1.831,8.344)%
  --(1.833,8.344)--(1.834,8.344)--(1.835,8.344)--(1.836,8.344)--(1.837,8.344)--(1.838,8.344)%
  --(1.839,8.344)--(1.840,8.344)--(1.842,8.344)--(1.843,8.344)--(1.844,8.344)--(1.845,8.344)%
  --(1.847,8.344)--(1.848,8.344)--(1.849,8.344)--(1.851,8.344)--(1.852,8.344)--(1.853,8.344)%
  --(1.855,8.344)--(1.856,8.344)--(1.857,8.344)--(1.858,8.344)--(1.859,8.344)--(1.861,8.344)%
  --(1.862,8.344)--(1.863,8.344)--(1.864,8.344)--(1.865,8.344)--(1.866,8.344)--(1.867,8.344)%
  --(1.868,8.344)--(1.869,8.344)--(1.871,8.344)--(1.872,8.344)--(1.873,8.344)--(1.874,8.344)%
  --(1.876,8.344)--(1.877,8.344)--(1.878,8.344)--(1.880,8.344)--(1.881,8.344)--(1.882,8.344)%
  --(1.884,8.344)--(1.885,8.344)--(1.886,8.344)--(1.887,8.344)--(1.889,8.344)--(1.890,8.344)%
  --(1.891,8.344)--(1.892,8.344)--(1.893,8.344)--(1.894,8.344)--(1.895,8.344)--(1.896,8.344)%
  --(1.897,8.344)--(1.899,8.344)--(1.900,8.344)--(1.901,8.344)--(1.902,8.344)--(1.904,8.344)%
  --(1.905,8.344)--(1.906,8.344)--(1.908,8.344)--(1.909,8.344)--(1.910,8.344)--(1.912,8.344)%
  --(1.913,8.344)--(1.914,8.344)--(1.915,8.344)--(1.916,8.344)--(1.918,8.344)--(1.919,8.344)%
  --(1.920,8.344)--(1.921,8.344)--(1.922,8.344)--(1.923,8.344)--(1.924,8.344)--(1.925,8.344)%
  --(1.927,8.344)--(1.928,8.344)--(1.929,8.344)--(1.930,8.344)--(1.931,8.344)--(1.933,8.344)%
  --(1.934,8.344)--(1.935,8.344)--(1.937,8.344)--(1.938,8.344)--(1.939,8.344)--(1.941,8.344)%
  --(1.942,8.344)--(1.943,8.344)--(1.944,8.344)--(1.946,8.344)--(1.947,8.344)--(1.948,8.344)%
  --(1.949,8.344)--(1.950,8.344)--(1.951,8.344)--(1.952,8.344)--(1.953,8.344)--(1.954,8.344)%
  --(1.956,8.344)--(1.957,8.344)--(1.958,8.344)--(1.959,8.344)--(1.961,8.344)--(1.962,8.344)%
  --(1.963,8.344)--(1.964,8.344)--(1.966,8.344)--(1.967,8.344)--(1.968,8.344)--(1.970,8.344)%
  --(1.971,8.344)--(1.972,8.344)--(1.974,8.344)--(1.975,8.344)--(1.976,8.344)--(1.977,8.344)%
  --(1.978,8.344)--(1.979,8.344)--(1.980,8.344)--(1.981,8.344)--(1.982,8.344)--(1.984,8.344)%
  --(1.985,8.344)--(1.986,8.344)--(1.987,8.344)--(1.988,8.344)--(1.990,8.344)--(1.991,8.344)%
  --(1.992,8.344)--(1.994,8.344)--(1.995,8.344)--(1.996,8.344)--(1.998,8.344)--(1.999,8.344)%
  --(2.000,8.344)--(2.002,8.344)--(2.003,8.344)--(2.004,8.344)--(2.005,8.344)--(2.006,8.344)%
  --(2.007,8.344)--(2.008,8.344)--(2.009,8.344)--(2.010,8.344)--(2.012,8.344)--(2.013,8.344)%
  --(2.014,8.344)--(2.015,8.344)--(2.016,8.344)--(2.018,8.344)--(2.019,8.344)--(2.020,8.344)%
  --(2.021,8.344)--(2.023,8.344)--(2.024,8.344)--(2.025,8.344)--(2.027,8.344)--(2.028,8.344)%
  --(2.029,8.344)--(2.031,8.344)--(2.032,8.344)--(2.033,8.344)--(2.034,8.344)--(2.035,8.344)%
  --(2.036,8.344)--(2.037,8.344)--(2.038,8.344)--(2.040,8.344)--(2.041,8.344)--(2.042,8.344)%
  --(2.043,8.344)--(2.044,8.344)--(2.045,8.344)--(2.047,8.344)--(2.048,8.344)--(2.049,8.344)%
  --(2.050,8.344)--(2.052,8.344)--(2.053,8.344)--(2.054,8.344)--(2.056,8.344)--(2.057,8.344)%
  --(2.058,8.344)--(2.060,8.344)--(2.061,8.344)--(2.062,8.344)--(2.063,8.344)--(2.064,8.344)%
  --(2.065,8.344)--(2.066,8.344)--(2.068,8.344)--(2.069,8.344)--(2.070,8.344)--(2.071,8.344)%
  --(2.072,8.344)--(2.073,8.344)--(2.074,8.344)--(2.076,8.344)--(2.077,8.344)--(2.078,8.344)%
  --(2.080,8.344)--(2.081,8.344)--(2.082,8.344)--(2.084,8.344)--(2.085,8.344)--(2.086,8.344)%
  --(2.088,8.344)--(2.089,8.344)--(2.090,8.344)--(2.091,8.344)--(2.092,8.344)--(2.093,8.344)%
  --(2.094,8.344)--(2.096,8.344)--(2.097,8.344)--(2.098,8.344)--(2.099,8.344)--(2.100,8.344)%
  --(2.101,8.344)--(2.102,8.344)--(2.104,8.344)--(2.105,8.344)--(2.106,8.344)--(2.107,8.344)%
  --(2.109,8.344)--(2.110,8.344)--(2.111,8.344)--(2.113,8.344)--(2.114,8.344)--(2.115,8.344)%
  --(2.117,8.344)--(2.118,8.344)--(2.119,8.344)--(2.120,8.344)--(2.121,8.344)--(2.122,8.344)%
  --(2.124,8.344)--(2.125,8.344)--(2.126,8.344)--(2.127,8.344)--(2.128,8.344)--(2.129,8.344)%
  --(2.130,8.344)--(2.131,8.344)--(2.133,8.344)--(2.134,8.344)--(2.135,8.344)--(2.136,8.344)%
  --(2.138,8.344)--(2.139,8.344)--(2.140,8.344)--(2.142,8.344)--(2.143,8.344)--(2.144,8.344)%
  --(2.146,8.344)--(2.147,8.344)--(2.148,8.344)--(2.149,8.344)--(2.150,8.344)--(2.152,8.344)%
  --(2.153,8.344)--(2.154,8.344)--(2.155,8.344)--(2.156,8.344)--(2.157,8.344)--(2.158,8.344)%
  --(2.159,8.344)--(2.161,8.344)--(2.162,8.344)--(2.163,8.344)--(2.164,8.344)--(2.166,8.344)%
  --(2.167,8.344)--(2.168,8.344)--(2.170,8.344)--(2.171,8.344)--(2.172,8.344)--(2.174,8.344)%
  --(2.175,8.344)--(2.176,8.344)--(2.177,8.344)--(2.178,8.344)--(2.180,8.344)--(2.181,8.344)%
  --(2.182,8.344)--(2.183,8.344)--(2.184,8.344)--(2.185,8.344)--(2.186,8.344)--(2.187,8.344)%
  --(2.188,8.344)--(2.190,8.344)--(2.191,8.344)--(2.192,8.344)--(2.193,8.344)--(2.195,8.344)%
  --(2.196,8.344)--(2.197,8.344)--(2.199,8.344)--(2.200,8.344)--(2.201,8.344)--(2.203,8.344)%
  --(2.204,8.344)--(2.205,8.344)--(2.206,8.344)--(2.208,8.344)--(2.209,8.344)--(2.210,8.344)%
  --(2.211,8.344)--(2.212,8.344)--(2.213,8.344)--(2.214,8.344)--(2.215,8.344)--(2.216,8.344)%
  --(2.218,8.344)--(2.219,8.344)--(2.220,8.344)--(2.221,8.344)--(2.223,8.344)--(2.224,8.344)%
  --(2.225,8.344)--(2.226,8.344)--(2.228,8.344)--(2.229,8.344)--(2.230,8.344)--(2.232,8.344)%
  --(2.233,8.344)--(2.234,8.344)--(2.236,8.344)--(2.237,8.344)--(2.238,8.344)--(2.239,8.344)%
  --(2.240,8.344)--(2.241,8.344)--(2.242,8.344)--(2.243,8.344)--(2.244,8.344)--(2.246,8.344)%
  --(2.247,8.344)--(2.248,8.344)--(2.249,8.344)--(2.250,8.344)--(2.252,8.344)--(2.253,8.344)%
  --(2.254,8.344)--(2.256,8.344)--(2.257,8.344)--(2.258,8.344)--(2.260,8.344)--(2.261,8.344)%
  --(2.262,8.344)--(2.264,8.344)--(2.265,8.344)--(2.266,8.344)--(2.267,8.344)--(2.268,8.344)%
  --(2.269,8.344)--(2.270,8.344)--(2.271,8.344)--(2.272,8.344)--(2.274,8.344)--(2.275,8.344)%
  --(2.276,8.344)--(2.277,8.344)--(2.278,8.344)--(2.279,8.344)--(2.281,8.344)--(2.282,8.344)%
  --(2.283,8.344)--(2.285,8.344)--(2.286,8.344)--(2.287,8.344)--(2.289,8.344)--(2.290,8.344)%
  --(2.291,8.344)--(2.293,8.344)--(2.294,8.344)--(2.295,8.344)--(2.296,8.344)--(2.297,8.344)%
  --(2.298,8.344)--(2.299,8.344)--(2.300,8.344)--(2.302,8.344)--(2.303,8.344)--(2.304,8.344)%
  --(2.305,8.344)--(2.306,8.344)--(2.307,8.344)--(2.309,8.344)--(2.310,8.344)--(2.311,8.344)%
  --(2.312,8.344)--(2.314,8.344)--(2.315,8.344)--(2.316,8.344)--(2.318,8.344)--(2.319,8.344)%
  --(2.320,8.344)--(2.322,8.344)--(2.323,8.344)--(2.324,8.344)--(2.325,8.344)--(2.326,8.344)%
  --(2.327,8.344)--(2.328,8.344)--(2.330,8.344)--(2.331,8.344)--(2.332,8.344)--(2.333,8.344)%
  --(2.334,8.344)--(2.335,8.344)--(2.336,8.344)--(2.338,8.344)--(2.339,8.344)--(2.340,8.344)%
  --(2.342,8.344)--(2.343,8.344)--(2.344,8.344)--(2.346,8.344)--(2.347,8.344)--(2.348,8.344)%
  --(2.350,8.344)--(2.351,8.344)--(2.352,8.344)--(2.353,8.344)--(2.354,8.344)--(2.355,8.344)%
  --(2.356,8.344)--(2.358,8.344)--(2.359,8.344)--(2.360,8.344)--(2.361,8.344)--(2.362,8.344)%
  --(2.363,8.344)--(2.364,8.344)--(2.366,8.344)--(2.367,8.344)--(2.368,8.344)--(2.369,8.344)%
  --(2.371,8.344)--(2.372,8.344)--(2.373,8.344)--(2.375,8.344)--(2.376,8.344)--(2.377,8.344)%
  --(2.379,8.344)--(2.380,8.344)--(2.381,8.344)--(2.382,8.344)--(2.383,8.344)--(2.384,8.344)%
  --(2.386,8.344)--(2.387,8.344)--(2.388,8.344)--(2.389,8.344)--(2.390,8.344)--(2.391,8.344)%
  --(2.392,8.344)--(2.393,8.344)--(2.395,8.344)--(2.396,8.344)--(2.397,8.344)--(2.398,8.344)%
  --(2.400,8.344)--(2.401,8.344)--(2.402,8.344)--(2.404,8.344)--(2.405,8.344)--(2.406,8.344)%
  --(2.408,8.344)--(2.409,8.344)--(2.410,8.344)--(2.411,8.344)--(2.412,8.344)--(2.414,8.344)%
  --(2.415,8.344)--(2.416,8.344)--(2.417,8.344)--(2.418,8.344)--(2.419,8.344)--(2.420,8.344)%
  --(2.421,8.344)--(2.422,8.344)--(2.424,8.344)--(2.425,8.344)--(2.426,8.344)--(2.428,8.344)%
  --(2.429,8.344)--(2.430,8.344)--(2.432,8.344)--(2.433,8.344)--(2.434,8.344)--(2.436,8.344)%
  --(2.437,8.344)--(2.438,8.344)--(2.439,8.344)--(2.440,8.344)--(2.442,8.344)--(2.443,8.344)%
  --(2.444,8.344)--(2.445,8.344)--(2.446,8.344)--(2.447,8.344)--(2.448,8.344)--(2.449,8.344)%
  --(2.450,8.344)--(2.452,8.344)--(2.453,8.344)--(2.454,8.344)--(2.455,8.344)--(2.457,8.344)%
  --(2.458,8.344)--(2.459,8.344)--(2.461,8.344)--(2.462,8.344)--(2.463,8.344)--(2.465,8.344)%
  --(2.466,8.344)--(2.467,8.344)--(2.468,8.344)--(2.470,8.344)--(2.471,8.344)--(2.472,8.344)%
  --(2.473,8.344)--(2.474,8.344)--(2.475,8.344)--(2.476,8.344)--(2.477,8.344)--(2.478,8.344)%
  --(2.480,8.344)--(2.481,8.344)--(2.482,8.344)--(2.483,8.344)--(2.484,8.344)--(2.486,8.344)%
  --(2.487,8.344)--(2.488,8.344)--(2.490,8.344)--(2.491,8.344)--(2.492,8.344)--(2.494,8.344)%
  --(2.495,8.344)--(2.496,8.344)--(2.498,8.344)--(2.499,8.344)--(2.500,8.344)--(2.501,8.344)%
  --(2.502,8.344)--(2.503,8.344)--(2.504,8.344)--(2.505,8.344)--(2.506,8.344)--(2.508,8.344)%
  --(2.509,8.344)--(2.510,8.344)--(2.511,8.344)--(2.512,8.344)--(2.514,8.344)--(2.515,8.344)%
  --(2.516,8.344)--(2.518,8.344)--(2.519,8.344)--(2.520,8.344)--(2.522,8.344)--(2.523,8.344)%
  --(2.524,8.344)--(2.526,8.344)--(2.527,8.344)--(2.528,8.344)--(2.529,8.344)--(2.530,8.344)%
  --(2.531,8.344)--(2.532,8.344)--(2.533,8.344)--(2.534,8.344)--(2.536,8.344)--(2.537,8.344)%
  --(2.538,8.344)--(2.539,8.344)--(2.540,8.344)--(2.541,8.344)--(2.543,8.344)--(2.544,8.344)%
  --(2.545,8.344)--(2.547,8.344)--(2.548,8.344)--(2.549,8.344)--(2.551,8.344)--(2.552,8.344)%
  --(2.553,8.344)--(2.555,8.344)--(2.556,8.344)--(2.557,8.344)--(2.558,8.344)--(2.559,8.344)%
  --(2.560,8.344)--(2.561,8.344)--(2.562,8.344)--(2.563,8.344)--(2.565,8.344)--(2.566,8.344)%
  --(2.567,8.344)--(2.568,8.344)--(2.569,8.344)--(2.571,8.344)--(2.572,8.344)--(2.573,8.344)%
  --(2.574,8.344)--(2.576,8.344)--(2.577,8.344)--(2.578,8.344)--(2.580,8.344)--(2.581,8.344)%
  --(2.582,8.344)--(2.584,8.344)--(2.585,8.344)--(2.586,8.344)--(2.587,8.344)--(2.588,8.344)%
  --(2.589,8.344)--(2.590,8.344)--(2.592,8.344)--(2.593,8.344)--(2.594,8.344)--(2.595,8.344)%
  --(2.596,8.344)--(2.597,8.344)--(2.598,8.344)--(2.600,8.344)--(2.601,8.344)--(2.602,8.344)%
  --(2.604,8.344)--(2.605,8.344)--(2.606,8.344)--(2.608,8.344)--(2.609,8.344)--(2.610,8.344)%
  --(2.611,8.344)--(2.613,8.344)--(2.614,8.344)--(2.615,8.344)--(2.616,8.344)--(2.617,8.344)%
  --(2.618,8.344)--(2.620,8.344)--(2.621,8.344)--(2.622,8.344)--(2.623,8.344)--(2.624,8.344)%
  --(2.625,8.344)--(2.626,8.344)--(2.627,8.344)--(2.629,8.344)--(2.630,8.344)--(2.631,8.344)%
  --(2.633,8.344)--(2.634,8.344)--(2.635,8.344)--(2.637,8.344)--(2.638,8.344)--(2.639,8.344)%
  --(2.641,8.344)--(2.642,8.344)--(2.643,8.344)--(2.644,8.344)--(2.645,8.344)--(2.646,8.344)%
  --(2.648,8.344)--(2.649,8.344)--(2.650,8.344)--(2.651,8.344)--(2.652,8.344)--(2.653,8.344)%
  --(2.654,8.344)--(2.655,8.344)--(2.657,8.344)--(2.658,8.344)--(2.659,8.344)--(2.660,8.344)%
  --(2.662,8.344)--(2.663,8.344)--(2.664,8.344)--(2.666,8.344)--(2.667,8.344)--(2.668,8.344)%
  --(2.670,8.344)--(2.671,8.344)--(2.672,8.344)--(2.673,8.344)--(2.674,8.344)--(2.676,8.344)%
  --(2.677,8.344)--(2.678,8.344)--(2.679,8.344)--(2.680,8.344)--(2.681,8.344)--(2.682,8.344)%
  --(2.683,8.344)--(2.684,8.344)--(2.686,8.344)--(2.687,8.344)--(2.688,8.344)--(2.689,8.344)%
  --(2.691,8.344)--(2.692,8.344)--(2.693,8.344)--(2.695,8.344)--(2.696,8.344)--(2.697,8.344)%
  --(2.699,8.344)--(2.700,8.344)--(2.701,8.344)--(2.702,8.344)--(2.704,8.344)--(2.705,8.344)%
  --(2.706,8.344)--(2.707,8.344)--(2.708,8.344)--(2.709,8.344)--(2.710,8.344)--(2.711,8.344)%
  --(2.712,8.344)--(2.714,8.344)--(2.715,8.344)--(2.716,8.344)--(2.717,8.344)--(2.719,8.344)%
  --(2.720,8.344)--(2.721,8.344)--(2.723,8.344)--(2.724,8.344)--(2.725,8.344)--(2.727,8.344)%
  --(2.728,8.344)--(2.729,8.344)--(2.730,8.344)--(2.732,8.344)--(2.733,8.344)--(2.734,8.344)%
  --(2.735,8.344)--(2.736,8.344)--(2.737,8.344)--(2.738,8.344)--(2.739,8.344)--(2.740,8.344)%
  --(2.742,8.344)--(2.743,8.344)--(2.744,8.344)--(2.745,8.344)--(2.746,8.344)--(2.748,8.344)%
  --(2.749,8.344)--(2.750,8.344)--(2.752,8.344)--(2.753,8.344)--(2.754,8.344)--(2.756,8.344)%
  --(2.757,8.344)--(2.758,8.344)--(2.759,8.344)--(2.761,8.344)--(2.762,8.344)--(2.763,8.344)%
  --(2.764,8.344)--(2.765,8.344)--(2.766,8.344)--(2.767,8.344)--(2.768,8.344)--(2.770,8.344)%
  --(2.771,8.344)--(2.772,8.344)--(2.773,8.344)--(2.774,8.344)--(2.776,8.344)--(2.777,8.344)%
  --(2.778,8.344)--(2.779,8.344)--(2.781,8.344)--(2.782,8.344)--(2.783,8.344)--(2.785,8.344)%
  --(2.786,8.344)--(2.787,8.344)--(2.789,8.344)--(2.790,8.344)--(2.791,8.344)--(2.792,8.344)%
  --(2.793,8.344)--(2.794,8.344)--(2.795,8.344)--(2.796,8.344)--(2.797,8.344)--(2.799,8.344)%
  --(2.800,8.344)--(2.801,8.344)--(2.802,8.344)--(2.803,8.344)--(2.805,8.344)--(2.806,8.344)%
  --(2.807,8.344)--(2.809,8.344)--(2.810,8.344)--(2.811,8.344)--(2.813,8.344)--(2.814,8.344)%
  --(2.815,8.344)--(2.817,8.344)--(2.818,8.344)--(2.819,8.344)--(2.820,8.344)--(2.821,8.344)%
  --(2.822,8.344)--(2.823,8.344)--(2.824,8.344)--(2.825,8.344)--(2.827,8.344)--(2.828,8.344)%
  --(2.829,8.344)--(2.830,8.344)--(2.831,8.344)--(2.833,8.344)--(2.834,8.344)--(2.835,8.344)%
  --(2.836,8.344)--(2.838,8.344)--(2.839,8.344)--(2.840,8.344)--(2.842,8.344)--(2.843,8.344)%
  --(2.844,8.344)--(2.846,8.344)--(2.847,8.344)--(2.848,8.344)--(2.849,8.344)--(2.850,8.344)%
  --(2.851,8.344)--(2.852,8.344)--(2.853,8.344)--(2.855,8.344)--(2.856,8.344)--(2.857,8.344)%
  --(2.858,8.344)--(2.859,8.344)--(2.860,8.344)--(2.862,8.344)--(2.863,8.344)--(2.864,8.344)%
  --(2.865,8.344)--(2.867,8.344)--(2.868,8.344)--(2.869,8.344)--(2.871,8.344)--(2.872,8.344)%
  --(2.873,8.344)--(2.875,8.344)--(2.876,8.344)--(2.877,8.344)--(2.878,8.344)--(2.879,8.344)%
  --(2.880,8.344)--(2.881,8.344)--(2.883,8.344)--(2.884,8.344)--(2.885,8.344)--(2.886,8.344)%
  --(2.887,8.344)--(2.888,8.344)--(2.889,8.344)--(2.891,8.344)--(2.892,8.344)--(2.893,8.344)%
  --(2.895,8.344)--(2.896,8.344)--(2.897,8.344)--(2.899,8.344)--(2.900,8.344)--(2.901,8.344)%
  --(2.903,8.344)--(2.904,8.344)--(2.905,8.344)--(2.906,8.344)--(2.907,8.344)--(2.908,8.344)%
  --(2.909,8.344)--(2.911,8.344)--(2.912,8.344)--(2.913,8.344)--(2.914,8.344)--(2.915,8.344)%
  --(2.916,8.344)--(2.917,8.344)--(2.919,8.344)--(2.920,8.344)--(2.921,8.344)--(2.922,8.344)%
  --(2.924,8.344)--(2.925,8.344)--(2.926,8.344)--(2.928,8.344)--(2.929,8.344)--(2.930,8.344)%
  --(2.932,8.344)--(2.933,8.344)--(2.934,8.344)--(2.935,8.344)--(2.936,8.344)--(2.938,8.344)%
  --(2.939,8.344)--(2.940,8.344)--(2.941,8.344)--(2.942,8.344)--(2.943,8.344)--(2.944,8.344)%
  --(2.945,8.344)--(2.946,8.344)--(2.948,8.344)--(2.949,8.344)--(2.950,8.344)--(2.951,8.344)%
  --(2.953,8.344)--(2.954,8.344)--(2.955,8.344)--(2.957,8.344)--(2.958,8.344)--(2.959,8.344)%
  --(2.961,8.344)--(2.962,8.344)--(2.963,8.344)--(2.964,8.344)--(2.966,8.344)--(2.967,8.344)%
  --(2.968,8.344)--(2.969,8.344)--(2.970,8.344)--(2.971,8.344)--(2.972,8.344)--(2.973,8.344)%
  --(2.974,8.344)--(2.976,8.344)--(2.977,8.344)--(2.978,8.344)--(2.979,8.344)--(2.981,8.344)%
  --(2.982,8.344)--(2.983,8.344)--(2.985,8.344)--(2.986,8.344)--(2.987,8.344)--(2.989,8.344)%
  --(2.990,8.344)--(2.991,8.344)--(2.992,8.344)--(2.994,8.344)--(2.995,8.344)--(2.996,8.344)%
  --(2.997,8.344)--(2.998,8.344)--(2.999,8.344)--(3.000,8.344)--(3.001,8.344)--(3.002,8.344)%
  --(3.004,8.344)--(3.005,8.344)--(3.006,8.344)--(3.007,8.344)--(3.008,8.344)--(3.010,8.344)%
  --(3.011,8.344)--(3.012,8.344)--(3.014,8.344)--(3.015,8.344)--(3.016,8.344)--(3.018,8.344)%
  --(3.019,8.344)--(3.020,8.344)--(3.022,8.344)--(3.023,8.344)--(3.024,8.344)--(3.025,8.344)%
  --(3.026,8.344)--(3.027,8.344)--(3.028,8.344)--(3.029,8.344)--(3.030,8.344)--(3.032,8.344)%
  --(3.033,8.344)--(3.034,8.344)--(3.035,8.344)--(3.036,8.344)--(3.038,8.344)--(3.039,8.344)%
  --(3.040,8.344)--(3.042,8.344)--(3.043,8.344)--(3.044,8.344)--(3.046,8.344)--(3.047,8.344)%
  --(3.048,8.344)--(3.050,8.344)--(3.051,8.344)--(3.052,8.344)--(3.053,8.344)--(3.054,8.344)%
  --(3.055,8.344)--(3.056,8.344)--(3.057,8.344)--(3.058,8.344)--(3.060,8.344)--(3.061,8.344)%
  --(3.062,8.344)--(3.063,8.344)--(3.064,8.344)--(3.065,8.344)--(3.067,8.344)--(3.068,8.344)%
  --(3.069,8.344)--(3.071,8.344)--(3.072,8.344)--(3.073,8.344)--(3.075,8.344)--(3.076,8.344)%
  --(3.077,8.344)--(3.079,8.344)--(3.080,8.344)--(3.081,8.344)--(3.082,8.344)--(3.083,8.344)%
  --(3.084,8.344)--(3.085,8.344)--(3.086,8.344)--(3.088,8.344)--(3.089,8.344)--(3.090,8.344)%
  --(3.091,8.344)--(3.092,8.344)--(3.093,8.344)--(3.095,8.344)--(3.096,8.344)--(3.097,8.344)%
  --(3.099,8.344)--(3.100,8.344)--(3.101,8.344)--(3.102,8.344)--(3.104,8.344)--(3.105,8.344)%
  --(3.106,8.344)--(3.108,8.344)--(3.109,8.344)--(3.110,8.344)--(3.111,8.344)--(3.112,8.344)%
  --(3.113,8.344)--(3.114,8.344)--(3.116,8.344)--(3.117,8.344)--(3.118,8.344)--(3.119,8.344)%
  --(3.120,8.344)--(3.121,8.344)--(3.122,8.344)--(3.124,8.344)--(3.125,8.344)--(3.126,8.344)%
  --(3.128,8.344)--(3.129,8.344)--(3.130,8.344)--(3.132,8.344)--(3.133,8.344)--(3.134,8.344)%
  --(3.136,8.344)--(3.137,8.344)--(3.138,8.344)--(3.139,8.344)--(3.140,8.344)--(3.141,8.344)%
  --(3.143,8.344)--(3.144,8.344)--(3.145,8.344)--(3.146,8.344)--(3.147,8.344)--(3.148,8.344)%
  --(3.149,8.344)--(3.150,8.344)--(3.152,8.344)--(3.153,8.344)--(3.154,8.344)--(3.155,8.344)%
  --(3.157,8.344)--(3.158,8.344)--(3.159,8.344)--(3.161,8.344)--(3.162,8.344)--(3.163,8.344)%
  --(3.165,8.344)--(3.166,8.344)--(3.167,8.344)--(3.168,8.344)--(3.169,8.344)--(3.171,8.344)%
  --(3.172,8.344)--(3.173,8.344)--(3.174,8.344)--(3.175,8.344)--(3.176,8.344)--(3.177,8.344)%
  --(3.178,8.344)--(3.179,8.344)--(3.181,8.344)--(3.182,8.344)--(3.183,8.344)--(3.185,8.344)%
  --(3.186,8.344)--(3.187,8.344)--(3.189,8.344)--(3.190,8.344)--(3.191,8.344)--(3.193,8.344)%
  --(3.194,8.344)--(3.195,8.344)--(3.196,8.344)--(3.197,8.344)--(3.199,8.344)--(3.200,8.344)%
  --(3.201,8.344)--(3.202,8.344)--(3.203,8.344)--(3.204,8.344)--(3.205,8.344)--(3.206,8.344)%
  --(3.208,8.344)--(3.209,8.344)--(3.210,8.344)--(3.211,8.344)--(3.212,8.344)--(3.214,8.344)%
  --(3.215,8.344)--(3.216,8.344)--(3.218,8.344)--(3.219,8.344)--(3.220,8.344)--(3.222,8.344)%
  --(3.223,8.344)--(3.224,8.344)--(3.225,8.344)--(3.227,8.344)--(3.228,8.344)--(3.229,8.344)%
  --(3.230,8.344)--(3.231,8.344)--(3.232,8.344)--(3.233,8.344)--(3.234,8.344)--(3.235,8.344)%
  --(3.237,8.344)--(3.238,8.344)--(3.239,8.344)--(3.240,8.344)--(3.242,8.344)--(3.243,8.344)%
  --(3.244,8.344)--(3.246,8.344)--(3.247,8.344)--(3.248,8.344)--(3.250,8.344)--(3.251,8.344)%
  --(3.252,8.344)--(3.253,8.344)--(3.255,8.344)--(3.256,8.344)--(3.257,8.344)--(3.258,8.344)%
  --(3.259,8.344)--(3.260,8.344)--(3.261,8.344)--(3.262,8.344)--(3.263,8.344)--(3.265,8.344)%
  --(3.266,8.344)--(3.267,8.344)--(3.268,8.344)--(3.269,8.344)--(3.271,8.344)--(3.272,8.344)%
  --(3.273,8.344)--(3.275,8.344)--(3.276,8.344)--(3.277,8.344)--(3.279,8.344)--(3.280,8.344)%
  --(3.281,8.344)--(3.283,8.344)--(3.284,8.344)--(3.285,8.344)--(3.286,8.344)--(3.287,8.344)%
  --(3.288,8.344)--(3.289,8.344)--(3.290,8.344)--(3.291,8.344)--(3.293,8.344)--(3.294,8.344)%
  --(3.295,8.344)--(3.296,8.344)--(3.297,8.344)--(3.299,8.344)--(3.300,8.344)--(3.301,8.344)%
  --(3.303,8.344)--(3.304,8.344)--(3.305,8.344)--(3.306,8.344)--(3.308,8.344)--(3.309,8.344)%
  --(3.310,8.344)--(3.312,8.344)--(3.313,8.344)--(3.314,8.344)--(3.315,8.344)--(3.316,8.344)%
  --(3.317,8.344)--(3.318,8.344)--(3.320,8.344)--(3.321,8.344)--(3.322,8.344)--(3.323,8.344)%
  --(3.324,8.344)--(3.325,8.344)--(3.326,8.344)--(3.328,8.344)--(3.329,8.344)--(3.330,8.344)%
  --(3.332,8.344)--(3.333,8.344)--(3.334,8.344)--(3.336,8.344)--(3.337,8.344)--(3.338,8.344)%
  --(3.340,8.344)--(3.341,8.344)--(3.342,8.344)--(3.343,8.344)--(3.344,8.344)--(3.345,8.344)%
  --(3.346,8.344)--(3.348,8.344)--(3.349,8.344)--(3.350,8.344)--(3.351,8.344)--(3.352,8.344)%
  --(3.353,8.344)--(3.354,8.344)--(3.356,8.344)--(3.357,8.344)--(3.358,8.344)--(3.359,8.344)%
  --(3.361,8.344)--(3.362,8.344)--(3.363,8.344)--(3.365,8.344)--(3.366,8.344)--(3.367,8.344)%
  --(3.369,8.344)--(3.370,8.344)--(3.371,8.344)--(3.372,8.344)--(3.373,8.344)--(3.374,8.344)%
  --(3.376,8.344)--(3.377,8.344)--(3.378,8.344)--(3.379,8.344)--(3.380,8.344)--(3.381,8.344)%
  --(3.382,8.344)--(3.383,8.344)--(3.385,8.344)--(3.386,8.344)--(3.387,8.344)--(3.389,8.344)%
  --(3.390,8.344)--(3.391,8.344)--(3.393,8.344)--(3.394,8.344)--(3.395,8.344)--(3.397,8.344)%
  --(3.398,8.344)--(3.399,8.344)--(3.400,8.344)--(3.401,8.344)--(3.402,8.344)--(3.404,8.344)%
  --(3.405,8.344)--(3.406,8.344)--(3.407,8.344)--(3.408,8.344)--(3.409,8.344)--(3.410,8.344)%
  --(3.411,8.344)--(3.412,8.344)--(3.414,8.344)--(3.415,8.344)--(3.416,8.344)--(3.418,8.344)%
  --(3.419,8.344)--(3.420,8.344)--(3.422,8.344)--(3.423,8.344)--(3.424,8.344)--(3.426,8.344)%
  --(3.427,8.344)--(3.428,8.344)--(3.429,8.344)--(3.431,8.344)--(3.432,8.344)--(3.433,8.344)%
  --(3.434,8.344)--(3.435,8.344)--(3.436,8.344)--(3.437,8.344)--(3.438,8.344)--(3.439,8.344)%
  --(3.440,8.344)--(3.442,8.344)--(3.443,8.344)--(3.444,8.344)--(3.446,8.344)--(3.447,8.344)%
  --(3.448,8.344)--(3.449,8.344)--(3.451,8.344)--(3.452,8.344)--(3.453,8.344)--(3.455,8.344)%
  --(3.456,8.344)--(3.457,8.344)--(3.459,8.344)--(3.460,8.344)--(3.461,8.344)--(3.462,8.344)%
  --(3.463,8.344)--(3.464,8.344)--(3.465,8.344)--(3.466,8.344)--(3.467,8.344)--(3.468,8.344)%
  --(3.470,8.344)--(3.471,8.344)--(3.472,8.344)--(3.473,8.344)--(3.475,8.344)--(3.476,8.344)%
  --(3.477,8.344)--(3.479,8.344)--(3.480,8.344)--(3.481,8.344)--(3.483,8.344)--(3.484,8.344)%
  --(3.485,8.344)--(3.487,8.344)--(3.488,8.344)--(3.489,8.344)--(3.490,8.344)--(3.491,8.344)%
  --(3.492,8.344)--(3.493,8.344)--(3.494,8.344)--(3.495,8.344)--(3.496,8.344)--(3.498,8.344)%
  --(3.499,8.344)--(3.500,8.344)--(3.501,8.344)--(3.503,8.344)--(3.504,8.344)--(3.505,8.344)%
  --(3.507,8.344)--(3.508,8.344)--(3.509,8.344)--(3.510,8.344)--(3.512,8.344)--(3.513,8.344)%
  --(3.514,8.344)--(3.516,8.344)--(3.517,8.344)--(3.518,8.344)--(3.519,8.344)--(3.520,8.344)%
  --(3.521,8.344)--(3.522,8.344)--(3.523,8.344)--(3.524,8.344)--(3.526,8.344)--(3.527,8.344)%
  --(3.528,8.344)--(3.529,8.344)--(3.530,8.344)--(3.532,8.344)--(3.533,8.344)--(3.534,8.344)%
  --(3.536,8.344)--(3.537,8.344)--(3.538,8.344)--(3.540,8.344)--(3.541,8.344)--(3.542,8.344)%
  --(3.543,8.344)--(3.545,8.344)--(3.546,8.344)--(3.547,8.344)--(3.548,8.344)--(3.549,8.344)%
  --(3.550,8.344)--(3.552,8.344)--(3.553,8.344)--(3.554,8.344)--(3.555,8.344)--(3.556,8.344)%
  --(3.557,8.344)--(3.558,8.344)--(3.560,8.344)--(3.561,8.344)--(3.562,8.344)--(3.563,8.344)%
  --(3.565,8.344)--(3.566,8.344)--(3.567,8.344)--(3.569,8.344)--(3.570,8.344)--(3.571,8.344)%
  --(3.573,8.344)--(3.574,8.344)--(3.575,8.344)--(3.576,8.344)--(3.577,8.344)--(3.578,8.344)%
  --(3.580,8.344)--(3.581,8.344)--(3.582,8.344)--(3.583,8.344)--(3.584,8.344)--(3.585,8.344)%
  --(3.586,8.344)--(3.587,8.344)--(3.589,8.344)--(3.590,8.344)--(3.591,8.344)--(3.592,8.344)%
  --(3.594,8.344)--(3.595,8.344)--(3.597,8.344)--(3.598,8.344)--(3.599,8.344)--(3.601,8.344)%
  --(3.602,8.344)--(3.603,8.344)--(3.604,8.344)--(3.605,8.344)--(3.606,8.344)--(3.608,8.344)%
  --(3.609,8.344)--(3.610,8.344)--(3.611,8.344)--(3.612,8.344)--(3.613,8.344)--(3.614,8.344)%
  --(3.615,8.344)--(3.616,8.344)--(3.618,8.344)--(3.619,8.344)--(3.620,8.344)--(3.622,8.344)%
  --(3.623,8.344)--(3.624,8.344)--(3.626,8.344)--(3.627,8.344)--(3.628,8.344)--(3.630,8.344)%
  --(3.631,8.344)--(3.632,8.344)--(3.633,8.344)--(3.634,8.344)--(3.636,8.344)--(3.637,8.344)%
  --(3.638,8.344)--(3.639,8.344)--(3.640,8.344)--(3.641,8.344)--(3.642,8.344)--(3.643,8.344)%
  --(3.644,8.344)--(3.646,8.344)--(3.647,8.344)--(3.648,8.344)--(3.650,8.344)--(3.651,8.344)%
  --(3.652,8.344)--(3.653,8.344)--(3.655,8.344)--(3.656,8.344)--(3.657,8.344)--(3.659,8.344)%
  --(3.660,8.344)--(3.661,8.344)--(3.662,8.344)--(3.663,8.344)--(3.665,8.344)--(3.666,8.344)%
  --(3.667,8.344)--(3.668,8.344)--(3.669,8.344)--(3.670,8.344)--(3.671,8.344)--(3.672,8.344)%
  --(3.674,8.344)--(3.675,8.344)--(3.676,8.344)--(3.677,8.344)--(3.679,8.344)--(3.680,8.344)%
  --(3.681,8.344)--(3.683,8.344)--(3.684,8.344)--(3.685,8.344)--(3.686,8.344)--(3.688,8.344)%
  --(3.689,8.344)--(3.690,8.344)--(3.691,8.344)--(3.693,8.344)--(3.694,8.344)--(3.695,8.344)%
  --(3.696,8.344)--(3.697,8.344)--(3.698,8.344)--(3.699,8.344)--(3.700,8.344)--(3.702,8.344)%
  --(3.703,8.344)--(3.704,8.344)--(3.705,8.344)--(3.706,8.344)--(3.708,8.344)--(3.709,8.344)%
  --(3.710,8.344)--(3.712,8.344)--(3.713,8.344)--(3.714,8.344)--(3.716,8.344)--(3.717,8.344)%
  --(3.718,8.344)--(3.719,8.344)--(3.721,8.344)--(3.722,8.344)--(3.723,8.344)--(3.724,8.344)%
  --(3.725,8.344)--(3.726,8.344)--(3.727,8.344)--(3.728,8.344)--(3.730,8.344)--(3.731,8.344)%
  --(3.732,8.344)--(3.733,8.344)--(3.734,8.344)--(3.736,8.344)--(3.737,8.344)--(3.738,8.344)%
  --(3.739,8.344)--(3.741,8.344)--(3.742,8.344)--(3.743,8.344)--(3.745,8.344)--(3.746,8.344)%
  --(3.747,8.344)--(3.749,8.344)--(3.750,8.344)--(3.751,8.344)--(3.752,8.344)--(3.753,8.344)%
  --(3.754,8.344)--(3.755,8.344)--(3.756,8.344)--(3.758,8.344)--(3.759,8.344)--(3.760,8.344)%
  --(3.761,8.344)--(3.762,8.344)--(3.763,8.344)--(3.765,8.344)--(3.766,8.344)--(3.767,8.344)%
  --(3.769,8.344)--(3.770,8.344)--(3.771,8.344)--(3.773,8.344)--(3.774,8.344)--(3.775,8.344)%
  --(3.777,8.344)--(3.778,8.344)--(3.779,8.344)--(3.780,8.344)--(3.781,8.344)--(3.782,8.344)%
  --(3.783,8.344)--(3.784,8.344)--(3.786,8.344)--(3.787,8.344)--(3.788,8.344)--(3.789,8.344)%
  --(3.790,8.344)--(3.791,8.344)--(3.793,8.344)--(3.794,8.344)--(3.795,8.344)--(3.796,8.344)%
  --(3.798,8.344)--(3.799,8.344)--(3.800,8.344)--(3.802,8.344)--(3.803,8.344)--(3.804,8.344)%
  --(3.806,8.344)--(3.807,8.344)--(3.808,8.344)--(3.809,8.344)--(3.810,8.344)--(3.811,8.344)%
  --(3.812,8.344)--(3.813,8.344)--(3.815,8.344)--(3.816,8.344)--(3.817,8.344)--(3.818,8.344)%
  --(3.819,8.344)--(3.820,8.344)--(3.822,8.344)--(3.823,8.344)--(3.824,8.344)--(3.826,8.344)%
  --(3.827,8.344)--(3.828,8.344)--(3.829,8.344)--(3.831,8.344)--(3.832,8.344)--(3.833,8.344)%
  --(3.835,8.344)--(3.836,8.344)--(3.837,8.344)--(3.838,8.344)--(3.839,8.344)--(3.840,8.344)%
  --(3.841,8.344)--(3.843,8.344)--(3.844,8.344)--(3.845,8.344)--(3.846,8.344)--(3.847,8.344)%
  --(3.848,8.344)--(3.849,8.344)--(3.851,8.344)--(3.852,8.344)--(3.853,8.344)--(3.855,8.344)%
  --(3.856,8.344)--(3.857,8.344)--(3.859,8.344)--(3.860,8.344)--(3.861,8.344)--(3.862,8.344)%
  --(3.864,8.344)--(3.865,8.344)--(3.866,8.344)--(3.867,8.344)--(3.868,8.344)--(3.869,8.344)%
  --(3.871,8.344)--(3.872,8.344)--(3.873,8.344)--(3.874,8.344)--(3.875,8.344)--(3.876,8.344)%
  --(3.877,8.344)--(3.878,8.344)--(3.880,8.344)--(3.881,8.344)--(3.882,8.344)--(3.884,8.344)%
  --(3.885,8.344)--(3.886,8.344)--(3.888,8.344)--(3.889,8.344)--(3.890,8.344)--(3.892,8.344)%
  --(3.893,8.344)--(3.894,8.344)--(3.895,8.344)--(3.896,8.344)--(3.897,8.344)--(3.899,8.344)%
  --(3.900,8.344)--(3.901,8.344)--(3.902,8.344)--(3.903,8.344)--(3.904,8.344)--(3.905,8.344)%
  --(3.906,8.344)--(3.908,8.344)--(3.909,8.344)--(3.910,8.344)--(3.911,8.344)--(3.913,8.344)%
  --(3.914,8.344)--(3.915,8.344)--(3.917,8.344)--(3.918,8.344)--(3.919,8.344)--(3.921,8.344)%
  --(3.922,8.344)--(3.923,8.344)--(3.924,8.344)--(3.925,8.344)--(3.927,8.344)--(3.928,8.344)%
  --(3.929,8.344)--(3.930,8.344)--(3.931,8.344)--(3.932,8.344)--(3.933,8.344)--(3.934,8.344)%
  --(3.935,8.344)--(3.937,8.344)--(3.938,8.344)--(3.939,8.344)--(3.941,8.344)--(3.942,8.344)%
  --(3.943,8.344)--(3.944,8.344)--(3.946,8.344)--(3.947,8.344)--(3.948,8.344)--(3.950,8.344)%
  --(3.951,8.344)--(3.952,8.344)--(3.954,8.344)--(3.955,8.344)--(3.956,8.344)--(3.957,8.344)%
  --(3.958,8.344)--(3.959,8.344)--(3.960,8.344)--(3.961,8.344)--(3.962,8.344)--(3.963,8.344)%
  --(3.965,8.344)--(3.966,8.344)--(3.967,8.344)--(3.969,8.344)--(3.970,8.344)--(3.971,8.344)%
  --(3.972,8.344)--(3.974,8.344)--(3.975,8.344)--(3.976,8.344)--(3.978,8.344)--(3.979,8.344)%
  --(3.980,8.344)--(3.982,8.344)--(3.983,8.344)--(3.984,8.344)--(3.985,8.344)--(3.986,8.344)%
  --(3.987,8.344)--(3.988,8.344)--(3.989,8.344)--(3.990,8.344)--(3.991,8.344)--(3.993,8.344)%
  --(3.994,8.344)--(3.995,8.344)--(3.996,8.344)--(3.998,8.344)--(3.999,8.344)--(4.000,8.344)%
  --(4.001,8.344)--(4.003,8.344)--(4.004,8.344)--(4.005,8.344)--(4.007,8.344)--(4.008,8.344)%
  --(4.009,8.344)--(4.011,8.344)--(4.012,8.344)--(4.013,8.344)--(4.014,8.344)--(4.015,8.344)%
  --(4.016,8.344)--(4.017,8.344)--(4.018,8.344)--(4.019,8.344)--(4.021,8.344)--(4.022,8.344)%
  --(4.023,8.344)--(4.024,8.344)--(4.025,8.344)--(4.027,8.344)--(4.028,8.344)--(4.029,8.344)%
  --(4.031,8.344)--(4.032,8.344)--(4.033,8.344)--(4.035,8.344)--(4.036,8.344)--(4.037,8.344)%
  --(4.038,8.344)--(4.040,8.344)--(4.041,8.344)--(4.042,8.344)--(4.043,8.344)--(4.044,8.344)%
  --(4.045,8.344)--(4.046,8.344)--(4.047,8.344)--(4.049,8.344)--(4.050,8.344)--(4.051,8.344)%
  --(4.052,8.344)--(4.053,8.344)--(4.054,8.344)--(4.056,8.344)--(4.057,8.344)--(4.058,8.344)%
  --(4.060,8.344)--(4.061,8.344)--(4.062,8.344)--(4.064,8.344)--(4.065,8.344)--(4.066,8.344)%
  --(4.068,8.344)--(4.069,8.344)--(4.070,8.344)--(4.071,8.344)--(4.072,8.344)--(4.073,8.344)%
  --(4.074,8.344)--(4.075,8.344)--(4.077,8.344)--(4.078,8.344)--(4.079,8.344)--(4.080,8.344)%
  --(4.081,8.344)--(4.082,8.344)--(4.084,8.344)--(4.085,8.344)--(4.086,8.344)--(4.087,8.344)%
  --(4.089,8.344)--(4.090,8.344)--(4.091,8.344)--(4.093,8.344)--(4.094,8.344)--(4.095,8.344)%
  --(4.097,8.344)--(4.098,8.344)--(4.099,8.344)--(4.100,8.344)--(4.101,8.344)--(4.102,8.344)%
  --(4.103,8.344)--(4.105,8.344)--(4.106,8.344)--(4.107,8.344)--(4.108,8.344)--(4.109,8.344)%
  --(4.110,8.344)--(4.111,8.344)--(4.113,8.344)--(4.114,8.344)--(4.115,8.344)--(4.117,8.344)%
  --(4.118,8.344)--(4.119,8.344)--(4.120,8.344)--(4.122,8.344)--(4.123,8.344)--(4.124,8.344)%
  --(4.126,8.344)--(4.127,8.344)--(4.128,8.344)--(4.129,8.344)--(4.130,8.344)--(4.132,8.344)%
  --(4.133,8.344)--(4.134,8.344)--(4.135,8.344)--(4.136,8.344)--(4.137,8.344)--(4.138,8.344)%
  --(4.139,8.344)--(4.141,8.344)--(4.142,8.344)--(4.143,8.344)--(4.144,8.344)--(4.146,8.344)%
  --(4.147,8.344)--(4.148,8.344)--(4.150,8.344)--(4.151,8.344)--(4.152,8.344)--(4.153,8.344)%
  --(4.155,8.344)--(4.156,8.344)--(4.157,8.344)--(4.158,8.344)--(4.159,8.344)--(4.161,8.344)%
  --(4.162,8.344)--(4.163,8.344)--(4.164,8.344)--(4.165,8.344)--(4.166,8.344)--(4.167,8.344)%
  --(4.168,8.344)--(4.170,8.344)--(4.171,8.344)--(4.172,8.344)--(4.173,8.344)--(4.175,8.344)%
  --(4.176,8.344)--(4.177,8.344)--(4.179,8.344)--(4.180,8.344)--(4.181,8.344)--(4.183,8.344)%
  --(4.184,8.344)--(4.185,8.344)--(4.186,8.344)--(4.188,8.344)--(4.189,8.344)--(4.190,8.344)%
  --(4.191,8.344)--(4.192,8.344)--(4.193,8.344)--(4.194,8.344)--(4.195,8.344)--(4.196,8.344)%
  --(4.198,8.344)--(4.199,8.344)--(4.200,8.344)--(4.201,8.344)--(4.203,8.344)--(4.204,8.344)%
  --(4.205,8.344)--(4.207,8.344)--(4.208,8.344)--(4.209,8.344)--(4.210,8.344)--(4.212,8.344)%
  --(4.213,8.344)--(4.214,8.344)--(4.215,8.344)--(4.217,8.344)--(4.218,8.344)--(4.219,8.344)%
  --(4.220,8.344)--(4.221,8.344)--(4.222,8.344)--(4.223,8.344)--(4.224,8.344)--(4.225,8.344)%
  --(4.227,8.344)--(4.228,8.344)--(4.229,8.344)--(4.230,8.344)--(4.232,8.344)--(4.233,8.344)%
  --(4.234,8.344)--(4.236,8.344)--(4.237,8.344)--(4.238,8.344)--(4.240,8.344)--(4.241,8.344)%
  --(4.242,8.344)--(4.244,8.344)--(4.245,8.344)--(4.246,8.344)--(4.247,8.344)--(4.248,8.344)%
  --(4.249,8.344)--(4.250,8.344)--(4.251,8.344)--(4.252,8.344)--(4.253,8.344)--(4.255,8.344)%
  --(4.256,8.344)--(4.257,8.344)--(4.258,8.344)--(4.260,8.344)--(4.261,8.344)--(4.262,8.344)%
  --(4.263,8.344)--(4.265,8.344)--(4.266,8.344)--(4.267,8.344)--(4.269,8.344)--(4.270,8.344)%
  --(4.271,8.344)--(4.273,8.344)--(4.274,8.344)--(4.275,8.344)--(4.276,8.344)--(4.277,8.344)%
  --(4.278,8.344)--(4.279,8.344)--(4.280,8.344)--(4.281,8.344)--(4.282,8.344)--(4.284,8.344)%
  --(4.285,8.344)--(4.286,8.344)--(4.287,8.344)--(4.289,8.344)--(4.290,8.344)--(4.291,8.344)%
  --(4.293,8.344)--(4.294,8.344)--(4.295,8.344)--(4.296,8.344)--(4.298,8.344)--(4.299,8.344)%
  --(4.300,8.344)--(4.302,8.344)--(4.303,8.344)--(4.304,8.344)--(4.305,8.344)--(4.306,8.344)%
  --(4.307,8.344)--(4.308,8.344)--(4.309,8.344)--(4.311,8.344)--(4.312,8.344)--(4.313,8.344)%
  --(4.314,8.344)--(4.315,8.344)--(4.316,8.344)--(4.318,8.344)--(4.319,8.344)--(4.320,8.344)%
  --(4.322,8.344)--(4.323,8.344)--(4.324,8.344)--(4.326,8.344)--(4.327,8.344)--(4.328,8.344)%
  --(4.330,8.344)--(4.331,8.344)--(4.332,8.344)--(4.333,8.344)--(4.334,8.344)--(4.335,8.344)%
  --(4.336,8.344)--(4.337,8.344)--(4.338,8.344)--(4.340,8.344)--(4.341,8.344)--(4.342,8.344)%
  --(4.343,8.344)--(4.344,8.344)--(4.345,8.344)--(4.347,8.344)--(4.348,8.344)--(4.349,8.344)%
  --(4.351,8.344)--(4.352,8.344)--(4.353,8.344)--(4.355,8.344)--(4.356,8.344)--(4.357,8.344)%
  --(4.359,8.344)--(4.360,8.344)--(4.361,8.344)--(4.362,8.344)--(4.363,8.344)--(4.364,8.344)%
  --(4.366,8.344)--(4.367,8.344)--(4.368,8.344)--(4.369,8.344)--(4.370,8.344)--(4.371,8.344)%
  --(4.372,8.344)--(4.373,8.344)--(4.375,8.344)--(4.376,8.344)--(4.377,8.344)--(4.379,8.344)%
  --(4.380,8.344)--(4.381,8.344)--(4.382,8.344)--(4.384,8.344)--(4.385,8.344)--(4.386,8.344)%
  --(4.388,8.344)--(4.389,8.344)--(4.390,8.344)--(4.391,8.344)--(4.392,8.344)--(4.393,8.344)%
  --(4.394,8.344)--(4.396,8.344)--(4.397,8.344)--(4.398,8.344)--(4.399,8.344)--(4.400,8.344)%
  --(4.401,8.344)--(4.402,8.344)--(4.404,8.344)--(4.405,8.344)--(4.406,8.344)--(4.408,8.344)%
  --(4.409,8.344)--(4.410,8.344)--(4.412,8.344)--(4.413,8.344)--(4.414,8.344)--(4.416,8.344)%
  --(4.417,8.344)--(4.418,8.344)--(4.419,8.344)--(4.420,8.344)--(4.422,8.344)--(4.423,8.344)%
  --(4.424,8.344)--(4.425,8.344)--(4.426,8.344)--(4.427,8.344)--(4.428,8.344)--(4.429,8.344)%
  --(4.431,8.344)--(4.432,8.344)--(4.433,8.344)--(4.434,8.344)--(4.436,8.344)--(4.437,8.344)%
  --(4.438,8.344)--(4.439,8.344)--(4.441,8.344)--(4.442,8.344)--(4.443,8.344)--(4.445,8.344)%
  --(4.446,8.344)--(4.447,8.344)--(4.448,8.344)--(4.449,8.344)--(4.451,8.344)--(4.452,8.344)%
  --(4.453,8.344)--(4.454,8.344)--(4.455,8.344)--(4.456,8.344)--(4.457,8.344)--(4.458,8.344)%
  --(4.459,8.344)--(4.461,8.344)--(4.462,8.344)--(4.463,8.344)--(4.464,8.344)--(4.466,8.344)%
  --(4.467,8.344)--(4.468,8.344)--(4.470,8.344)--(4.471,8.344)--(4.473,8.344)--(4.474,8.344)%
  --(4.475,8.344)--(4.476,8.344)--(4.478,8.344)--(4.479,8.344)--(4.480,8.344)--(4.481,8.344)%
  --(4.482,8.344)--(4.483,8.344)--(4.484,8.344)--(4.485,8.344)--(4.486,8.344)--(4.488,8.344)%
  --(4.489,8.344)--(4.490,8.344)--(4.491,8.344)--(4.492,8.344)--(4.494,8.344)--(4.495,8.344)%
  --(4.496,8.344)--(4.498,8.344)--(4.499,8.344)--(4.500,8.344)--(4.502,8.344)--(4.503,8.344)%
  --(4.504,8.344)--(4.505,8.344)--(4.507,8.344)--(4.508,8.344)--(4.509,8.344)--(4.510,8.344)%
  --(4.511,8.344)--(4.512,8.344)--(4.513,8.344)--(4.514,8.344)--(4.515,8.344)--(4.516,8.344)%
  --(4.518,8.344)--(4.519,8.344)--(4.520,8.344)--(4.521,8.344)--(4.523,8.344)--(4.524,8.344)%
  --(4.525,8.344)--(4.527,8.344)--(4.528,8.344)--(4.529,8.344)--(4.531,8.344)--(4.532,8.344)%
  --(4.533,8.344)--(4.535,8.344)--(4.536,8.344)--(4.537,8.344)--(4.538,8.344)--(4.539,8.344)%
  --(4.540,8.344)--(4.541,8.344)--(4.542,8.344)--(4.544,8.344)--(4.545,8.344)--(4.546,8.344)%
  --(4.547,8.344)--(4.548,8.344)--(4.549,8.344)--(4.551,8.344)--(4.552,8.344)--(4.553,8.344)%
  --(4.555,8.344)--(4.556,8.344)--(4.557,8.344)--(4.558,8.344)--(4.560,8.344)--(4.561,8.344)%
  --(4.562,8.344)--(4.564,8.344)--(4.565,8.344)--(4.566,8.344)--(4.567,8.344)--(4.568,8.344)%
  --(4.569,8.344)--(4.570,8.344)--(4.571,8.344)--(4.572,8.344)--(4.574,8.344)--(4.575,8.344)%
  --(4.576,8.344)--(4.577,8.344)--(4.578,8.344)--(4.580,8.344)--(4.581,8.344)--(4.582,8.344)%
  --(4.584,8.344)--(4.585,8.344)--(4.586,8.344)--(4.588,8.344)--(4.589,8.344)--(4.590,8.344)%
  --(4.592,8.344)--(4.593,8.344)--(4.594,8.344)--(4.595,8.344)--(4.596,8.344)--(4.597,8.344)%
  --(4.598,8.344)--(4.600,8.344)--(4.601,8.344)--(4.602,8.344)--(4.603,8.344)--(4.604,8.344)%
  --(4.605,8.344)--(4.606,7.087)--(4.607,6.217)--(4.609,5.549)--(4.610,4.976)--(4.610,4.434)%
  --(4.612,3.867)--(4.613,3.211)--(4.614,2.366)--(4.614,2.219)--(4.614,2.066)--(4.614,1.902)%
  --(4.614,1.728)--(4.614,1.545)--(4.615,1.344)--(4.615,1.131)--(4.615,1.022)--(4.616,1.022)%
  --(4.617,1.022)--(4.618,1.022)--(4.619,1.022)--(4.620,1.022)--(4.621,1.022)--(4.622,1.022)%
  --(4.623,1.022)--(4.624,1.022)--(4.625,1.022)--(4.626,1.022)--(4.627,1.022)--(4.628,1.022)%
  --(4.629,1.022)--(4.630,1.022)--(4.631,1.022)--(4.632,1.022)--(4.633,1.022)--(4.635,1.022)%
  --(4.636,1.022)--(4.637,1.022)--(4.638,1.022)--(4.639,1.022)--(4.640,1.022)--(4.641,1.022)%
  --(4.642,1.022)--(4.643,1.022)--(4.644,1.022)--(4.645,1.022)--(4.646,1.022)--(4.647,1.022)%
  --(4.648,1.022)--(4.649,1.022)--(4.650,1.022)--(4.651,1.022)--(4.652,1.022)--(4.653,1.022)%
  --(4.654,1.022)--(4.655,1.022)--(4.656,1.022)--(4.657,1.022)--(4.658,1.022)--(4.659,1.022)%
  --(4.660,1.022)--(4.661,1.022)--(4.662,1.022)--(4.663,1.022)--(4.665,1.022)--(4.666,1.022)%
  --(4.667,1.022)--(4.668,1.022)--(4.669,1.022)--(4.670,1.022)--(4.671,1.022)--(4.672,1.022)%
  --(4.673,1.022)--(4.674,1.022)--(4.675,1.022)--(4.677,1.022)--(4.678,1.022)--(4.679,1.022)%
  --(4.680,1.022)--(4.681,1.022)--(4.682,1.022)--(4.683,1.022)--(4.684,1.022)--(4.685,1.022)%
  --(4.686,1.022)--(4.687,1.022)--(4.688,1.022)--(4.689,1.022)--(4.690,1.022)--(4.691,1.022)%
  --(4.692,1.022)--(4.693,1.022)--(4.694,1.022)--(4.695,1.022)--(4.696,1.022)--(4.697,1.022)%
  --(4.698,1.022)--(4.699,1.022)--(4.700,1.022)--(4.701,1.022)--(4.702,1.022)--(4.703,1.022)%
  --(4.704,1.022)--(4.705,1.022)--(4.706,1.022)--(4.707,1.022)--(4.708,1.022)--(4.709,1.022)%
  --(4.710,1.022)--(4.711,1.022)--(4.712,1.022)--(4.713,1.022)--(4.714,1.022)--(4.715,1.022)%
  --(4.716,1.022)--(4.717,1.022)--(4.718,1.022)--(4.719,1.022)--(4.720,1.022)--(4.722,1.022)%
  --(4.723,1.022)--(4.724,1.022)--(4.725,1.022)--(4.726,1.022)--(4.727,1.022)--(4.729,1.022)%
  --(4.730,1.022)--(4.731,1.022)--(4.732,1.022)--(4.733,1.022)--(4.734,1.022)--(4.735,1.022)%
  --(4.736,1.022)--(4.737,1.022)--(4.738,1.022)--(4.739,1.022)--(4.740,1.022)--(4.741,1.022)%
  --(4.742,1.022)--(4.743,1.022)--(4.745,1.022)--(4.746,1.022)--(4.748,1.022)--(4.749,1.022)%
  --(4.750,1.022)--(4.751,1.022)--(4.753,1.022)--(4.754,1.022)--(4.755,1.022)--(4.757,1.022)%
  --(4.758,1.022)--(4.759,1.022)--(4.760,1.022)--(4.761,1.022)--(4.762,1.022)--(4.763,1.022)%
  --(4.764,1.022)--(4.765,1.022)--(4.766,1.022)--(4.767,1.022)--(4.768,1.022)--(4.769,1.022)%
  --(4.770,1.022)--(4.771,1.022)--(4.772,1.022)--(4.773,1.022)--(4.774,1.022)--(4.775,1.022)%
  --(4.776,1.022)--(4.777,1.022)--(4.778,1.022)--(4.779,1.022)--(4.781,1.022)--(4.782,1.022)%
  --(4.783,1.022)--(4.785,1.022)--(4.786,1.022)--(4.787,1.022)--(4.788,1.022)--(4.789,1.022)%
  --(4.790,1.022)--(4.791,1.022)--(4.792,1.022)--(4.793,1.022)--(4.794,1.022)--(4.795,1.022)%
  --(4.796,1.022)--(4.797,1.022)--(4.798,1.022)--(4.799,1.022)--(4.800,1.022)--(4.801,1.022)%
  --(4.802,1.022)--(4.803,1.022)--(4.804,1.022)--(4.805,1.022)--(4.806,1.022)--(4.807,1.022)%
  --(4.808,1.022)--(4.809,1.022)--(4.811,1.022)--(4.812,1.022)--(4.813,1.022)--(4.814,1.022)%
  --(4.815,1.022)--(4.816,1.022)--(4.817,1.022)--(4.818,1.022)--(4.819,1.022)--(4.820,1.022)%
  --(4.821,1.022)--(4.822,1.022)--(4.823,1.022)--(4.824,1.022)--(4.825,1.022)--(4.826,1.022)%
  --(4.827,1.022)--(4.828,1.022)--(4.829,1.022)--(4.830,1.022)--(4.831,1.022)--(4.832,1.022)%
  --(4.833,1.022)--(4.834,1.022)--(4.835,1.022)--(4.837,1.022)--(4.838,1.022)--(4.839,1.022)%
  --(4.840,1.022)--(4.841,1.022)--(4.842,1.022)--(4.843,1.022)--(4.844,1.022)--(4.845,1.022)%
  --(4.846,1.022)--(4.847,1.022)--(4.848,1.022)--(4.849,1.022)--(4.850,1.022)--(4.851,1.022)%
  --(4.852,1.022)--(4.853,1.022)--(4.854,1.022)--(4.855,1.022)--(4.856,1.022)--(4.857,1.022)%
  --(4.858,1.022)--(4.859,1.022)--(4.860,1.022)--(4.861,1.022)--(4.862,1.022)--(4.863,1.022)%
  --(4.864,1.022)--(4.865,1.022)--(4.866,1.022)--(4.868,1.022)--(4.869,1.022)--(4.870,1.022)%
  --(4.871,1.022)--(4.872,1.022)--(4.873,1.022)--(4.874,1.022)--(4.875,1.022)--(4.876,1.022)%
  --(4.877,1.022)--(4.878,1.022)--(4.879,1.022)--(4.881,1.022)--(4.882,1.022)--(4.883,1.022)%
  --(4.884,1.022)--(4.885,1.022)--(4.886,1.022)--(4.887,1.022)--(4.888,1.022)--(4.889,1.022)%
  --(4.890,1.022)--(4.891,1.022)--(4.892,1.022)--(4.893,1.022)--(4.894,1.022)--(4.895,1.022)%
  --(4.896,1.022)--(4.897,1.022)--(4.898,1.022)--(4.899,1.022)--(4.901,1.022)--(4.902,1.022)%
  --(4.903,1.022)--(4.904,1.022)--(4.905,1.022)--(4.906,1.022)--(4.907,1.022)--(4.908,1.022)%
  --(4.909,1.022)--(4.910,1.022)--(4.911,1.022)--(4.912,1.022)--(4.913,1.022)--(4.914,1.022)%
  --(4.915,1.022)--(4.916,1.022)--(4.917,1.022)--(4.918,1.022)--(4.919,1.022)--(4.920,1.022)%
  --(4.921,1.022)--(4.922,1.022)--(4.923,1.022)--(4.924,1.022)--(4.926,1.022)--(4.927,1.022)%
  --(4.928,1.022)--(4.929,1.022)--(4.931,1.022)--(4.932,1.022)--(4.933,1.022)--(4.934,1.022)%
  --(4.935,1.022)--(4.936,1.022)--(4.937,1.022)--(4.938,1.022)--(4.939,1.022)--(4.940,1.022)%
  --(4.941,1.022)--(4.942,1.022)--(4.943,1.022)--(4.944,1.022)--(4.945,1.022)--(4.946,1.022)%
  --(4.947,1.022)--(4.948,1.022)--(4.949,1.022)--(4.950,1.022)--(4.951,1.022)--(4.952,1.022)%
  --(4.953,1.022)--(4.955,1.022)--(4.956,1.022)--(4.957,1.022)--(4.958,1.022)--(4.959,1.022)%
  --(4.960,1.022)--(4.961,1.022)--(4.962,1.022)--(4.963,1.022)--(4.964,1.022)--(4.965,1.022)%
  --(4.966,1.022)--(4.967,1.022)--(4.968,1.022)--(4.969,1.022)--(4.970,1.022)--(4.971,1.022)%
  --(4.972,1.022)--(4.973,1.022)--(4.974,1.022)--(4.975,1.022)--(4.976,1.022)--(4.977,1.022)%
  --(4.978,1.022)--(4.979,1.022)--(4.980,1.022)--(4.981,1.022)--(4.982,1.022)--(4.983,1.022)%
  --(4.984,1.022)--(4.985,1.022)--(4.987,1.022)--(4.989,1.022)--(4.990,1.022)--(4.992,1.022)%
  --(4.993,1.022)--(4.994,1.022)--(4.995,1.022)--(4.996,1.022)--(4.997,1.022)--(4.998,1.022)%
  --(4.999,1.022)--(5.000,1.022)--(5.001,1.022)--(5.002,1.022)--(5.003,1.022)--(5.004,1.022)%
  --(5.005,1.022)--(5.006,1.022)--(5.007,1.022)--(5.008,1.022)--(5.010,1.022)--(5.011,1.022)%
  --(5.012,1.022)--(5.013,1.022)--(5.014,1.022)--(5.015,1.022)--(5.016,1.022)--(5.017,1.022)%
  --(5.018,1.022)--(5.020,1.022)--(5.021,1.022)--(5.022,1.022)--(5.023,1.022)--(5.024,1.022)%
  --(5.025,1.022)--(5.026,1.022)--(5.027,1.022)--(5.028,1.022)--(5.029,1.022)--(5.030,1.022)%
  --(5.031,1.022)--(5.032,1.022)--(5.033,1.022)--(5.034,1.022)--(5.035,1.022)--(5.036,1.022)%
  --(5.037,1.022)--(5.038,1.022)--(5.039,1.022)--(5.040,1.022)--(5.041,1.022)--(5.043,1.022)%
  --(5.044,1.022)--(5.045,1.022)--(5.047,1.022)--(5.048,1.022)--(5.049,1.022)--(5.050,1.022)%
  --(5.051,1.022)--(5.052,1.022)--(5.053,1.022)--(5.054,1.022)--(5.055,1.022)--(5.056,1.022)%
  --(5.057,1.022)--(5.058,1.022)--(5.059,1.022)--(5.060,1.022)--(5.061,1.022)--(5.062,1.022)%
  --(5.063,1.022)--(5.064,1.022)--(5.065,1.022)--(5.066,1.022)--(5.067,1.022)--(5.068,1.022)%
  --(5.069,1.022)--(5.070,1.022)--(5.071,1.022)--(5.073,1.022)--(5.074,1.022)--(5.075,1.022)%
  --(5.076,1.022)--(5.077,1.022)--(5.078,1.022)--(5.079,1.022)--(5.080,1.022)--(5.081,1.022)%
  --(5.082,1.022)--(5.083,1.022)--(5.084,1.022)--(5.085,1.022)--(5.086,1.022)--(5.087,1.022)%
  --(5.088,1.022)--(5.089,1.022)--(5.090,1.022)--(5.091,1.022)--(5.092,1.022)--(5.093,1.022)%
  --(5.094,1.022)--(5.095,1.022)--(5.096,1.022)--(5.097,1.022)--(5.099,1.022)--(5.100,1.022)%
  --(5.101,1.022)--(5.102,1.022)--(5.103,1.022)--(5.104,1.022)--(5.105,1.022)--(5.106,1.022)%
  --(5.107,1.022)--(5.108,1.022)--(5.109,1.022)--(5.110,1.022)--(5.111,1.022)--(5.112,1.022)%
  --(5.113,1.022)--(5.114,1.022)--(5.115,1.022)--(5.116,1.022)--(5.117,1.022)--(5.118,1.022)%
  --(5.119,1.022)--(5.120,1.022)--(5.121,1.022)--(5.122,1.022)--(5.123,1.022)--(5.124,1.022)%
  --(5.125,1.022)--(5.126,1.022)--(5.127,1.022)--(5.128,1.022)--(5.130,1.022)--(5.131,1.022)%
  --(5.132,1.022)--(5.133,1.022)--(5.134,1.022)--(5.135,1.022)--(5.136,1.022)--(5.137,1.022)%
  --(5.138,1.022)--(5.139,1.022)--(5.140,1.022)--(5.141,1.022)--(5.142,1.022)--(5.143,1.022)%
  --(5.144,1.022)--(5.145,1.022)--(5.146,1.022)--(5.147,1.022)--(5.148,1.022)--(5.149,1.022)%
  --(5.150,1.022)--(5.151,1.022)--(5.152,1.022)--(5.153,1.022)--(5.154,1.022)--(5.155,1.022)%
  --(5.156,1.022)--(5.157,1.022)--(5.158,1.022)--(5.159,1.022)--(5.160,1.022)--(5.161,1.022)%
  --(5.163,1.022)--(5.164,1.022)--(5.165,1.022)--(5.166,1.022)--(5.167,1.022)--(5.168,1.022)%
  --(5.169,1.022)--(5.170,1.022)--(5.171,1.022)--(5.172,1.022)--(5.173,1.022)--(5.174,1.022)%
  --(5.175,1.022)--(5.176,1.022)--(5.177,1.022)--(5.178,1.022)--(5.179,1.022)--(5.180,1.022)%
  --(5.181,1.022)--(5.182,1.022)--(5.183,1.022)--(5.184,1.022)--(5.185,1.022)--(5.186,1.022)%
  --(5.188,1.022)--(5.189,1.022)--(5.190,1.022)--(5.191,1.022)--(5.193,1.022)--(5.194,1.022)%
  --(5.195,1.022)--(5.196,1.022)--(5.197,1.022)--(5.198,1.022)--(5.199,1.022)--(5.200,1.022)%
  --(5.201,1.022)--(5.202,1.022)--(5.203,1.022)--(5.204,1.022)--(5.205,1.022)--(5.206,1.022)%
  --(5.207,1.022)--(5.208,1.022)--(5.209,1.022)--(5.210,1.022)--(5.211,1.022)--(5.212,1.022)%
  --(5.213,1.022)--(5.214,1.022)--(5.215,1.022)--(5.217,1.022)--(5.218,1.022)--(5.219,1.022)%
  --(5.220,1.022)--(5.221,1.022)--(5.222,1.022)--(5.223,1.022)--(5.224,1.022)--(5.225,1.022)%
  --(5.226,1.022)--(5.227,1.022)--(5.228,1.022)--(5.229,1.022)--(5.230,1.022)--(5.231,1.022)%
  --(5.232,1.022)--(5.233,1.022)--(5.234,1.022)--(5.235,1.022)--(5.236,1.022)--(5.237,1.022)%
  --(5.238,1.022)--(5.239,1.022)--(5.240,1.022)--(5.241,1.022)--(5.242,1.022)--(5.243,1.022)%
  --(5.244,1.022)--(5.245,1.022)--(5.247,1.022)--(5.248,1.022)--(5.250,1.022)--(5.251,1.022)%
  --(5.252,1.022)--(5.253,1.022)--(5.254,1.022)--(5.255,1.022)--(5.256,1.022)--(5.257,1.022)%
  --(5.258,1.022)--(5.259,1.022)--(5.260,1.022)--(5.261,1.022)--(5.262,1.022)--(5.263,1.022)%
  --(5.264,1.022)--(5.265,1.022)--(5.266,1.022)--(5.268,1.022)--(5.269,1.022)--(5.271,1.022)%
  --(5.272,1.022)--(5.273,1.022)--(5.274,1.022)--(5.275,1.022)--(5.276,1.022)--(5.277,1.022)%
  --(5.278,1.022)--(5.280,1.022)--(5.281,1.022)--(5.282,1.022)--(5.283,1.022)--(5.284,1.022)%
  --(5.285,1.022)--(5.286,1.022)--(5.287,1.022)--(5.288,1.022)--(5.289,1.022)--(5.290,1.022)%
  --(5.291,1.022)--(5.292,1.022)--(5.293,1.022)--(5.294,1.022)--(5.295,1.022)--(5.296,1.022)%
  --(5.297,1.022)--(5.298,1.022)--(5.299,1.022)--(5.300,1.022)--(5.301,1.022)--(5.302,1.022)%
  --(5.303,1.022)--(5.304,1.022)--(5.306,1.022)--(5.307,1.022)--(5.308,1.022)--(5.309,1.022)%
  --(5.310,1.022)--(5.311,1.022)--(5.312,1.022)--(5.313,1.022)--(5.314,1.022)--(5.315,1.022)%
  --(5.316,1.022)--(5.317,1.022)--(5.318,1.022)--(5.319,1.022)--(5.320,1.022)--(5.321,1.022)%
  --(5.322,1.022)--(5.323,1.022)--(5.324,1.022)--(5.325,1.022)--(5.326,1.022)--(5.327,1.022)%
  --(5.328,1.022)--(5.329,1.022)--(5.330,1.022)--(5.331,1.022)--(5.333,1.022)--(5.334,1.022)%
  --(5.335,1.022)--(5.336,1.022)--(5.337,1.022)--(5.338,1.022)--(5.339,1.022)--(5.340,1.022)%
  --(5.341,1.022)--(5.342,1.022)--(5.343,1.022)--(5.344,1.022)--(5.345,1.022)--(5.346,1.022)%
  --(5.347,1.022)--(5.348,1.022)--(5.349,1.022)--(5.350,1.022)--(5.351,1.022)--(5.352,1.022)%
  --(5.353,1.022)--(5.354,1.022)--(5.355,1.022)--(5.356,1.022)--(5.357,1.022)--(5.358,1.022)%
  --(5.359,1.022)--(5.360,1.022)--(5.361,1.022)--(5.363,1.022)--(5.364,1.022)--(5.365,1.022)%
  --(5.366,1.022)--(5.367,1.022)--(5.368,1.022)--(5.369,1.022)--(5.370,1.022)--(5.371,1.022)%
  --(5.372,1.022)--(5.373,1.022)--(5.374,1.022)--(5.375,1.022)--(5.376,1.022)--(5.377,1.022)%
  --(5.378,1.022)--(5.379,1.022)--(5.380,1.022)--(5.381,1.022)--(5.382,1.022)--(5.383,1.022)%
  --(5.384,1.022)--(5.385,1.022)--(5.386,1.022)--(5.387,1.022)--(5.388,1.022)--(5.389,1.022)%
  --(5.390,1.022)--(5.391,1.022)--(5.392,1.022)--(5.393,1.022)--(5.394,1.022)--(5.395,1.022)%
  --(5.397,1.022)--(5.398,1.022)--(5.399,1.022)--(5.400,1.022)--(5.401,1.022)--(5.402,1.022)%
  --(5.403,1.022)--(5.404,1.022)--(5.405,1.022)--(5.406,1.022)--(5.407,1.022)--(5.408,1.022)%
  --(5.409,1.022)--(5.410,1.022)--(5.411,1.022)--(5.412,1.022)--(5.413,1.022)--(5.414,1.022)%
  --(5.415,1.022)--(5.416,1.022)--(5.417,1.022)--(5.418,1.022)--(5.419,1.022)--(5.421,1.022)%
  --(5.422,1.022)--(5.423,1.022)--(5.425,1.022)--(5.426,1.022)--(5.427,1.022)--(5.428,1.022)%
  --(5.429,1.022)--(5.430,1.022)--(5.431,1.022)--(5.432,1.022)--(5.433,1.022)--(5.434,1.022)%
  --(5.435,1.022)--(5.436,1.022)--(5.437,1.022)--(5.438,1.022)--(5.439,1.022)--(5.440,1.022)%
  --(5.441,1.022)--(5.442,1.022)--(5.443,1.022)--(5.444,1.022)--(5.445,1.022)--(5.446,1.022)%
  --(5.447,1.022)--(5.448,1.022)--(5.450,1.022)--(5.451,1.022)--(5.452,1.022)--(5.454,1.022)%
  --(5.455,1.022)--(5.456,1.022)--(5.457,1.022)--(5.458,1.022)--(5.459,1.022)--(5.460,1.022)%
  --(5.461,1.022)--(5.462,1.022)--(5.463,1.022)--(5.464,1.022)--(5.465,1.022)--(5.466,1.022)%
  --(5.467,1.022)--(5.468,1.022)--(5.469,1.022)--(5.470,1.022)--(5.471,1.022)--(5.472,1.022)%
  --(5.473,1.022)--(5.474,1.022)--(5.475,1.022)--(5.476,1.022)--(5.477,1.022)--(5.479,1.022)%
  --(5.480,1.022)--(5.481,1.022)--(5.483,1.022)--(5.484,1.022)--(5.485,1.022)--(5.486,1.022)%
  --(5.487,1.022)--(5.488,1.022)--(5.489,1.022)--(5.490,1.022)--(5.491,1.022)--(5.492,1.022)%
  --(5.493,1.022)--(5.494,1.022)--(5.495,1.022)--(5.496,1.022)--(5.497,1.022)--(5.498,1.022)%
  --(5.499,1.022)--(5.500,1.022)--(5.501,1.022)--(5.502,1.022)--(5.503,1.022)--(5.504,1.022)%
  --(5.505,1.022)--(5.507,1.022)--(5.508,1.022)--(5.509,1.022)--(5.510,1.022)--(5.512,1.022)%
  --(5.513,1.022)--(5.514,1.022)--(5.515,1.022)--(5.516,1.022)--(5.517,1.022)--(5.518,1.022)%
  --(5.519,1.022)--(5.520,1.022)--(5.521,1.022)--(5.522,1.022)--(5.523,1.022)--(5.524,1.022)%
  --(5.525,1.022)--(5.526,1.022)--(5.527,1.022)--(5.528,1.022)--(5.529,1.022)--(5.530,1.022)%
  --(5.531,1.022)--(5.532,1.022)--(5.533,1.022)--(5.534,1.022)--(5.535,1.022)--(5.537,1.022)%
  --(5.538,1.022)--(5.539,1.022)--(5.540,1.022)--(5.541,1.022)--(5.542,1.022)--(5.543,1.022)%
  --(5.544,1.022)--(5.545,1.022)--(5.546,1.022)--(5.547,1.022)--(5.548,1.022)--(5.549,1.022)%
  --(5.550,1.022)--(5.551,1.022)--(5.552,1.022)--(5.553,1.022)--(5.554,1.022)--(5.555,1.022)%
  --(5.556,1.022)--(5.557,1.022)--(5.558,1.022)--(5.559,1.022)--(5.560,1.022)--(5.561,1.022)%
  --(5.562,1.022)--(5.563,1.022)--(5.564,1.022)--(5.566,1.022)--(5.567,1.022)--(5.568,1.022)%
  --(5.569,1.022)--(5.570,1.022)--(5.571,1.022)--(5.572,1.022)--(5.573,1.022)--(5.574,1.022)%
  --(5.575,1.022)--(5.576,1.022)--(5.577,1.022)--(5.578,1.022)--(5.579,1.022)--(5.580,1.022)%
  --(5.581,1.022)--(5.582,1.022)--(5.583,1.022)--(5.584,1.022)--(5.585,1.022)--(5.586,1.022)%
  --(5.587,1.022)--(5.588,1.022)--(5.589,1.022)--(5.590,1.022)--(5.591,1.022)--(5.592,1.022)%
  --(5.593,1.022)--(5.594,1.022)--(5.595,1.022)--(5.596,1.022)--(5.597,1.022)--(5.599,1.022)%
  --(5.600,1.022)--(5.601,1.022)--(5.602,1.022)--(5.603,1.022)--(5.604,1.022)--(5.605,1.022)%
  --(5.606,1.022)--(5.607,1.022)--(5.608,1.022)--(5.609,1.022)--(5.610,1.022)--(5.611,1.022)%
  --(5.612,1.022)--(5.613,1.022)--(5.614,1.022)--(5.615,1.022)--(5.616,1.022)--(5.617,1.022)%
  --(5.618,1.022)--(5.619,1.022)--(5.620,1.022)--(5.621,1.022)--(5.622,1.022)--(5.623,1.022)%
  --(5.625,1.022)--(5.626,1.022)--(5.627,1.022)--(5.628,1.022)--(5.629,1.022)--(5.630,1.022)%
  --(5.631,1.022)--(5.632,1.022)--(5.633,1.022)--(5.634,1.022)--(5.635,1.022)--(5.636,1.022)%
  --(5.637,1.022)--(5.638,1.022)--(5.639,1.022)--(5.640,1.022)--(5.641,1.022)--(5.642,1.022)%
  --(5.643,1.022)--(5.644,1.022)--(5.645,1.022)--(5.646,1.022)--(5.647,1.022)--(5.648,1.022)%
  --(5.649,1.022)--(5.650,1.022)--(5.652,1.022)--(5.653,1.022)--(5.654,1.022)--(5.655,1.022)%
  --(5.656,1.022)--(5.657,1.022)--(5.658,1.022)--(5.659,1.022)--(5.660,1.022)--(5.661,1.022)%
  --(5.662,1.022)--(5.663,1.022)--(5.664,1.022)--(5.665,1.022)--(5.666,1.022)--(5.667,1.022)%
  --(5.668,1.022)--(5.669,1.022)--(5.670,1.022)--(5.671,1.022)--(5.672,1.022)--(5.673,1.022)%
  --(5.674,1.022)--(5.675,1.022)--(5.676,1.022)--(5.677,1.022)--(5.678,1.022)--(5.679,1.022)%
  --(5.680,1.022)--(5.681,1.022)--(5.683,1.022)--(5.684,1.022)--(5.685,1.022)--(5.686,1.022)%
  --(5.687,1.022)--(5.688,1.022)--(5.689,1.022)--(5.690,1.022)--(5.691,1.022)--(5.692,1.022)%
  --(5.693,1.022)--(5.694,1.022)--(5.695,1.022)--(5.696,1.022)--(5.697,1.022)--(5.698,1.022)%
  --(5.699,1.022)--(5.700,1.022)--(5.701,1.022)--(5.702,1.022)--(5.703,1.022)--(5.704,1.022)%
  --(5.705,1.022)--(5.706,1.022)--(5.707,1.022)--(5.708,1.022)--(5.709,1.022)--(5.710,1.022)%
  --(5.711,1.022)--(5.712,1.022)--(5.713,1.022)--(5.714,1.022)--(5.716,1.022)--(5.717,1.022)%
  --(5.718,1.022)--(5.719,1.022)--(5.720,1.022)--(5.721,1.022)--(5.722,1.022)--(5.723,1.022)%
  --(5.724,1.022)--(5.725,1.022)--(5.726,1.022)--(5.727,1.022)--(5.728,1.022)--(5.729,1.022)%
  --(5.730,1.022)--(5.731,1.022)--(5.732,1.022)--(5.733,1.022)--(5.734,1.022)--(5.735,1.022)%
  --(5.736,1.022)--(5.737,1.022)--(5.738,1.022)--(5.739,1.022)--(5.741,1.022)--(5.742,1.022)%
  --(5.743,1.022)--(5.744,1.022)--(5.746,1.022)--(5.747,1.022)--(5.748,1.022)--(5.749,1.022)%
  --(5.750,1.022)--(5.751,1.022)--(5.752,1.022)--(5.753,1.022)--(5.754,1.022)--(5.755,1.022)%
  --(5.756,1.022)--(5.757,1.022)--(5.758,1.022)--(5.759,1.022)--(5.760,1.022)--(5.761,1.022)%
  --(5.762,1.022)--(5.763,1.022)--(5.764,1.022)--(5.765,1.022)--(5.766,1.022)--(5.767,1.022)%
  --(5.768,1.022)--(5.770,1.022)--(5.771,1.022)--(5.772,1.022)--(5.773,1.022)--(5.774,1.022)%
  --(5.775,1.022)--(5.776,1.022)--(5.777,1.022)--(5.778,1.022)--(5.779,1.022)--(5.780,1.022)%
  --(5.781,1.022)--(5.782,1.022)--(5.783,1.022)--(5.784,1.022)--(5.785,1.022)--(5.786,1.022)%
  --(5.787,1.022)--(5.788,1.022)--(5.789,1.022)--(5.790,1.022)--(5.791,1.022)--(5.792,1.022)%
  --(5.793,1.022)--(5.794,1.022)--(5.795,1.022)--(5.796,1.022)--(5.797,1.022)--(5.798,1.022)%
  --(5.800,1.022)--(5.801,1.022)--(5.802,1.022)--(5.803,1.022)--(5.804,1.022)--(5.805,1.022)%
  --(5.806,1.022)--(5.807,1.022)--(5.808,1.022)--(5.809,1.022)--(5.810,1.022)--(5.811,1.022)%
  --(5.812,1.022)--(5.813,1.022)--(5.814,1.022)--(5.815,1.022)--(5.816,1.022)--(5.817,1.022)%
  --(5.818,1.022)--(5.819,1.022)--(5.820,1.022)--(5.822,1.022)--(5.824,1.022)--(5.825,1.022)%
  --(5.826,1.022)--(5.827,1.022)--(5.828,1.022)--(5.829,1.022)--(5.830,1.022)--(5.831,1.022)%
  --(5.833,1.022)--(5.834,1.022)--(5.835,1.022)--(5.836,1.022)--(5.837,1.022)--(5.838,1.022)%
  --(5.839,1.022)--(5.840,1.022)--(5.841,1.022)--(5.842,1.022)--(5.843,1.022)--(5.844,1.022)%
  --(5.845,1.022)--(5.846,1.022)--(5.847,1.022)--(5.848,1.022)--(5.849,1.022)--(5.850,1.022)%
  --(5.851,1.022)--(5.852,1.022)--(5.853,1.022)--(5.854,1.022)--(5.855,1.022)--(5.857,1.022)%
  --(5.858,1.022)--(5.859,1.022)--(5.860,1.022)--(5.861,1.022)--(5.862,1.022)--(5.863,1.022)%
  --(5.864,1.022)--(5.865,1.022)--(5.866,1.022)--(5.867,1.022)--(5.868,1.022)--(5.869,1.022)%
  --(5.870,1.022)--(5.871,1.022)--(5.872,1.022)--(5.873,1.022)--(5.874,1.022)--(5.875,1.022)%
  --(5.876,1.022)--(5.877,1.022)--(5.878,1.022)--(5.879,1.022)--(5.880,1.022)--(5.881,1.022)%
  --(5.882,1.022)--(5.883,1.022)--(5.884,1.022)--(5.885,1.022)--(5.887,1.022)--(5.888,1.022)%
  --(5.889,1.022)--(5.890,1.022)--(5.891,1.022)--(5.892,1.022)--(5.893,1.022)--(5.894,1.022)%
  --(5.895,1.022)--(5.896,1.022)--(5.897,1.022)--(5.898,1.022)--(5.899,1.022)--(5.900,1.022)%
  --(5.901,1.022)--(5.902,1.022)--(5.903,1.022)--(5.904,1.022)--(5.905,1.022)--(5.906,1.022)%
  --(5.907,1.022)--(5.908,1.022)--(5.909,1.022)--(5.911,1.022)--(5.912,1.022)--(5.913,1.022)%
  --(5.914,1.022)--(5.915,1.022)--(5.916,1.022)--(5.917,1.022)--(5.918,1.022)--(5.920,1.022)%
  --(5.921,1.022)--(5.922,1.022)--(5.923,1.022)--(5.924,1.022)--(5.925,1.022)--(5.926,1.022)%
  --(5.927,1.022)--(5.928,1.022)--(5.929,1.022)--(5.930,1.022)--(5.931,1.022)--(5.932,1.022)%
  --(5.933,1.022)--(5.934,1.022)--(5.935,1.022)--(5.936,1.022)--(5.937,1.022)--(5.938,1.022)%
  --(5.939,1.022)--(5.940,1.022)--(5.941,1.022)--(5.942,1.022)--(5.943,1.022)--(5.945,1.022)%
  --(5.946,1.022)--(5.947,1.022)--(5.948,1.022)--(5.950,1.022)--(5.951,1.022)--(5.952,1.022)%
  --(5.953,1.022)--(5.954,1.022)--(5.955,1.022)--(5.956,1.022)--(5.957,1.022)--(5.958,1.022)%
  --(5.959,1.022)--(5.960,1.022)--(5.961,1.022)--(5.962,1.022)--(5.963,1.022)--(5.964,1.022)%
  --(5.965,1.022)--(5.966,1.022)--(5.967,1.022)--(5.968,1.022)--(5.969,1.022)--(5.970,1.022)%
  --(5.971,1.022)--(5.972,1.022)--(5.974,1.022)--(5.975,1.022)--(5.976,1.022)--(5.977,1.022)%
  --(5.979,1.022)--(5.980,1.022)--(5.981,1.022)--(5.982,1.022)--(5.983,1.022)--(5.984,1.022)%
  --(5.985,1.022)--(5.986,1.022)--(5.987,1.022)--(5.988,1.022)--(5.989,1.022)--(5.990,1.022)%
  --(5.991,1.022)--(5.992,1.022)--(5.993,1.022)--(5.994,1.022)--(5.995,1.022)--(5.996,1.022)%
  --(5.997,1.022)--(5.998,1.022)--(5.999,1.022)--(6.000,1.022)--(6.001,1.022)--(6.003,1.022)%
  --(6.004,1.022)--(6.005,1.022)--(6.007,1.022)--(6.008,1.022)--(6.009,1.022)--(6.010,1.022)%
  --(6.011,1.022)--(6.012,1.022)--(6.013,1.022)--(6.014,1.022)--(6.015,1.022)--(6.016,1.022)%
  --(6.017,1.022)--(6.018,1.022)--(6.019,1.022)--(6.020,1.022)--(6.021,1.022)--(6.022,1.022)%
  --(6.023,1.022)--(6.024,1.022)--(6.025,1.022)--(6.026,1.022)--(6.027,1.022)--(6.028,1.022)%
  --(6.029,1.022)--(6.031,1.022)--(6.032,1.022)--(6.033,1.022)--(6.035,1.022)--(6.036,1.022)%
  --(6.037,1.022)--(6.038,1.022)--(6.039,1.022)--(6.040,1.022)--(6.041,1.022)--(6.042,1.022)%
  --(6.043,1.022)--(6.044,1.022)--(6.045,1.022)--(6.046,1.022)--(6.047,1.022)--(6.048,1.022)%
  --(6.049,1.022)--(6.050,1.022)--(6.051,1.022)--(6.052,1.022)--(6.053,1.022)--(6.054,1.022)%
  --(6.055,1.022)--(6.056,1.022)--(6.057,1.022)--(6.058,1.022)--(6.060,1.022)--(6.061,1.022)%
  --(6.062,1.022)--(6.064,1.022)--(6.065,1.022)--(6.066,1.022)--(6.067,1.022)--(6.068,1.022)%
  --(6.069,1.022)--(6.070,1.022)--(6.071,1.022)--(6.072,1.022)--(6.073,1.022)--(6.074,1.022)%
  --(6.075,1.022)--(6.076,1.022)--(6.077,1.022)--(6.078,1.022)--(6.080,1.022)--(6.081,1.022)%
  --(6.082,1.022)--(6.083,1.022)--(6.084,1.022)--(6.085,1.022)--(6.086,1.022)--(6.087,1.022)%
  --(6.089,1.022)--(6.090,1.022)--(6.091,1.022)--(6.093,1.022)--(6.094,1.022)--(6.095,1.022)%
  --(6.096,1.022)--(6.098,1.022)--(6.099,1.022)--(6.100,1.022)--(6.101,1.022)--(6.102,1.022)%
  --(6.103,1.022)--(6.104,1.022)--(6.105,1.022)--(6.106,1.022)--(6.107,1.022)--(6.108,1.022)%
  --(6.109,1.022)--(6.110,1.022)--(6.112,1.022)--(6.113,1.022)--(6.114,1.022)--(6.115,1.022)%
  --(6.117,1.022)--(6.118,1.022)--(6.119,1.022)--(6.121,1.022)--(6.122,1.022)--(6.123,1.022)%
  --(6.124,1.022)--(6.126,1.022)--(6.127,1.022)--(6.128,1.022)--(6.129,1.022)--(6.131,1.022)%
  --(6.132,1.022)--(6.133,1.022)--(6.134,1.022)--(6.135,1.022)--(6.136,1.022)--(6.137,1.022)%
  --(6.138,1.022)--(6.139,1.022)--(6.140,1.022)--(6.141,1.022)--(6.143,1.022)--(6.144,1.022)%
  --(6.145,1.022)--(6.146,1.022)--(6.147,1.022)--(6.148,1.022)--(6.150,1.022)--(6.151,1.022)%
  --(6.152,1.022)--(6.153,1.022)--(6.155,1.022)--(6.156,1.022)--(6.157,1.022)--(6.158,1.022)%
  --(6.159,1.022)--(6.160,1.022)--(6.161,1.022)--(6.162,1.022)--(6.163,1.022)--(6.164,1.022)%
  --(6.165,1.022)--(6.166,1.022)--(6.167,1.022)--(6.168,1.022)--(6.169,1.022)--(6.170,1.022)%
  --(6.171,1.022)--(6.172,1.022)--(6.173,1.022)--(6.174,1.022)--(6.175,1.022)--(6.176,1.022)%
  --(6.178,1.022)--(6.179,1.022)--(6.180,1.022)--(6.181,1.022)--(6.183,1.022)--(6.184,1.022)%
  --(6.185,1.022)--(6.186,1.022)--(6.187,1.022)--(6.188,1.022)--(6.189,1.022)--(6.190,1.022)%
  --(6.191,1.022)--(6.192,1.022)--(6.193,1.022)--(6.194,1.022)--(6.195,1.022)--(6.196,1.022)%
  --(6.197,1.022)--(6.198,1.022)--(6.199,1.022)--(6.200,1.022)--(6.201,1.022)--(6.202,1.022)%
  --(6.203,1.022)--(6.204,1.022)--(6.205,1.022)--(6.207,1.022)--(6.208,1.022)--(6.209,1.022)%
  --(6.211,1.022)--(6.212,1.022)--(6.213,1.022)--(6.214,1.022)--(6.215,1.022)--(6.216,1.022)%
  --(6.217,1.022)--(6.218,1.022)--(6.219,1.022)--(6.220,1.022)--(6.221,1.022)--(6.222,1.022)%
  --(6.223,1.022)--(6.224,1.022)--(6.225,1.022)--(6.226,1.022)--(6.227,1.022)--(6.228,1.022)%
  --(6.229,1.022)--(6.230,1.022)--(6.231,1.022)--(6.232,1.022)--(6.233,1.022)--(6.234,1.022)%
  --(6.236,1.022)--(6.237,1.022)--(6.239,1.022)--(6.240,1.022)--(6.241,1.022)--(6.242,1.022)%
  --(6.243,1.022)--(6.244,1.022)--(6.245,1.022)--(6.246,1.022)--(6.247,1.022)--(6.248,1.022)%
  --(6.249,1.022)--(6.250,1.022)--(6.251,1.022)--(6.252,1.022)--(6.253,1.022)--(6.254,1.022)%
  --(6.255,1.022)--(6.256,1.022)--(6.257,1.022)--(6.258,1.022)--(6.259,1.022)--(6.260,1.022)%
  --(6.261,1.022)--(6.262,1.022)--(6.264,1.022)--(6.265,1.022)--(6.267,1.022)--(6.268,1.022)%
  --(6.269,1.022)--(6.270,1.022)--(6.272,1.022)--(6.273,1.022)--(6.274,1.022)--(6.275,1.022)%
  --(6.276,1.022)--(6.277,1.022)--(6.278,1.022)--(6.279,1.022)--(6.280,1.022)--(6.281,1.022)%
  --(6.282,1.022)--(6.283,1.022)--(6.284,1.022)--(6.285,1.022)--(6.286,1.022)--(6.288,1.022)%
  --(6.289,1.022)--(6.291,1.022)--(6.292,1.022)--(6.293,1.022)--(6.294,1.022)--(6.295,1.022)%
  --(6.296,1.022)--(6.297,1.022)--(6.299,1.022)--(6.300,1.022)--(6.301,1.022)--(6.302,1.022)%
  --(6.303,1.022)--(6.304,1.022)--(6.305,1.022)--(6.306,1.022)--(6.307,1.022)--(6.308,1.022)%
  --(6.309,1.022)--(6.310,1.022)--(6.311,1.022)--(6.312,1.022)--(6.313,1.022)--(6.314,1.022)%
  --(6.315,1.022)--(6.316,1.022)--(6.317,1.022)--(6.318,1.022)--(6.319,1.022)--(6.320,1.022)%
  --(6.321,1.022)--(6.322,1.022)--(6.324,1.022)--(6.325,1.022)--(6.326,1.022)--(6.327,1.022)%
  --(6.328,1.022)--(6.329,1.022)--(6.330,1.022)--(6.331,1.022)--(6.332,1.022)--(6.333,1.022)%
  --(6.334,1.022)--(6.335,1.022)--(6.336,1.022)--(6.337,1.022)--(6.338,1.022)--(6.339,1.022)%
  --(6.340,1.022)--(6.341,1.022)--(6.342,1.022)--(6.343,1.022)--(6.344,1.022)--(6.345,1.022)%
  --(6.346,1.022)--(6.347,1.022)--(6.348,1.022)--(6.349,1.022)--(6.350,1.022)--(6.351,1.022)%
  --(6.352,1.022)--(6.353,1.022)--(6.354,1.022)--(6.355,1.022)--(6.356,1.022)--(6.357,1.022)%
  --(6.358,1.022)--(6.359,1.022)--(6.360,1.022)--(6.361,1.022)--(6.362,1.022)--(6.363,1.022)%
  --(6.364,1.022)--(6.365,1.022)--(6.366,1.022)--(6.367,1.022)--(6.368,1.022)--(6.369,1.022)%
  --(6.370,1.022)--(6.371,1.022)--(6.372,1.022)--(6.373,1.022)--(6.374,1.022)--(6.375,1.022)%
  --(6.376,1.022)--(6.377,1.022)--(6.378,1.022)--(6.379,1.022)--(6.381,1.022)--(6.382,1.022)%
  --(6.383,1.022)--(6.384,1.022)--(6.385,1.022)--(6.386,1.022)--(6.387,1.022)--(6.389,1.022)%
  --(6.391,1.022)--(6.392,1.022)--(6.393,1.022)--(6.394,1.022)--(6.395,1.022)--(6.396,1.022)%
  --(6.397,1.022)--(6.398,1.022)--(6.399,1.022)--(6.400,1.022)--(6.401,1.022)--(6.402,1.022)%
  --(6.403,1.022)--(6.404,1.022)--(6.405,1.022)--(6.406,1.022)--(6.408,1.022)--(6.409,1.022)%
  --(6.410,1.022)--(6.411,1.022)--(6.413,1.022)--(6.414,1.022)--(6.415,1.022)--(6.416,1.022)%
  --(6.418,1.022)--(6.420,1.022)--(6.421,1.022)--(6.422,1.022)--(6.423,1.022)--(6.424,1.022)%
  --(6.425,1.022)--(6.426,1.022)--(6.427,1.022)--(6.428,1.022)--(6.429,1.022)--(6.430,1.022)%
  --(6.431,1.022)--(6.432,1.022)--(6.433,1.022)--(6.434,1.022)--(6.436,1.022)--(6.437,1.022)%
  --(6.438,1.022)--(6.439,1.022)--(6.441,1.022)--(6.442,1.022)--(6.443,1.022)--(6.444,1.022)%
  --(6.445,1.022)--(6.447,1.022)--(6.448,1.022)--(6.449,1.022)--(6.450,1.022)--(6.451,1.022)%
  --(6.452,1.022)--(6.453,1.022)--(6.454,1.022)--(6.455,1.022)--(6.456,1.022)--(6.457,1.022)%
  --(6.458,1.022)--(6.459,1.022)--(6.460,1.022)--(6.462,1.022)--(6.464,1.022)--(6.465,1.022)%
  --(6.466,1.022)--(6.467,1.022)--(6.468,1.022)--(6.470,1.022)--(6.471,1.022)--(6.472,1.022)%
  --(6.473,1.022)--(6.474,1.022)--(6.476,1.022)--(6.477,1.022)--(6.478,1.022)--(6.479,1.022)%
  --(6.480,1.022)--(6.481,1.022)--(6.482,1.022)--(6.483,1.022)--(6.484,1.022)--(6.485,1.022)%
  --(6.486,1.022)--(6.487,1.022)--(6.488,1.022)--(6.489,1.022)--(6.490,1.022)--(6.492,1.022)%
  --(6.493,1.022)--(6.494,1.022)--(6.495,1.022)--(6.496,1.022)--(6.497,1.022)--(6.499,1.022)%
  --(6.500,1.022)--(6.501,1.022)--(6.502,1.022)--(6.503,1.022)--(6.504,1.022)--(6.505,1.022)%
  --(6.506,1.022)--(6.507,1.022)--(6.508,1.022)--(6.509,1.022)--(6.510,1.022)--(6.511,1.022)%
  --(6.512,1.022)--(6.513,1.022)--(6.514,1.022)--(6.515,1.022)--(6.516,1.022)--(6.517,1.022)%
  --(6.518,1.022)--(6.520,1.022)--(6.521,1.022)--(6.523,1.022)--(6.524,1.022)--(6.525,1.022)%
  --(6.526,1.022)--(6.527,1.022)--(6.528,1.022)--(6.529,1.022)--(6.530,1.022)--(6.531,1.022)%
  --(6.532,1.022)--(6.533,1.022)--(6.534,1.022)--(6.535,1.022)--(6.536,1.022)--(6.537,1.022)%
  --(6.538,1.022)--(6.539,1.022)--(6.540,1.022)--(6.541,1.022)--(6.542,1.022)--(6.543,1.022)%
  --(6.544,1.022)--(6.545,1.022)--(6.546,1.022)--(6.547,1.022)--(6.548,1.022)--(6.549,1.022)%
  --(6.550,1.022)--(6.551,1.022)--(6.552,1.022)--(6.553,1.022)--(6.554,1.022)--(6.556,1.022)%
  --(6.557,1.022)--(6.558,1.022)--(6.559,1.022)--(6.560,1.022)--(6.561,1.022)--(6.562,1.022)%
  --(6.564,1.022)--(6.565,1.022)--(6.566,1.022)--(6.567,1.022)--(6.568,1.022)--(6.569,1.022)%
  --(6.570,1.022)--(6.571,1.022)--(6.572,1.022)--(6.573,1.022)--(6.574,1.022)--(6.575,1.022)%
  --(6.577,1.022)--(6.578,1.022)--(6.579,1.022)--(6.581,1.022)--(6.582,1.022)--(6.584,1.022)%
  --(6.585,1.022)--(6.587,1.022)--(6.588,1.022)--(6.589,1.022)--(6.591,1.022)--(6.592,1.022)%
  --(6.593,1.022)--(6.594,1.022)--(6.595,1.022)--(6.596,1.022)--(6.597,1.022)--(6.598,1.022)%
  --(6.599,1.022)--(6.600,1.022)--(6.601,1.022)--(6.602,1.022)--(6.603,1.022)--(6.604,1.022)%
  --(6.605,1.022)--(6.606,1.022)--(6.607,1.022)--(6.608,1.022)--(6.609,1.022)--(6.610,1.022)%
  --(6.611,1.022)--(6.612,1.022)--(6.613,1.022)--(6.614,1.022)--(6.615,1.022)--(6.617,1.022)%
  --(6.618,1.022)--(6.619,1.022)--(6.620,1.022)--(6.621,1.022)--(6.622,1.022)--(6.623,1.022)%
  --(6.624,1.022)--(6.625,1.022)--(6.626,1.022)--(6.627,1.022)--(6.628,1.022)--(6.629,1.022)%
  --(6.630,1.022)--(6.631,1.022)--(6.632,1.022)--(6.633,1.022)--(6.634,1.022)--(6.635,1.022)%
  --(6.636,1.022)--(6.637,1.022)--(6.638,1.022)--(6.639,1.022)--(6.640,1.022)--(6.641,1.022)%
  --(6.643,1.022)--(6.644,1.022)--(6.645,1.022)--(6.646,1.022)--(6.647,1.022)--(6.648,1.022)%
  --(6.649,1.022)--(6.650,1.022)--(6.651,1.022)--(6.652,1.022)--(6.653,1.022)--(6.654,1.022)%
  --(6.655,1.022)--(6.656,1.022)--(6.657,1.022)--(6.658,1.022)--(6.659,1.022)--(6.660,1.022)%
  --(6.661,1.022)--(6.662,1.022)--(6.663,1.022)--(6.664,1.022)--(6.665,1.022)--(6.666,1.022)%
  --(6.667,1.022)--(6.668,1.022)--(6.669,1.022)--(6.670,1.022)--(6.671,1.022)--(6.673,1.022)%
  --(6.675,1.022)--(6.676,1.022)--(6.678,1.022)--(6.679,1.022)--(6.680,1.022)--(6.681,1.022)%
  --(6.682,1.022)--(6.683,1.022)--(6.684,1.022)--(6.685,1.022)--(6.686,1.022)--(6.687,1.022)%
  --(6.688,1.022)--(6.689,1.022)--(6.690,1.022)--(6.691,1.022)--(6.692,1.022)--(6.693,1.022)%
  --(6.694,1.022)--(6.696,1.022)--(6.697,1.022)--(6.698,1.022)--(6.699,1.022)--(6.700,1.022)%
  --(6.701,1.022)--(6.702,1.022)--(6.703,1.022)--(6.704,1.022)--(6.705,1.022)--(6.707,1.022)%
  --(6.708,1.022)--(6.709,1.022)--(6.710,1.022)--(6.711,1.022)--(6.712,1.022)--(6.713,1.022)%
  --(6.714,1.022)--(6.715,1.022)--(6.716,1.022)--(6.717,1.022)--(6.718,1.022)--(6.719,1.022)%
  --(6.720,1.022)--(6.721,1.022)--(6.722,1.022)--(6.723,1.022)--(6.724,1.022)--(6.725,1.022)%
  --(6.726,1.022)--(6.727,1.022)--(6.728,1.022)--(6.729,1.022)--(6.731,1.022)--(6.732,1.022)%
  --(6.733,1.022)--(6.734,1.022)--(6.736,1.022)--(6.737,1.022)--(6.738,1.022)--(6.739,1.022)%
  --(6.740,1.022)--(6.741,1.022)--(6.742,1.022)--(6.743,1.022)--(6.744,1.022)--(6.745,1.022)%
  --(6.746,1.022)--(6.747,1.022)--(6.748,1.022)--(6.749,1.022)--(6.750,1.022)--(6.751,1.022)%
  --(6.752,1.022)--(6.753,1.022)--(6.754,1.022)--(6.755,1.022)--(6.756,1.022)--(6.757,1.022)%
  --(6.758,1.022)--(6.760,1.022)--(6.761,1.022)--(6.762,1.022)--(6.764,1.022)--(6.765,1.022)%
  --(6.766,1.022)--(6.767,1.022)--(6.768,1.022)--(6.769,1.022)--(6.770,1.022)--(6.771,1.022)%
  --(6.772,1.022)--(6.773,1.022)--(6.774,1.022)--(6.775,1.022)--(6.776,1.022)--(6.777,1.022)%
  --(6.778,1.022)--(6.779,1.022)--(6.780,1.022)--(6.781,1.022)--(6.782,1.022)--(6.783,1.022)%
  --(6.784,1.022)--(6.785,1.022)--(6.786,1.022)--(6.788,1.022)--(6.789,1.022)--(6.790,1.022)%
  --(6.792,1.022)--(6.793,1.022)--(6.794,1.022)--(6.795,1.022)--(6.797,1.022)--(6.798,1.022)%
  --(6.799,1.022)--(6.800,1.022)--(6.801,1.022)--(6.802,1.022)--(6.803,1.022)--(6.804,1.022)%
  --(6.805,1.022)--(6.806,1.022)--(6.807,1.022)--(6.809,1.022)--(6.810,1.022)--(6.811,1.022)%
  --(6.812,1.022)--(6.813,1.022)--(6.814,1.022)--(6.816,1.022)--(6.817,1.022)--(6.818,1.022)%
  --(6.819,1.022)--(6.820,1.022)--(6.821,1.022)--(6.822,1.022)--(6.824,1.022)--(6.825,1.022)%
  --(6.826,1.022)--(6.827,1.022)--(6.828,1.022)--(6.829,1.022)--(6.830,1.022)--(6.831,1.022)%
  --(6.832,1.022)--(6.833,1.022)--(6.834,1.022)--(6.835,1.022)--(6.836,1.022)--(6.837,1.022)%
  --(6.838,1.022)--(6.839,1.022)--(6.840,1.022)--(6.841,1.022)--(6.842,1.022)--(6.843,1.022)%
  --(6.844,1.022)--(6.845,1.022)--(6.846,1.022)--(6.847,1.022)--(6.849,1.022)--(6.850,1.022)%
  --(6.851,1.022)--(6.852,1.022)--(6.853,1.022)--(6.854,1.022)--(6.855,1.022)--(6.856,1.022)%
  --(6.857,1.022)--(6.858,1.022)--(6.859,1.022)--(6.860,1.022)--(6.861,1.022)--(6.862,1.022)%
  --(6.863,1.022)--(6.864,1.022)--(6.865,1.022)--(6.866,1.022)--(6.867,1.022)--(6.868,1.022)%
  --(6.869,1.022)--(6.870,1.022)--(6.871,1.022)--(6.873,1.022)--(6.874,1.022)--(6.875,1.022)%
  --(6.876,1.022)--(6.877,1.022)--(6.878,1.022)--(6.879,1.022)--(6.880,1.022)--(6.881,1.022)%
  --(6.882,1.022)--(6.883,1.022)--(6.884,1.022)--(6.885,1.022)--(6.886,1.022)--(6.887,1.022)%
  --(6.888,1.022)--(6.889,1.022)--(6.890,1.022)--(6.891,1.022)--(6.892,1.022)--(6.893,1.022)%
  --(6.894,1.022)--(6.895,1.022)--(6.896,1.022)--(6.897,1.022)--(6.898,1.022)--(6.899,1.022)%
  --(6.900,1.022)--(6.901,1.022)--(6.902,1.022)--(6.903,1.022)--(6.904,1.022)--(6.906,1.022)%
  --(6.907,1.022)--(6.908,1.022)--(6.909,1.022)--(6.910,1.022)--(6.911,1.022)--(6.912,1.022)%
  --(6.913,1.022)--(6.914,1.022)--(6.915,1.022)--(6.916,1.022)--(6.917,1.022)--(6.918,1.022)%
  --(6.919,1.022)--(6.920,1.022)--(6.921,1.022)--(6.922,1.022)--(6.923,1.022)--(6.924,1.022)%
  --(6.925,1.022)--(6.926,1.022)--(6.927,1.022)--(6.928,1.022)--(6.929,1.022)--(6.930,1.022)%
  --(6.931,1.022)--(6.933,1.022)--(6.934,1.022)--(6.935,1.022)--(6.937,1.022)--(6.938,1.022)%
  --(6.939,1.022)--(6.940,1.022)--(6.942,1.022)--(6.943,1.022)--(6.944,1.022)--(6.945,1.022)%
  --(6.946,1.022)--(6.947,1.022)--(6.948,1.022)--(6.949,1.022)--(6.950,1.022)--(6.951,1.022)%
  --(6.952,1.022)--(6.954,1.022)--(6.955,1.022)--(6.956,1.022)--(6.957,1.022)--(6.958,1.022)%
  --(6.959,1.022)--(6.960,1.022)--(6.962,1.022)--(6.963,1.022)--(6.964,1.022)--(6.966,1.022)%
  --(6.967,1.022)--(6.968,1.022)--(6.969,1.022)--(6.970,1.022)--(6.971,1.022)--(6.972,1.022)%
  --(6.973,1.022)--(6.974,1.022)--(6.975,1.022)--(6.976,1.022)--(6.977,1.022)--(6.978,1.022)%
  --(6.979,1.022)--(6.980,1.022)--(6.981,1.022)--(6.982,1.022)--(6.983,1.022)--(6.984,1.022)%
  --(6.985,1.022)--(6.986,1.022)--(6.987,1.022)--(6.988,1.022)--(6.989,1.022)--(6.991,1.022)%
  --(6.992,1.022)--(6.993,1.022)--(6.994,1.022)--(6.995,1.022)--(6.996,1.022)--(6.997,1.022)%
  --(6.998,1.022)--(6.999,1.022)--(7.000,1.022)--(7.001,1.022)--(7.002,1.022)--(7.003,1.022)%
  --(7.004,1.022)--(7.005,1.022)--(7.006,1.022)--(7.007,1.022)--(7.008,1.022)--(7.009,1.022)%
  --(7.010,1.022)--(7.011,1.022)--(7.012,1.022)--(7.013,1.022)--(7.014,1.022)--(7.015,1.022)%
  --(7.016,1.022)--(7.017,1.022)--(7.018,1.022)--(7.019,1.022)--(7.020,1.022)--(7.021,1.022)%
  --(7.022,1.022)--(7.024,1.022)--(7.025,1.022)--(7.026,1.022)--(7.027,1.022)--(7.028,1.022)%
  --(7.029,1.022)--(7.030,1.022)--(7.031,1.022)--(7.032,1.022)--(7.033,1.022)--(7.034,1.022)%
  --(7.035,1.022)--(7.036,1.022)--(7.037,1.022)--(7.038,1.022)--(7.039,1.022)--(7.041,1.022)%
  --(7.042,1.022)--(7.043,1.022)--(7.044,1.022)--(7.045,1.022)--(7.046,1.022)--(7.047,1.022)%
  --(7.048,1.022)--(7.050,1.022)--(7.051,1.022)--(7.052,1.022)--(7.053,1.022)--(7.054,1.022)%
  --(7.055,1.022)--(7.057,1.022)--(7.058,1.022)--(7.059,1.022)--(7.060,1.022)--(7.061,1.022)%
  --(7.062,1.022)--(7.063,1.022)--(7.064,1.022)--(7.065,1.022)--(7.066,1.022)--(7.067,1.022)%
  --(7.068,1.022)--(7.069,1.022)--(7.070,1.022)--(7.071,1.022)--(7.072,1.022)--(7.074,1.022)%
  --(7.075,1.022)--(7.076,1.022)--(7.077,1.022)--(7.078,1.022)--(7.079,1.022)--(7.080,1.022)%
  --(7.082,1.022)--(7.083,1.022)--(7.084,1.022)--(7.085,1.022)--(7.086,1.022)--(7.088,1.022)%
  --(7.090,1.022)--(7.091,1.022)--(7.092,1.022)--(7.093,1.022)--(7.094,1.022)--(7.095,1.022)%
  --(7.096,1.022)--(7.097,1.022)--(7.098,1.022)--(7.100,1.022)--(7.101,1.022)--(7.102,1.022)%
  --(7.103,1.022)--(7.104,1.022)--(7.105,1.022)--(7.107,1.022)--(7.108,1.022)--(7.109,1.022)%
  --(7.110,1.022)--(7.111,1.022)--(7.112,1.022)--(7.113,1.022)--(7.114,1.022)--(7.115,1.022)%
  --(7.116,1.022)--(7.117,1.022)--(7.118,1.022)--(7.119,1.022)--(7.120,1.022)--(7.121,1.022)%
  --(7.122,1.022)--(7.123,1.022)--(7.124,1.022)--(7.125,1.022)--(7.126,1.022)--(7.127,1.022)%
  --(7.128,1.022)--(7.129,1.022)--(7.130,1.022)--(7.131,1.022)--(7.132,1.022)--(7.133,1.022)%
  --(7.134,1.022)--(7.135,1.022)--(7.136,1.022)--(7.137,1.022)--(7.139,1.022)--(7.140,1.022)%
  --(7.141,1.022)--(7.142,1.022)--(7.143,1.022)--(7.144,1.022)--(7.145,1.022)--(7.146,1.022)%
  --(7.147,1.022)--(7.148,1.022)--(7.149,1.022)--(7.150,1.022)--(7.151,1.022)--(7.152,1.022)%
  --(7.153,1.022)--(7.154,1.022)--(7.155,1.022)--(7.156,1.022)--(7.157,1.022)--(7.158,1.022)%
  --(7.159,1.022)--(7.160,1.022)--(7.161,1.022)--(7.162,1.022)--(7.163,1.022)--(7.164,1.022)%
  --(7.166,1.022)--(7.167,1.022)--(7.168,1.022)--(7.170,1.022)--(7.171,1.022)--(7.172,1.022)%
  --(7.173,1.022)--(7.174,1.022)--(7.175,1.022)--(7.176,1.022)--(7.177,1.022)--(7.178,1.022)%
  --(7.179,1.022)--(7.180,1.022)--(7.181,1.022)--(7.182,1.022)--(7.183,1.022)--(7.184,1.022)%
  --(7.185,1.022)--(7.186,1.022)--(7.187,1.022)--(7.188,1.022)--(7.189,1.022)--(7.190,1.022)%
  --(7.191,1.022)--(7.192,1.022)--(7.193,1.022)--(7.195,1.022)--(7.196,1.022)--(7.197,1.022)%
  --(7.199,1.022)--(7.200,1.022)--(7.201,1.022)--(7.202,1.022)--(7.203,1.022)--(7.205,1.022)%
  --(7.206,1.022)--(7.207,1.022)--(7.208,1.022)--(7.209,1.022)--(7.210,1.022)--(7.211,1.022)%
  --(7.212,1.022)--(7.213,1.022)--(7.214,1.022)--(7.215,1.022)--(7.216,1.022)--(7.217,1.022)%
  --(7.218,1.022)--(7.219,1.022)--(7.220,1.022)--(7.221,1.022)--(7.222,1.022)--(7.224,1.022)%
  --(7.225,1.022)--(7.226,1.022)--(7.228,1.022)--(7.229,1.022)--(7.230,1.022)--(7.231,1.022)%
  --(7.233,1.022)--(7.234,1.022)--(7.235,1.022)--(7.236,1.022)--(7.237,1.022)--(7.238,1.022)%
  --(7.239,1.022)--(7.240,1.022)--(7.241,1.022)--(7.242,1.022)--(7.243,1.022)--(7.244,1.022)%
  --(7.245,1.022)--(7.247,1.022)--(7.248,1.022)--(7.249,1.022)--(7.250,1.022)--(7.252,1.022)%
  --(7.253,1.022)--(7.254,1.022)--(7.255,1.022)--(7.257,1.022)--(7.258,1.022)--(7.259,1.022)%
  --(7.260,1.022)--(7.262,1.022)--(7.264,1.022)--(7.265,1.022)--(7.266,1.022)--(7.267,1.022)%
  --(7.268,1.022)--(7.269,1.022)--(7.270,1.022)--(7.271,1.022)--(7.272,1.022)--(7.273,1.022)%
  --(7.274,1.022)--(7.275,1.022)--(7.276,1.022)--(7.277,1.022)--(7.278,1.022)--(7.280,1.022)%
  --(7.281,1.022)--(7.282,1.022)--(7.283,1.022)--(7.285,1.022)--(7.286,1.022)--(7.287,1.022)%
  --(7.288,1.022)--(7.289,1.022)--(7.290,1.022)--(7.291,1.022)--(7.292,1.022)--(7.294,1.022)%
  --(7.295,1.022)--(7.296,1.022)--(7.297,1.022)--(7.298,1.022)--(7.299,1.022)--(7.300,1.022)%
  --(7.301,1.022)--(7.302,1.022)--(7.304,1.022)--(7.305,1.022)--(7.306,1.022)--(7.307,1.022)%
  --(7.308,1.022)--(7.309,1.022)--(7.311,1.022)--(7.312,1.022)--(7.313,1.022)--(7.314,1.022)%
  --(7.315,1.022)--(7.316,1.022)--(7.317,1.022)--(7.318,1.022)--(7.319,1.022)--(7.320,1.022)%
  --(7.321,1.022)--(7.322,1.022)--(7.323,1.022)--(7.324,1.022)--(7.325,1.022)--(7.326,1.022)%
  --(7.327,1.022)--(7.328,1.022)--(7.329,1.022)--(7.330,1.022)--(7.331,1.022)--(7.332,1.022)%
  --(7.333,1.022)--(7.334,1.022)--(7.335,1.022)--(7.336,1.022)--(7.337,1.022)--(7.338,1.022)%
  --(7.339,1.022)--(7.340,1.022)--(7.341,1.022)--(7.343,1.022)--(7.344,1.022)--(7.345,1.022)%
  --(7.346,1.022)--(7.347,1.022)--(7.348,1.022)--(7.350,1.022)--(7.351,1.022)--(7.352,1.022)%
  --(7.353,1.022)--(7.354,1.022)--(7.355,1.022)--(7.356,1.022)--(7.357,1.022)--(7.358,1.022)%
  --(7.359,1.022)--(7.360,1.022)--(7.362,1.022)--(7.363,1.022)--(7.364,1.022)--(7.365,1.022)%
  --(7.366,1.022)--(7.367,1.022)--(7.369,1.022)--(7.370,1.022)--(7.371,1.022)--(7.373,1.022)%
  --(7.374,1.022)--(7.376,1.022)--(7.377,1.022)--(7.378,1.022)--(7.379,1.022)--(7.380,1.022)%
  --(7.381,1.022)--(7.382,1.022)--(7.383,1.022)--(7.384,1.022)--(7.385,1.022)--(7.386,1.022)%
  --(7.387,1.022)--(7.388,1.022)--(7.389,1.022)--(7.390,1.022)--(7.391,1.022)--(7.392,1.022)%
  --(7.393,1.022)--(7.394,1.022)--(7.395,1.022)--(7.396,1.022)--(7.397,1.022)--(7.398,1.022)%
  --(7.399,1.022)--(7.400,1.022)--(7.401,1.022)--(7.403,1.022)--(7.404,1.022)--(7.405,1.022)%
  --(7.406,1.022)--(7.407,1.022)--(7.408,1.022)--(7.409,1.022)--(7.410,1.022)--(7.411,1.022)%
  --(7.412,1.022)--(7.413,1.022)--(7.414,1.022)--(7.415,1.022)--(7.416,1.022)--(7.417,1.022)%
  --(7.418,1.022)--(7.419,1.022)--(7.420,1.022)--(7.421,1.022)--(7.422,1.022)--(7.423,1.022)%
  --(7.424,1.022)--(7.425,1.022)--(7.426,1.022)--(7.427,1.022)--(7.429,1.022)--(7.430,1.022)%
  --(7.431,1.022)--(7.432,1.022)--(7.433,1.022)--(7.434,1.022)--(7.435,1.022)--(7.436,1.022)%
  --(7.437,1.022)--(7.438,1.022)--(7.439,1.022)--(7.440,1.022)--(7.441,1.022)--(7.442,1.022)%
  --(7.443,1.022)--(7.444,1.022)--(7.445,1.022)--(7.446,1.022)--(7.447,1.022)--(7.448,1.022)%
  --(7.449,1.022)--(7.450,1.022)--(7.451,1.022)--(7.452,1.022)--(7.453,1.022)--(7.454,1.022)%
  --(7.455,1.022)--(7.456,1.022)--(7.458,1.022)--(7.459,1.022)--(7.460,1.022)--(7.461,1.022)%
  --(7.462,1.022)--(7.463,1.022)--(7.464,1.022)--(7.465,1.022)--(7.466,1.022)--(7.467,1.022)%
  --(7.468,1.022)--(7.470,1.022)--(7.471,1.022)--(7.472,1.022)--(7.473,1.022)--(7.474,1.022)%
  --(7.475,1.022)--(7.476,1.022)--(7.477,1.022)--(7.478,1.022)--(7.479,1.022)--(7.480,1.022)%
  --(7.482,1.022)--(7.483,1.022)--(7.484,1.022)--(7.485,1.022)--(7.486,1.022)--(7.487,1.022)%
  --(7.488,1.022)--(7.489,1.022)--(7.490,1.022)--(7.492,1.022)--(7.493,1.022)--(7.494,1.022)%
  --(7.495,1.022)--(7.496,1.022)--(7.497,1.022)--(7.498,1.022)--(7.499,1.022)--(7.500,1.022)%
  --(7.501,1.022)--(7.502,1.022)--(7.503,1.022)--(7.504,1.022)--(7.505,1.022)--(7.506,1.022)%
  --(7.507,1.022)--(7.508,1.022)--(7.509,1.022)--(7.510,1.022)--(7.511,1.022)--(7.512,1.022)%
  --(7.513,1.022)--(7.514,1.022)--(7.515,1.022)--(7.517,1.022)--(7.518,1.022)--(7.520,1.022)%
  --(7.521,1.022)--(7.522,1.022)--(7.523,1.022)--(7.524,1.022)--(7.525,1.022)--(7.526,1.022)%
  --(7.527,1.022)--(7.528,1.022)--(7.529,1.022)--(7.530,1.022)--(7.531,1.022)--(7.532,1.022)%
  --(7.533,1.022)--(7.534,1.022)--(7.535,1.022)--(7.536,1.022)--(7.537,1.022)--(7.538,1.022)%
  --(7.539,1.022)--(7.540,1.022)--(7.541,1.022)--(7.542,1.022)--(7.543,1.022)--(7.545,1.022)%
  --(7.546,1.022)--(7.547,1.022)--(7.548,1.022)--(7.549,1.022)--(7.550,1.022)--(7.552,1.022)%
  --(7.553,1.022)--(7.554,1.022)--(7.555,1.022)--(7.557,1.022)--(7.558,1.022)--(7.559,1.022)%
  --(7.560,1.022)--(7.561,1.022)--(7.562,1.022)--(7.563,1.022)--(7.564,1.022)--(7.565,1.022)%
  --(7.566,1.022)--(7.567,1.022)--(7.569,1.022)--(7.571,1.022)--(7.572,1.022)--(7.573,1.022)%
  --(7.574,1.022)--(7.576,1.091)--(7.577,1.183)--(7.578,1.249)--(7.579,1.314)--(7.580,1.378)%
  --(7.581,1.440)--(7.582,1.499)--(7.583,1.557)--(7.583,1.613)--(7.585,1.666)--(7.585,1.714)%
  --(7.586,1.761)--(7.587,1.803)--(7.588,1.844)--(7.589,1.882)--(7.590,1.917)--(7.591,1.949)%
  --(7.592,1.979)--(7.593,2.007)--(7.594,2.031)--(7.595,2.053)--(7.595,2.072)--(7.597,2.089)%
  --(7.598,2.102)--(7.599,2.113)--(7.600,2.121)--(7.601,2.126)--(7.602,2.127)--(7.603,2.125)%
  --(7.605,2.120)--(7.606,2.114)--(7.607,2.106)--(7.608,2.098)--(7.609,2.088)--(7.610,2.078)%
  --(7.611,2.067)--(7.612,2.056)--(7.613,2.046)--(7.614,2.037)--(7.615,2.028)--(7.615,2.021)%
  --(7.616,2.015)--(7.617,2.010)--(7.618,2.007)--(7.619,2.006)--(7.620,2.007)--(7.621,2.010)%
  --(7.622,2.016)--(7.623,2.024)--(7.624,2.034)--(7.625,2.047)--(7.626,2.064)--(7.627,2.082)%
  --(7.628,2.104)--(7.629,2.132)--(7.630,2.162)--(7.631,2.196)--(7.632,2.231)--(7.634,2.266)%
  --(7.635,2.301)--(7.636,2.338)--(7.637,2.375)--(7.638,2.413)--(7.639,2.450)--(7.640,2.488)%
  --(7.641,2.525)--(7.642,2.557)--(7.643,2.589)--(7.644,2.620)--(7.645,2.649)--(7.646,2.677)%
  --(7.647,2.703)--(7.648,2.727)--(7.648,2.750)--(7.649,2.772)--(7.650,2.791)--(7.651,2.809)%
  --(7.652,2.825)--(7.653,2.839)--(7.654,2.850)--(7.655,2.860)--(7.656,2.867)--(7.657,2.872)%
  --(7.658,2.874)--(7.660,2.874)--(7.661,2.871)--(7.662,2.865)--(7.663,2.858)--(7.664,2.849)%
  --(7.665,2.839)--(7.666,2.827)--(7.667,2.815)--(7.669,2.802)--(7.670,2.789)--(7.671,2.779)%
  --(7.672,2.768)--(7.673,2.759)--(7.674,2.750)--(7.675,2.743)--(7.676,2.737)--(7.676,2.732)%
  --(7.677,2.728)--(7.678,2.726)--(7.679,2.726)--(7.680,2.728)--(7.681,2.731)--(7.682,2.737)%
  --(7.683,2.745)--(7.684,2.756)--(7.685,2.768)--(7.686,2.785)--(7.687,2.804)--(7.689,2.825)%
  --(7.690,2.849)--(7.691,2.876)--(7.692,2.905)--(7.693,2.936)--(7.694,2.968)--(7.695,3.001)%
  --(7.697,3.035)--(7.698,3.069)--(7.699,3.103)--(7.700,3.132)--(7.701,3.160)--(7.702,3.188)%
  --(7.703,3.214)--(7.704,3.239)--(7.705,3.262)--(7.706,3.284)--(7.707,3.305)--(7.708,3.324)%
  --(7.709,3.341)--(7.709,3.357)--(7.710,3.370)--(7.711,3.382)--(7.712,3.391)--(7.714,3.399)%
  --(7.714,3.403)--(7.716,3.406)--(7.717,3.405)--(7.718,3.401)--(7.719,3.392)--(7.720,3.381)%
  --(7.721,3.368)--(7.723,3.354)--(7.724,3.339)--(7.725,3.322)--(7.726,3.305)--(7.728,3.287)%
  --(7.729,3.270)--(7.730,3.256)--(7.731,3.243)--(7.732,3.230)--(7.733,3.218)--(7.734,3.207)%
  --(7.735,3.197)--(7.736,3.188)--(7.737,3.179)--(7.737,3.172)--(7.738,3.166)--(7.739,3.161)%
  --(7.740,3.158)--(7.741,3.155)--(7.742,3.155)--(7.744,3.156)--(7.744,3.158)--(7.746,3.164)%
  --(7.747,3.173)--(7.748,3.184)--(7.749,3.197)--(7.750,3.210)--(7.751,3.233)--(7.752,3.265)%
  --(7.754,3.298)--(7.755,3.331)--(7.756,3.365)--(7.757,3.400)--(7.758,3.434)--(7.759,3.465)%
  --(7.760,3.495)--(7.761,3.525)--(7.762,3.554)--(7.763,3.581)--(7.764,3.607)--(7.765,3.631)%
  --(7.766,3.653)--(7.767,3.674)--(7.768,3.693)--(7.769,3.709)--(7.770,3.722)--(7.771,3.734)%
  --(7.772,3.742)--(7.773,3.748)--(7.774,3.751)--(7.775,3.751)--(7.776,3.747)--(7.777,3.740)%
  --(7.779,3.730)--(7.780,3.720)--(7.781,3.709)--(7.782,3.696)--(7.783,3.683)--(7.784,3.670)%
  --(7.785,3.657)--(7.786,3.644)--(7.787,3.632)--(7.788,3.621)--(7.789,3.612)--(7.790,3.603)%
  --(7.791,3.596)--(7.792,3.590)--(7.793,3.585)--(7.794,3.582)--(7.795,3.581)--(7.796,3.581)%
  --(7.797,3.583)--(7.798,3.587)--(7.799,3.592)--(7.800,3.601)--(7.801,3.611)--(7.802,3.624)%
  --(7.803,3.638)--(7.805,3.659)--(7.806,3.682)--(7.807,3.708)--(7.808,3.736)--(7.810,3.761)%
  --(7.810,3.789)--(7.812,3.817)--(7.813,3.846)--(7.814,3.876)--(7.815,3.906)--(7.816,3.937)%
  --(7.817,3.967)--(7.818,3.995)--(7.819,4.022)--(7.820,4.048)--(7.821,4.072)--(7.822,4.095)%
  --(7.823,4.117)--(7.824,4.136)--(7.825,4.153)--(7.826,4.168)--(7.827,4.181)--(7.828,4.190)%
  --(7.829,4.197)--(7.830,4.201)--(7.831,4.202)--(7.832,4.199)--(7.833,4.194)--(7.834,4.183)%
  --(7.836,4.168)--(7.837,4.149)--(7.839,4.128)--(7.840,4.111)--(7.841,4.093)--(7.842,4.075)%
  --(7.843,4.057)--(7.844,4.039)--(7.845,4.021)--(7.846,4.004)--(7.847,3.988)--(7.848,3.975)%
  --(7.849,3.962)--(7.850,3.951)--(7.851,3.941)--(7.852,3.933)--(7.853,3.927)--(7.854,3.923)%
  --(7.855,3.921)--(7.856,3.921)--(7.857,3.923)--(7.858,3.927)--(7.859,3.934)--(7.860,3.944)%
  --(7.861,3.957)--(7.862,3.972)--(7.864,3.990)--(7.865,4.012)--(7.866,4.037)--(7.867,4.064)%
  --(7.869,4.093)--(7.870,4.124)--(7.871,4.156)--(7.872,4.189)--(7.874,4.223)--(7.875,4.251)%
  --(7.876,4.278)--(7.877,4.305)--(7.878,4.331)--(7.879,4.354)--(7.880,4.376)--(7.881,4.396)%
  --(7.882,4.414)--(7.883,4.430)--(7.884,4.443)--(7.885,4.454)--(7.886,4.462)--(7.887,4.467)%
  --(7.888,4.469)--(7.889,4.467)--(7.890,4.463)--(7.891,4.454)--(7.893,4.442)--(7.894,4.426)%
  --(7.895,4.407)--(7.896,4.389)--(7.897,4.369)--(7.899,4.348)--(7.900,4.326)--(7.901,4.305)%
  --(7.902,4.283)--(7.903,4.262)--(7.904,4.242)--(7.905,4.225)--(7.906,4.209)--(7.907,4.194)%
  --(7.908,4.181)--(7.909,4.169)--(7.910,4.160)--(7.911,4.152)--(7.912,4.146)--(7.913,4.142)%
  --(7.914,4.141)--(7.915,4.142)--(7.916,4.145)--(7.918,4.151)--(7.919,4.160)--(7.920,4.171)%
  --(7.921,4.185)--(7.922,4.207)--(7.924,4.232)--(7.925,4.260)--(7.927,4.291)--(7.928,4.322)%
  --(7.929,4.354)--(7.931,4.386)--(7.932,4.419)--(7.933,4.445)--(7.934,4.471)--(7.935,4.496)%
  --(7.936,4.519)--(7.937,4.541)--(7.938,4.560)--(7.939,4.577)--(7.941,4.593)--(7.942,4.606)%
  --(7.942,4.616)--(7.944,4.624)--(7.945,4.628)--(7.946,4.630)--(7.947,4.629)--(7.948,4.624)%
  --(7.949,4.617)--(7.951,4.604)--(7.952,4.587)--(7.953,4.567)--(7.955,4.545)--(7.956,4.525)%
  --(7.957,4.504)--(7.958,4.482)--(7.959,4.461)--(7.960,4.439)--(7.961,4.418)--(7.962,4.398)%
  --(7.963,4.379)--(7.965,4.362)--(7.965,4.347)--(7.967,4.333)--(7.968,4.321)--(7.969,4.310)%
  --(7.970,4.302)--(7.971,4.295)--(7.972,4.291)--(7.973,4.289)--(7.974,4.289)--(7.975,4.292)%
  --(7.976,4.297)--(7.977,4.306)--(7.979,4.317)--(7.980,4.331)--(7.981,4.348)--(7.982,4.369)%
  --(7.984,4.391)--(7.985,4.416)--(7.986,4.442)--(7.987,4.469)--(7.989,4.498)--(7.990,4.526)%
  --(7.991,4.555)--(7.992,4.578)--(7.993,4.601)--(7.995,4.622)--(7.996,4.642)--(7.996,4.659)%
  --(7.998,4.675)--(7.999,4.689)--(8.000,4.700)--(8.001,4.709)--(8.002,4.715)--(8.003,4.718)%
  --(8.004,4.719)--(8.005,4.717)--(8.006,4.712)--(8.008,4.703)--(8.009,4.692)--(8.010,4.674)%
  --(8.012,4.652)--(8.013,4.627)--(8.015,4.600)--(8.016,4.575)--(8.018,4.550)--(8.019,4.525)%
  --(8.020,4.501)--(8.021,4.483)--(8.023,4.466)--(8.024,4.450)--(8.025,4.436)--(8.026,4.423)%
  --(8.027,4.413)--(8.028,4.404)--(8.029,4.398)--(8.030,4.393)--(8.031,4.390)--(8.032,4.390)%
  --(8.033,4.392)--(8.035,4.397)--(8.036,4.404)--(8.037,4.414)--(8.038,4.427)--(8.040,4.446)%
  --(8.041,4.469)--(8.043,4.495)--(8.044,4.524)--(8.046,4.550)--(8.047,4.577)--(8.049,4.604)%
  --(8.050,4.630)--(8.051,4.651)--(8.052,4.670)--(8.053,4.689)--(8.054,4.706)--(8.055,4.720)%
  --(8.056,4.732)--(8.057,4.743)--(8.059,4.751)--(8.060,4.757)--(8.061,4.761)--(8.062,4.762)%
  --(8.063,4.761)--(8.064,4.756)--(8.066,4.749)--(8.067,4.740)--(8.068,4.727)--(8.069,4.711)%
  --(8.071,4.693)--(8.072,4.673)--(8.073,4.651)--(8.075,4.629)--(8.076,4.606)--(8.077,4.584)%
  --(8.079,4.562)--(8.080,4.545)--(8.081,4.528)--(8.082,4.513)--(8.083,4.500)--(8.084,4.488)%
  --(8.085,4.478)--(8.087,4.470)--(8.088,4.464)--(8.089,4.460)--(8.090,4.458)--(8.091,4.459)%
  --(8.092,4.461)--(8.093,4.466)--(8.095,4.473)--(8.096,4.484)--(8.097,4.496)--(8.098,4.512)%
  --(8.100,4.530)--(8.101,4.550)--(8.103,4.572)--(8.104,4.595)--(8.105,4.618)--(8.107,4.641)%
  --(8.108,4.664)--(8.109,4.683)--(8.110,4.701)--(8.111,4.717)--(8.113,4.731)--(8.114,4.744)%
  --(8.115,4.755)--(8.116,4.764)--(8.117,4.771)--(8.118,4.776)--(8.119,4.778)--(8.120,4.779)%
  --(8.122,4.777)--(8.123,4.773)--(8.124,4.766)--(8.125,4.756)--(8.127,4.745)--(8.128,4.730)%
  --(8.129,4.713)--(8.131,4.694)--(8.132,4.674)--(8.133,4.654)--(8.135,4.633)--(8.136,4.613)%
  --(8.138,4.593)--(8.139,4.578)--(8.140,4.563)--(8.141,4.550)--(8.142,4.538)--(8.143,4.528)%
  --(8.145,4.520)--(8.146,4.514)--(8.147,4.509)--(8.148,4.507)--(8.149,4.506)--(8.150,4.508)%
  --(8.151,4.511)--(8.153,4.517)--(8.154,4.526)--(8.155,4.537)--(8.157,4.550)--(8.158,4.566)%
  --(8.159,4.583)--(8.161,4.601)--(8.162,4.621)--(8.163,4.641)--(8.165,4.661)--(8.166,4.681)%
  --(8.167,4.700)--(8.169,4.715)--(8.170,4.729)--(8.171,4.742)--(8.172,4.753)--(8.173,4.762)%
  --(8.174,4.770)--(8.175,4.776)--(8.177,4.780)--(8.178,4.782)--(8.179,4.782)--(8.180,4.780)%
  --(8.181,4.776)--(8.183,4.768)--(8.184,4.759)--(8.185,4.747)--(8.187,4.733)--(8.188,4.719)%
  --(8.189,4.703)--(8.191,4.686)--(8.192,4.669)--(8.193,4.651)--(8.195,4.634)--(8.196,4.618)%
  --(8.197,4.603)--(8.198,4.590)--(8.200,4.578)--(8.201,4.568)--(8.202,4.559)--(8.203,4.552)%
  --(8.204,4.546)--(8.205,4.542)--(8.207,4.540)--(8.208,4.540)--(8.209,4.542)--(8.210,4.546)%
  --(8.211,4.552)--(8.213,4.561)--(8.214,4.573)--(8.216,4.586)--(8.217,4.602)--(8.218,4.616)%
  --(8.220,4.632)--(8.221,4.648)--(8.222,4.664)--(8.223,4.681)--(8.225,4.696)--(8.226,4.712)%
  --(8.227,4.726)--(8.228,4.737)--(8.230,4.748)--(8.231,4.757)--(8.232,4.765)--(8.233,4.771)%
  --(8.234,4.775)--(8.236,4.778)--(8.237,4.779)--(8.238,4.777)--(8.239,4.774)--(8.241,4.769)%
  --(8.242,4.762)--(8.243,4.750)--(8.245,4.737)--(8.247,4.721)--(8.248,4.705)--(8.249,4.690)%
  --(8.251,4.676)--(8.252,4.661)--(8.253,4.647)--(8.254,4.634)--(8.256,4.621)--(8.257,4.609)%
  --(8.258,4.598)--(8.259,4.589)--(8.260,4.581)--(8.261,4.575)--(8.263,4.570)--(8.264,4.567)%
  --(8.265,4.565)--(8.266,4.565)--(8.268,4.568)--(8.269,4.572)--(8.270,4.578)--(8.272,4.586)%
  --(8.273,4.596)--(8.274,4.608)--(8.276,4.622)--(8.277,4.637)--(8.279,4.653)--(8.280,4.669)%
  --(8.282,4.685)--(8.283,4.701)--(8.285,4.716)--(8.286,4.727)--(8.287,4.738)--(8.288,4.747)%
  --(8.289,4.755)--(8.291,4.761)--(8.292,4.766)--(8.293,4.770)--(8.294,4.771)--(8.296,4.771)%
  --(8.297,4.769)--(8.298,4.766)--(8.300,4.760)--(8.301,4.752)--(8.303,4.741)--(8.304,4.729)%
  --(8.306,4.715)--(8.307,4.703)--(8.308,4.690)--(8.309,4.676)--(8.311,4.663)--(8.312,4.650)%
  --(8.313,4.638)--(8.315,4.627)--(8.316,4.616)--(8.317,4.608)--(8.318,4.600)--(8.320,4.594)%
  --(8.321,4.589)--(8.322,4.586)--(8.323,4.584)--(8.324,4.584)--(8.326,4.586)--(8.327,4.589)%
  --(8.329,4.595)--(8.330,4.602)--(8.331,4.610)--(8.333,4.621)--(8.334,4.634)--(8.336,4.647)%
  --(8.337,4.661)--(8.339,4.675)--(8.340,4.690)--(8.342,4.704)--(8.343,4.717)--(8.344,4.727)%
  --(8.346,4.736)--(8.347,4.744)--(8.348,4.751)--(8.349,4.756)--(8.351,4.760)--(8.352,4.762)%
  --(8.353,4.763)--(8.355,4.762)--(8.356,4.760)--(8.357,4.755)--(8.359,4.750)--(8.360,4.741)%
  --(8.362,4.730)--(8.363,4.718)--(8.365,4.705)--(8.366,4.693)--(8.368,4.682)--(8.369,4.670)%
  --(8.370,4.659)--(8.371,4.648)--(8.373,4.638)--(8.374,4.628)--(8.375,4.620)--(8.377,4.613)%
  --(8.378,4.608)--(8.379,4.604)--(8.380,4.601)--(8.382,4.599)--(8.383,4.599)--(8.384,4.601)%
  --(8.386,4.604)--(8.387,4.609)--(8.389,4.616)--(8.390,4.625)--(8.392,4.634)--(8.393,4.645)%
  --(8.394,4.656)--(8.396,4.668)--(8.397,4.680)--(8.399,4.692)--(8.400,4.703)--(8.401,4.714)%
  --(8.403,4.724)--(8.404,4.732)--(8.405,4.739)--(8.407,4.744)--(8.408,4.749)--(8.409,4.752)%
  --(8.411,4.754)--(8.412,4.754)--(8.413,4.753)--(8.415,4.750)--(8.416,4.746)--(8.417,4.740)%
  --(8.419,4.733)--(8.420,4.724)--(8.422,4.713)--(8.424,4.702)--(8.425,4.690)--(8.427,4.678)%
  --(8.428,4.666)--(8.430,4.655)--(8.431,4.644)--(8.432,4.636)--(8.434,4.629)--(8.435,4.623)%
  --(8.436,4.618)--(8.438,4.615)--(8.439,4.612)--(8.440,4.611)--(8.442,4.612)--(8.443,4.614)%
  --(8.445,4.617)--(8.446,4.622)--(8.447,4.628)--(8.449,4.638)--(8.451,4.649)--(8.453,4.661)%
  --(8.455,4.674)--(8.456,4.686)--(8.458,4.698)--(8.459,4.708)--(8.461,4.718)--(8.462,4.726)%
  --(8.464,4.732)--(8.465,4.737)--(8.466,4.741)--(8.468,4.744)--(8.469,4.746)--(8.470,4.746)%
  --(8.472,4.745)--(8.473,4.742)--(8.475,4.738)--(8.476,4.733)--(8.478,4.726)--(8.479,4.717)%
  --(8.481,4.708)--(8.482,4.698)--(8.484,4.687)--(8.485,4.677)--(8.487,4.666)--(8.488,4.656)%
  --(8.490,4.647)--(8.492,4.640)--(8.493,4.634)--(8.494,4.629)--(8.495,4.626)--(8.497,4.623)%
  --(8.498,4.622)--(8.500,4.621)--(8.501,4.622)--(8.502,4.625)--(8.504,4.629)--(8.505,4.634)%
  --(8.507,4.641)--(8.508,4.649)--(8.510,4.658)--(8.512,4.667)--(8.513,4.677)--(8.515,4.687)%
  --(8.516,4.697)--(8.518,4.707)--(8.520,4.715)--(8.521,4.721)--(8.522,4.727)--(8.524,4.731)%
  --(8.525,4.735)--(8.526,4.737)--(8.528,4.738)--(8.529,4.738)--(8.530,4.737)--(8.532,4.734)%
  --(8.533,4.730)--(8.535,4.724)--(8.537,4.718)--(8.538,4.710)--(8.540,4.701)--(8.541,4.692)%
  --(8.543,4.683)--(8.544,4.673)--(8.546,4.664)--(8.548,4.656)--(8.549,4.648)--(8.551,4.643)%
  --(8.552,4.638)--(8.553,4.634)--(8.555,4.632)--(8.556,4.630)--(8.558,4.630)--(8.559,4.631)%
  --(8.560,4.632)--(8.562,4.636)--(8.563,4.641)--(8.565,4.646)--(8.567,4.653)--(8.568,4.661)%
  --(8.570,4.669)--(8.571,4.678)--(8.573,4.686)--(8.574,4.695)--(8.576,4.703)--(8.578,4.710)%
  --(8.579,4.716)--(8.581,4.721)--(8.582,4.725)--(8.583,4.728)--(8.585,4.730)--(8.586,4.731)%
  --(8.588,4.731)--(8.589,4.730)--(8.590,4.727)--(8.592,4.723)--(8.594,4.718)--(8.595,4.712)%
  --(8.597,4.705)--(8.598,4.698)--(8.600,4.690)--(8.601,4.682)--(8.603,4.674)--(8.605,4.667)%
  --(8.606,4.660)--(8.608,4.654)--(8.609,4.648)--(8.610,4.644)--(8.612,4.641)--(8.613,4.639)%
  --(8.615,4.637)--(8.616,4.637)--(8.618,4.638)--(8.619,4.640)--(8.621,4.643)--(8.622,4.648)%
  --(8.624,4.653)--(8.626,4.660)--(8.627,4.668)--(8.629,4.674)--(8.630,4.682)--(8.632,4.689)%
  --(8.633,4.696)--(8.635,4.702)--(8.636,4.708)--(8.638,4.713)--(8.639,4.718)--(8.641,4.721)%
  --(8.642,4.723)--(8.644,4.725)--(8.645,4.725)--(8.647,4.725)--(8.648,4.723)--(8.650,4.720)%
  --(8.651,4.716)--(8.653,4.711)--(8.655,4.704)--(8.657,4.697)--(8.658,4.689)--(8.660,4.681)%
  --(8.662,4.673)--(8.664,4.665)--(8.666,4.658)--(8.667,4.654)--(8.669,4.650)--(8.670,4.647)%
  --(8.672,4.645)--(8.673,4.643)--(8.675,4.643)--(8.676,4.644)--(8.678,4.646)--(8.679,4.649)%
  --(8.681,4.653)--(8.682,4.658)--(8.684,4.664)--(8.686,4.670)--(8.687,4.676)--(8.689,4.683)%
  --(8.690,4.690)--(8.692,4.696)--(8.694,4.702)--(8.696,4.707)--(8.697,4.712)--(8.699,4.715)%
  --(8.700,4.717)--(8.701,4.719)--(8.703,4.720)--(8.704,4.719)--(8.706,4.718)--(8.708,4.716)%
  --(8.709,4.713)--(8.711,4.708)--(8.713,4.703)--(8.715,4.696)--(8.716,4.690)--(8.718,4.684)%
  --(8.719,4.678)--(8.721,4.672)--(8.722,4.667)--(8.724,4.662)--(8.725,4.658)--(8.727,4.654)%
  --(8.729,4.651)--(8.730,4.650)--(8.732,4.649)--(8.733,4.649)--(8.734,4.649)--(8.736,4.651)%
  --(8.738,4.654)--(8.739,4.658)--(8.741,4.662)--(8.743,4.667)--(8.744,4.673)--(8.746,4.679)%
  --(8.748,4.686)--(8.750,4.692)--(8.751,4.698)--(8.753,4.703)--(8.755,4.707)--(8.757,4.710)%
  --(8.758,4.713)--(8.760,4.714)--(8.761,4.715)--(8.763,4.715)--(8.764,4.714)--(8.766,4.712)%
  --(8.767,4.709)--(8.769,4.705)--(8.771,4.700)--(8.773,4.695)--(8.775,4.689)--(8.776,4.683)%
  --(8.778,4.678)--(8.779,4.673)--(8.781,4.668)--(8.783,4.664)--(8.784,4.660)--(8.786,4.657)%
  --(8.787,4.655)--(8.789,4.654)--(8.790,4.653)--(8.792,4.653)--(8.793,4.654)--(8.795,4.656)%
  --(8.797,4.659)--(8.798,4.663)--(8.800,4.667)--(8.802,4.672)--(8.804,4.677)--(8.805,4.683)%
  --(8.807,4.688)--(8.809,4.694)--(8.811,4.698)--(8.813,4.703)--(8.814,4.706)--(8.816,4.708)%
  --(8.818,4.710)--(8.819,4.711)--(8.821,4.711)--(8.822,4.710)--(8.824,4.708)--(8.826,4.706)%
  --(8.827,4.703)--(8.829,4.698)--(8.831,4.693)--(8.833,4.687)--(8.835,4.681)--(8.837,4.675)%
  --(8.839,4.670)--(8.841,4.666)--(8.843,4.662)--(8.845,4.660)--(8.846,4.658)--(8.848,4.657)%
  --(8.849,4.657)--(8.851,4.658)--(8.853,4.659)--(8.854,4.661)--(8.856,4.664)--(8.858,4.668)%
  --(8.860,4.673)--(8.862,4.678)--(8.864,4.683)--(8.866,4.689)--(8.868,4.694)--(8.870,4.698)%
  --(8.872,4.702)--(8.873,4.704)--(8.875,4.706)--(8.877,4.707)--(8.878,4.707)--(8.880,4.707)%
  --(8.882,4.706)--(8.883,4.704)--(8.885,4.701)--(8.887,4.697)--(8.889,4.693)--(8.891,4.688)%
  --(8.893,4.683)--(8.895,4.678)--(8.897,4.673)--(8.899,4.669)--(8.901,4.666)--(8.903,4.663)%
  --(8.904,4.662)--(8.906,4.661)--(8.908,4.660)--(8.909,4.661)--(8.911,4.662)--(8.913,4.664)%
  --(8.914,4.666)--(8.916,4.670)--(8.919,4.674)--(8.921,4.679)--(8.923,4.684)--(8.925,4.689)%
  --(8.927,4.693)--(8.929,4.697)--(8.931,4.700)--(8.932,4.702)--(8.934,4.703)--(8.936,4.704)%
  --(8.937,4.704)--(8.939,4.703)--(8.941,4.702)--(8.943,4.700)--(8.944,4.697)--(8.946,4.694)%
  --(8.948,4.689)--(8.951,4.685)--(8.953,4.680)--(8.955,4.676)--(8.957,4.672)--(8.959,4.669)%
  --(8.961,4.666)--(8.962,4.665)--(8.964,4.664)--(8.966,4.663)--(8.968,4.664)--(8.969,4.665)%
  --(8.971,4.666)--(8.973,4.669)--(8.975,4.671)--(8.977,4.676)--(8.979,4.680)--(8.982,4.685)%
  --(8.984,4.690)--(8.986,4.693)--(8.988,4.696)--(8.990,4.699)--(8.992,4.700)--(8.994,4.701)%
  --(8.995,4.701)--(8.997,4.701)--(8.999,4.700)--(9.001,4.698)--(9.003,4.696)--(9.004,4.693)%
  --(9.006,4.690)--(9.008,4.686)--(9.010,4.683)--(9.012,4.679)--(9.014,4.676)--(9.015,4.673)%
  --(9.017,4.670)--(9.019,4.668)--(9.021,4.667)--(9.023,4.666)--(9.025,4.666)--(9.026,4.666)%
  --(9.028,4.667)--(9.030,4.669)--(9.032,4.671)--(9.034,4.674)--(9.036,4.677)--(9.037,4.680)%
  --(9.039,4.683)--(9.041,4.687)--(9.043,4.690)--(9.045,4.693)--(9.047,4.695)--(9.049,4.697)%
  --(9.051,4.698)--(9.052,4.699)--(9.054,4.699)--(9.056,4.698)--(9.058,4.697)--(9.059,4.696)%
  --(9.061,4.693)--(9.063,4.691)--(9.065,4.688)--(9.067,4.685)--(9.069,4.682)--(9.071,4.679)%
  --(9.073,4.676)--(9.075,4.673)--(9.077,4.671)--(9.078,4.670)--(9.080,4.669)--(9.082,4.668)%
  --(9.084,4.668)--(9.086,4.669)--(9.087,4.670)--(9.089,4.672)--(9.091,4.674)--(9.093,4.677)%
  --(9.095,4.680)--(9.097,4.683)--(9.099,4.686)--(9.101,4.688)--(9.103,4.691)--(9.105,4.693)%
  --(9.107,4.695)--(9.108,4.696)--(9.110,4.697)--(9.112,4.697)--(9.114,4.696)--(9.116,4.695)%
  --(9.118,4.694)--(9.119,4.692)--(9.122,4.690)--(9.124,4.687)--(9.126,4.684)--(9.128,4.681)%
  --(9.130,4.679)--(9.131,4.676)--(9.133,4.674)--(9.135,4.673)--(9.137,4.671)--(9.139,4.671)%
  --(9.141,4.670)--(9.142,4.670)--(9.144,4.671)--(9.146,4.672)--(9.148,4.674)--(9.150,4.676)%
  --(9.152,4.679)--(9.155,4.682)--(9.157,4.684)--(9.159,4.687)--(9.161,4.690)--(9.164,4.692)%
  --(9.166,4.694)--(9.168,4.695)--(9.170,4.695)--(9.171,4.695)--(9.173,4.694)--(9.175,4.693)%
  --(9.177,4.692)--(9.179,4.690)--(9.181,4.687)--(9.183,4.685)--(9.185,4.682)--(9.187,4.680)%
  --(9.189,4.678)--(9.192,4.676)--(9.194,4.674)--(9.196,4.673)--(9.198,4.672)--(9.200,4.672)%
  --(9.201,4.672)--(9.203,4.673)--(9.205,4.674)--(9.207,4.676)--(9.210,4.678)--(9.212,4.680)%
  --(9.214,4.683)--(9.216,4.685)--(9.218,4.687)--(9.220,4.689)--(9.222,4.691)--(9.223,4.692)%
  --(9.225,4.693)--(9.227,4.693)--(9.229,4.693)--(9.231,4.693)--(9.233,4.692)--(9.235,4.691)%
  --(9.237,4.689)--(9.239,4.687)--(9.242,4.684)--(9.244,4.682)--(9.247,4.679)--(9.249,4.677)%
  --(9.251,4.676)--(9.254,4.674)--(9.256,4.674)--(9.258,4.673)--(9.260,4.674)--(9.261,4.674)%
  --(9.263,4.675)--(9.266,4.677)--(9.268,4.679)--(9.270,4.681)--(9.272,4.683)--(9.274,4.685)%
  --(9.276,4.687)--(9.278,4.688)--(9.280,4.690)--(9.282,4.691)--(9.284,4.692)--(9.286,4.692)%
  --(9.288,4.692)--(9.290,4.691)--(9.292,4.691)--(9.294,4.689)--(9.296,4.688)--(9.299,4.686)%
  --(9.301,4.683)--(9.304,4.681)--(9.306,4.679)--(9.309,4.677)--(9.311,4.676)--(9.313,4.675)%
  --(9.315,4.675)--(9.317,4.675)--(9.319,4.675)--(9.321,4.676)--(9.323,4.677)--(9.326,4.679)%
  --(9.328,4.681)--(9.330,4.683)--(9.333,4.685)--(9.335,4.687)--(9.337,4.688)--(9.340,4.690)%
  --(9.343,4.690)--(9.344,4.691)--(9.346,4.691)--(9.349,4.690)--(9.351,4.689)--(9.353,4.688)%
  --(9.355,4.687)--(9.357,4.685)--(9.359,4.683)--(9.362,4.682)--(9.363,4.680)--(9.366,4.679)%
  --(9.368,4.677)--(9.370,4.677)--(9.372,4.676)--(9.374,4.676)--(9.376,4.676)--(9.378,4.676)%
  --(9.380,4.677)--(9.382,4.678)--(9.384,4.680)--(9.387,4.682)--(9.390,4.684)--(9.393,4.686)%
  --(9.396,4.687)--(9.398,4.689)--(9.400,4.689)--(9.403,4.690)--(9.405,4.690)--(9.407,4.689)%
  --(9.409,4.689)--(9.411,4.688)--(9.413,4.686)--(9.416,4.685)--(9.419,4.683)--(9.421,4.681)%
  --(9.424,4.679)--(9.426,4.678)--(9.429,4.677)--(9.431,4.677)--(9.434,4.677)--(9.436,4.677)%
  --(9.438,4.678)--(9.440,4.679)--(9.442,4.680)--(9.445,4.681)--(9.447,4.683)--(9.450,4.685)%
  --(9.453,4.686)--(9.455,4.687)--(9.458,4.688)--(9.460,4.689)--(9.463,4.689)--(9.465,4.689)%
  --(9.467,4.688)--(9.469,4.687)--(9.471,4.686)--(9.474,4.685)--(9.476,4.683)--(9.479,4.681)%
  --(9.482,4.680)--(9.484,4.679)--(9.487,4.678)--(9.489,4.678)--(9.492,4.678)--(9.494,4.678)%
  --(9.496,4.678)--(9.498,4.679)--(9.500,4.680)--(9.503,4.681)--(9.506,4.683)--(9.508,4.684)%
  --(9.511,4.686)--(9.513,4.687)--(9.516,4.688)--(9.519,4.688)--(9.521,4.688)--(9.523,4.688)%
  --(9.525,4.687)--(9.527,4.686)--(9.530,4.686)--(9.533,4.684)--(9.535,4.683)--(9.538,4.681)%
  --(9.541,4.680)--(9.543,4.679)--(9.546,4.679)--(9.548,4.678)--(9.551,4.678)--(9.553,4.679)%
  --(9.555,4.679)--(9.557,4.680)--(9.560,4.681)--(9.563,4.682)--(9.566,4.684)--(9.569,4.685)%
  --(9.572,4.686)--(9.574,4.687)--(9.577,4.687)--(9.579,4.687)--(9.581,4.687)--(9.584,4.687)%
  --(9.586,4.686)--(9.588,4.685)--(9.591,4.684)--(9.593,4.683)--(9.596,4.682)--(9.598,4.681)%
  --(9.601,4.680)--(9.603,4.679)--(9.606,4.679)--(9.608,4.679)--(9.610,4.679)--(9.613,4.680)%
  --(9.615,4.680)--(9.617,4.681)--(9.620,4.682)--(9.622,4.683)--(9.625,4.684)--(9.627,4.685)%
  --(9.630,4.686)--(9.632,4.686)--(9.635,4.687)--(9.637,4.687)--(9.640,4.687)--(9.642,4.686)%
  --(9.644,4.686)--(9.647,4.685)--(9.649,4.684)--(9.651,4.683)--(9.654,4.682)--(9.657,4.681)%
  --(9.659,4.680)--(9.662,4.680)--(9.664,4.680)--(9.667,4.679)--(9.669,4.680)--(9.672,4.680)%
  --(9.674,4.681)--(9.676,4.681)--(9.679,4.682)--(9.681,4.683)--(9.684,4.684)--(9.686,4.685)%
  --(9.689,4.686)--(9.691,4.686)--(9.694,4.686)--(9.697,4.686)--(9.699,4.686)--(9.702,4.685)%
  --(9.704,4.685)--(9.707,4.684)--(9.709,4.683)--(9.711,4.682)--(9.714,4.682)--(9.717,4.681)%
  --(9.719,4.680)--(9.722,4.680)--(9.724,4.680)--(9.727,4.680)--(9.729,4.680)--(9.732,4.681)%
  --(9.734,4.682)--(9.737,4.682)--(9.739,4.683)--(9.742,4.684)--(9.744,4.685)--(9.747,4.685)%
  --(9.749,4.686)--(9.752,4.686)--(9.754,4.686)--(9.757,4.686)--(9.759,4.685)--(9.762,4.685)%
  --(9.764,4.684)--(9.767,4.683)--(9.770,4.683)--(9.772,4.682)--(9.775,4.681)--(9.777,4.681)%
  --(9.780,4.680)--(9.782,4.680)--(9.785,4.680)--(9.787,4.681)--(9.790,4.681)--(9.792,4.682)%
  --(9.795,4.682)--(9.798,4.683)--(9.800,4.684)--(9.803,4.684)--(9.805,4.685)--(9.808,4.685)%
  --(9.810,4.685)--(9.813,4.685)--(9.815,4.685)--(9.818,4.685)--(9.820,4.684)--(9.823,4.684)%
  --(9.826,4.683)--(9.829,4.682)--(9.831,4.682)--(9.834,4.681)--(9.838,4.681)--(9.841,4.681)%
  --(9.844,4.681)--(9.846,4.681)--(9.849,4.681)--(9.851,4.682)--(9.854,4.683)--(9.857,4.683)%
  --(9.859,4.684)--(9.862,4.684)--(9.865,4.685)--(9.867,4.685)--(9.870,4.685)--(9.873,4.685)%
  --(9.876,4.685)--(9.878,4.684)--(9.881,4.684)--(9.884,4.683)--(9.887,4.683)--(9.889,4.682)%
  --(9.892,4.682)--(9.894,4.681)--(9.897,4.681)--(9.900,4.681)--(9.902,4.681)--(9.905,4.681)%
  --(9.908,4.682)--(9.911,4.682)--(9.913,4.683)--(9.916,4.683)--(9.919,4.684)--(9.922,4.685)%
  --(9.925,4.685)--(9.929,4.685)--(9.932,4.685)--(9.934,4.684)--(9.937,4.684)--(9.940,4.684)%
  --(9.942,4.683)--(9.945,4.683)--(9.948,4.682)--(9.951,4.682)--(9.954,4.681)--(9.957,4.681)%
  --(9.959,4.681)--(9.962,4.681)--(9.965,4.682)--(9.968,4.682)--(9.971,4.683)--(9.974,4.683)%
  --(9.976,4.684)--(9.979,4.684)--(9.982,4.684)--(9.984,4.685)--(9.987,4.685)--(9.990,4.685)%
  --(9.993,4.684)--(9.995,4.684)--(9.998,4.684)--(10.001,4.683)--(10.005,4.682)--(10.008,4.682)%
  --(10.011,4.682)--(10.014,4.681)--(10.017,4.682)--(10.020,4.682)--(10.023,4.682)--(10.026,4.682)%
  --(10.029,4.683)--(10.032,4.683)--(10.035,4.684)--(10.037,4.684)--(10.040,4.684)--(10.043,4.684)%
  --(10.046,4.684)--(10.049,4.684)--(10.051,4.684)--(10.054,4.684)--(10.057,4.683)--(10.060,4.683)%
  --(10.064,4.682)--(10.067,4.682)--(10.071,4.682)--(10.074,4.682)--(10.077,4.682)--(10.080,4.682)%
  --(10.083,4.682)--(10.086,4.683)--(10.089,4.683)--(10.092,4.683)--(10.095,4.684)--(10.098,4.684)%
  --(10.102,4.684)--(10.105,4.684)--(10.108,4.684)--(10.111,4.684)--(10.114,4.683)--(10.117,4.683)%
  --(10.120,4.683)--(10.123,4.682)--(10.126,4.682)--(10.129,4.682)--(10.132,4.682)--(10.135,4.682)%
  --(10.138,4.682)--(10.141,4.682)--(10.144,4.683)--(10.147,4.683)--(10.150,4.683)--(10.154,4.684)%
  --(10.157,4.684)--(10.160,4.684)--(10.163,4.684)--(10.167,4.684)--(10.170,4.684)--(10.173,4.683)%
  --(10.176,4.683)--(10.179,4.683)--(10.182,4.682)--(10.186,4.682)--(10.189,4.682)--(10.193,4.682)%
  --(10.197,4.682)--(10.200,4.682)--(10.203,4.683)--(10.206,4.683)--(10.209,4.683)--(10.212,4.684)%
  --(10.215,4.684)--(10.218,4.684)--(10.221,4.684)--(10.224,4.684)--(10.227,4.684)--(10.230,4.683)%
  --(10.233,4.683)--(10.237,4.683)--(10.240,4.682)--(10.244,4.682)--(10.247,4.682)--(10.251,4.682)%
  --(10.254,4.682)--(10.257,4.682)--(10.261,4.683)--(10.264,4.683)--(10.267,4.683)--(10.271,4.684)%
  --(10.274,4.684)--(10.277,4.684)--(10.281,4.684)--(10.284,4.684)--(10.288,4.683)--(10.291,4.683)%
  --(10.294,4.683)--(10.297,4.683)--(10.300,4.682)--(10.304,4.682)--(10.308,4.682)--(10.312,4.682)%
  --(10.315,4.683)--(10.318,4.683)--(10.321,4.683)--(10.325,4.683)--(10.328,4.683)--(10.331,4.684)%
  --(10.334,4.684)--(10.337,4.684)--(10.341,4.684)--(10.344,4.683)--(10.347,4.683)--(10.350,4.683)%
  --(10.353,4.683)--(10.357,4.683)--(10.360,4.682)--(10.363,4.682)--(10.366,4.682)--(10.370,4.682)%
  --(10.373,4.683)--(10.376,4.683)--(10.380,4.683)--(10.383,4.683)--(10.387,4.683)--(10.391,4.684)%
  --(10.395,4.684)--(10.399,4.684)--(10.402,4.683)--(10.405,4.683)--(10.409,4.683)--(10.413,4.683)%
  --(10.417,4.683)--(10.420,4.682)--(10.424,4.682)--(10.427,4.682)--(10.431,4.683)--(10.434,4.683)%
  --(10.438,4.683)--(10.442,4.683)--(10.446,4.683)--(10.449,4.684)--(10.453,4.684)--(10.457,4.683)%
  --(10.460,4.683)--(10.464,4.683)--(10.467,4.683)--(10.471,4.683);
\gpsetdashtype{gp dt solid}
\draw[gp path] (1.504,8.381)--(1.504,0.985)--(10.475,0.985)--(10.475,8.381)--cycle;
%% coordinates of the plot area
\gpdefrectangularnode{gp plot 1}{\pgfpoint{1.504cm}{0.985cm}}{\pgfpoint{10.475cm}{8.381cm}}
\end{tikzpicture}
%% gnuplot variables

%\end{figure}



The efficiency of this FIS was determined by using it as the control mechanism in the simulation. The goal was
to approach the realistic theoretical limits calculated above and reach this goal with minimum force.
\Cref{tab:finalres} shows the settling time, final position of cart 2 and the calculated $J$ value for
the controller's performance. \Cref{f:forceplot} shows the controller's output force over time for each
step of the simulation. 

It is clear that the controller expends less energy than a traditional bang-bang type controller would in the
oscillation damping process. It can be seen that the FIS responds quickly to control needs and applies the
needed force with low-latency. 

Much work was done to hand-tune this controller to perform optimally, thus the work was undertaken to develop
a genetic algorithm to produce similar results autonomously. Automating the tuning process ultimately produces
a controller which is as good as the hand-tuned controller, but requires little to no effort from the
programmer.

\begin{table} \centering \caption{Results from hand-tuned FIS.} \label{tab:finalres} \begin{tabular}{|c|c|c|}
\hline $t_f$ & $x_2(500)$ & $J$ \\ \hline \SI{32.303}{\second} & \SI{99.999821}{\metre} & 0.32328 \\ \hline
\end{tabular} \end{table}

\section{Genetic Algorithm}\label{s:ga} The performance of the FIS depends directly
on the value of each parameter of the membership functions. These values were hand-tuned by a time-consuming
process of trial and error. To quicken this process, a genetic algorithm was utilized to autonomously tune the
membership functions and approach an optimal solution. 

As previously discussed, a genetic algorithm is a computational mechanism which imitates evolutionary behavior
to achieve optimality. It consists of a population of individuals which undergoes a process similar to natural
selection, reproduction, and mutation over the course of a number of generations\cite{cordon:01bk}.  Selection
is attained by evaluating the fitness of each individual FIS according to \cref{e:cost}. The individual which
most effectively minimizes the cost function is considered most fit for the control environment.

In order to manipulate the FIS structure in an algorithmic manner, it is necessary to represent it as a
genetic individual. Since the values of the parameters of the membership functions of the FIS have
significant impact on the control performance, it was decided to manipulate only these parameters with
the algorithm; however, of the thirty-one parameters which comprise this FIS model, many are trivial to
the overall performance. It is desirable to reduce the number of parameters to facilitate the
optimization process.

\subsection{State Reduction} To simplify the genetic tuning of the parameters, the symmetry of the system
was exploited. The parameters were reduced from thirty-one to seven. 

The position membership functions were
simplified to only three parameters by defining a center point (center) between far and close functions, a
distance from the center point at which the far and close functions will be valued at 0 and 1 respectively
(iTrap1), and half of the base of the triangular membership function which decides when the car is very close
to the wall (iTriBase1). The velocity membership function parameters were reduced to two parameters. One
parameter describes the distance from 0 to the point at which each of the velocities reach 1 (iTrap2). The
other parameter is defined as half of the base of the zero velocity triangular membership function
(iTriBase2). The output force membership functions were reduced similarly by allowing the negative and
positive membership functions to become trapezoidal (oTrap and oTriBase). 

\begin{itemize} \item Position Membership Function Parameter \begin{displaymath} \mathrm{Far:}\quad
\begin{bmatrix} 0\\0\\(center-iTrap1)\\(center+iTrap1) \end{bmatrix}, \quad
    \mathrm{Close:}\quad \begin{bmatrix} (center-iTrap1)\\(center+iTrap1)\\99.9\\100
\end{bmatrix}, \end{displaymath} \begin{displaymath} \mathrm{VeryClose:}\quad \begin{bmatrix}
(100-iTriBase1)\\100\\ (100+iTriBase1) \end{bmatrix} \end{displaymath}
 
 \item Velocity Membership Function Parameters \begin{displaymath} \mathrm{Negative:}\quad
     \begin{bmatrix} -6\\-6\\(0)-iTrap2)\\0 \end{bmatrix}, \quad \mathrm{Zero:}\quad
     \begin{bmatrix} (0-iTriBase2)\\0\\ (0+iTriBase2) \end{bmatrix}, \end{displaymath}
         \begin{displaymath} \mathrm{Positive:}\quad \begin{bmatrix} 0\\ (0+iTrap2)\\6\\6
         \end{bmatrix} \end{displaymath}
 
\item Control Force Membership Function Parameters \begin{displaymath} \mathrm{Negative:}\quad
    \begin{bmatrix} -2\\ (-1-oTrap)\\ (-1+oTrap)\\0 \end{bmatrix}, \quad \mathrm{Zero:}\quad
    \begin{bmatrix} (0-oTriBase)\\0\\ (1+oTriBase) \end{bmatrix}, \end{displaymath}
        \begin{displaymath} \mathrm{Positive:}\quad \begin{bmatrix} 0\\ (1-oTrap)\\
    (1+oTrap)\\2 \end{bmatrix} \end{displaymath} \end{itemize}
 
These state reductions allow an individual to be defined by a single vector of seven variables.
\begin{itemize} \item Individual Definition \begin{displaymath} \begin{bmatrix} iTrap1\\ center\\ iTriBase1\\
iTrap2 \\iTriBase2 \\oTrap \\oTriBase \end{bmatrix} \end{displaymath} \end{itemize}
 
\subsection{Population Initialization} An initial population is generated by assigning random values to each
of the individual parameters within given ranges. iTrap1, iTriBase1, and oTriBase are allowed to vary between
0.05 and 1. iTrap2 and iTriBase2 are allowed to vary between 0.05 and 2. Center values fall between 45 and 55,
and oTrap between 0.05 and 0.95. Twenty individuals comprise a population. Each individual is evaluated for
fitness and brought up for selection to produce a new generation.

\subsection{Parent Selection and Reproduction} A new generation consists of three individuals which remain
unchanged from the previous generation, called elite individuals, ten children which are created from
recombination of two parents each, five individuals which are created from mutating recombined children, and
two individuals randomly defined from the previously defined ranges.  The parents for this population consist
of the three elite individuals as well as seven more individuals. Each of the seven is selected by randomly
choosing three individuals from the previous population, selecting the most fit, and returning the other two.
This tournament style of selection is repeated until all parents are selected.

Reproduction occurs by blended crossover process with an $\alpha$ modification (BLX-$\alpha$), by selecting a
new parameter $x_i'$ from the range $[x_{min}-I\alpha,x_{max}+I\alpha]$, where \begin{displaymath}
x_{min}=min(x_i^1,x_i^2)\quad \mathrm{and} \quad x_{max}=max(x_i^1,x_i^1) \end{displaymath}
\nomenclature[1\(xa\)]{$x^j_i$}{Parameter of genetic individual} \nomenclature{$I$}{Genetic selection
interval} \nomenclature[g]{$\alpha$}{Genetic parameter modifier} Parents are defined as \begin{displaymath}
C^1=(x_1^1,x_2^1,\cdots,x_7^1)\quad\mathrm{and}\quad C^2=(x_1^2,x_2^2,\cdots,x_7^2) \end{displaymath}
\nomenclature{$C^i$}{Genetic parent} \nomenclature{$d$}{Genetic interval buffer distance} \begin{displaymath}
I=\frac{x_{max}-x{min}}{b_i-a_i} \end{displaymath} \begin{displaymath} \alpha=min(d^1,d^2) \end{displaymath}
\begin{displaymath} d^1=x_{min}-a_i\quad\mathrm{and}\quad d^2=b_i-x_{max} \end{displaymath}

The interval $[a_i,b_i]$ is the parameter-specific range. This mechanism allows the algorithm to create a
child from two parents which is a blend of both parents, while still expanding the search space. As a
population generally converges on a solution, so too do the children of the population.

\subsection{Mutation} Five of the recombined children are selected by random sampling and then two randomly
sampled parameters are selected from each of these child for mutation. Mutation is defined to be nonuniform
such that the mutation has a smaller effect in later generations as follows: \begin{displaymath} x_i'=
    \begin{cases} a_i+\Delta(t,x_i-a_i),& \text{if }\tau=0\\ b_i-\Delta(t,b_i-x_i),& \text{if }\tau=1
    \end{cases} \end{displaymath} where $\tau$ represents a coin flip such that $P(\tau=1)=P(\tau=0)=0.5$
    \nomenclature[g]{$\tau$}{Coin flip value} \nomenclature[g]{$\Delta(t,x)$}{Genetic mutation function}
    \begin{displaymath} \Delta(t,x)=x(1-\lambda(1-\frac{t}{t_{max}})^b) \end{displaymath}
        \nomenclature[g]{$\lambda$}{Random constant}\noindent where $t$ is the current generation, and
        $t_{max}$ is the maximum number of generations. The variable $\lambda$ is a random value from the
        interval $[0,1]$. The function $\Delta$ computes a value in the range $[0,x]$ such that the
        probability of returning a zero increases as the algorithm advances. The value of $b$ determines the
        impact of the time on the probability distribution of $\Delta$. The value of $b$ is set to 1.5 for
         the algorithms used in this research.

\section{Results} Running the algorithm for 50 generations yields a FIS which
performs as well as the hand-tuned FIS from~\cref{ss:simperf}. This result converges out of the evolution
process quickly as can be seen in \cref{f:popfitness}. Though the algorithm finds a near optimal
solution quickly, it continues to search similar solutions, selecting the best individuals each time to
produce progressively better fit children each generation. \Cref{f:popaverage} shows the average
fitness of generation. It is clear from this plot that the algorithm produces many unfit children in its
search for optimality. This satisfies the need of a good algorithm to expand the search area to eliminate
premature convergence.

The results of the performance of the most fit individual produced by the genetic algorithm are shown in
\cref{tab:garesult}. It is shown in \cref{f:accforcecomp} that the output force of the GA-tuned controller is
very similar to the hand-tuned version. Examining \cref{f:fiscomp} shows that the GA has changed the
membership functions appreciably within the allotted envelopes while still yielding a near-optimal result.

\begin{table} \centering \caption{Final Results from algorithm-generated FIS}\label{tab:garesult}
\begin{tabular}{|c|c|c|} \hline $t_f$ & $x_2(500)$ & $J$ \\\hline \SI{32.308}{\second} &
\SI{99.999999}{\metre} &  0.32311 \\\hline \end{tabular} \end{table}

\begin{figure}
    \centering
    \resizebox{0.75\textwidth}{!}{\begin{tikzpicture}[gnuplot]
%% generated with GNUPLOT 5.0p3 (Lua 5.1; terminal rev. 99, script rev. 100)
%% Thu 29 Mar 2018 12:41:20 AM EDT
\gpmonochromelines
\path (0.000,0.000) rectangle (12.500,8.750);
\gpcolor{color=gp lt color border}
\gpsetlinetype{gp lt border}
\gpsetdashtype{gp dt solid}
\gpsetlinewidth{1.00}
\draw[gp path] (1.504,0.985)--(1.684,0.985);
\draw[gp path] (11.947,0.985)--(11.767,0.985);
\node[gp node right] at (1.320,0.985) {$0.3$};
\draw[gp path] (1.504,1.909)--(1.684,1.909);
\draw[gp path] (11.947,1.909)--(11.767,1.909);
\node[gp node right] at (1.320,1.909) {$0.35$};
\draw[gp path] (1.504,2.834)--(1.684,2.834);
\draw[gp path] (11.947,2.834)--(11.767,2.834);
\node[gp node right] at (1.320,2.834) {$0.4$};
\draw[gp path] (1.504,3.758)--(1.684,3.758);
\draw[gp path] (11.947,3.758)--(11.767,3.758);
\node[gp node right] at (1.320,3.758) {$0.45$};
\draw[gp path] (1.504,4.683)--(1.684,4.683);
\draw[gp path] (11.947,4.683)--(11.767,4.683);
\node[gp node right] at (1.320,4.683) {$0.5$};
\draw[gp path] (1.504,5.608)--(1.684,5.608);
\draw[gp path] (11.947,5.608)--(11.767,5.608);
\node[gp node right] at (1.320,5.608) {$0.55$};
\draw[gp path] (1.504,6.532)--(1.684,6.532);
\draw[gp path] (11.947,6.532)--(11.767,6.532);
\node[gp node right] at (1.320,6.532) {$0.6$};
\draw[gp path] (1.504,7.457)--(1.684,7.457);
\draw[gp path] (11.947,7.457)--(11.767,7.457);
\node[gp node right] at (1.320,7.457) {$0.65$};
\draw[gp path] (1.504,8.381)--(1.684,8.381);
\draw[gp path] (11.947,8.381)--(11.767,8.381);
\node[gp node right] at (1.320,8.381) {$0.7$};
\draw[gp path] (1.504,0.985)--(1.504,1.165);
\draw[gp path] (1.504,8.381)--(1.504,8.201);
\node[gp node center] at (1.504,0.677) {$0$};
\draw[gp path] (2.548,0.985)--(2.548,1.165);
\draw[gp path] (2.548,8.381)--(2.548,8.201);
\node[gp node center] at (2.548,0.677) {$5$};
\draw[gp path] (3.593,0.985)--(3.593,1.165);
\draw[gp path] (3.593,8.381)--(3.593,8.201);
\node[gp node center] at (3.593,0.677) {$10$};
\draw[gp path] (4.637,0.985)--(4.637,1.165);
\draw[gp path] (4.637,8.381)--(4.637,8.201);
\node[gp node center] at (4.637,0.677) {$15$};
\draw[gp path] (5.681,0.985)--(5.681,1.165);
\draw[gp path] (5.681,8.381)--(5.681,8.201);
\node[gp node center] at (5.681,0.677) {$20$};
\draw[gp path] (6.726,0.985)--(6.726,1.165);
\draw[gp path] (6.726,8.381)--(6.726,8.201);
\node[gp node center] at (6.726,0.677) {$25$};
\draw[gp path] (7.770,0.985)--(7.770,1.165);
\draw[gp path] (7.770,8.381)--(7.770,8.201);
\node[gp node center] at (7.770,0.677) {$30$};
\draw[gp path] (8.814,0.985)--(8.814,1.165);
\draw[gp path] (8.814,8.381)--(8.814,8.201);
\node[gp node center] at (8.814,0.677) {$35$};
\draw[gp path] (9.858,0.985)--(9.858,1.165);
\draw[gp path] (9.858,8.381)--(9.858,8.201);
\node[gp node center] at (9.858,0.677) {$40$};
\draw[gp path] (10.903,0.985)--(10.903,1.165);
\draw[gp path] (10.903,8.381)--(10.903,8.201);
\node[gp node center] at (10.903,0.677) {$45$};
\draw[gp path] (11.947,0.985)--(11.947,1.165);
\draw[gp path] (11.947,8.381)--(11.947,8.201);
\node[gp node center] at (11.947,0.677) {$50$};
\draw[gp path] (1.504,8.381)--(1.504,0.985)--(11.947,0.985)--(11.947,8.381)--cycle;
\node[gp node center,rotate=-270] at (0.246,4.683) {Cost};
\node[gp node center] at (6.725,0.215) {Generation};
\gpsetlinewidth{2.00}
\draw[gp path] (1.713,8.297)--(1.922,7.211)--(2.131,7.211)--(2.339,4.890)--(2.548,4.890)%
  --(2.757,2.212)--(2.966,2.212)--(3.175,2.091)--(3.384,2.091)--(3.593,1.666)--(3.801,1.666)%
  --(4.010,1.436)--(4.219,1.436)--(4.428,1.436)--(4.637,1.436)--(4.846,1.436)--(5.055,1.436)%
  --(5.263,1.422)--(5.472,1.422)--(5.681,1.413)--(5.890,1.413)--(6.099,1.413)--(6.308,1.413)%
  --(6.517,1.413)--(6.726,1.413)--(6.934,1.413)--(7.143,1.413)--(7.352,1.413)--(7.561,1.413)%
  --(7.770,1.413)--(7.979,1.413)--(8.188,1.413)--(8.396,1.413)--(8.605,1.413)--(8.814,1.413)%
  --(9.023,1.413)--(9.232,1.413)--(9.441,1.413)--(9.650,1.413)--(9.858,1.413)--(10.067,1.413)%
  --(10.276,1.413)--(10.485,1.413)--(10.694,1.413)--(10.903,1.413)--(11.112,1.413)--(11.320,1.413)%
  --(11.529,1.413)--(11.738,1.412)--(11.947,1.412);
\gpsetlinewidth{1.00}
\draw[gp path] (1.504,8.381)--(1.504,0.985)--(11.947,0.985)--(11.947,8.381)--cycle;
%% coordinates of the plot area
\gpdefrectangularnode{gp plot 1}{\pgfpoint{1.504cm}{0.985cm}}{\pgfpoint{11.947cm}{8.381cm}}
\end{tikzpicture}
%% gnuplot variables
}
    \caption{Best fit individual by generation.}\label{f:popfitness}
\end{figure}

\begin{figure}
    \centering
    \resizebox{0.75\textwidth}{!}{\begin{tikzpicture}[gnuplot]
%% generated with GNUPLOT 5.0p3 (Lua 5.1; terminal rev. 99, script rev. 100)
%% Thu 29 Mar 2018 12:41:20 AM EDT
\gpmonochromelines
\path (0.000,0.000) rectangle (12.500,8.750);
\gpcolor{color=gp lt color border}
\gpsetlinetype{gp lt border}
\gpsetdashtype{gp dt solid}
\gpsetlinewidth{1.00}
\draw[gp path] (1.320,0.985)--(1.500,0.985);
\draw[gp path] (11.947,0.985)--(11.767,0.985);
\node[gp node right] at (1.136,0.985) {$0$};
\draw[gp path] (1.320,2.042)--(1.500,2.042);
\draw[gp path] (11.947,2.042)--(11.767,2.042);
\node[gp node right] at (1.136,2.042) {$50$};
\draw[gp path] (1.320,3.098)--(1.500,3.098);
\draw[gp path] (11.947,3.098)--(11.767,3.098);
\node[gp node right] at (1.136,3.098) {$100$};
\draw[gp path] (1.320,4.155)--(1.500,4.155);
\draw[gp path] (11.947,4.155)--(11.767,4.155);
\node[gp node right] at (1.136,4.155) {$150$};
\draw[gp path] (1.320,5.211)--(1.500,5.211);
\draw[gp path] (11.947,5.211)--(11.767,5.211);
\node[gp node right] at (1.136,5.211) {$200$};
\draw[gp path] (1.320,6.268)--(1.500,6.268);
\draw[gp path] (11.947,6.268)--(11.767,6.268);
\node[gp node right] at (1.136,6.268) {$250$};
\draw[gp path] (1.320,7.324)--(1.500,7.324);
\draw[gp path] (11.947,7.324)--(11.767,7.324);
\node[gp node right] at (1.136,7.324) {$300$};
\draw[gp path] (1.320,8.381)--(1.500,8.381);
\draw[gp path] (11.947,8.381)--(11.767,8.381);
\node[gp node right] at (1.136,8.381) {$350$};
\draw[gp path] (1.320,0.985)--(1.320,1.165);
\draw[gp path] (1.320,8.381)--(1.320,8.201);
\node[gp node center] at (1.320,0.677) {$0$};
\draw[gp path] (2.383,0.985)--(2.383,1.165);
\draw[gp path] (2.383,8.381)--(2.383,8.201);
\node[gp node center] at (2.383,0.677) {$5$};
\draw[gp path] (3.445,0.985)--(3.445,1.165);
\draw[gp path] (3.445,8.381)--(3.445,8.201);
\node[gp node center] at (3.445,0.677) {$10$};
\draw[gp path] (4.508,0.985)--(4.508,1.165);
\draw[gp path] (4.508,8.381)--(4.508,8.201);
\node[gp node center] at (4.508,0.677) {$15$};
\draw[gp path] (5.571,0.985)--(5.571,1.165);
\draw[gp path] (5.571,8.381)--(5.571,8.201);
\node[gp node center] at (5.571,0.677) {$20$};
\draw[gp path] (6.634,0.985)--(6.634,1.165);
\draw[gp path] (6.634,8.381)--(6.634,8.201);
\node[gp node center] at (6.634,0.677) {$25$};
\draw[gp path] (7.696,0.985)--(7.696,1.165);
\draw[gp path] (7.696,8.381)--(7.696,8.201);
\node[gp node center] at (7.696,0.677) {$30$};
\draw[gp path] (8.759,0.985)--(8.759,1.165);
\draw[gp path] (8.759,8.381)--(8.759,8.201);
\node[gp node center] at (8.759,0.677) {$35$};
\draw[gp path] (9.822,0.985)--(9.822,1.165);
\draw[gp path] (9.822,8.381)--(9.822,8.201);
\node[gp node center] at (9.822,0.677) {$40$};
\draw[gp path] (10.884,0.985)--(10.884,1.165);
\draw[gp path] (10.884,8.381)--(10.884,8.201);
\node[gp node center] at (10.884,0.677) {$45$};
\draw[gp path] (11.947,0.985)--(11.947,1.165);
\draw[gp path] (11.947,8.381)--(11.947,8.201);
\node[gp node center] at (11.947,0.677) {$50$};
\draw[gp path] (1.320,8.381)--(1.320,0.985)--(11.947,0.985)--(11.947,8.381)--cycle;
\node[gp node center,rotate=-270] at (0.246,4.683) {Cost};
\node[gp node center] at (6.633,0.215) {Generation};
\gpsetlinewidth{2.00}
\draw[gp path] (1.533,7.455)--(1.745,6.939)--(1.958,4.454)--(2.170,3.322)--(2.383,3.934)%
  --(2.595,2.881)--(2.808,1.525)--(3.020,1.652)--(3.233,1.266)--(3.445,1.191)--(3.658,2.313)%
  --(3.870,1.333)--(4.083,1.527)--(4.296,1.713)--(4.508,1.427)--(4.721,1.645)--(4.933,1.527)%
  --(5.146,1.922)--(5.358,2.322)--(5.571,2.245)--(5.783,1.494)--(5.996,1.182)--(6.208,1.370)%
  --(6.421,1.152)--(6.634,1.708)--(6.846,1.229)--(7.059,1.391)--(7.271,2.229)--(7.484,1.448)%
  --(7.696,1.027)--(7.909,1.600)--(8.121,2.021)--(8.334,1.369)--(8.546,2.894)--(8.759,1.263)%
  --(8.971,1.152)--(9.184,1.962)--(9.397,1.323)--(9.609,1.012)--(9.822,1.017)--(10.034,1.088)%
  --(10.247,2.234)--(10.459,1.960)--(10.672,1.337)--(10.884,1.523)--(11.097,1.002)--(11.309,1.311)%
  --(11.522,1.021)--(11.734,1.016)--(11.947,1.291);
\gpsetlinewidth{1.00}
\draw[gp path] (1.320,8.381)--(1.320,0.985)--(11.947,0.985)--(11.947,8.381)--cycle;
%% coordinates of the plot area
\gpdefrectangularnode{gp plot 1}{\pgfpoint{1.320cm}{0.985cm}}{\pgfpoint{11.947cm}{8.381cm}}
\end{tikzpicture}
%% gnuplot variables
}
    \caption{Individual fitness average by generation.}\label{f:popaverage}
\end{figure}

\begin{figure}
    \centering
    \resizebox{0.75\textwidth}{!}{\input{tikz/accforcecomp}}
    \caption{The force signals over time from the hand- and GA-tuned controllers.}\label{f:accforcecomp}
\end{figure}

\begin{figure}
    \begin{subfigmatrix}{2}
        \subfigure[$x_2$ input partition before GA
        tuning]{\resizebox{0.49\textwidth}{!}{\input{tikz/accx2plotcomp}}}
        \subfigure[$x_2$ input partition after GA
        tuning]{\resizebox{0.49\textwidth}{!}{\begin{tikzpicture}[gnuplot]
%% generated with GNUPLOT 5.0p3 (Lua 5.1; terminal rev. 99, script rev. 100)
%% Wed 28 Mar 2018 07:50:11 PM EDT
\gpmonochromelines
\path (0.000,0.000) rectangle (12.500,8.750);
\gpcolor{color=gp lt color border}
\gpsetlinetype{gp lt border}
\gpsetdashtype{gp dt solid}
\gpsetlinewidth{1.00}
\draw[gp path] (1.012,1.263)--(1.192,1.263);
\draw[gp path] (11.947,1.263)--(11.767,1.263);
\node[gp node right] at (0.828,1.263) {$0$};
\draw[gp path] (1.012,2.557)--(1.192,2.557);
\draw[gp path] (11.947,2.557)--(11.767,2.557);
\node[gp node right] at (0.828,2.557) {$0.2$};
\draw[gp path] (1.012,3.851)--(1.192,3.851);
\draw[gp path] (11.947,3.851)--(11.767,3.851);
\node[gp node right] at (0.828,3.851) {$0.4$};
\draw[gp path] (1.012,5.146)--(1.192,5.146);
\draw[gp path] (11.947,5.146)--(11.767,5.146);
\node[gp node right] at (0.828,5.146) {$0.6$};
\draw[gp path] (1.012,6.440)--(1.192,6.440);
\draw[gp path] (11.947,6.440)--(11.767,6.440);
\node[gp node right] at (0.828,6.440) {$0.8$};
\draw[gp path] (1.012,7.734)--(1.192,7.734);
\draw[gp path] (11.947,7.734)--(11.767,7.734);
\node[gp node right] at (0.828,7.734) {$1$};
\draw[gp path] (1.012,0.616)--(1.012,0.796);
\draw[gp path] (1.012,8.381)--(1.012,8.201);
\node[gp node center] at (1.012,0.308) {$0$};
\draw[gp path] (2.835,0.616)--(2.835,0.796);
\draw[gp path] (2.835,8.381)--(2.835,8.201);
\node[gp node center] at (2.835,0.308) {$20$};
\draw[gp path] (4.657,0.616)--(4.657,0.796);
\draw[gp path] (4.657,8.381)--(4.657,8.201);
\node[gp node center] at (4.657,0.308) {$40$};
\draw[gp path] (6.480,0.616)--(6.480,0.796);
\draw[gp path] (6.480,8.381)--(6.480,8.201);
\node[gp node center] at (6.480,0.308) {$60$};
\draw[gp path] (8.302,0.616)--(8.302,0.796);
\draw[gp path] (8.302,8.381)--(8.302,8.201);
\node[gp node center] at (8.302,0.308) {$80$};
\draw[gp path] (10.125,0.616)--(10.125,0.796);
\draw[gp path] (10.125,8.381)--(10.125,8.201);
\node[gp node center] at (10.125,0.308) {$100$};
\draw[gp path] (11.947,0.616)--(11.947,0.796);
\draw[gp path] (11.947,8.381)--(11.947,8.201);
\node[gp node center] at (11.947,0.308) {$120$};
\draw[gp path] (1.012,8.381)--(1.012,0.616)--(11.947,0.616)--(11.947,8.381)--cycle;
\draw[gp path] (7.995,4.036)--(7.995,4.960)--(11.763,4.960)--(11.763,4.036)--cycle;
\gpfill{color=gp lt color border,gp pattern 0,pattern color=.} (1.012,1.263)--(1.012,7.734)--(5.543,7.734)--(5.584,1.263)%
    --(10.099,1.263)--(10.125,1.263)--(10.150,1.263)--cycle;
\draw[gp path] (1.012,1.263)--(1.012,7.734)--(5.543,7.734)--(5.584,1.263)--(10.099,1.263)%
  --(10.125,1.263)--(10.150,1.263);
\gpfill{color=gp lt color border,gp pattern 1,pattern color=.} (1.012,1.263)--(1.012,1.263)--(5.543,1.263)--(5.584,7.734)%
    --(10.099,7.734)--(10.125,1.263)--(10.150,1.263)--cycle;
\gpsetdashtype{gp dt 2}
\draw[gp path] (1.012,1.263)--(5.543,1.263)--(5.584,7.734)--(10.099,7.734)--(10.125,1.263)%
  --(10.150,1.263);
\gpfill{color=gp lt color border,opacity=0.25} (1.012,1.263)--(1.012,1.263)--(5.543,1.263)--(5.584,1.263)%
    --(10.099,1.263)--(10.125,7.734)--(10.150,1.263)--cycle;
\gpsetdashtype{gp dt 3}
\draw[gp path] (1.012,1.263)--(5.543,1.263)--(5.584,1.263)--(10.099,1.263)--(10.125,7.734)%
  --(10.150,1.263);
\gpfill{color=gpbgfillcolor} (7.995,4.036)--(11.763,4.036)--(11.763,4.960)--(7.995,4.960)--cycle;
\gpsetdashtype{gp dt solid}
\draw[gp path] (7.995,4.036)--(7.995,4.960)--(11.763,4.960)--(11.763,4.036)--cycle;
\node[gp node right] at (10.479,4.806) {Far};
\gpfill{color=gp lt color border,gp pattern 0,pattern color=.} (10.663,4.729)--(11.579,4.729)--(11.579,4.883)--(10.663,4.883)--cycle;
\draw[gp path] (10.663,4.729)--(11.579,4.729)--(11.579,4.883)--(10.663,4.883)--cycle;
\node[gp node right] at (10.479,4.498) {Close};
\gpfill{color=gp lt color border,gp pattern 1,pattern color=.} (10.663,4.421)--(11.579,4.421)--(11.579,4.575)--(10.663,4.575)--cycle;
\gpsetdashtype{gp dt 2}
\draw[gp path] (10.663,4.421)--(11.579,4.421)--(11.579,4.575)--(10.663,4.575)--cycle;
\node[gp node right] at (10.479,4.190) {VeryClose};
\gpfill{color=gp lt color border,opacity=0.25} (10.663,4.113)--(11.579,4.113)--(11.579,4.267)--(10.663,4.267)--cycle;
\gpsetdashtype{gp dt 3}
\draw[gp path] (10.663,4.113)--(11.579,4.113)--(11.579,4.267)--(10.663,4.267)--cycle;
\gpsetdashtype{gp dt solid}
\draw[gp path] (1.012,8.381)--(1.012,0.616)--(11.947,0.616)--(11.947,8.381)--cycle;
%% coordinates of the plot area
\gpdefrectangularnode{gp plot 1}{\pgfpoint{1.012cm}{0.616cm}}{\pgfpoint{11.947cm}{8.381cm}}
\end{tikzpicture}
%% gnuplot variables
}}
        \subfigure[$\dot{x}_2$ input partition before GA
        tuning]{\resizebox{0.49\textwidth}{!}{\begin{tikzpicture}[gnuplot]
%% generated with GNUPLOT 5.0p3 (Lua 5.1; terminal rev. 99, script rev. 100)
%% Wed 28 Mar 2018 10:34:02 PM EDT
\gpmonochromelines
\path (0.000,0.000) rectangle (12.500,8.750);
\gpcolor{color=gp lt color border}
\gpsetlinetype{gp lt border}
\gpsetdashtype{gp dt solid}
\gpsetlinewidth{1.00}
\draw[gp path] (1.012,1.263)--(1.192,1.263);
\draw[gp path] (11.947,1.263)--(11.767,1.263);
\node[gp node right] at (0.828,1.263) {$0$};
\draw[gp path] (1.012,2.557)--(1.192,2.557);
\draw[gp path] (11.947,2.557)--(11.767,2.557);
\node[gp node right] at (0.828,2.557) {$0.2$};
\draw[gp path] (1.012,3.851)--(1.192,3.851);
\draw[gp path] (11.947,3.851)--(11.767,3.851);
\node[gp node right] at (0.828,3.851) {$0.4$};
\draw[gp path] (1.012,5.146)--(1.192,5.146);
\draw[gp path] (11.947,5.146)--(11.767,5.146);
\node[gp node right] at (0.828,5.146) {$0.6$};
\draw[gp path] (1.012,6.440)--(1.192,6.440);
\draw[gp path] (11.947,6.440)--(11.767,6.440);
\node[gp node right] at (0.828,6.440) {$0.8$};
\draw[gp path] (1.012,7.734)--(1.192,7.734);
\draw[gp path] (11.947,7.734)--(11.767,7.734);
\node[gp node right] at (0.828,7.734) {$1$};
\draw[gp path] (1.012,0.616)--(1.012,0.796);
\draw[gp path] (1.012,8.381)--(1.012,8.201);
\node[gp node center] at (1.012,0.308) {$-2$};
\draw[gp path] (2.379,0.616)--(2.379,0.796);
\draw[gp path] (2.379,8.381)--(2.379,8.201);
\node[gp node center] at (2.379,0.308) {$-1.5$};
\draw[gp path] (3.746,0.616)--(3.746,0.796);
\draw[gp path] (3.746,8.381)--(3.746,8.201);
\node[gp node center] at (3.746,0.308) {$-1$};
\draw[gp path] (5.113,0.616)--(5.113,0.796);
\draw[gp path] (5.113,8.381)--(5.113,8.201);
\node[gp node center] at (5.113,0.308) {$-0.5$};
\draw[gp path] (6.480,0.616)--(6.480,0.796);
\draw[gp path] (6.480,8.381)--(6.480,8.201);
\node[gp node center] at (6.480,0.308) {$0$};
\draw[gp path] (7.846,0.616)--(7.846,0.796);
\draw[gp path] (7.846,8.381)--(7.846,8.201);
\node[gp node center] at (7.846,0.308) {$0.5$};
\draw[gp path] (9.213,0.616)--(9.213,0.796);
\draw[gp path] (9.213,8.381)--(9.213,8.201);
\node[gp node center] at (9.213,0.308) {$1$};
\draw[gp path] (10.580,0.616)--(10.580,0.796);
\draw[gp path] (10.580,8.381)--(10.580,8.201);
\node[gp node center] at (10.580,0.308) {$1.5$};
\draw[gp path] (11.947,0.616)--(11.947,0.796);
\draw[gp path] (11.947,8.381)--(11.947,8.201);
\node[gp node center] at (11.947,0.308) {$2$};
\draw[gp path] (1.012,8.381)--(1.012,0.616)--(11.947,0.616)--(11.947,8.381)--cycle;
\draw[gp path] (8.179,4.036)--(8.179,4.960)--(11.763,4.960)--(11.763,4.036)--cycle;
\gpfill{color=gp lt color border,gp pattern 0,pattern color=.} (1.012,1.263)--(1.012,7.734)--(5.716,7.734)--(6.480,1.263)%
    --(7.243,1.263)--(11.947,1.263)--(11.947,1.263)--cycle;
\draw[gp path] (1.012,1.263)--(1.012,7.734)--(5.716,7.734)--(6.480,1.263)--(7.243,1.263)%
  --(11.947,1.263);
\gpfill{color=gp lt color border,gp pattern 1,pattern color=.} (1.012,1.263)--(1.012,1.263)--(5.716,1.263)--(6.480,7.734)%
    --(7.243,1.263)--(11.947,1.263)--(11.947,1.263)--cycle;
\gpsetdashtype{gp dt 2}
\draw[gp path] (1.012,1.263)--(5.716,1.263)--(6.480,7.734)--(7.243,1.263)--(11.947,1.263);
\gpfill{color=gp lt color border,opacity=0.25} (1.012,1.263)--(1.012,1.263)--(5.716,1.263)--(6.480,1.263)%
    --(7.243,7.734)--(11.947,7.734)--(11.947,1.263)--cycle;
\gpsetdashtype{gp dt 3}
\draw[gp path] (1.012,1.263)--(5.716,1.263)--(6.480,1.263)--(7.243,7.734)--(11.947,7.734)%
  --(11.947,1.263);
\gpfill{color=gpbgfillcolor} (8.179,4.036)--(11.763,4.036)--(11.763,4.960)--(8.179,4.960)--cycle;
\gpsetdashtype{gp dt solid}
\draw[gp path] (8.179,4.036)--(8.179,4.960)--(11.763,4.960)--(11.763,4.036)--cycle;
\node[gp node right] at (10.479,4.806) {Negative};
\gpfill{color=gp lt color border,gp pattern 0,pattern color=.} (10.663,4.729)--(11.579,4.729)--(11.579,4.883)--(10.663,4.883)--cycle;
\draw[gp path] (10.663,4.729)--(11.579,4.729)--(11.579,4.883)--(10.663,4.883)--cycle;
\node[gp node right] at (10.479,4.498) {Zero};
\gpfill{color=gp lt color border,gp pattern 1,pattern color=.} (10.663,4.421)--(11.579,4.421)--(11.579,4.575)--(10.663,4.575)--cycle;
\gpsetdashtype{gp dt 2}
\draw[gp path] (10.663,4.421)--(11.579,4.421)--(11.579,4.575)--(10.663,4.575)--cycle;
\node[gp node right] at (10.479,4.190) {Positive};
\gpfill{color=gp lt color border,opacity=0.25} (10.663,4.113)--(11.579,4.113)--(11.579,4.267)--(10.663,4.267)--cycle;
\gpsetdashtype{gp dt 3}
\draw[gp path] (10.663,4.113)--(11.579,4.113)--(11.579,4.267)--(10.663,4.267)--cycle;
\gpsetdashtype{gp dt solid}
\draw[gp path] (1.012,8.381)--(1.012,0.616)--(11.947,0.616)--(11.947,8.381)--cycle;
%% coordinates of the plot area
\gpdefrectangularnode{gp plot 1}{\pgfpoint{1.012cm}{0.616cm}}{\pgfpoint{11.947cm}{8.381cm}}
\end{tikzpicture}
%% gnuplot variables
}}
        \subfigure[$\dot{x}_2$ input partition after GA
        tuning]{\resizebox{0.49\textwidth}{!}{\input{tikz/gaaccx2dotplot}}}
        \subfigure[Force output partition before GA
        tuning]{\resizebox{0.49\textwidth}{!}{\input{tikz/accoutplotcomp}}}
        \subfigure[Force output partition after GA
        tuning]{\resizebox{0.49\textwidth}{!}{\input{tikz/gaaccoutplot}}}
    \end{subfigmatrix} \caption{FIS comparison between hand- and GA-tuned membership
    functions}\label{f:fiscomp}
\end{figure}

\subsection{Genetic Adaptability}
These results demonstrate the ability of the genetic algorithm to tune a FIS
to near-optimal performance. All tuning and development heretofore was done with an unchanging system setup of
known masses connected by a known spring. Although a robust controller\cite{cohen:01jgcd}, introducing changes
to the masses of each car significantly alters the performance of the FIS as it was carefully tuned to only a
certain envelope; however, utilizing the genetic algorithm to generate an optimum FIS for each new system
setup is an efficient method of developing good active controllers. This is demonstrated by changing the
masses $m_1$ and $m_2$ to \SI{2}{\kilogram} and \SI{4}{\kilogram} respectively. They are again changed to
\SI{4}{\kilogram} and \SI{8}{\kilogram}, and finally \SI{4}{\kilogram} and \SI{16}{\kilogram}. The algorithm
was deployed for each case to optimize a controller for that envelope. After fifty generations of evolution,
the genetically optimized FIS performed within 3\% of the theoretical rigid body limit in all four cases.
These results are displayed in \cref{tab:gacomp}.  \begin{table} \centering \caption{Genetic algorithm
    FIS performance compared to the hand-tuned FIS and rigid body limit.} \label{tab:gacomp}
    \begin{tabular}{|c|c|c|c|c|} \cline{2-5} \multicolumn{1}{c|}{} & \multicolumn{4}{|c|}{Mass 1
        (\si{\kilogram}), Mass 2 (\si{\kilogram})} \\\cline{2-5} \multicolumn{1}{c|}{} & \SI{1}{\kilogram},
        \SI{2}{\kilogram} & \SI{2}{\kilogram}, \SI{4}{\kilogram} & \SI{4}{\kilogram}, \SI{8}{\kilogram} &
        \SI{4}{\kilogram}, \SI{16}{\kilogram} \\\hline Theoretical Limit & 0.3191 & 0.4553 & 0.6438 & 0.8312
        \\\hline GA FIS & 0.3231 & 0.4562 & 0.6579 & 0.8434 \\\hline Hand-tuned FIS & 0.3233 & 0.6125 & 5.3072
                        & 3.3240 \\\hline\hline GA Error & \multicolumn{1}{|d|}{1.3\%} &
        \multicolumn{1}{|d|}{0.2\%} & \multicolumn{1}{|d|}{2.2\%} & \multicolumn{1}{|d|}{1.5\%} \\\hline
        Hand-tuned Error & \multicolumn{1}{|d|}{1.3\%} & \multicolumn{1}{|d|}{34.5\%} &
        \multicolumn{1}{|d|}{724.4\%} & \multicolumn{1}{|d|}{299.9\%} \\\hline \end{tabular} \end{table}

It is easily seen in these results that the use of the genetic algorithm is advantageous in the autonomous
development of near optimal FIS controllers. Given a generic control architecture, the genetic algorithm is
able to tune a FIS rapidly and accurately for a varied set of circumstances.

\section{Conclusions} Fuzzy logic provides a robust framework for control. It has
been demonstrated that proper fuzzy control is efficient and computationally inexpensive. The inherent
vagueness of set membership and linguistic operation of fuzzy logic allows the controller to mimic expert
human control. This superior control, however, comes with a steep cost in FIS development. Hand-tuning a FIS
is time-consuming and tedious.


The use of the genetic algorithm facilitates FIS development. Once a FIS has been developed for a general type
of control situation, it is relatively simple to define the FIS as a genetic element and automate the tuning
through the evolutionary process. These results imply that if a general fuzzy controller is developed for a
family of control situations, then a genetic algorithm can be implemented to tune each FIS to its specific
task. The tuning, therefore, can be accomplished by someone with no expertise in the control of the situation.
As the computation is quick, efficient control could be widely distributed due also to the low-cost of
development.


