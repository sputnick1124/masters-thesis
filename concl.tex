\chapter{Conclusions}

\section{Future Work}
It is the author's hope that the work will be continued in the laboratoy and eventually applied to hardware.
The use of ROS to implement the control architecture allows this control architecture to operate in physical
hardware as has been shown in other projects in the laboratory environment in recent projects. The barrier to
actualizing the hardware platform is the current lack of a reliable indoor positioning system. The lab used to
facilitate this research contains a motion capture system, but it has only recently been used enough to
understand how to integrate it into flying platforms repeatably, thus enabling further hardware testing of
this project.

Also, there is still much work which could be done with regards to the components of the controller itself.
There are many corollaries to the controller which were not much explored due to the limited scope of the
thesis. The computer vision algorithm is very simplistic and could be expanded greatly to produce a much more
robust and versatile application. Also, the EKF is virtually untuned and could stand a great deal of work to
produce a cleaner estimate signal.

Finally, integration of this work with other projects within the UAV Master Labs at the University of
Cincinnati would facilitate this project in being used in real applications. The controller could rather
easily be ported to be a module within the  $\mathbbm{FlyMASTER}$ framework. This would ease the integration
of this work with other projects in the future and work towards the goal of unifying many projects in the lab
towards a greater whole.

\section{Summary}
Genetic Fuzzy Systems are not a panacea. They are not magic. They are not a silver bullet. These are truths
which can be said of any architecture, paradigm, or approach. It becomes the job of an engineer to know the
breadth of tools at their disposal and be able to chose the best for a job. It is the proposition of this
thesis that FL systems should be considered as first class citizens in the tool chest of any controls
engineer. They are more than capable of performing the low-level control tasks required in dynamics problems
and are remarkably well-suited to make decisions at higher levels as well. The ability of FLS to translate the
intuition of expert controls designers into machine-executable signals is a unique benefit that is worth
exploring for a wide array of problems we face every day.

It was shown in this work that the combination of manual construction and genetic tuning of a FLS produces a
controller which performs near-optimally in a highly non-linear system. The intuition gained from manual
interaction with system translates directly into controller development very well. It was also shown that a
FLS which is learned almost solely via a GA can attain results which meet any number of physically realizable
specifications. The expert intuition in these cases emerges in the development of a fitness (or cost) function
which accurately describes the desired behavior. Finally, it was shown that purely hand-tuned system can
perform remarkably well. This system is perhaps far from optimal, but has the distinction of being
implementable on affordable hardware in real time. The integation with ROS also facilitates its use in a wide
array of robotics applications across many domains

