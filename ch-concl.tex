\chapter{Conclusions}

\section{Summary}
Genetic Fuzzy Systems are not a silver bullet solution to every problem; however, this can be said of any
architecture, paradigm, or approach. It becomes the job of an engineer to know the breadth of tools at their
disposal and be able to chose the best for a job. It is the proposition of this thesis that FL systems should
be considered as first-class citizens in the tool chest of any controls engineer. They are more than capable
of performing the low-level control tasks required in dynamics problems and are remarkably well-suited to make
decisions at higher levels as well. The ability of FLS to translate the intuition of expert controls designers
into machine-executable signals is a unique benefit that is worth exploring for a wide array of problems we
face every day.

It was shown in this work that the combination of manual construction and genetic tuning of a FLS produces a
controller which performs near-optimally in a highly non-linear system. It was also shown that a
FLS which is learned almost solely via a GA can attain results which meet any number of physically realizable
specifications. The expert intuition in these cases emerges in the development of a fitness (or cost) function
which accurately describes the desired behavior. Finally, it was shown that a purely hand-tuned system can
perform remarkably well. This system is perhaps far from optimal, but has the distinction of being
implementable on affordable hardware in real time. 

In conclusion, fuzzy logic systems and genetic fuzzy systems are valuable tools in the controls engineering
field and should be considered as viable alternatives when working with many systems. Indeed, there are many
applications in which FL control provides a nearly optimal controller and does so with minimal computational
overhead and little control effort. 

\section{Future Work}
This thesis opens the door to many opportunities for future work. First, there is still much work which could
be done with regards to the components of the controller itself. There are many corollaries to the controller
which remain unexplored. The computer vision algorithm is simplified and could be expanded greatly to produce
a much more robust and versatile application. Also, the EKF is minimally tuned, more work could be done to
produce a cleaner estimate signal.

Second, this thesis showed the effectiveness of landing a FL-controlled multirotor aircraft on a moving
platform. It is the author's belief that layering a GA-automated tuning process to the simulation would result
in even better performance. Some preliminary work has been done in developing a cost function which will aid
the development of evolutionary approaches in the future.

Third, it is the author's hope that the work will be continued in the laboratory and eventually be applied to
hardware.  The use of ROS to implement the control architecture allows this control architecture to operate in
physical hardware, as has been shown in other projects in the laboratory environment in recent projects.

Finally, integration of this work with other projects within the UAV Master Labs at the University of
Cincinnati would facilitate this project in being used in real applications. The controller could rather
easily be ported into the $\mathbbm{FlyMASTER}$ framework as a controller module. This would ease the
integration of this work with other projects in the future and work towards the goal of unifying many projects
in the lab towards a greater whole.

