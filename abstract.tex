Aerospace applications are composed of many dynamic systems which are coupled, nonlinear, and difficult to
control. Fuzzy logic (FL) systems  provides a means by which to encode expert knowledge into a set of rules
which can produce highly nonlinear control signals; this is possible because FL, like many other soft
computational methods is a universal approximator. While FL systems excel at encapsulating expert knowledge
bases, when coupled with genetic algorithms (GA), they can learn the knowledge base from evolutionary
repetition. It is the goal of this work to present the efficacy of hybrid genetic fuzzy systems (GFS) in a
variety of applications.

First, a sample problem presented at the 1990 American Control Conference is used to demonstrate the
robustness of FL control as well as the utility of GAs in the learning process. The results are a controller
that is far more resistant to even large changes in the plant dynamics compared to a linear controller.

The next problem applies the same approach to an elevator actuator for pitch control of an F-4 Phantom. This
controller is tuned for a nominal case and ten subjected to the same plant with degraded aerodynamic
coefficients. It is compared to a well-tuned PID controller.

The effort culminates in a practical application of a FL system to guide a small unmanned aerial system (sUAS)
to a precision landing on a target platform moving with uncertain velocity. This was accomplished using a
custom developed Python for GFS control in conjunction with Robot Operating System (ROS) and a simulation
environment called Gazebo. Heavy emphasis was placed on using only software components which an be easily
implemented on popular hardware platforms. ROS was critical to meeting this goal, as well as the open source
flight controller project PX4. Many unanticipated drawbacks surfaced using this approach, and are discussed in
detail. A great effort was made to apply GA learning to the final controller, but the end result is a
hand-tuned controller controlled which is able to perform the task admirably.

